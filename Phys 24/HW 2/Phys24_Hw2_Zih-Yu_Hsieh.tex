%Phys24_Hw2_Zih-Yu_Hsieh.tex

\documentclass{article}
\usepackage{graphicx} % Required for inserting images
\usepackage[margin = 2.54cm]{geometry}
\usepackage[most]{tcolorbox}

\newtcolorbox{myBox}[3]{
arc=5mm,
lower separated=false,
fonttitle=\bfseries,
%colbacktitle=green!10,
%coltitle=green!50!black,
enhanced,
attach boxed title to top left={xshift=0.5cm,
        yshift=-2mm},
colframe=blue!50!black,
colback=blue!10
}

\usepackage{amsmath}
\usepackage{amssymb}
\usepackage{verbatim}
\usepackage[utf8]{inputenc}
%\linespread{1.5}
\usepackage{fancyhdr}

\newtheorem{definition}{Definition}
\newtheorem{proposition}{Proposition}
\newtheorem{theorem}{Theorem}
\newtheorem{question}{Question}

\title{Phys 24 HW2}
\author{Zih-Yu Hsieh}

\fancypagestyle{plain}{%
   \fancyhead[L]\textbf{Hsieh}
   \renewcommand{\headrulewidth}{0pt}
}

\begin{document}
\maketitle

\section*{1}
\begin{myBox}[]{}
    \begin{question}
        Suppose the current I that flows in the circuit in the figure below is
        20 amperes. The distance between the wires is 5 cm. How large
        is the force, per meter of length, that pushes horizontally on one of
        the wires?
    \end{question}

    \begin{center}
        \includegraphics*[width=40mm]{Purcell 6.33.png}
    \end{center}
\end{myBox}

\textbf{Pf:}

Assume the wires are long enough such that one can approximate the magnetic field using a straight thread of current with indefinite length.
For the problem, let the wire (on the right in the graph) be pointing in the $\hat{z}$ direction, and use cylindrical coordinate for it.
Our first goal is to establish the magnetic field generated by the wire on the right.

\hfill

First, let $\bar{B}=B_r\hat{r}+B_\phi\hat{\phi}+B_z\hat{z}$, which $B_r,B_\phi,B_z$ are dependent on the position.
Because we assume the wire is straight and indefinitely long, then it is not dependent on the $z$ coordinate; also, it implies that $B_z=0$ (meaning no magnetic field toward the z-axis, due to infinite symmetry).

Then, due to rotational symmetry, regardless of the $\phi$ coordinate, the magnetic field would look identical, thus $B_z, B_\phi$ is only dependent on radius $r$.
So, $\bar{B}=B_r\hat{r}+B_\phi\hat{\phi}$.

\hfill

Now, construct a regular cylinder with center axis being the wire with radius $r$ and height $l$. Which, by Maxwell's Equation, $\int_S\bar{B}\cdot d\bar{s} = 0$.
If consider the top and bottom disk of the cylinder (normal vector in $\hat{z}$ direction), then since $\hat{z}$ and $\bar{B}$ are orthogonal (one is in $\hat{z}$, while the other is in the plane made by $\hat{r},\hat{\phi}$),
thus the magnetic flux through the top and bottom is $0$.

We just need to consider the flux on the side, which since the side has normal vector $\hat{r}$, the flux is given as follow:
$$\int_{S}\bar{B}\cdot d\bar{s} = \int_{S}(B_r\hat{r}+B_\phi\hat{\phi})\cdot \hat{r}dA = \int_{S}B_rdA$$
Notice that since $r$ is fixed, $B_r$ is a constant; also, since the total flux is $0$, then the above quantity is also $0$.
Which, because the side has nonzero area (given as $2\pi rl$), then for the quantity to be $0$, we need $B_r = 0$.
Thus, $\bar{B}=B_\phi\hat{\phi}$.

\hfill

Then, construct an Amperian loop, with center at the wire and radius $r$. Again, by Maxwell's Equation, $\int_{\partial A}\bar{B}\cdot d\bar{s}=\mu_0I$.
Which, $\int_{\partial A}\bar{B}\cdot d\bar{s} = \int_{0}^{2\pi}B_\phi\hat{\phi}\cdot r\hat{\phi}d\phi = B_\phi 2\pi r$ (since $r$ is fixed, $B_\phi$ is again a constant).

So, $B_\phi 2\pi r = \mu_0 I$, which $B_\phi = \frac{\mu_0I}{2\pi r}$.

Hence, the magnetic field of the wire on the right is $\bar{B}=\frac{\mu_0I}{2\pi r}\hat{\phi}$.

\hfill

\textbf{Plug in:}

Before that, since both wires are straight, then the small change in current component $d\hat{r}$ and the displacement vector $\bar{r}-\bar{r}'$ on the wire itself are actually parallel,
hence $d\bar{r}\times (\bar{r}-\bar{r}')=\bar{0}$, showing that a straight wire is not exerting force on itself by its own magnetic field.

\hfill

Thus, to consider the affect on the left wire, only need to consider the magnetic field generated by the right wire, with $\bar{B}=\frac{\mu_0I}{2\pi r}\hat{\phi}$.

If consider just the force on one meter of the left wire, since the left wire has current $I$ traveling in the $-\hat{z}$ direction (opposite from the right wire),
then the force on it is given by:
$$\bar{F}=I\int_{0}^{1}d\bar{r}\times \bar{B}(\bar{r}) = I\int_{0}^{1}-dr\hat{z}\times \hat{\phi}\frac{\mu_0I}{2\pi r} = \frac{\mu_0I^2}{2\pi r}\int_{0}^{1}dr (\hat{\phi}\times \hat{z}) = \frac{\mu_0I^2}{2\pi r}\hat{r}$$
Given that $r=5cm = 0.05m$, $I=20A$, and $\mu_0 = 1.257\cdot 10^{-6}N\cdot A^{-2}$, the magnitude of the force is:
$$\|\bar{F}\|=\frac{\mu_0I^2}{2\pi r}=\frac{1.257\cdot 10^{-6}\cdot (20)^2}{2\pi \cdot 0.05} \approx 0.0016 N$$
So, the magnitude of force per meter of length, is about $1.6 \cdot 10^{-3}N/m$.

\hfill

\hfill

\section*{2}
\begin{myBox}[]{}
    \begin{question}
        A long copper rod 8 cm in diameter has an off-center cylindrical
        hole, as shown below, down its full length. This conductor
        carries a current of 900 amps flowing in the direction “into the
        paper.” What is the direction, and strength in gauss, of the magnetic
        field at the point P that lies on the axis of the outer cylinder?
    \end{question}

    \begin{center}
        \includegraphics*[width=40mm]{Purcell 6.37.png}
    \end{center}
\end{myBox}

\textbf{Pf:}

Instead of thinking it as having $900A$ of current flowing into the paper under the remaining region,
we can think of it as two currents combined - $900A$ going into the page with the larger copper rod, 
and $900A$ coming out of the page with the smaller hole (so in the hole region, the net current is $0$, identical to have noting flowing through it).

Let the direction pointing into the page be $\hat{z}$, and use the cylindrical coordinate.

Again, based on the same symmetry argument done in \textbf{Question 1}, we can assume the magnetic field is purely in angular direction $\hat{\phi}$.

\hfill

\textbf{Field of current flowing into the page:}

Construct an Amperian loop with radius $r<4cm$ (diameter $d<8cm$) centered at $P$ (big disk's center) and travel in $\hat{\phi}$ direction.

Since the wire has $I=900A$ flowing into the page, while the cross-section area of the whole disk is $\pi\cdot 4 cm^2$, then the amount of current enclosed by
the loop is $I\cdot \frac{\pi r^2}{\pi \cdot 4^2} = \frac{r^2}{4^2}I$.

Now, by Maxwell's Equation, since $\int_{\partial A}\bar{B}\cdot d\bar{s}=\mu_0 I$ ($I$ is the enclosed current, here it is $\frac{r^2}{4}I$), while the magnetic field $\bar{B}=B_\phi\hat{\phi}$, and the spacial differential $d\bar{s}=r\hat{\phi}d\phi$,
then the integral is given by:
$$\int_{\partial A}\bar{B}\cdot d\bar{s}=\int_{0}^{2\pi}(A_\phi\hat{\phi})\cdot rd\phi \hat{\phi} = 2\pi r B_\phi$$
Which, $B_\phi$ is again a constant given that $r$ is fixed. So, $2\pi rB_\phi = \frac{\mu_0r^2}{4^2}I$, then $B_\phi = \frac{\mu_0r}{32\pi}I$.

Since point $P$ is at the center, then $r=0$, thus $B_\phi = 0$, showing that the current going into the page is not producing a net field at point $P$.

\hfill

\textbf{Field of current flowing out the page:}

Now, construct an Amperian loop wih radius $r= 2cm$ (diameter $d= 4 cm$) concentric to the small hole, and travel in $\hat{\phi}$ direction.

Since the wire has $I=900A$ flowing out the page (or $-I=-900A$ flowing into the page), the ammount of current enclosed by the loop,
is the whole current (everything in the small hole), which is $-I$.

Again, by Maxwell's equation, $\int_{\partial A}\bar{B}\cdot d\bar{s}=-\mu_0 I$ (because with respect to the corrdinate, the current here is negative), while using similar equation as the previous section:
$$\int_{\partial A}\bar{B}\cdot d\bar{s}=\int_{0}^{2\pi}(A_\phi\hat{\phi})\cdot rd\phi \hat{\phi} = 2\pi r B_\phi$$
Which, $B_\phi$ is again a constant given that $r$ is fixed. So, $2\pi rB_\phi = -\mu_0I$, then $B_\phi = -\frac{\mu_0}{2\pi r}I$.
Since point $P$ is $2cm$ away from the center, then $\bar{B}(P)=-\frac{\mu_0}{2\pi \cdot 2}I\hat{\phi}$, and since $\hat{z}$ is pointing into the page,
here $\hat{\phi}$ is pointing to the right, or $\bar{B}(P)$ is pointing to the left.

\hfill

Add the two fields up, since the current pointing into the page (provided by the outer cylinder) provide $\bar{B}=\bar{0}$, then the field is purely provided by the current out of the page (by the hole), or $\bar{B}(P)=-\frac{\mu_0}{2\pi \cdot 2}I\hat{\phi}$.

Given that $\mu_0=1.257\cdot 10^{-6}N\cdot A^{-2}$, $I=900A$, $2$ above is in $cm$, thus convert to $0.02m$, then the magnitude of the field is:
$$\|\bar{B}(P)\|=\frac{\mu_0I}{2\pi\cdot 0.02} = \frac{1.257\cdot 10^{-6}\cdot 900}{2\pi \cdot 0.02} = 0.009 N/(A\cdot m)$$
Since $1G = 10^{-4}T = 10^{-4}N/(A\cdot m)$, while $\|\bar{B}(P)\| = 9\cdot 10^{-3} N/(A\cdot m)=90 \cdot 10^{-4}T$,
then $\|\bar{B}(P)\| = 90G$.

Also, from the previous part, it showed that the direction of the field is to the left of the page.

\break

\section*{3}
\begin{myBox}[]{}
    \begin{question}
        A round wire of radius $r_0$ carries a current $I$ distributed uniformly
        over the cross section of the wire. Let the axis of the wire be the $z$
        axis, with $\hat{z}$ the direction of the current. Show that a vector potential 
        of the form $\bar{A}=A_0\hat{z}(x^2+y^2)$ will correctly give the magnetic
        field $\bar{B}$ of this current at all points inside the wire. What is the value
        of the constant, $A_0$?
    \end{question}
\end{myBox}

\textbf{Pf:}

Given that $\bar{A}(\bar{r})=A_0\hat{z}(x^2+y^2)$ in cartesian coordinate, or $\bar{A}(\bar{r})=A_0r^2\hat{z} = 0\hat{r}+0\hat{\phi}+A_0r^2\hat{z}$ in cylindrical coordinate.
Which, consider the curl in cylindrical coordinate, the following is true:
$$\nabla\times \bar{A}=\frac{1}{r}\left(\frac{\partial A_z}{\partial \phi}-\frac{\partial A_\phi}{\partial z}\right)\hat{r}+\left(\frac{\partial A_z}{\partial r}-\frac{\partial A_r}{\partial z}\right)\hat{\phi}+\frac{1}{r}\left(\frac{\partial rA_\phi}{\partial r}-\frac{\partial A_r}{\partial \phi}\right)\hat{r}$$
$$=\frac{1}{r}\left(0-0\right)\hat{r}+\left(\frac{\partial}{\partial r}(A_0r^2)-0\right)\hat{\phi}+\frac{1}{r}\left(\frac{\partial r\cdot 0}{\partial r}-0\right)\hat{r} =2A_0r\hat{\phi}$$
So, $\bar{B}(\bar{r})=\nabla\times \bar{A}=2A_0r\hat{\phi}$.

\hfill

Now, using the same argument done in \textbf{Question 1}, we can argue that $\bar{B}=B_\phi(r)\hat{\phi}$, solely in the $\phi$ direction, and the magnitued is only dependent on the radial coordinate.
Then, construct an Amperian loop with radius $r<r_0$, concentric to the round wire's axis going in $\hat{\phi}$ direction. Then, based on Maxwell's Equation, we can establish the follow:
$$\int_{\partial A}\bar{B}\cdot d\bar{s}=\mu_0 I_{enc}$$
Which, $\int_{\partial A}\bar{B}\cdot d\bar{s}=\|\bar{B}\|\cdot 2\pi r = 2\pi rB_\phi(r)$ (since $\bar{B}$ is parallel to the moving direction, while the magnitude is constant).

Also, since the loop encloses area $\pi r^2$, while the total area of the wire $\pi r_{0}^2$, so the enclosed current is:
$$I_{enc}=\frac{\pi r^2}{\pi r_0^2}I = \frac{r^2}{r_0^2}I$$
Then, the following is true:
$$2\pi rB_\phi(r)=\mu_0 I_{enc}=\mu_0\frac{r^2}{r_0^2}I,\quad B_\phi(r)=\frac{\mu_0r}{2\pi r_0^2}I$$
Hence, $\bar{B}(\bar{r})=\frac{\mu_0r}{2\pi r_0^2}I\hat{\phi}=2A_0r\hat{\phi}$, so: 
$$A_0 = \frac{\mu_0I}{4\pi r_0^2}$$
So, we can say the given vector potential agrees with the 

\break

\section*{4}
\begin{myBox}[]{}
    \begin{question}
        In class, we calculate the vector potential $\bar{A}(\bar{r})$ and the magnetic induction $\bar{B}(\bar{r})$ of a circular
        thread of current. The current density is given in cylindrical coordinates as
        $$\bar{J}(\bar{r})=I\delta(\rho-R)\delta(z)\hat{\phi}$$
        See lecture 4, page 1.44 and continued. The calculation of the vector potential $\bar{A}(\bar{r})$ leads to
        an ellipitic integral, which cannot be solved analytically. Estimate the integral for the limit
        $\rho>> R$, by a Taylor expansion. Show that in this case a dipole field emerges!

    \end{question}
\end{myBox}

\end{document}