%Phys24_Hw2_Zih-Yu_Hsieh.tex

\documentclass{article}
\usepackage{graphicx} % Required for inserting images
\usepackage[margin = 2.54cm]{geometry}
\usepackage[most]{tcolorbox}

\newtcolorbox{myBox}[3]{
arc=5mm,
lower separated=false,
fonttitle=\bfseries,
%colbacktitle=green!10,
%coltitle=green!50!black,
enhanced,
attach boxed title to top left={xshift=0.5cm,
        yshift=-2mm},
colframe=blue!50!black,
colback=blue!10
}

\usepackage{amsmath}
\usepackage{amssymb}
\usepackage{verbatim}
\usepackage[utf8]{inputenc}
%\linespread{1.5}
\usepackage{fancyhdr}

\newtheorem{definition}{Definition}
\newtheorem{proposition}{Proposition}
\newtheorem{theorem}{Theorem}
\newtheorem{question}{Question}

\title{Phys 24 HW2}
\author{Zih-Yu Hsieh}

\fancypagestyle{plain}{%
   \fancyhead[L]\textbf{Hsieh}
   \renewcommand{\headrulewidth}{0pt}
}

\begin{document}
\maketitle

\section*{1}
\begin{myBox}[]{}
    \begin{question}
        Suppose the current I that flows in the circuit in the figure below is
        20 amperes. The distance between the wires is 5 cm. How large
        is the force, per meter of length, that pushes horizontally on one of
        the wires?
    \end{question}

    \begin{center}
        \includegraphics*[width=40mm]{Purcell 6.33.png}
    \end{center}
\end{myBox}

\break

\section*{2}
\begin{myBox}[]{}
    \begin{question}
        A long copper rod 8 cm in diameter has an off-center cylindrical
        hole, as shown in Fig. 6.43, down its full length. This conductor
        carries a current of 900 amps flowing in the direction “into the
        paper.” What is the direction, and strength in gauss, of the magnetic
        field at the point P that lies on the axis of the outer cylinder?
    \end{question}

    \begin{center}
        \includegraphics*[width=40mm]{Purcell 6.37.png}
    \end{center}
\end{myBox}

\break

\section*{3}
\begin{myBox}[]{}
    \begin{question}
        A round wire of radius $r_0$ carries a current $I$ distributed uniformly
        over the cross section of the wire. Let the axis of the wire be the $z$

        axis, with $\hat{z}$ the direction of the current. Show that a vector poten-
        tial of the form $\bar{A}=A_0\hat{z}(x^2+y^2)$ will correctly give the magnetic

        field $\bar{B}$ of this current at all points inside the wire. What is the value
        of the constant, $A_0$?
    \end{question}
\end{myBox}

\break

\section*{4}
\begin{myBox}[]{}
    \begin{question}
        In class, we calculate the vector potential $\bar{A}(\bar{r})$ and the magnetic induction $\bar{B}(\bar{r})$ of a circular
        thread of current. The current density is given in cylindrical coordinates as
        $$\bar{J}(\bar{r})=I\delta(\rho-R)\delta(z)\hat{\phi}$$
        See lecture 4, page 1.44 and continued. The calculation of the vector potential $\bar{A}(\bar{r})$ leads to
        an ellipitic integral, which cannot be solved analytically. Estimate the integral for the limit
        $\rho>> R$, by a Taylor expansion. Show that in this case a dipole field emerges!

    \end{question}
\end{myBox}

\end{document}