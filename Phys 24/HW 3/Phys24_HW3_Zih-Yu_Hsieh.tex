%Phys24_HW3_Zih-Yu_Hsieh.tex

\documentclass{article}
\usepackage{graphicx} % Required for inserting images
\usepackage[margin = 2.54cm]{geometry}
\usepackage[most]{tcolorbox}

\newtcolorbox{myBox}[3]{
arc=5mm,
lower separated=false,
fonttitle=\bfseries,
%colbacktitle=green!10,
%coltitle=green!50!black,
enhanced,
attach boxed title to top left={xshift=0.5cm,
        yshift=-2mm},
colframe=blue!50!black,
colback=blue!10
}

\usepackage{amsmath}
\usepackage{amssymb}
\usepackage{verbatim}
\usepackage[utf8]{inputenc}
\linespread{1.2}

\newtheorem{definition}{Definition}
\newtheorem{proposition}{Proposition}
\newtheorem{theorem}{Theorem}
\newtheorem{question}{Question}

\title{Phys 24 HW3}
\author{Zih-Yu Hsieh}

\begin{document}
\maketitle

\section*{1}
\begin{myBox}[]{}
    \begin{question}
        Purcell 7.21:

        What is the maximum electromotive force induced in a coil of
        4000 turns, average radius 12 cm, rotating at 30 revolutions per
        second in the earth’s magnetic field where the field intensity is
        0.5 gauss?
    \end{question}
\end{myBox}

\textbf{Pf:}

If assume the field is in $\hat{x}$ direction, with strength $0.5$ gauss (or $5\cdot 10^{-5} T$), and the coil has a normal always pointing in the xy-plane.
Since it rotates 30 revolution every second, then frequency $f=30 Hz$, or the angular frequency $\omega = 2\pi f = 2\pi \cdot 30 = 60\pi$.
Then, assume at initial $t=0$, the normal is pointing in $\hat{x}$ direction, then the normal $\bar{n}$ can be parametrized as $\bar{n}=(\cos(\omega t),\sin (\omega t),0)$.

\hfill

Then, since the coil has 4000 turns, and the average radius is $12 cm = 0.12 m$, if consider the magnetic flux through each turn, we get:
$$\phi = \int_A\bar{B}\cdot d\bar{s} = \int_A B\hat{x} \cdot\bar{n} dA = B\int_A\hat{x}\cdot (\cos(\omega t),\sin(\omega t),0)dA = B\cos(\omega t)\int_AdA$$
(Note: above equation is true, since the field can be treated as constant direction that's not changing).

Given that $\int_A dA$ is the area of the coil, which is given by $\pi \cdot (0.12)^2$, hence the magnetic flux for each turn is $\phi = B\cos(\omega t)\pi \cdot (0.12)^2$.
With $4000$ turns, the total magnetic flux is $\phi_{tot}=4000 \cdot \pi \cdot (0.12)^2 B\cos(\omega t)$.

\hfill

Now, since $\varepsilon=-\frac{d}{dt}\phi_{tot} = 4000\cdot\pi\cdot(0.12)^2B\omega\sin(\omega t)$, and $|\sin(\omega t)\leq 1$, then the maximum of $\varepsilon$ is given as:
$$\max|\varepsilon| =4000\cdot \pi \cdot (0.12)^2 B\omega $$
With $B=5\cdot 10^{-5}T$ (or unit in $\frac{V\cdot s}{m^2}$), $\omega = 60\pi\ rad/s$, then the maximum EMF is given by:
$$\max|\varepsilon| = 4000\cdot\pi\cdot((0.12)^2m^2)\cdot \left(5\cdot 10^{-5}\ \frac{V\cdot s}{m^2}\right)\cdot (60\pi\ rad/s) \approx 1.7055 V$$
So, the maximum EMF is approximately $1.7055 V$.

\break

\section*{2}
\begin{myBox}[]{}
    \begin{question}
        Purcell 7.22:

        In the central region of a solenoid that is connected to a radiofrequency 
        power source, the magnetic field oscillates at $2.5\cdot 10^6$
        cycles per second with an amplitude of 4 gauss. What is the amplitude of 
        the oscillating electric field at a point 3 cm from the axis?

        (This point lies within the region where the magnetic field is nearly
        uniform.)
    \end{question}
\end{myBox}

\textbf{Pf:}

Given the oscillation happens at $2.5\cdot 10^6$ cycles per second (frequency), then the angular frequency $\omega = 2.5 \cdot 10^6\pi$.
Then, assume at $t=0$, the magnetic field strength is $0$, the field strength can be modeled as:
$$B(t)=4\cdot 10^{-4}\sin(\omega t)$$
(Note: $4$ gauss is $4\cdot 10^{-4} T$). Which, we treat the direction of magnetic field as $\hat{z}$ for simplicity.

\hfill

Now, consider an imaginary conducting circle with radius $r$ concentric to the axis of the solenoid pointing in $\hat{z}$ (and $r$ is small compared to the solenoid's radius).
Through this imaginary conducting loop, WLOG, we can assume when an electric field is generated, it is always pointing in $\hat{\phi}$ direction
(since the current is traveling in $\hat{\phi}$ direction, which current density is pointing in the same direction; and, because $\bar{J}=\sigma\bar{E}$, can assume electric field is also pointing in the same direction).
Which, let $\bar{E}=E(t)\hat{\phi}$, we can calculate the EMF as:
$$\varepsilon = \int_{\partial A}\bar{E}\cdot d\bar{r} = E(t) 2\pi r$$
(Note: $\bar{E}$ and $d\bar{r}$ are always parallel, since on the circle we constructed, both are in $\hat{\phi}$ direction; and, can assume $\bar{E}$ has constant magnitude by symmetry, when the time $t$ is fixed).

\hfill

Then, consider the magnetic flux through this conducting loop: Since $\bar{B}(t)=B(t)\hat{z}$, and the normal of the circle is also in $\hat{z}$ direction, then the flux is given as follow:
$$\phi = \int_A\bar{B}(t)\cdot d\bar{s}=B(t) \pi r^2 = \pi r^2 \cdot (4\cdot 10^{-4})\sin(\omega t)$$
(Note: because $\bar{B}(t)$ and $d\bar{s}$ are parallel, and $B(t)$ is assumed to be constant, hence the flux is field strnegth times area of the loop).

Hence, EMF is also given as follow:
$$E(t)2\pi r=\varepsilon=-\frac{d}{dt}\phi = -\pi r^2\cdot (4\cdot 10^{-4})\omega \cos(\omega t)$$
Which, we can derive the following:
$$E(t)=-\frac{r}{2}(4\cdot 10^{-4})\omega \cos(\omega t)$$
Since $|\cos(\omega t)|\leq 1$, then the maximum of $E(t)$ is given as:
$$\max|E(t)|=\frac{r}{2}\cdot (4\cdot 10^{-4})\omega$$

\hfill

Now, with $r=3\ cm=0.03\ m$, $4\cdot 10^{-4}$ is given as $T$ (or $\frac{N\cdot s}{C\cdot m}$), and $\omega = 2.5\cdot 10^6\ rad/s$, we get:
$$\max|E(t)|=\frac{0.03 m}{2}\left(4\cdot 10^{-4} \frac{N\cdot s}{C\cdot m}\right)(2.5\cdot 10^6\ rad/s)\approx 15 \frac{N}{C}$$
Hence, the amplitude of the oscillating electric field is given by $15\ N/C$.

\break

\section*{3}
\begin{myBox}[]{}
    \begin{question}
        Purcell 7.25:

        A long straight stationary wire is parallel to the $y$ axis and passes
        through the point $z = h$ on the $z$ axis. A current $I$ flows in this
        wire, returning by a remote conductor whose field we may neglect.

        Lying in the $xy$ plane is a square loop with two of its sides, of
        length $b$, parallel to the long wire. This loop slides with constant
        speed $v$ in the $\hat{x}$ direction. 
        
        Find the magnitude of the electromotive
        force induced in the loop at the moment when the center of the
        loop crosses the $y$ axis.
    \end{question}
\end{myBox}

\textbf{Pf:}

Since calculating the general behavior is tedious, we'll just focus on the local behavior.

Assume at time $t=0$, the center of the loop crosses the $y$ axis, and at time $t=\Delta t$, the loop has a displacement $v\cdot \Delta t$ toward the $x$ direction.

If the change $\Delta t$ is small enough, we can treat the magnetic field at the position of area encountering the change as constant.
Which, the changes happen at $x=-\frac{b}{2}$, and $x=\frac{b}{2}$.

\hfill

\begin{figure}[h!]
    \begin{center}
        \includegraphics*[width=100mm]{Diagram for magnetic field.jpg}
        \caption{Diagram of Magnetic Field near the two ends of the Loop}
    \end{center}
\end{figure}

Based on the graph, at the two location, the radial distance from the wire is $r=\sqrt{(\frac{b}{2})^2+h^2} = \frac{1}{2}\sqrt{b^2+4h^2}$,
which the field strength is given by $B=\frac{\mu_0I}{2\pi r}$, and pointing in $\hat{\phi}$ direction.

However, to calculate the flux through the area (with normal $\hat{z}$), we need to consider the component of the fiel that is orthogonal to the area (pointing in $\hat{z}$ direction):
At $x=-\frac{b}{2}$, if the angle is given by $\phi$, then $\sin(\phi)=\frac{b/2}{r}$, which the component of the field that's orthogonal, is given by:
$$\bar{B}^\perp_{-\frac{b}{2}}=-B\cdot \sin(\phi)\hat{z} = -\frac{\mu_0I}{2\pi r^2}\cdot \frac{b}{2}\hat{z} = -\frac{\mu_0Ib}{4\pi r^2}\hat{z}$$
Similarly, at $x=\frac{b}{2}$, since the angle is the same, the only difference is now the component is pointing upward. So:
$$\bar{B}^\perp_{\frac{b}{2}}=\frac{\mu_0Ib}{4\pi r^2}\hat{z}$$

\hfill

Now, let $\hat{z}$ be the normal vector of the loop, then during the time $\Delta t$, the loop decreases an area of $b\cdot v\Delta t$ at $x=-\frac{b}{2}$,
which the change in flux is given by:
$$\Delta\phi_{-\frac{b}{2}}=\bar{B}^\perp_{-\frac{b}{2}}\cdot\left(-b\cdot v\Delta t\hat{z}\right) = \frac{\mu_0Ib^2\cdot v\Delta t}{4\pi r^2}$$
(Note: The area is negative, because there is a decrease in area).

Similarly, for the corresponding increase of area $b\cdot v\Delta t$ at $x=\frac{b}{2}$, the change in flux is given by:
$$\Delta\phi_{\frac{b}{2}}=\bar{B}^\perp_{\frac{b}{2}}\cdot\left(b\cdot v\Delta t\hat{z}\right)=\frac{\mu_0Ib^2\cdot v\Delta t}{4\pi r^2}$$
Hence, the total change in flux is $\Delta \phi = \Delta\phi_{-\frac{b}{2}}+\Delta\phi_{\frac{b}{2}}=\frac{\mu_0Ib^2\cdot v\Delta t}{2\pi r^2}$. Then, the instantaneous rate of change in flux is:
$$\frac{d\phi}{dt} = \lim_{\Delta t\rightarrow 0^+}\frac{\Delta \phi}{\Delta t}=\lim_{\Delta t\rightarrow 0^+}\frac{\mu_0Ib^2\cdot v\Delta t}{2\pi r^2}\cdot \frac{1}{\Delta t}=\frac{\mu_0Ib^2v}{2\pi r^2}$$
Which, the induced EMF is given by (Recall: $r=\frac{1}{2}\sqrt{b^2+4h^2}$):
$$|\varepsilon|=\left|-\frac{d\phi}{dt}\right|=\frac{\mu_0Ib^2v}{2\pi (\frac{1}{2}\sqrt{b^2+4h^2})^2} = \frac{2\mu_0Ib^2v}{b^2+4h^2}$$


\break

\section*{4}
\begin{myBox}[]{}
    \begin{question}
        Purcell 7.26:

        A metal crossbar of mass $m$ slides without friction on two long
        parallel conducting rails a distance $b$ apart (see Figure below). 
        
        A resistor $R$ is connected across the rails at one end; compared with $R$, 
        the resistance of bar and rails is negligible. 
        
        There is a uniform field $B$ perpendicular to the plane of the figure. 
        At time $t = 0$ the crossbar is given a velocity $v_0$ toward the right. 
        What happens afterward?

        \begin{itemize}
            \item[(a)]Does the rod ever stop moving? If so, when?
            \item[(b)]How far does it go?
            \item[(c)] How about conservation of energy?
        \end{itemize}
    \end{question}

    \begin{center}
        \includegraphics*[width=80mm]{7.25.png}
    \end{center}
\end{myBox}

\textbf{Pf:}

Before heading into the problem, we need some preliminary setup to understand the relationship of the force, position, velocity, and acceleration.

\hfill

WLOG, assume the magnetic field is pointing out of the plane. Let the rightward direction be $\hat{x}$, upward direction be $\hat{y}$, and the direction out of the page be $\hat{z}$.

Let the left most position of the rail be position $x=0$, and say initially the crossbar is at distance $x=x_0$, and at time $t$, 
the cross bar is at distance $x(t)$.

Then, the magnetic flux at time $t$ is given by $\phi=\int_A\bar{B}\cdot d\bar{s} = B\cdot x(t)b$ 
(Note: since $\bar{B}$ and $d\bar{s}$ the normal of the area is always parallel, or pointing out of the page, while $\bar{B}$ has constant magnitude,
then the flux is field strength $\times$ area).
Which, the EMF at time $t$ is given by: $\varepsilon(t)=-\frac{d}{dt}\phi = -Bb\frac{dx}{dt} = -Bb\cdot v(t)$ (where $v(t)$ is the velocity to the right).

So, with the resistance $R$, at time $t$, the induced current is $I_{ind}=\frac{\varepsilon}{R} = -\frac{Bb}{R}v(t)$.
(Note: Based on Lenz's Law, if $v(t)>0$ - when the crossbar is traveling to the right, we have $\frac{d\phi}{dt}>0$, hence with the normal pointing out of the page,
the current is going clockwise, with magnitde $\frac{Bb}{R}v(t)$).

\hfill

Now, notice that if consider the top rail as $l=0$, and the bottom rail as $l=b$, with the field $\bar{B}$ pointing out of the page ($\hat{z}$ direction), the force on the crossbar is given by:
$$\bar{F}=\int_c I_{ind}d\bar{r}\times \bar{B} = \int_{l=0}^{b}\frac{Bb}{R}v(t)dl \cdot B (-\hat{y}\times \hat{z}) = -\frac{B^2b}{R}v(t)\hat{x}\int_{l=0}^{b}dl = -\frac{B^2b^2}{R}v(t)\hat{x}$$

Let $a(t)=\frac{dv}{dt}$ be the acceleration toward the right direction for time $t>0$. Since $\bar{F}=m\bar{a}$, then $\frac{dv}{dt}=\bar{a}=-\frac{B^2b^2}{mR}v(t)\hat{x}$, or $\frac{dv}{dt}=-\frac{B^2b^2}{mR}v(t)$.

\hfill

\begin{itemize}
    \item[(a)] Based on the preliminary derivation above, since $\frac{dv}{dt}=-\frac{B^2b^2}{mR}v$, then the solution to the differential equation is:
    $$v(t)=Ce^{-\frac{B^2b^2}{mR}t}$$
    Which, for $v(0)=v_0=Ce^0$, $C=v_0$. So $v(t)=v_0e^{-\frac{B^2b^2}{mR}t}$.
    
    By the property of exponential function, $v(t)\neq 0$, hence the crossbar never stops in finite amount of time.

    \hfill

    \item[(b)] If consider the total distance the crossbar can travel, for any $t_1>0$, consider the following:
    $$\Delta x = \int_{t=0}^{t_1}v(t)dt = \int_{t=0}^{t_1}v_0e^{-\frac{B^2b^2}{mR}t}dt = -\frac{v_0mR}{B^2b^2}e^{-\frac{B^2b^2}{mR}t}\bigg|_{0}^{t_1} = \frac{v_0mR}{B^2b^2}\left(1-e^{-\frac{B^2b^2}{mR}t_1}\right)$$
    Then, if take the limit as $t_1$ approaches infinity, the above term becomes $\frac{v_0mR}{B^2b^2}$,
    hence this is the largest distance the crossbar can travel. It travels distance of at most $\frac{v_0mR}{B^2b^2}$.

    \hfill

    \item[(c)] The initial kinetic energy is given by $\frac{1}{2}mv_0^2$. If there are no other external sources dissipating the energy, then the energy would be dissipated by the resistor.
    
    Notice that the power dissipated by the resistor is given by $P=I^2R$, which over time the energy dissipated is given by:
    $$\int_{t=0}^{\infty}I^2Rdt = \int_{t=0}^{\infty}\left(-\frac{Bb}{R}v(t)\right)^2R dt = \frac{B^2b^2v_0^2}{R^2}\int_{t=0}^{\infty}e^{-\frac{2B^2b^2}{mR}t}dt$$
    $$=\frac{B^2b^2v_0^2}{R^2}\cdot\left(-\frac{mR}{2B^2b^2}\right)e^{-\frac{2B^2b^2}{mR}t}\bigg|_{0}^{\infty} = \frac{mv_0^2}{2}$$
    Hence, the total energy dissipated forthe whole time interval matches with the initial kinetic energy in the system,
    showing that energy is conserved.
    
\end{itemize}

\break

\section*{5}
\begin{myBox}[]{}
    \begin{question}
        Purcell 7.27:

        An infinite solenoid with radius $b$ has $n$ turns per unit length. The
        current varies in time according to $I(t) = I_0\cos(\omega t)$ (with positive 
        defined as shown in the Figure below). 
        
        A ring with radius $r < b$ and resistance $R$ is centered on the solenoid’s axis, 
        with its plane perpendicular to the axis.

        \begin{itemize}
            \item[(a)] What is the induced current in the ring?
            \item[(b)] A given little piece of the ring will feel a magnetic force. For
            what values of t is this force maximum?
            \item[(c)] What is the effect of the force on the ring? That is, does the
            force cause the ring to translate, spin, flip over, stretch/
            shrink, etc.?
        \end{itemize}
    \end{question}

    \begin{center}
        \includegraphics*[width=30mm]{7.26.png}
    \end{center}
\end{myBox}

\textbf{Pf:}

Let the solenoid's axis be in the $\hat{z}$ direction (pointing upward), and use cylindrical coordinates for the problem.

Recall that in lecture, the magnetic field of the interior of the solenoid with current $I$ passing through, is given by $\bar{B}=\mu_0 nI\hat{z}$,
where $n$ is the number of turns per unit length. Hence, consider time dependence, $\bar{B}(t)=\mu_0nI(t)\hat{z}$.

\begin{itemize}
    \item[(a)] Consider the magnetic flux $\phi = \int_A \bar{B}\cdot d\bar{s}$. At any instant $t$, since $\bar{B}$ is nearly a constant near the center of the solenoid,
    while both $\bar{B}$ and $d\bar{s}$ (normal of the area of the ring) are both pointing in $\hat{z}$ direction. Hence, it can be simplified to $\phi=B(t)\cdot \pi r^2$
    (which, $B(t)=\mu_0nI(t)$, the magnitude of magnetic field; and, $\pi r^2$ is the area of the ring), or $\phi(t)=\mu_0n\cdot\pi r^2 I(t)$.
    Which, consider time dependence, we can yield the EMF by:
    $$\varepsilon=-\frac{d}{dt}\phi = -\mu_0n\pi r^2\frac{dI}{dt}$$
    With theresistance $R$ of the ring, then by Ohm's Law $\varepsilon=I_{ind}R$, we yield:
    $$I_{ind}=\frac{\varepsilon}{R}=-\frac{\mu_0n\pi r^2}{R}\cdot\frac{dI}{dt}$$
    So, with $I(t)=I_0\cos(\omega t)$, which $I'(t)=-\omega I_0\sin(\omega t)$, the induced current in the ring is $I_{ind}=\frac{\mu_0n\pi r^2}{R}\cdot\omega I_0\sin(\omega t)$.

    \hfill

    \item[(b)] Given the circular ring, at any instant the current is always traveling in $\hat{\phi}$ direction. Hence, given a small portion of the angle $d\phi$,
    the corresponding force on the component is given by $I_{ind} d\bar{r}\times \bar{B}$.
    Which, $I_{ind}(t)=\frac{\mu_0n\pi r^2\cdot \omega I_0}{R}\sin(\omega t)$, $d\bar{r}=rd\phi \hat{\phi}$, and $\bar{B}=\mu_0nI(t)\hat{z} = \mu_0nI_0\cos(\omega t)\hat{z}$.
    Then, the force component can be calculated as:
    $$d\bar{F}=\frac{\mu_0n\pi r^2\cdot \omega I_0}{R}\sin(\omega t)\cdot rd\phi \cdot\mu_0nI_0\cos(\omega t)(\hat{\phi}\times \hat{z})$$
    In case to maximize the force component with respect to time, it suffices to maximize $\sin(\omega t)\cos(\omega t)=\frac{1}{2}\sin(2\omega t)$.
    Which, the maximum of the amplitude occurs when the input $2\omega t=\frac{(2n+1)}{2}\pi = n\pi + \frac{\pi}{2}$ (for $n\in\mathbb{Z}$).
    Hence, for $t=\frac{2n+1}{4\omega}\pi$, the force component caused by magnetic field on the ring is maximized. (not regarding the direction here).
    
    \hfill

    \item[(c)] The force would stretch / shrink the ring in radial direction. Since the current in the wire is going in the $\phi$ direction, this is also 
    the change in direction $d\bar{r}$ when finding the force on the whole ring. Then, because $\bar{B}$ is pointing in $\hat{z}$ direction,
    then each small component of force has direction being the same as $d\bar{r}\times \bar{B} = rd\phi\cdot B(\hat{\phi}\times \hat{z}) = rd\phi\cdot B\hat{r}$, 
    which the force is solely in radial direction at each location.

    Then, because when the induced current occurs, we can assume the current has equal strength at each location,
    and the magnetic field strength is also nearly constant near the ring, then the force component at each location has the same magnitude.
    So, the net force on the ring is $0$ (since the ring has rotation symmetry, which the force in radial direction cancels each other out),
    but it creates a force that stretches or shinks the ring in radial direction.
\end{itemize}

\hfill

\hfill

\section*{6}
\begin{myBox}[]{}
    \begin{question}
        Purcell 7.33:

        A nonconducting thin ring of radius a carries a static charge q.
        This ring is in a magnetic field of strength $B_0$, parallel to the ring’s
        axis, and is supported so that it is free to rotate about that axis.

        If the field is switched off, how much angular momentum will be
        added to the ring? Supposing the mass of the ring to be $m$, show
        that the ring, if initially at rest, will acquire an angular velocity
        $\omega=\frac{qB_0}{2m}$.
    \end{question}
\end{myBox}

\textbf{Pf:}

Suppose the field strength $B(t)$ is modeled as $B(0)=B_0\neq 0$, $B(t_1)=0$ for some $t_1>0$, and for all $t\geq t_1$, $B(t)=0$ (assume it's differentiable on the interval $(0,t_1)$),
and both $\bar{B}$ and the axis are poiting in $\hat{z}$ direction (so $\bar{B}=B(t)\hat{z}$).

At any moment $t\in (0,t_1)$, the magnetic flux is given as $\phi=\int_A\bar{B}\cdot d\bar{s}=B(t)\cdot \pi a^2$ 
(since $\bar{B}$ and the normal $d\bar{s}$ is always parallel, and field strength across the ring is constant, then the flux is field strength $\times$ area of the ring).
Hence, the EMF $\varepsilon = -\frac{d}{dt}\phi = -\frac{dB}{dt}\cdot \pi a^2$.

\hfill

Also, notice that WLOG, we can assume when the electric field is generated in the ring, the field strength is constant at anywhere in the ring, and it's always pointing in the $\phi$ direction of the ring, based on symmetry.
Let $\bar{E}=E(t)\hat{\phi}$, which the EMF is also calculated as $\varepsilon = \int_{\partial A}\bar{E}\cdot d\bar{r} = E(t)\cdot 2\pi a$ (since $\bar{E}$ and $d\bar{r}$ are all in $\phi$ direction on the ring, and $E(t)$ is constant
at fixed time $t$, so the integral is field strength $\times$ length of circumference of the ring).

Combining the two equations above, we get:
$$E(t)\cdot 2\pi a=-\frac{dB}{dt}\cdot \pi a^2,\quad E(t)=-\frac{a}{2}\cdot\frac{dB}{dt}$$
Hence, $\bar{E}=-\frac{a}{2}\cdot \frac{dB}{dt}$.

\hfill

Now, if assume the charge $q$ is distributed uniformly around the ring, then for each small portion of angle $d\phi$, the corresponding charge is given by $dq = \frac{q}{2\pi a}\cdot a\cdot d\phi = \frac{q}{2\pi} d\phi$.
Then, the force on the component is given by $d\bar{F}=dq\cdot \bar{E} = -\frac{qa\cdot d\phi}{4\pi}\cdot \frac{dB}{dt}\hat{\phi}$, which the torque on the component (with position about the axis given by $\bar{r}=a\hat{r}$),
is then given by $d\bar{\tau}=\bar{r}\times d\bar{F} = -\frac{qa^2\cdot d\phi}{4\pi}\cdot \frac{dB}{dt}(\hat{r}\times \hat{\phi}) = -\frac{qa^2\cdot d\phi}{4\pi}\cdot \frac{dB}{dt}\hat{z}$.

So, the total torque is given as:
$$\bar{\tau}=\int_{\phi=0}^{2\pi}d\bar{\tau}=\int_{0}^{2\pi}-\frac{qa^2\cdot d\phi}{4\pi}\cdot \frac{dB}{dt}\hat{z} = -\frac{qa^2}{4\pi}\cdot \frac{dB}{dt}\hat{z} \cdot 2\pi = \frac{-qa^2}{2}\cdot \frac{dB}{dt}\hat{z}$$
Recall that $\bar{\tau} = I\bar{\alpha}$, and for a circular ring with radius $a$ and mass $m$, the moment of inertia $I=ma^2$. Hence, the angular acceleration is given by:
$$\bar{\alpha}=\frac{1}{I}\bar{\tau} = \frac{1}{ma^2}\cdot \left(\frac{-qa^2}{2}\cdot \frac{dB}{dt}\hat{z}\right) = \frac{-q}{2m}\cdot \frac{dB}{dt}\hat{z}$$
Then, the change in angular velocity over time interval $(0,t_1)$ is then given by:
$$\Delta\bar{\omega}=\int_{t=0}^{t_1}\bar{\alpha}\hat{z} dt = \frac{-q}{2m}\int_{t=0}^{t_1}\frac{dB}{dt} dt \hat{z} = -\frac{q}{2m}(B(t_1)-B(0))\hat{z} = -\frac{q}{2m}(0-B_0)\hat{z} = \frac{qB_0}{2m}\hat{z}$$
With the initial angular velocity being $\bar{w}=\bar{0}$ (since it's initially at rest), then the angular velocity at $t_1$ is given by $\Delta\bar{\omega} = \frac{qB_0}{2m}\hat{z}$.

And, since after $t_1$, $\frac{dB}{dt}=0$ (never changing after decrease to $0$), then $\bar{\alpha}=\bar{0}$, angular velocity is no longer increasing.
So, the final angular velocity is $\omega = \frac{qB_0}{2m}$.

\end{document}