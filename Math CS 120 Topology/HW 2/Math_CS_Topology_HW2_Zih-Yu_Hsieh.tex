%Math_CS_Topology_HW2_Zih-Yu_Hsieh.tex

\documentclass{article}
\usepackage{graphicx} % Required for inserting images
\usepackage[margin = 2.54cm]{geometry}
\usepackage[most]{tcolorbox}

\newtcolorbox{myBox}[3]{
arc=5mm,
lower separated=false,
fonttitle=\bfseries,
%colbacktitle=green!10,
%coltitle=green!50!black,
enhanced,
attach boxed title to top left={xshift=0.5cm,
        yshift=-2mm},
colframe=blue!50!black,
colback=blue!10
}

\usepackage{amsmath}
\usepackage{amssymb}
\usepackage{verbatim}
\usepackage[utf8]{inputenc}
%\linespread{1.5}

\newtheorem{definition}{Definition}
\newtheorem{proposition}{Proposition}
\newtheorem{theorem}{Theorem}
\newtheorem{question}{Question}

\title{Math CS Topology HW2}
\author{Zih-Yu Hsieh}

\begin{document}
\maketitle

\section*{1}
\begin{myBox}[]{}
    \begin{question}
        Let $A^*$ denote the set of limit points of $A$. Prove this satisfies:
        \begin{itemize}
            \item $\emptyset^*=\emptyset$
            \item $x\notin \{x\}^*$
            \item $A^{**}\subseteq A\cup A^*$
            \item $(A\cup B)^*=A^*\cup B^*$
        \end{itemize}
    \end{question}
\end{myBox}

\textbf{Pf:}

All the below proofs are based on a nonempty topological space $X$.

\begin{itemize}
    \item[1.] To prove that $\emptyset^*=\emptyset$, we'll use contradiction: Suppose there exists $x\in X$ with $x\in \emptyset^*$,
    then by definition, every open neighborhood $U$ of $x$, the intersection $\emptyset \cap (U\setminus \{x\})\neq \emptyset$.

    However, since every set $A$ satisfies $\emptyset \cap A=\emptyset$, the above condition is a contradiction. Therefore, there's no such $x\in X$ satisfying $x\in\emptyset^*$,
    thus $\emptyset^*=\emptyset$.

    \hfill

    \item[2.] To prove that $x\notin\{x\}^*$, consider any arbitrary open neighborhood $U$ of $x$: Since $x\notin U\setminus\{x\}$, 
    then $\{x\}\cap (U\setminus \{x\})=\emptyset$. Thus, $x$ is not a limit point of $\{x\}$, or $x\notin \{x\}^*$.

    \hfill

    \item[3.] To prove that $A^{**}\subseteq A\cup A^*$, consider any $x\in A^{**}$:
    
    If $x\in A$, then $x\in A\cup A^*$.

    Else, if $x\notin A$, by definition, for every open neightborhood $U$ of $x$, there exists $y\in A^*\cap (U\setminus\{x\})$, 
    which $y\in A^*$ and $y\in U$, thus $U$ is an open neighborhood of $y$.

    Then, since $y$ is a limit point of $A$, then there exists $a\in A\cap (U\setminus \{y\})$, which $a\in A$ and $a\in U$.

    Yet, since $x\notin A$, so $a\neq x$, thus $a\in U\setminus \{x\}$, proving that $A\cap (U\setminus\{x\})\neq \emptyset$.

    Since every open neighborhood of $x$ satisfies $A\cap (U\setminus\{x\})\neq \emptyset$, then $x$ is a limit point of $A$, thus $x\in A^*\subseteq A\cup A^*$.

    So, regardless of the case, $x\in A^{**}$ implies $x\in A\cup A^*$, thus $A^{**}\subseteq A\cup A^*$.

    \hfill

    \item[4.] To prove that $(A\cup B)^* = A^*\cup B^*$, consider the following:
    
    First, $A^*\cup B^*\subseteq (A\cup B)^*$: Since $A,B\subseteq (A\cup B)$, then if $x\in A^*$, every open neighborhood $U$ of $x$ satisfies $A\cap (U\setminus\{x\})\neq \emptyset$,
    thus $(A\cup B)\cap (U\setminus\{x\})\neq \emptyset$, showing that $x\in (A\cup B)^*$, or $A^*\subseteq (A\cup B)^*$. 
    Applying the same logic on $B^*$, we'll get $B^*\subseteq (A\cup B)^*$, hence $(A^*\cup B^*)\subseteq (A\cup B)^*$.

    \hfill

    Now, to prove that $(A\cup B)^* \subseteq (A^*\cup B^*)$, we'll approach by contradiction:

    Suppose $(A\cup B)^*\not\subseteq (A^*\cup B^*)$, there exists $x\in (A\cup B)^*$, while $x\notin (A^*\cup B^*)$.

    Then, since $x\notin A^*$, there exists open neighborhood $U_1$ of $x$, with $A\cap (U_1\setminus\{x\})=\emptyset$;
    similarly, since $x\notin B^*$, there exists open neighborhood $U_2$ of $x$, with $B\cap (U_2\setminus\{x\})=\emptyset$.

    Now, consider $U=U_1\cap U_2$: It is an open set, and since $x\in U_1$ and $x\in U_2$, then $x\in (U_1\cap U_2)=U$, thus $U$ is an open neighborhood of $x$.

    However, since $U\subseteq U_1$, then $A\cap (U\setminus\{x\})\subseteq A\cap (U_1\setminus\{x\})=\emptyset$; similarly, $U\subseteq U_2$ implies $B\cap (U\setminus\{x\})\subseteq B\cap (U_2\setminus\{x\})=\emptyset$.

    So, $(A\cup B)\cap (U\setminus\{x\})=(A\cap (U\setminus\{x\}))\cup (B\cap (U\setminus\{x\}))=\emptyset$. Yet, if $x\in (A\cup B)^*$, 
    then every open neighborhood of $x$ should have nonempty intersection with $(A\cup B)$, while not including $x$.

    So, this is a contradiction. Hence, $(A\cup B)^*\subseteq (A^*\cup B^*)$.

    \hfill

    With the above two statements, $(A\cup B)^*=A^*\cup B^*$.
\end{itemize}

\hfill

\hfill

\section*{2}
\begin{myBox}[]{}
    \begin{question}
        Prove that the boundary operation satisfies:
        \begin{itemize}
            \item $\partial A=\partial(X\setminus A)$
            \item $\partial\partial A\subseteq \partial A$
            \item $\partial(A\cup B)\subseteq \partial A\cup \partial B$
            \item $A\subseteq B\implies \partial A\subseteq (B\cup \partial B)$
        \end{itemize}
    \end{question}
\end{myBox}

\textbf{Pf:}

\begin{itemize}
    \item[1.] Given any set $A\subseteq X$, since $\partial A=\overline{A}\cap \overline{X\setminus A}$ and $\partial(X\setminus A)=\overline{X\setminus A}\cap \overline{X\setminus(X\setminus A)}=\overline{X\setminus A}\cap \overline{A}$,
    thus $\partial A=\partial (X\setminus A)$.

    \hfill

    \item[2.] Given $\partial A=\overline{A}\cap \overline{X\setminus A}$, since $\overline{A}$ and $\overline{X\setminus A}$ are both closed, the $\partial A$ is closed (intersection of arbitrary closed set is closed).
    Which, $\overline{\partial A}=\partial A$. Hence, $\partial \partial A=\overline{\partial A}\cap \overline{X\setminus \partial A}\subseteq \overline{\partial A}=\partial A$.
    So, $\partial\partial A\subseteq \partial A$.

    \hfill

    \item[3.] For all $x\in \partial(A\cup B)$, $x\in \overline{A\cup B} = \overline{A}\cup \overline{B}$, and $x\in \overline{X\setminus(A\cup B)}=\overline{(X\setminus A)\cap (X\setminus B)}\subseteq (\overline{X\setminus A}\cap \overline{X\setminus B})$. 
    Then, there are two cases to consider:
    
    First, if $x\in \overline{A}$, since $x\in (\overline{X\setminus A}\cap \overline{X\setminus B})\subseteq \overline{X\setminus A}$, then $x\in \partial A = \overline{A}\cap\overline{X\setminus A}$.

    Else, if $x\in \overline{B}$, since $x\in (\overline{X\setminus A}\cap \overline{X\setminus B})\subseteq \overline{X\setminus B}$, then $x\in \partial B = \overline{B}\cap\overline{X\setminus B}$.

    In either case, $x\in \partial A\cup \partial B$, thus we can conclude that $\partial(A\cup B)\subseteq \partial A\cup \partial B$.

    \hfill

    \item[4.] Suppose $A\subseteq B$, then $\overline{A}\subseteq \overline{B}$. Thus, $\partial A=\overline{A}\cap\overline{X\setminus A}\subseteq \overline{A}\subseteq \overline{B}$. 
    Now, for all $x\in \partial A\subseteq \overline{B}$, there are two cases:
    
    If $x\in B$, then $x\in (B\cup \partial B)$.

    Else, if $x\notin B$, then $x\in X\setminus B\subseteq \overline{X\setminus B}$. With the statement that $x\in \overline{B}$, $x\in \overline{B}\cap \overline{X\setminus B} = \partial B$.
    Hence again, $x\in (B\cup \partial B)$.

    So, regardless of the case, $x\in \partial A$ implies $x\in (B\cup \partial B)$, thus $\partial A\subseteq (B\cup \partial B)$.
\end{itemize}

\break

\section*{3}
\begin{myBox}[]{}
    \begin{question}
        Let A be a set in a topological space. Prove that the closure of the interior of
        the closure of the interior of A equals the closure of the interior of A.
    \end{question}
\end{myBox}

\textbf{Pf:}

Let $B=\overline{A^\circ}$ (the closure of the interior of $A$), which $\overline{B^\circ}$ is the closure of the interior of closure of the interior of $A$.

Notice that since $B^\circ \subseteq B$, then $\overline{B^\circ}\subseteq \overline{B}$; and since $B=\overline{A^\circ}$, which is already closed, then $\overline{B}=B$.
Thus, $\overline{B^\circ}\subseteq \overline{B}=B$.

\hfill

Also, since $A^\circ \subseteq \overline{A^\circ}=B$ while $A^\circ$ is open, then $A^\circ \subseteq B^\circ$; hence, $B=\overline{A^\circ}\subseteq \overline{B^\circ}$.

\hfill

Combining both criteria, $B=\overline{B^\circ}$, so the Closure of the Interior of $A$, equals to the Closure of the Interior of the Closure of the Interior of $A$.

\hfill

\hfill

\section*{4}
\begin{myBox}[]{}
    \begin{question}
        Give an example of a topological space that is not Hausdorff, but still has
        the property that every sequence converges to at most one point. Prove your
        answer is correct. I suggest using the “countable complement” (also called
        “cocountable”) topology.
    \end{question}
\end{myBox}

\textbf{Pf:}

Given not countable set $X$, and consider the Countable Complement Topology on $X$ (which, $U\subseteq X$ is open iff either $X\setminus U$ is at most countable, or $U=\emptyset$).

\hfill

\hfill

\textbf{The Cocountable Topology on $X$ is not Hausdorff:}

We'll prove by contradiction. Suppose the given topology is Hausdorff, then for all $x,y\in X$ with $x\neq y$, there exists disjoint open neighborhood $U,V\subseteq X$,
with $x\in U$ and $y\in V$ (which $y\notin U$ and $x\notin V$).

First, since $U$ is open, then $X\setminus U$ is at most countable, according to the definition of cocountable topology.

Then, since $U,V$ are disjoint, then every point $z\in V$ satisfies $z\notin U$, or $z\in X\setminus U$. Hence, $V\subseteq X\setminus U$, which implies $V$ is also countable (subset of at most countable set is at most countable).

However, this implies that $X\setminus V$ is not countable: If $X\setminus V$ is countable, then $V\cup (X\setminus V)=X$ is countable, which contradicts the assumption that $X$ is uncountable.

Yet, if $X\setminus V$ is not countable, $V$ is no longer open, which again contradicts our assumption that $V$ is open.

Thus, the initial assumption is false, the Cocountable Topology on $X$ is not Hausdorff.

\hfill

\hfill

\textbf{Type of convergent sequences in Cocountable Topology:}

We'll prove that the sequence $(x_n)_{n\in\mathbb{N}}\subset X$ converges iff it is eventually constant (which, after some index $k$, $n\geq k$ implies $x_n=x$ for some $x\in X$).

\hfill

To prove the forward implication, we'll use contradiction.

Suppose there exists a sequence $(x_n)_{n\in\mathbb{N}}$ that's not eventually constant, but still converges to $x\in X$.

Since it's not eventually constant, for all $N>0$, there exists index $n\geq N$, with $x_n \neq x$. 

Which, consider an arbitrary open neighborhood $U$ of $x$, and consider the set $V=\{x\}\cup (U\setminus (x_n)_{n\in\mathbb{N}})$:

Taking the complement:
$$X\setminus V = X\setminus(\{x\}\cup (U\setminus (x_n)_{n\in\mathbb{N}})) = (X\setminus \{x\})\cap (X\setminus(U\setminus (x_n)_{n\in\mathbb{N}}))$$
$$= (X\setminus\{x\})\cap ((x_n)_{n\in\mathbb{N}}\cup (X\setminus U))$$
Notice that the set $(x_n)_{n\in\mathbb{N}}$ is at most countable, and since $U$ is open, $X\setminus U$ is also at most countable.
Thus, the set $(x_n)_{n\in\mathbb{N}}\cup (X\setminus U)$ is at most countable, which $X\setminus V$ as a subset of $(x_n)_{n\in\mathbb{N}}\cup (X\setminus U)$, must also be at most countable.

Hence, $V$ is actually an open set, which since $x\in V$, it is an open neighborhood of $x$.

However, for all $N>0$, there exists $n\geq N$, with $x_n\neq x$, which $x_n\notin (U\setminus (x_n)_{n\in\mathbb{N}})$, and $x_n\notin \{x\}$,
hence $x_n \notin V=\{x\}\cup (U\setminus (x_n)_{n\in\mathbb{N}})$. This contradictts with the fact that $(x_n)_{n\in\mathbb{N}}$ converges to $x$,
since there should exist $N$, with $n\geq N$ implies $x_n \in V$.

So, the assumption must be false, $(x_n)_{n\in\mathbb{N}}\subset X$ converges implies it is eventually constant.

\hfill

To prove the converse, suppose the sequence $(x_n)_{n\in\mathbb{N}}$ is eventually constant, there exists $x\in X$ and index $N$, with $n\geq N$ implies $x_n=x$.

Which, for all open neighborhood $U$ of $x$, choose the index $N$, for all $n\geq N$ satisfies $x_n=x \in U$, thus by the definition of convergence,
$\lim_{n\rightarrow\infty}x_n=x$.

\hfill

\hfill

\textbf{Limit of Converging Sequence has at most one limit:}

In previous section, we've proven that a sequence $(x_n)_{n\in\mathbb{N}}\subset X$ converges iff it is eventually constant.
Then, there exists an index $N$, such that $n\geq N$ implies $x_n=x$. This implies that the limit is actually unique:

For all $x'\in X$ with $x'\neq x$, consider any open neighborhood $U$ of $x'$: Since $X\setminus U$ is at most countable, then $X\setminus(U\setminus\{x\})=\{x\}\cup (X\setminus U)$ is also at most countable.
Thus, the set $U\setminus \{x\}$ is open under cocountable topology, and $x'\in U\setminus\{x\}$ (since $x'\in U$ and $x'\neq x$).
Hence, $U\setminus\{x\}$ is an open neighborhood of $x'$.

However, if consider all index $n\geq N$, since $x_n=x$, then $x_n \notin U\setminus\{x\}$, this indicates that $(x_n)_{n\in\mathbb{N}}$ is not converging to $x'$.

Since for all $x'\in X$ with $x'\neq x$, it is not a limit of $(x_n)_{n\in\mathbb{N}}$, then the only limit is $x$,
indicating that there is at most one limit for $(x_n)_{n\in\mathbb{N}}$.


So, under Cocountable Topology for an uncountable set $X$, even though the space is not Hausdorff, but the limit is still unique.
\end{document}