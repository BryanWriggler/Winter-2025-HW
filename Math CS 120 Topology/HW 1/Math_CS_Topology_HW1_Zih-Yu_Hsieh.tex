%Math_CS_Topology_HW1_Zih-Yu_Hsieh.tex

\documentclass{article}
\usepackage{graphicx} % Required for inserting images
\usepackage[margin = 2.54cm]{geometry}

\usepackage{amsmath}
\usepackage{amssymb}
\usepackage{verbatim}
\usepackage[utf8]{inputenc}
%\linespread{1.5}

\newtheorem{definition}{Definition}
\newtheorem{proposition}{Proposition}
\newtheorem{theorem}{Theorem}
\newtheorem{question}{Question}

\title{Latex Template}
\author{Zih-Yu Hsieh}

\begin{document}
\maketitle

\section*{1}
\begin{question}
    In a parallel universe, a topological space is defined in terms of closed sets,
    and an open set is defined to be the complement of a closed set (rather than
    the other way around). Complete the parallel universe definition: a topological
    space is a set of points $ X$ and a collection of “closed” sets satisfying the following
    axioms. Explain your answer, but you do not need a rigorous proof.
\end{question}

\textbf{Ans:}

The collection of "closed" sets $\mathcal{C}$ would satisfy the following axioms:
\begin{itemize}
    \item[(1)] $\emptyset,\ X\in \mathcal{C}$. 
    
    The reason is that for the common definition of topology, 
    empty set $\emptyset$ and $X$ (the whole space) are considered open, which their complement - $X$ and $\emptyset$ -
    are considered as closed. In case to be consistent with the common definition, the two sets need to be closed.

    \hfill

    \item[(2)] For finite collection $C_1,...,C_n \in \mathcal{C}$, the union $\left(\bigcup_{i=1}^{n}C_i\right)\in \mathcal{C}$.
    
    The motivation comes from common definition of topology, where the intersection of finite collection of open sets are open.
    Which, the intersection of complements is the same as complement of union, which for finitely many open sets,
    their intersection can be thought of as intersection of finitely many "closed sets' complements", which is the complement of union of these closed sets.
    
    According to the definition in (2), this union is closed (since only finitely many is considered), then it's complement is open,
    thus the intersection of finitely many open sets is open, which is consistent with our definition.

    \hfill

    \item[(3)] For arbitrary collection of closed sets $\mathcal{F}\subseteq \mathcal{C}$, the intersection $\bigcap \mathcal{F}\in \mathcal{C}$.
    
    The reason is again from the common definition of topology, where the union of arbitrary collection of open sets are open.
    Which, the union of complements is equivalent to complement of intersection, so union of arbitrary open sets,
    is in fact union of arbitrary closed sets' complements, which is the complement of intersection of these closed sets.

    Then, according to the definition in (3), this intersection is closed, which its complement - the union of arbitrary open sets, is therefore open.
    
\end{itemize}

\break

\section*{2}
\begin{question}
    For this question, a “neighborhood” of a point is a not necessarily open
    set $ N$ such that $x\in U\subseteq N$ for some open set $U$. In a parallel universe, a
    topological space is defined in terms of neighborhoods, and an open set is defined
    to be a set that is a neighborhood of all of its points. Complete the parallel
    universe definition: a topological space is a set of points $X$ and for every $x\in X$
    a collection of “neighborhoods” that satisfy the following axioms. Explain your
    answer, but you do not need a rigorous proof.
\end{question}

\textbf{Ans:}

For all $x\in X$, a collection of "neighborhoods" $N_x$ would satisfy the following axioms:
\begin{itemize}
    \item[(1)]$N_x\neq \emptyset$.
    
    The reason behind this, is to ensure every element is covered by some open sets, so the following definition is meaningful.

    \hfill

    \item[(2)] For all $B\in N_x$, if $B\subseteq U\subseteq X$, then $U\in N_x$
    
    The motivation of this, is to ensure $X$ is open, and the union of open sets is open.

    Since $X$ contains all possible subset, it contains a neighborhood of any $x\in X$, which this definition
    imposes $X$ to be a neighborhood of all points, which is open by the given definition.

    Also, For any collection of open sets, every of them is a neighborhood of elements contained in themselves.
    Which, since the union contains all open sets, it automatically becomes a neighborhood of all elements in every
    collected open sets, which it is a neighborhood of all its element, forcing the union to also be open.

    \hfill

    \item[(3)] For a finite collection $B_1,...,B_n\in N_x$, the intersection $\bigcap_{i=1}^{n}B_i \in N_x$.
    
    The idea behind this, is to ensure nonempty finite intersection of open sets are open.
    If a finite collection of sets are open, then every set is a neighborhood of every of its points.
    Then, for every point in the intersection, since every open set containing the intersection is a neighborhood of chosen point,
    then their finite intersection is also a neighborhood of the chosen point according to definition in (3). If the point is arbitrary, the the intersection is a 
    neighborhood of all points in the intersection, making the finite intersection open.
\end{itemize}

Even though not explicitly stated, but $\emptyset$ is vacuously satisfying the condition of being an open set,
since it contains no points, the given condition is vacuously true.



\break

\section*{3}
\begin{question}
    Give an example of a countably infinite basis for the standard topology on R, 
    and prove it is a basis.
\end{question}

\textbf{Pf:}

For all $q\in \mathbb{Q}$, let the collection $V_q=\{(q-r,q+r)\ |\ r\in\mathbb{Q},\ r>0\}$, 
and define the collection of open sets $\mathcal{B}=\bigcup_{q\in \mathbb{Q}}V_q$.

Since collection of positive integer is countable, thus for each $q\in\mathbb{Q}$, there are countable open intervals collected in $V_q$;
and, since the rationals $\mathbb{Q}$ is countable, $\mathcal{B}=\bigcup_{q\in \mathbb{Q}}V_q$ is the union of countable collections of countable sets,
which is also countable.

\hfill

\textbf{$\mathcal{B}$ is a basis:}

There are two criteria to check:
\begin{itemize}
    \item[1.] For all $x\in\mathbb{R}$, since $\mathbb{Q}$ is dense in $\mathbb{R}$, then choose $\epsilon>0$ with $\epsilon <1$,
    there exists rational number $q\in (x-\epsilon,x+\epsilon)$, which $|q-x|<\epsilon<1$.
    Then, $x\in (q-1,q+1)$ (since its distance away from $q$ is strictly less than $1$), while $1\in \mathbb{Q}$ and $1>0$,
    thus $(q-1,q+1)\in V_q \subset \mathcal{B}$.

    So, every element $x\in\mathbb{R}$ is covered by an open interval in $\mathcal{B}$.

    \hfill

    \item[2.] For all $B_1,B_2\in\mathcal{B}$, there exists $q,q',r,r'\in \mathbb{Q}$ with $r,r'>0$, 
    such that $B_1=(q-r,q+r)$ and $B_2=(q'-r',q'+r')$.

    Now, consider $B_1\cap B_2$: if it is empty, then we don't need further proof;
    else, if it is nonempty, we'll show that $B_1\cap B_2$ is actually an element of $\mathcal{B}$.

    Given that $B_1\cap B_2\neq \emptyset$, then there exists $x\in B_1\cap B_2$, which $(q-r)<x<(q+r)$ and $(q'-r')<x<(q'+r')$.
    Now, define $m=\max\{(q-r),(q'-r')\}$ and $M=\min\{(q+r),(q'+r')\}$ (which both $m,M\in \mathbb{Q}$, since the elements $(q-r),(q'-r'),(q+r),(q'+r')\in\mathbb{Q}$),
    and let $q_0=\frac{m+M}{2}$, $r_0=\frac{M-m}{2}$.

    \hfill

    First, since $m,M\in\mathbb{Q}$, then $q_0,r_0\in\mathbb{Q}$; 
    also, $(q-r),(q'-r')<x$ implies $m <x$, and $(q+r),(q'+r')>x$ implies $M>x$, thus $m<x<M$, or $r_0=\frac{M-m}{2}>0$.
    Then, the set $B_3 = (q_0-r_0,q_0+r_0)\in V_{q_0}\subseteq\mathcal{B}$.

    Now, we'll prove that $B_3 = (B_1\cap B_2)$:

    (Note: $B_3 = (\frac{M+m}{2}-\frac{M-m}{2},\frac{M+m}{2}+\frac{M-m}{2}) = (m,M)$).
    \begin{itemize}
        \item[$\subseteq$:] For all $x\in B_3$, since $m<x<M$, then by the definition of $m$ and $M$,
        since $(q-r),(q'-r')\leq m$ and $(q+r),(q'+r')\geq M$, then:
        $$(q-r)\leq m<x<M\leq(q+r),\quad\quad (q'-r')\leq m<x<M\leq(q'+r')$$
        Thus, $x\in (q-r,q+r)=B_1$, and $x\in (q'-r',q'+r')=B_2$, or $x\in B_1\cap B_2$.
        This proves that $B_3\subseteq (B_1\cap B_2)$.

        \hfill

        \item[$\supseteq:$] For all $x\in (B_1\cap B_2)$, $x\in B_1$ implies $(q-r)<x<(q+r)$, and $x\in B_2$ implies $(q'-r')<x<(q'+r')$.
        Which, $(q-r),(q'-r')<x$ implies $m<x$ (since $m\in\{(q-r),(q'-r')\}$ by the definition of maximum) and $(q+r),(q'+r')>x$ implies $M>x$
        (again, since $M\in \{(q+r),(q'+r')\}$ by the definition of minimum).

        Thus, $m<x<M$, which $x\in (m,M) = B_3$, this proves that $(B_1\cap B_2)\subseteq B_3$.
    \end{itemize}

    With the above two implications, $(B_1\cap B_2)=B_3$, while $B_3\in \mathcal{B}$, thus every element in the intersection $(B_1\cap B_2)$
    is covered by an element $B_3=(B_1\cap B_2)\in\mathcal{B}$ that is a subset of the intersection (namely itself).
\end{itemize}
With the above two criteria, $\mathcal{B}$ is a basis of $\mathbb{R}$, and it is also countable.

\hfill

\textbf{$\mathcal{B}$ generates Standard Topology:}

Let $\mathcal{T_B}$ be the topology generated by $\mathcal{B}$, and let $\mathcal{T}$ be the standard topology on $\mathbb{R}$.

For all $U\in \mathcal{T_B}$, it is the union of some collection of basis elements in $\mathcal{B}$; but, every basis element in $\mathcal{B}$
is in fact an open set under standard topology, thus $\mathcal{B}\subseteq \mathcal{T}$, which union of any collection of basis elements in $\mathcal{B}$
is still an element of $\mathcal{T}$, thus $U\in \mathcal{T}$, or $\mathcal{T_B}\subseteq \mathcal{T}$.

\hfill

On the other hand, for all $V\in\mathcal{T}$, for any $x\in V$, there exists $\epsilon>0$, such that $(x-\epsilon,x+\epsilon)\subseteq V$.
Which, $(x-\frac{\epsilon}{2},x+\frac{\epsilon}{2})\subseteq (x-\epsilon,x+\epsilon)\subseteq V$ by definition.

Since $\mathbb{Q}$ is dense on $\mathbb{R}$, then there exists a rational number $q\in (x-\frac{\epsilon}{2},x+\frac{\epsilon}{2})$, with the following being true:
$$(x-\frac{\epsilon}{2})<q<(x+\frac{\epsilon}{2}),\quad -\frac{\epsilon}{2}<(q-x)<\frac{\epsilon}{2}, \quad |q-x|<\frac{\epsilon}{2}$$
Again, by denseness of $\mathbb{Q}$ on $\mathbb{R}$, there exists $r\in\mathbb{Q}$, with $0\leq |q-x| <r<\frac{\epsilon}{2}$.

Now, consider the following open interval $(q-r,q+r)\in V_q\subseteq \mathcal{B}$:

Since $|q-x|<r$, then $x\in (q-r,q+r)$; also, for all $z\in (q-r,q+r)$, $|q-z|<r<\frac{\epsilon}{2}$, while $|q-x|<\frac{\epsilon}{2}$, thus:
$$|x-z| = |(x-q)+(q-z)| \leq |x-q|+|q-z| < \frac{\epsilon}{2}+\frac{\epsilon}{2} = \epsilon$$
Thus, $z\in (x-\epsilon,x+\epsilon)$, or $(q-r,q+r)\subseteq(x-\epsilon,x+\epsilon)\subseteq V$.
So, for every $x\in V$, there exists basis element $B\in\mathcal{B}$, with $x\in B\subseteq V$, which $V$ is open under the topology generated by $\mathcal{B}$,
or $V\in \mathcal{T_B}$. Hence, $\mathcal{T}\subseteq \mathcal{T_B}$.

\hfill

With the two inclusions above, $\mathcal{T}=\mathcal{T_B}$, which we can conclude that $\mathcal{B}$ is a basis of Standard Topology.


\break

\section*{4}
\begin{question}
    A topological space $X$ is called Fréchet if: for every pair of distinct points 
    $x, y \in X$ there exists an open neighborhood $U$ of $x$ such that $y\notin U$. Prove that 
    $X$ is Fréchet if and only if every singleton in $X$ is closed.
\end{question}

\textbf{Pf:}
\begin{itemize}
    \item[$\implies:$]
    Suppose $X$ is Fréchet, then for all distinct points $x,y\in X$, there exists open neighborhood $U$ of $x$, with $y\notin U$.
    This implies that for arbitrary $x\in X$, since for any $y\in X\setminus\{x\}$, there exists open neighborhood $U_y$ of $y$, with $x\notin U_y$,
    thus $x\in X\setminus U_y$. Now, consider the following set:
    $$K=\bigcap_{y\in X\setminus\{x\}}(X\setminus U_y)$$
    Since $x\in X\setminus U_y$ for all $y\in X\setminus\{x\}$, then $x\in K$; furthermore, for any $y\in X\setminus\{x\}$, 
    since $y\in U_y$ by definitio, then $y\notin X\setminus U_y$, thus $y\notin K$. Which, we can conclude that $K=\{x\}$.

    However, since for all $y\in X\setminus\{x\}$, $U_y$ is open, thus $X\setminus U_y$ is closed, so $K$ is an intersection of closed sets, 
    which is also closed. Thus, $K=\{x\}$ is closed for arbitrary $x\in X$, all singleton set in $X$ is closed.
    

    \hfill

    \item[$\impliedby:$] 
    Suppose every singleton set in $X$ is closed, then for all distinct $x,y\in X$, since $\{y\}$ is closed, $X\setminus\{y\}$ is open.
    Which, since $x\in X\setminus\{y\}$ while $y\notin X\setminus\{y\}$, then $X\setminus\{y\}$ is the desired open neighborhood of $x$ not containing $y$, 
    proving that $X$ is actually Fréchet.

\end{itemize}

\hfill

\section*{5}
\begin{question}
    Prove that if $X$ is finite then the only Hausdorff topology on $X$ is the discrete topology.
\end{question}

\textbf{Pf:}

Suppose $X$ is finite, and a topology $\mathcal{T}$ on $X$ is Hausdorff. Then, for any $x,y\in X$, there exists disjoint sets $U,V\in\mathcal{T}$
such that $x\in U$ and $y\in V$. Which, $y\notin U$ and $x\notin V$ in particular.

Now, for arbitrary $x\in X$, since $\mathcal{T}$ is a Hausdorff topology, then for any $y\in X\setminus\{x\}$, there exists $U_y\in \mathcal{T}$,
with $x\in U_y$ but $y\notin U_y$. (Note: Hausdorff imposes a stronger condition, but here we just need a set containing $x$ while not containing $y$).

\hfill

Then, consider the following set:
$$K=\bigcap_{y\in X\setminus\{x\}}U_y$$
By the definition above, since $x\in U_y$ for all $y\in X\setminus\{x\}$, then $x\in K$;
however, for any $y\in X\setminus\{x\}$, since $y\notin U_y$, then $y\notin K$. Thus, $K=\{x\}$.

Notice that since $X$ is finite, there are finitely many $y\in X\setminus\{x\}$, thus the collection $\{U_y\}_{y\in X\setminus\{x\}}$ is finite,
$K$ is in fact intersection of finitely many open sets, which is open (since $U_y\in \mathcal{T}$ for all $y\in X\setminus\{x\}$).

So, for arbitrary $x\in X$, the singleton set $\{x\}$ is open. Which, any subset $U\in \mathcal{P}(X)$ is union of singleton set of elements in $U$,
and by the statement that all singleton set is open, the union $U$ is also open.

\hfill

Thus, every subset of $X$ is open, showing that $\mathcal{P}(X)\subseteq \mathcal{T}$; and since $\mathcal{T}$ is a collection of open subsets in $X$, 
thus $\mathcal{T}\subseteq \mathcal{P}(X)$, or $\mathcal{T}=\mathcal{P}(X)$, showing that the topology is in fact discrete topology.


\end{document}