%Math_CS_Topology_HW3_Zih-Yu_Hsieh.tex

\documentclass{article}
\usepackage{graphicx} % Required for inserting images
\usepackage[margin = 2.54cm]{geometry}
\usepackage[most]{tcolorbox}

\newtcolorbox{myBox}[3]{
arc=5mm,
lower separated=false,
fonttitle=\bfseries,
%colbacktitle=green!10,
%coltitle=green!50!black,
enhanced,
attach boxed title to top left={xshift=0.5cm,
        yshift=-2mm},
colframe=blue!50!black,
colback=blue!10
}

\usepackage{amsmath}
\usepackage{amssymb}
\usepackage{verbatim}
\usepackage[utf8]{inputenc}
%\linespread{1.5}

\newtheorem{definition}{Definition}
\newtheorem{proposition}{Proposition}
\newtheorem{theorem}{Theorem}
\newtheorem{question}{Question}

\title{Math CS Topology HW3}
\author{Zih-Yu Hsieh}

\begin{document}
\maketitle

\section*{1}
\begin{myBox}[]{}
    \begin{question}
        Prove that every subspace of a Hausdorff space is Hausdorff.
    \end{question}
\end{myBox}

\textbf{Pf:}

Suppose $X$ is a Hausdorff topological space, and define the subspace topology on $A\subseteq X$.

Then, for all distinct elements $a,b\in A\subseteq X$, since $X$ is Hausdorff, there exists two disjoint open sets $U,V\subseteq X$, 
with $a\in U$ and $b\in V$ (which $a\notin V$ and $b\notin U$).

Then, since $a\in U\cap A$ (which is an open neighborhood of $a$ in subspace topology on $A$) and $b\in V\cap A$ (which is an open neighborhood of $b$ in subspace topology on $A$).

Also, since $U\cap V=\emptyset$, then $(U\cap A)\cap (V\cap A)=(U\cap V)\cap A=\emptyset$, hence the two open sets are disjoint.

\hfill

So, for any two points $a,b\in A$, there exists two disjoint open sets $(U\cap A)$ and $(V\cap A)$ under subspace topology,
such that $a\in (U\cap A)$ and $b\in (V\cap A)$, showing that $A$ is Hausdorff under subspace topology.

\hfill

\hfill

\section*{2}
\begin{myBox}[]{}
    \begin{question}
        Prove that every product of two Hausdorff spaces is Hausdorff.
    \end{question}
\end{myBox}

\textbf{Pf:}

Given $X,Y$ two Hausdorff topological space, and define the product topology on $X\times Y$.
Then, for any distinct elements $(x_1,y_1),(x_2,y_2)\in X\times Y$, either $x_1\neq x_2$ or $y_1\neq y_2$.

\hfill

If $x_1\neq x_2$, then since $X$ is Hausdorff, there exists two disjoint open sets $U,V\subseteq X$, with $x_1\in U$ and $x_2\in V$.
Then, consider the basis element $U\times Y$ and $V\times Y$: From the above construction, $(x_1,y_1)\in U\times Y$ while $(x_2,y_2)\in V\times Y$;
also, since $(U\times Y)\cap (V\times Y) = (U\cap V)\times (Y\cap Y) = \emptyset$ (since $U\cap V=\emptyset$), then the two basis elements of product topology are disjoint,
while each cover one of the given points.

Else, if $y_1\neq y_2$, by similar logic ($Y$ is Hausdorff), there exists two disjoint open sets $A,B\subseteq Y$, which $y_1\in A$ and $y_2\in B$.
Then, the basis element $X\times A$ and $X\times B$ satisfies $(X\times A)\cap (X\times B)=(X\cap X)\times (A\cap B)=\emptyset$ (since $A\cap B=\emptyset$ by assumption),
and $(x_1,y_1)\in X\times A$ while $(x_2,y_2)\in X\times B$, which again the disjoint basis elements each cover one of the given points.

\hfill

Regardless of the case, we'll find two open sets (basis elements) that cover $(x_1,y_1)$ and $(x_2,y_2)$ respectively,
showing that $X\times Y$ under product topology is also Hausdorff.

\break

\section*{3}
\begin{myBox}[]{}
    \begin{question}
        Give an example of a quotient of a Hausdorff space that is not Hausdorff.
    \end{question}
\end{myBox}

\textbf{Pf:}

Given $\mathbb{R}$ with the standard topology, and the following function $f:\mathbb{R}\rightarrow\mathbb{Z}$:
$$f(x)=\begin{cases}
    x & x\in\mathbb{Z}\\
    2n & x\notin \mathbb{Z},\ n\in\mathbb{Z},\ |x-2n|=\min\{|x-2k|\ :\ k\in\mathbb{Z}\}
\end{cases}$$
So, if $x\in\mathbb{Z}$, the function does nothing; else if $x\notin Z$, then it gets mapped to the closest even integer in $\mathbb{Z}$, which the function is surjective.
(Note: this is the Digital Line Topology covered in class).

\hfill

Now, if we apply the quotient topology on $\mathbb{Z}$ according to this surjective function, the space is not Hausdorff:

Consider the element $1\in\mathbb{Z}$. For all open neighborhood $V\subseteq \mathbb{Z}$ of $1$, $f^{-1}(V)\subseteq \mathbb{R}$ is open.

Which, since $1\in\mathbb{R}$ satisfies $f(1)=1$, then $1\in f^{-1}(V)$; and since $f^{-1}(V)\subseteq \mathbb{R}$ is open under standard topology,
there exists radius $r>0$, such that $(1-r,1+r)\subseteq f^{-1}(V)$ (also for simplicity, choose $r<1$).

Then, consider the element $(1-\frac{r}{2})\in (1-r,1+r)\subseteq f^{-1}(V)$: since $0<r<1$, then $0<(1-\frac{r}{2})<1<2$, which the closest even integer to $(1-\frac{r}{2})$ is $0$.
Thus, $f(1-\frac{r}{2})=0$, showing that $0\in V$.

So, any open neighborhood $V\subseteq \mathbb{Z}$ of $1$ must contain $0$, there doesn't exist two disjoint open subsets $U,V\subseteq \mathbb{Z}$
with $1\in V$ and $0\in U$, showing that $\mathbb{Z}$ is not Hausdorff under this quotient topology, while $\mathbb{R}$ is in fact Hausdorff under standard topology.

\hfill

\hfill

\section*{4}
\begin{myBox}[]{}
    \begin{question}
        Prove that a Möbius band can be obtained by gluing two of the edges of a
        triangle. This can be a topology-style proof that relies heavily on pictures.
    \end{question}
\end{myBox}

\textbf{Pf:}

The following graph shows how having the two edges of the triangle needs to be glued in opposite orientation, 
we can form a Möbius band out of it: By cutting the triangle that separates the two sides with opposite orientation,
we can glue the two original sides together with the orientation agrees with each other.

Now, it will form a rectangle with the diagonal being the original two edges of the triangle, 
and the edge created by the cut becomes the two opposite sides with opposite orientations, whch is topologically equivalent
to a Möbius band.

\begin{figure}[h!]
    \begin{center}
        \includegraphics*[width=100mm]{mobius.jpg}
        \caption{The Equivalence of Triangle, Rectangle, and Möbius Band under some Orientation}
    \end{center}
\end{figure}

\end{document}