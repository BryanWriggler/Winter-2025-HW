% Math_CS_Topology_HW4_Zih-Yu_Hsieh.tex

\documentclass{article}
\usepackage{graphicx} % Required for inserting images
\usepackage[margin = 2.54cm]{geometry}
\usepackage[most]{tcolorbox}

\newtcolorbox{myBox}[3]{
arc=5mm,
lower separated=false,
fonttitle=\bfseries,
%colbacktitle=green!10,
%coltitle=green!50!black,
enhanced,
attach boxed title to top left={xshift=0.5cm,
        yshift=-2mm},
colframe=blue!50!black,
colback=blue!10
}

\usepackage{amsmath}
\usepackage{amssymb}
\usepackage{verbatim}
\usepackage[utf8]{inputenc}
\linespread{1.2}

\newtheorem{definition}{Definition}
\newtheorem{proposition}{Proposition}
\newtheorem{theorem}{Theorem}
\newtheorem{question}{Question}

\title{Math CS Topology HW4}
\author{Zih-Yu Hsieh}

\begin{document}
\maketitle

\section*{1}
\begin{myBox}[]{}
    \begin{question}
        Let $f:X\rightarrow Y$ be a continuous map between topological spaces, and suppose
        $Y$ is Hausdorff. Prove that the graph $\{(x,f(x))\ |\ x\in X\}$ is a closed subset of
        $X\times Y$.
    \end{question}
\end{myBox}

\textbf{Pf:}

Let $G=\{(x,f(x))\ |\ x\in X\}$ be the graph. To prove that $G$ is closed, it is equivalent to show that $X\setminus G$ is open.

For all $(x,y)\in X\setminus G$, since the element is not in $G$, then $y\neq f(x)$. Then, by the Hausdorff Property of $Y$, 
there exists disjoint open subsets $U,V\subseteq Y$, such that $f(x)\in U$ and $y\in V$.

\hfill

Notice that because $f$ is continuous, then $f^{-1}(U)$ and $f^{-1}(V)$ are open; furthermore, since $U\cap V=\emptyset$, 
then $f^{-1}(U)\cap f^{-1}(V)=\emptyset$.

Now, consider the basis element $f^{-1}(U)\times V$: First, it is an open neighborhood of $(x,y)$, since $y\in V$ and $f(x)\in U$ (which implies $x\in f^{-1}(U)$);
futhermore, for all $a\in f^{-1}(U)$, since $f^{-1}(U)\cap f^{-1}(V)=\emptyset$, then $f(a)\notin V$. 
Hence, for all $(a,b)\in f^{-1}(U)\times V$, since $f(a)\notin V$, $(a,f(a))\notin f^{-1}(U)\times V$, hence $(a,b)\neq (a,f(a))$ (which $b\neq f(a)$),
showing that $(a,b)\notin G$.

Therefore, $(a,b)\in X\setminus G$, implying that $f^{-1}(U)\times V\subseteq G$.

\hfill

So, for all $(x,y)\in X\setminus G$, there exists a basis element $B$ (Note: $B=f^{-1}(U)\times V$ in the above construction),
such that $(x,y)\in B\subseteq X\setminus G$, hence $X\setminus G$ is open, showing that its complement $G$ is closed.

Hence, the graph $G=\{(x,f(x))\ |\ x\in X\}$ is closed.

\break

\section*{2}
\begin{myBox}[]{}
    \begin{question}
        Prove that if $X$ and $Y$ are nonempty topological spaces then 
        $X$ is homeomorphic to a subspace of $X\times Y$.
    \end{question}
\end{myBox}

\textbf{Pf:}

Since $Y$ is not empty, there exists $y_0\in Y$. Consider the following map $f:X\rightarrow X\times Y$, such that for all $x\in X$,
$f(x)=(x,y_0)$, which $f(X)=X\times \{y_0\}$ (since for all $x\in X$, $f(x)=(x,y_0)\in X\times \{y_0\}$, and for all $(x,y_0)\in X\times \{y_0\}$, $f(x)=(x,y_0)$).
So, we'll restrict the codomain to the set $X\times \{y_0\}$, letting $f:X\rightarrow X\times \{y_0\}$.

\hfill

\textbf{$f$ is Bijective:}

First, we've verified that $f(X)=X\times \{y_0\}$, hence restricting the codomain to the image had made the map surjective.

To verify injectivity, consider $x_1,x_2 \in X$: If $f(x_1)=f(x_2)$, then $(x_1,y_0)=(x_2,y_0)$, so $x_1=x_2$, proving that it's injective.

So, the map $f$ is bijective, and $f^{-1}:X\times \{y_0\}\rightarrow X$ satisfies $f(x,y_0)=x$.

\hfill

\textbf{$f$ is Continuous:}

For all ope subset $U'\subseteq X\times \{y_0\}$, there exists open subset $U\subseteq X\times Y$, with $U\cap (X\times \{y_0\}) = U'$.
Now, consider the preimage $f^{-1}(U')$: For all $x\in f^{-1}(U')$, since $f(x)=(x,y_0)\in U' \subseteq U$, there exists a basis element $A\times B$ 
(where $A\subseteq X$ and $B\subseteq Y$ are both open), such that $(x,y_0)\in A\times B \subseteq U$.
Which:
$$A\times \{y_0\} = (A\cap X)\times (B\cap \{y_0\}) = (A\times B)\cap (X\times \{y_0\}) \subseteq U\cap (X\times \{y_0\})=U'$$
So, $A\times \{y_0\}$ is an open subset of $X\times \{y_0\}$ under subspace topology, and $(A\times \{y_0\})\subseteq U'$.

Now, consider all $a\in A\subseteq X$: Since $f(a)=(a,y_0)\in (A\times \{y_0\})\subseteq U'$, then $a\in f^{-1}(U')$.
Hence, $A\subseteq f^{-1}(U')$. Also, recall that $x\in A$, hence $x\in A \subseteq f^{-1}(U')$. 

So, for every $x\in f^{-1}(U')$, there is an open subset $A\subseteq X$, with $x\in A\subseteq f^{-1}(U')$, showing that $f^{-1}(U')\subseteq X$ is open.

Therefore, we can conclude that $f$ is continuous, since every open subset of $X\times \{y_0\}$ the image, the preimage in $X$ is open.

\hfill

\textbf{$f^{-1}$ is Continuous:}

For all open subset $U\subseteq X$, notice that for all $(x,y_0)\in X\times \{y_0\}$, $f^{-1}(x,y_0)=x\in U$ if and only if $x\in U$,
hence the preimage $(f^{-1})^{-1}(U) = U\times \{y_0\}$. Which, consider $U\times Y$ an open subset of $X\times Y$, the following is true:
$$(U\times Y)\cap (X\times \{y_0\}) = (U\cap X)\times (Y\cap \{y_0\}) = U\times \{y_0\}$$
Hence, $U\times \{y_0\}$ is an intersection of $X\times \{y_0\}$ and $(U\times Y)$, proving that $U\times \{y_0\}$ is an open subset of $X\times \{y_0\}$ under subspace topology,
so the preimage of $U$ under $f^{-1}$, $(f^{-1})^{-1}(U) = U\times \{y_0\}$ is open, showing that $f^{-1}$ is continuous, since all open subset of $X$ has a preimage being open.

\hfill

Because $f^{-1}$ exists when restricting the codomain to $X\times \{y_0\}$, and both $f$ and $f^{-1}$ are continuous using the given topology,
hence $f$ is a homeomorphism, showing that $X$ and $X\times \{y_0\}$ (as a subspace of $X\times Y$) are homeomorphic.

\break

\section*{3}
\begin{myBox}[]{}
    \begin{question}
        If $X$ is a metric space, prove that the distance function $d:X\times X\rightarrow \mathbb{R}$ is
        continuous, where $X\times X$ has the product of the metric topologies.
    \end{question}
\end{myBox}

\textbf{Pf:}

For all open subset $U\subseteq \mathbb{R}$, consider the preimage $d^{-1}(U)\subseteq X\times X$:

For all $(x_1,x_2)\in d^{-1}(U)$, since $y=d(x_1,x_2)\in U$ while $U$ is open under standard topology of $\mathbb{R}$, 
then there exists $r>0$, such that $(y-\frac{r}{3},y+\frac{r}{3})\subseteq (y-r,y+r) \subseteq U$.

\hfill

Now, consider the basis element $\left(B_d(x_1,\frac{r}{3})\times B_d(x_2,\frac{r}{3})\right)\subseteq X\times X$ under product topology:
For all $(a,b)\in \left(B_d(x_1,\frac{r}{3})\times B_d(x_2,\frac{r}{3})\right)$, the following is true:
$$d(a,b) \leq d(a,x_1) + d(x_1,b) \leq d(a,x_1)+d(x_1,x_2)+d(x_2,b) < \frac{r}{3}+y+\frac{r}{3} = y+\frac{2r}{3}$$
(Note: the above is true, since $a\in B_d(x_1,\frac{r}{3})$ and $b\in B_d(x_2,\frac{r}{3})$).

Hence, since $y<y+\frac{2r}{3} < y+r$, then $d(a,b)=(y+\frac{2r}{3}) \in (y-r,y+r)\subseteq U$, showing that $(a,b)\in d^{-1}(U)$.
And, since the choice of $(a,b)\in \left(B_d(x_1,\frac{r}{3})\times B_d(x_2,\frac{r}{3})\right)$ is arbitrary, $\left(B_d(x_1,\frac{r}{3})\times B_d(x_2,\frac{r}{3})\right) \subseteq f^{-1}(U)$.

\hfill

So, for all $(x_1,x_2) \in d^{-1}(U) \subseteq X\times X$, there exists a basis element $\left(B_d(x_1,\frac{r}{3})\times B_d(x_2,\frac{r}{3})\right)$, 
such that $(x_1,x_2)\in \left(B_d(x_1,\frac{r}{3})\times B_d(x_2,\frac{r}{3})\right)\subseteq d^{-1}(U)$,
hence $d^{-1}(U)$ is open.

Which, we can conclude that $d:X\times X\rightarrow \mathbb{R}$ is continuous under product topology of $X\times X$ (based on metric topology of $X$), and standard topology of $\mathbb{R}$.

\break

\section*{4}
\begin{myBox}[]{}
    \begin{question}
        Let $x$ be a point in a metric space $X$. Prove that $\{y\in X\ |\ d(x,y)\leq 1\}$ is
        closed, but is not necessarily equal to the closure of the unit open ball $B(x, 1)$.
        (This is contrary to Exercise 5.14b in my copy of the textbook.)
    \end{question}
\end{myBox}

\textbf{Pf:}

To prove that $C=\{y\in X\ |\ d(x,y)\leq 1\}$ is closed, it suffices to prove that $X\setminus C = \{y\in X\ |\ d(x,y)>1\}$ is open.

For all $y\in X\setminus C$, $d(x,y)>1$. Which, consider $r=d(x,y)-1>0$, and the open ball $B(y,r)$:
For all $z\in B(y,r)$, $d(y,z) < r = d(x,y)-1$. Which, consider $d(x,z)$, the following is true:
$$d(x,y) \leq d(x,z)+d(y,z) < d(x,z)+d(x,y)-1$$
$$0 < d(x,z)-1,\quad 1<d(x,z)$$
Hence, we can conclude that $z\in X\setminus C$, which $y\in B(y,r)\subseteq (X\setminus C)$.

Since for all points in $X\setminus C$, there exists a basis element containing the point, that is a subset of $X\setminus C$,
then $X\setminus C$ is open, hence $C$ is closed.

\hfill

\textbf{Closure of Open Ball and Closed Ball could be Different:}

For any nonempty set $X$ with more than one element, consider the discrete metric $d:X\times X\rightarrow \mathbb{R}$ defined as follow:
$$d(x,y)=\begin{cases}
    1 & x=y\\
    0 & x\neq y
\end{cases}$$
For all $x\in X$, the ball $B(x,1)=\{x\}$, since for all $y\in X$ with $y\neq x$, $d(x,y)=1$, so $y \notin B(x,1)$ (since the distance is strictly smaller than 1).

Which, if we take the closed ball of distance $1$ around $x$, $CB(x,1)=\{y\in X\ |\ d(x,y)\leq 1\}=X$ (since everything has distance at most $1$ from $x$).

Yet, the closure of open ball with radius 1, is $\overline{B(x,1)}=\{x\}$, since under discrete metric,
$\{x\}$ is also a closed set containing itself, hence the closure (which is the intersection of closed set containing $\{x\}$) must be $\{x\}$,
because it is the smallest closed set containing itself.

Hence, $CB(x,1)\neq \overline{B(x,1)}$, showing that under extreme cases (like discrete metric), the two may not be the same.

\end{document}