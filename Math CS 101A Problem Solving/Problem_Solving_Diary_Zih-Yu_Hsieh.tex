%name: Problem_Solving_Diary_Zih-Yu_Hsieh.tex

\documentclass{article}
\usepackage{graphicx} % Required for inserting images
\usepackage[margin = 2.54cm]{geometry}

\usepackage{amsmath}
\usepackage{amssymb}
\usepackage{verbatim}

\usepackage{hyperref}

%\linespread{1.5}

\newtheorem{definition}{Definition}
\newtheorem{proposition}{Proposition}
\newtheorem{theorem}{Theorem}
\newtheorem{question}{Question}

\title{Math CS 101A Problem Solving Diary}
\author{Zih-Yu Hsieh (Bryan Hsieh)}

\hypersetup{
  colorlinks   = true,    % Colours links instead of ugly boxes
  urlcolor     = blue,    % Colour for external hyperlinks
  linkcolor    = blue,    % Colour of internal links
  citecolor    = red      % Colour of citations
}

\begin{document}
\maketitle

\section*{Contents}
\begin{itemize}
    \item \hyperlink{section.1}{A Mathematical Adventure in the Jungle}
    \begin{itemize}
        \item{} \hyperlink{subsection.1.1}{Problem 1}
        \item{} \hyperlink{subsection.1.2}{Problem 2}
        \item{} \hyperlink{subsection.1.3}{Problem 3}
        \item{} \hyperlink{subsection.1.4}{Problem 4}
        \item{} \hyperlink{subsection.1.5}{Problem 5}
        \item{} \hyperlink{subsection.1.6}{Problem 6}
        \item{} \hyperlink{subsection.1.7}{Problem 7}
    \end{itemize}
    \item \hyperlink{section.2}{Argument by Contradiction}
    \begin{itemize}
        \item{} \hyperlink{subsection.2.1}{Problem 1}
        \item{} \hyperlink{subsection.2.2}{Problem 2}
        \item{} \hyperlink{subsection.2.3}{Problem 3}
        \item{} \hyperlink{subsection.2.4}{Problem 4}
        \item{} \hyperlink{subsection.2.5}{Problem 5}
        \item{} \hyperlink{subsection.2.6}{Problem 6}
        \item{} \hyperlink{subsection.2.7}{Problem 7}
        \item{} \hyperlink{subsection.2.8}{Problem 8}
    \end{itemize}
\end{itemize}

\break

\section{A Mathematical Adventure in the Jungle}
\subsection{Problem 1}
The goal is to let everyone - you, your sister Maria, Mom, and Dad cross the bridge, with some restrictions:

1. At most 2 people can cross the bridge together

2. Cross the bridge requires a flashlight, and there is only one.

3. Everyone needs some time to cross the bridge: you need 5 minutes, Maria needs 10 minutes, Mom needs 20 minutes, and Dad needs 25 minutes.

4. If 2 people cross the bridge together, then it requires the time the slower person need to cross the bridge 
(e.g. if one needs 5 minutes while the other needs 10, then the two together need 10 minutes).

5. All people need to cross the bridge in 60 minutes.

For the first round, you and Maria cross the bridge together with the flashlight, which takes 10 minutes (longest of the two); 
then, you take the flashlight and go back alone, which takes 5 minutes. 
Now, Maria is successfully across the bridge, and the total time spent is 10+5=15 minutes.

For the second round, let your mom and dad cross the bridge together with the flashlight, which takes 25 minutes (dads time is longest); 
then, let Maria take the flashlight and go back alone, which takes 10 minutes. Now, Mom and Dad are across the bridge, 
and the total time spend is 15+25+10 = 50 minutes.

For the last round, you and Maria cross the bridge together with the flashlight, which takes 10 minutes. Then, everyone had crossed the bridge, 
the total time spend is 50+10=60 minutes.

So, it's possible to let everyone cross the bridge in 60 minutes.


\hfill

\subsection{Problem 2}
Now you have 3 pills in your palm, 1 from the yellow bottle and 2 from the blue bottle, and it's indistinguishable between the two type of pills; 
also, there are some remaining pills in each bottle still.
Here is the restriction: Dad needs to eat exactly 1 pill from the yellow bottle and 1 pill from the blue bottle simultaneously each day. No less, no more. 

So, what we can do is carefully pour another pill from the yellow bottle, so now there are 4 pills outside, 2 from the yellow and 2 from the blue bottle.

Then, break each pill in half, and let Dad eat half of the pill from all 4 of them. With this strategy, Dad swallowed exactly half of the given amount, 
which is half of 2 pills from yellow bottle and half of 2 pills from blue bottle, having exactly 1 pill from each bottle. This satisfies day 2, 
and the remaining other half of these 4 pills satisfies day 3 similarly. Note that you have 2 yellow and 2 blue pills remaining.
For day 4 and 5, you follow the original instructions, taking one blue and one yellow pill each day, finishing the exact amount by the end of day 5.

\hfill

\subsection{Problem 3}
There is a 3x3 grid formed by 9 sticks stuck on the ground. With a long rope, the goal is to connect all 9 sticks while only bending the rope 3 times. 
(In other words, connecting 3x3 grid points with 4 straight lines).

Instead of letting the rope always bend around the existed sticks, if we can bend the rope at the place without a stick, then the following graph is a solution:
\textbf{Insert Picture Here}

\hfill

\subsection{Problem 4}
Ok, so they tell you that they have two wooden containers of the same size. They fill one of them with water from the river and the other one with a sacred concoction 
so that both containers have the same amount of liquid. 
Then, they pour a small amount of the water into the container with the sacred liquid, mix well, and pour back some of this mix into the water container 
until both containers end up having the exact amount of liquid. Which container has the higher proportion of foreign liquid?

Observe the sacred liquid in the water container. It is not in the sacred liquid. In order for the sacred liquid container to have its starting volume, 
the rest of the volume in the sacred liquid container must be water, and with the same volume as the sacred liquid that’s in the water container. 
Therefore, they are equal, i.e. the amount of water in the sacred liquid container must equal the amount of sacred liquid in the water container.


\hfill

\subsection{Problem 5}
We were given a knife and two 50-ft ropes hanging on the ceiling of a 50-ft tall house. This house is 100 ft above the ground. Nobody can fall more than 10 ft, 
and we can’t cut the rope lengthwise. Then, to get back to the ground, we need a rope of at least 90 ft.

First, climb up one of the rope to the ceiling, and use less than 10 ft of the other rope to tie a loop on the ceiling. 
Now, hang yourself with the loop on the ceiling, cut the rope you climbed up while holding it (which provides 50 ft), then also cut the rope below the loop while holding it 
(which provides more than 40 ft, since only less than 10 ft is used for the loop).

Then, tie the 40 and 50 ft together, that’s a 90 ft long rope. Pass this rope through the loop, while the loop bisects the rope, then we have two 45 ft long ropes hanging on the ceiling, 
which we can safely get back down (as there is only a 5 ft gap from the floor with the 45 ft long ropes).

After sliding down, we can take the 90 ft long rope, and use that to safely get down to the ground, as there is only a 10 ft gap from the ground with the 90 ft long rope.


\hfill

\subsection{Problem 6}
Observe that the 25s will never move because they are always the biggest card (if it’s equal, than someone has 2 and the game ends). 
Once the 24s are out of the hands of the people with the 25s (if they were in the first place), they also never move for the same reason (as the 25s will never get passed, so there’s no bigger card in circulation). 
The same logic applies to all the smaller cards through 14. At this point, if someone hasn’t won yet, the only cards being passed are 13s and below. There’s 12 numbers being fixed, 
each with 2 cards, so 24 people have a card that never moves. Eventually, the 25th person gets a 13, which they hold since there isn’t a bigger card in circulation. 
There’s another 13 in circulation, which no one else will keep. Thus the 25th person gets it and has 2 13s. The game ends.


\hfill

\subsection{Problem 7}
Upstairs, there is a room with a light bulb that is turned off. It turns out that the switch to turn it on is on the first floor. But there are three switches and only one of them controls the bulb on the second floor. 
After so many days of solving interesting puzzles and problems, your sister challenges you to discover which switch controls the bulb, but you are only allowed to go upstairs once. 
How do you do it? (no fancy strings, telescopes, etc. allowed. You cannot see the upstairs room from downstairs. The light bulbs are standard 100-watt bulbs).

Figure out what the other two switches do by observing their effects downstairs. Alternatively if the bulb is incandescent, turn a switch on for a hot sec, then switch it off and another one on. If the light’s on, 
it's the switch that’s currently on. If the bulb’s hot, it's the switch that used to be on. If the bulb’s cold and off, it’s the other switch.


\hfill

\break

\section{Argument by Contradiction}
\subsection{Problem 1}

\textbf{Prove that $\sqrt{2}+\sqrt{3}+\sqrt{5}$ is an irrational number.}

\textbf{Pf:}

Suppose $q=\sqrt{2}+\sqrt{3}+\sqrt{5}$ is rational. First, a lower bound and upper bound of $q$ can be figured out as follow:
$$1 < 2 < 4 \implies 1 < \sqrt{2} < 2,\quad \quad 1 < 3 < 4 \implies 1 < \sqrt{3} < 2,\quad\quad 4 < 5 < 9 \implies 2<\sqrt{5}<3$$
Which implies that $4=(1+1+2) < (\sqrt{2}+\sqrt{3}+\sqrt{5})<(2+2+3)=7$, thus $4<q<7$.

\hfill

Now, consider the following derivation:
$$q=\sqrt{2}+\sqrt{3}+\sqrt{5} \implies (q-\sqrt{2})=(\sqrt{3}+\sqrt{5})\implies(q-\sqrt{2})^2=(\sqrt{3}+\sqrt{5})^2$$
$$\implies q^2-2\sqrt{2}q+2=3+2\sqrt{15}+5\implies (q^2-6)=(2\sqrt{2}q+2\sqrt{15})\implies (q^2-6)^2=(2\sqrt{2}q+2\sqrt{15})^2$$
$$\implies q^4-12q^2+36=8q^2+8\sqrt{30}q+60\implies (q^4-20q^2-24)=8\sqrt{30}q\implies (q^4-20q^2-24)^2=(8\sqrt{30}q)^2$$
$$\implies q^8+(-20)^2q^4+(24)^2 - 2\cdot 20q^6+2\cdot20\cdot24q^2-2\cdot24q^4=64\cdot30 q^2$$
$$\implies q^8-2\cdot 20q^6+(20^2-2\cdot24)q^4 + (40\cdot24-64\cdot30)q^2 + (24)^2=0$$

Which, let $p(x)=x^8-2\cdot 20x^6+(20^2-2\cdot24)x^4 + (40\cdot24-64\cdot30)x^2 + (24)^2$, $p(q)=0$, which $q$ is a rational root of $p(x)$. Thus, by Rational Root Theorem, $q=\frac{b}{a}$, where $a, b$ are integers such that $a\bigm|1$ (coefficient of highest degree term, $x^8$) and $b\bigm|(24)^2$ (constant term, $(24)^2$). 

Since $a$ divides $1$, $a=1$ or $a=-1$. Thus, $q=\frac{b}{a}$ can actually be expressed as $b$ or $-b$, showing that $q$ is an integer, and $q\bigm| (24)^2$. Furthermore, initially we've proven that $4<q<7$, the only possible integers are $q=5$ and $q=6$. Also, since $5$ is not a divisor of $(24)^2$, the only possibility is $q=6$.

\hfill

However, if $q=\sqrt{2}+\sqrt{3}+\sqrt{5}=6$, there is a contradiction: Consider squaring both sides, we get the following:
$$(\sqrt{2}+\sqrt{3}+\sqrt{5})^2=6^2,\quad 2+3+5+2(\sqrt{6}+\sqrt{10}+\sqrt{15}) = 36$$
$$2(\sqrt{6}+\sqrt{10}+\sqrt{15})=26,\quad \sqrt{6}+\sqrt{10}+\sqrt{15} = 13$$
However, we can also yield an upper bound for $\sqrt{6}+\sqrt{10}+\sqrt{15}$:
$$6<9\implies \sqrt{6}<3,\quad 10,15 < 16 \implies \sqrt{10},\sqrt{15} < 4$$
Thus, $(\sqrt{6}+\sqrt{10}+\sqrt{15}) < (3+4+4) = 11$, or $13 = (\sqrt{6}+\sqrt{10}+\sqrt{15}) < 11$, which is a contradiction.

\hfill

Now, we can conclude that the assumption is false, which $(\sqrt{2}+\sqrt{3}+\sqrt{5})$ is irrational.

\hfill

\subsection{Problem 2}
\textbf{Show that there are no positive integer solutions to the diophantine equation $x^2-y^2=1$.}

\textbf{Pf:}

Suppose there exists positive integer $x,y$ satisfying $x^2-y^2=1$.

Since both $x,y$ are positive, $x,y\geq 1$, which $(x+y) \geq 2>0$. 

Then, by factorization, $x^2-y^2=(x-y)(x+y)=1$, which both $(x-y)$ and $(x+y)$ are integers. Let $k=(x-y)$ be the integer, then $1=k(x+y)$, which implies that $1$ is divisible by $(x+y)$, a positive integer that is greater than 1. However, this is a contradiction, since the only positive integer that can divide 1 is 1 itself.

So, the assumption is false, there are no positive integer solutions to $x^2-y^2=1$.

\hfill

\subsection{Problem 3}
\textbf{Show that the equation $b^2+b+1=a^2$ has no positive integer solutions.}

\textbf{Pf:}

Suppose there exists positive integer $a,b$ satisfying $b^2+b+1=a^2$. Then, the following two equations are true:
$$(b-1)^2  = b^2-2b+1 = (b^2+b+1)-3b = a^2-3b$$
$$(b+1)^2 = b^2+2b+1=(b^2+b+1)+b = a^2+b$$
Since we assume $b>0$, $-3b<0$. Hence the following two inequalities are true:
$$(b-1)^2 = a^2-3b < a^2$$
$$a^2 < a^2+b = (b+1)^2$$
So, $(b-1)^2<a^2<(b+1)^2$.

\hfill

Now, because $b>0$, then $(b-1)\geq 0$ and $(b+1)>0$. Which, the above inequality implies the following:
$$(b-1) < a < (b+1)$$
Which, the only integer satisfying $(b-1)<a<(b+1)$ is $a=b$.

Then, plug the solution back to the equation, we get $b^2+b+1=a^2=b^2$, or $b+1=0$, thus $b=-1$. However, this contradicts with our assumption that $b$ is positive. 

So, the assumption is false, there doesn't exist positive integer solutions to the equation $b^2+b+1=a^2$.

\hfill

\subsection{Problem 4}
\textbf{Prove that there is no polynomial $P(x)=a_nx^n+...+a_0$ with integer coefficients $a_i$ and of degree $n$ at least 1 with the property that $P(0), P(1), P(2),...$ are all prime numbers.}

\textbf{Pf:}

Suppose there exists polynomial $P(x)=a_nx^n+...+a_0$ with degree $n\geq 1$ that satisfies $P(n)$ is a prime number for all $n\in\mathbb{N}$.

\hfill

Since $P(0) = a_n0^n+a_{n-1}0^{n-1}+...+a_10+a_0 = a_0$, and $P(0)$ is prime, then $a_0$ is prime (which $a_0>0$).

Then, for all positive integer $k$, since $ka_0$ is a positive integer, then: 
$$P(ka_0) = a_n(ka_0)^n+a_{n-1}(ka_0)^{n-1}+...+a_1(ka_0)+a_0$$
$$P(ka_0) = a_0(a_nk^na_0^{n-1}+a_{n-1}k^{n-1}a_0^{n-2}+...+a_1k+1)$$
Thus, $P(ka_0)$ is divisible by $a_0$; however, since $P(ka_0)$ is also a prime by assumption, then $P(ka_0)=a_0$ is enforced (if $P(ka_0)\neq a_0$, then the two are distinct primes while $a_0\bigm| P(ka_0)$, which contradicts).

\hfill

Now, consider $Q(x)=P(x)-a_0$: from the above section, every positive integer $k$ satisfies $Q(ka_0)=P(ka_0)-a_0 = a_0-a_0 = 0$, thus $ka_0$ is a root from $Q(ka_0)$, or $Q(ka_0)$ has infinite roots; However, since $Q(x)=P(x)-a_0$:
$$Q(x)=(a_nx^n+...+a_0)-a_0 = a_nx^n+...+n_1x$$
Which, $Q(x)$ is also a polynomial of degree $n\geq 1$. Then, by Fundamental Theorem of Algebra, $Q(x)$ has at most  $n$ distinct roots, which contradicts the above statement that $Q(x)$ has infinite roots.

\hfill

Thus, the assumption is false, such polynomial doesn't exist.

\hfill

\subsection{Problem 5}
\textbf{Show that there does not exist a function $f:\mathbf{Z}\rightarrow \{1,2,3\}$
satisfying $f(x) \neq f(y)$ for all $x, y \in \mathbf{Z}$ such that $|x-y| \in \{2, 3, 5\}$.}

\hfill

\textbf{Pf:}

Suppose there exists a function $f:\mathbf{Z}\rightarrow \{1,2,3\}$, such that for all $x,y\in\mathbf{Z}$ satisfying $|x-y|\in \{2, 3, 5\}$, $f(x)\neq f(y)$. Then, for all $n\in\mathbf{Z}$, consider $(n+2),(n+3),(n+5)\in\mathbf{Z}$:

\hfill

First, since $|(n+5)-n| = 5$, by assumption, this implies that $f(n+5)\neq f(n)$.

Then, since $|(n+2)-n|=2$ and $|(n+3)-n|=3$, by assumption it implies that $f(n+2)\neq f(n)$ and $f(n+3)\neq f(n)$.

Furthermore, since $|(n+5)-(n+2)|=3$ and $|(n+5)-(n+3)|=2$, by assumption it implies that $f(n+5)\neq f(n+2)$ and $f(n+5)\neq f(n+3)$.

\hfill

Now, consider the set $K=\{1,2,3\}\setminus \{f(n),f(n+5)\}$: since $f(n),f(n+5)\in\{1,2,3\}$ by how the function is defined, and $f(n)\neq f(n+5)$, then $K$ only has $(3-2)=1$ element. 

Then, since $f(n+2), f(n+3)$ both are elements of $\{1,2,3\}$, while the values are different from both $f(n)$ and $f(n+5)$, hence $f(n+2),f(n+3)\in K$. Which, $K$ has only 1 element implies $f(n+2)=f(n+3)$.

\hfill

Since for all $n\in\mathbf{Z}$, we have $f(n+2)=f(n+3)$, then here is the contradiction: 

Take $0$ and $1$ as input, which the following equations are true:
$$f(2)=f(0+2)=f(0+3)=f(3),\quad f(3)=f(1+2)=f(1+3)=f(4)$$
Which, we yield $f(2)=f(4)$; yet, since $|4-2|=2$, by assumption it implies $f(2)\neq f(4)$, which reaches a contradiction. So, the initial assumption is false, such function with given condition does not exist.

\hfill

\subsection{Problem 6}

\hfill

\subsection{Problem 7}
\textbf{Show that the interval $[0, 1]$ cannot be partitioned into two disjoint sets $A$ and $B$ such that $B = A + r$ for some real number $r$.}

\textbf{Pf:}

Suppose there exists $A, B$ that are disjoint, such that $B=A+r$ for some real number $r$ and $A\cup B=[0,1]$.

First, for all $a\in[0,1]$, $a\in A \iff (a+r)\in B$, so Without Loss of Generality, we can assume $r \geq 0$ (since if $r<0$, let $r'=-r >0$, then $A+r=B \iff A=B-r = B+r'$, which interchange $A$ and $B$ provides the same context). Also, both $A$ and $B$ are nonempty, since $A\neq \emptyset \iff B=(A+r)\neq \emptyset$, while $A\cup B=[0,1]$, so we need at least one of the set to be nonempty, which implies the other set is also nonempty.

Now, based on the setup, we can prove some statements in the following order:
\begin{itemize}
    \item[(1)] \textbf{$r>0$:} Suppose $r=0$, then $B = A+r = A$. However, in case for $A\cap B = A\cap A = A$ to be empty ($A$ and $B$ are disjoint), $A=\emptyset$. Yet, $B=A+r=\emptyset$, which $A\cup B=\emptyset$, which is a contradiction. Therefore, $r>0$.

    \item[(2)] $r<1$: Suppose $r\geq 1$. Then, since $A\neq \emptyset$, for all $a\in A \subseteq [0,1]$, $0\leq a$, thus $1 \leq r \leq (a+r)$. But, $(a+r)\in B\subseteq [0,1]$, thus $(a+r) \leq 1$, so $(a+r)=1$. Yet, this implies $a = (1-r)$, while $(1-r) \leq 1+(-1)=0$, which regarding $a \geq 0$ we get $a=0$. So, $A=\{0\}$ and $B=\{r\}$. Yet, $A\cup B\neq [0,1]$ based on this result, again it is a contradiction. Therefore, $r<1$.

    \item[(3)] $0\in A$ and $1\in B$: It is clear that $0,1\in[0,1]=A\cup B$, which they must be contained in one of the sets.

    Suppose $0\in B$, then since $0\in B\iff (0-r)\in A$, so $-r\in A$ (Note: if $b\in B$, then there exists $a\in A$ with $(a+r)=b$, thus $a=(b-r)$). However, based on (1), $r>0$ implies $-r<0$, which since $A\subseteq [0,1]$, it is a contradiction. Therefore, $0\notin B$, which $0\in A$.

    Now, suppose $1\in A$, then it implies $(1+r)\in B$. However, since $r>0$ is proven in (1), then $(1+r)>1$; yet, $(1+r)\in B\subseteq [0,1]$ indicates $(1+r)\leq 1$, which is a contradiction. Therefore, $1\in B$.

    \item[(4)] $r\in B$, and $r$ is the minimum of $B$: First, since $0\in A$ is proven in (3), then $0+r=r\in B$. Also, for all $b\in B$, there exists $a\in A$ with $(a+r)=b$. Since $0 \leq a$ (note: $A\in[0,1]$), then $r\leq (a+r)=b$, so $r$ is a lower bound of $B$. The two statements imply that $r$ is the minimum of $B$.

    \item[(5)] $[0,r)\subseteq A$: From (4), since $r$ is the minimum of $B$, then for all $a\in[0,r)\subseteq [0,1]$, $a<r$, which $a\notin B$. Thus, $a\in A$, proving that $[0,r)\subseteq A$.
\end{itemize}

Now, define the set $K_0 = [0,r)$, and for all $n\in\mathbb{N}$, define $K_{n+1}=K_{n}+r=[(n+1)r,(n+2)r)$ recursively. Also, let $k\in\mathbb{N}$ be the smallest positive integer such that $kr >1$ (which based on Archimedean's Property, $r>0$ implies the existence of $k$, and $k>1$ since $1\cdot r <1$).

Then, $K_k \not\subseteq [0,1]$, since $K_k=[kr,(k+1)r)$, while $kr >1$; also, $K_{k-1}\not\subseteq [0,1]$, since $K_{k-1}=[(k-1)r,kr)$, which by definition of $k$, $(k-1)r \leq 1 < kr$ (since $k$ is the minimum positive integer with $kr>1$, which $(k-1)<k$ doesn't satisfy the condition), so choose $b$ with $1<b<kr$, the $b\in K_{k-1}$ while $b\notin [0,1]$.

\hfill

Now, we can use induction to prove that for all $n\in\mathbb{N}$, if $K_n\subseteq [0,1]$, then $K_n\subseteq A$ or $K_n \subseteq B$ (must be a subset of one of the set, but not separate in two sets).

For base case $n=0$, in (5) we've proven that $K_0=[0,r) \subseteq A$, which the hypothesis is true.

Now, suppose for given $n\in\mathbb{N}$, $K_n\subseteq [0,1]$ implies $K_n\subseteq A$ or $K_n \subseteq B$: If $(n+1)$ satisfies $K_{n+1}=[(n+1)r,(n+2)r)\subseteq [0,1]$, then $0 \leq nr < (n+1)r \leq 1$, thus $K_n = [nr,(n+1)r)\subseteq [0,1]$.

By induction hypothesis, since $K_n\subseteq [0,1]$, then $K_n\subseteq A$ or $K_n\subseteq B$:

If $K_n\subseteq A$, then $K_{n+1}=(K_n+r)\subseteq (A+r) = B$, thus it is a subset of $B$.

Else if $K_n \subseteq B$, then $K_{n+1}\subseteq A$ must be true: If $(K_n+r)=K_{n+1}\subseteq B=(A+r)$, then it indicates $K_n\subseteq A$. Yet, by assumption $K_n\subseteq B$, while $A,B$ are disjoint. So, this is a contradiction, indicating that $K_{n+1}\subseteq A$.

This completes the induction.

\hfill

Finally, consider $1\in B$ (proven in (3)): By the definition of $k$ above, it is the smallest positive integer with $kr >1$, thus $(k-1)<k$ implies $(k-1)r \leq 1$, which $1 \in [(k-1)r,kr) = K_{k-1}$ (Note: $k>1$ has been proven, thus $(k-1)\geq 1$ and $(k-2)\geq 0$).

Since $1\in B$, then $(1-r)\in A$. Which, since $K_{k-2}=[(k-2)r, (k-1)r)$, with $0 \leq (k-2)r < (k-1)r \leq 1$, thus $K_{k-2}\subseteq [0,1)$, which implies that $K_{k-2}\subseteq A$ or $K_{k-2}\subseteq B$. Also, since $1\in [(k-1)r,kr)$, then $(k-1)r\leq 1 < kr$, which $(k-2)r \leq (1-r) < (k-1)r$. Thus, $(1-r) \in K_{k-2}$.

Since $(1-r)\in A$, then $K_{k-2}\subseteq A$ (or else if $K_{k-2}\subseteq B$, $(1-r)\in K_{k-2}\subseteq B$ would lead to a contradiction, since $A,B$ are disjoint).

However, here is a contradiction: since $(1-r)\in K_{k-2}=[(k-2)r,(k-1)r)$, then $(1-r)<(k-1)r$. Now, choose $a$ with $(1-r)<a<(k-1)r$, which $a\in K_{k-2}\subseteq A$, indicating $(a+r)\in B$; yet, the inequality implies $1<(a+r)<kr$, which $(a+r)\notin [0,1]$, and this is a contradiction.

So, the assumption must be false, $[0,1]$ can't be partitioned into two disjoint sets $A,B$ with $B=A+r$ for some real number $r$.

\hfill

\subsection{Problem 8}
\textbf{If $a, b, c, d, e$ are real numbers such that the equation
$ax^2+(b+c)x+(d+e)=0$
has real roots greater than 1, show that the equation
$ax^4+bx^3+cx^2+dx+e=0$
has at least one real root.}

\textbf{Pf:}

Suppose the given equation $ax^2+(b+c)x+(d+e)=0$ has real roots $\alpha,\beta>1$, but the equation $ax^4+bx^3+cx^2+dx+e=0$ has no real root.

\hfill

First, let $p(x)=ax^4+bx^3+cx^2+dx+e$, since it is assumed to have no real roots, then $p(x)\neq 0$ for all $x\in\mathbb{R}$. This implies that $p(x)$ is strictly above $0$ for all input $x\in\mathbb{R}$, or strictly below $0$: If there exists $x,y\in\mathbf{R}$, with $p(x)<0$ and $p(y)>0$, then by Intermediate Value Theorem, since $p(x)<0<p(y)$ and $x\neq y$, there exists real number $z$ in between $x,y$, with $p(z)=0$, which contradicts the assumption. Thus, $p(x)$ can't obtain both negative and positive outputs, which the graph of $p(x)$ is strictly above, or strictly below $0$.

\hfill

Now, for the given $\alpha >1$, since $a\alpha^2 + (b+c)\alpha + (d+e)=0$, then:
$$a\alpha^2 + c\alpha+e = -b\alpha-d=-(b\alpha+d)$$
$$-(a\alpha^2 + c\alpha+e)=(b\alpha+d)$$
Which, if we consider $x=\sqrt{\alpha}$ (Note: $\sqrt{\alpha}>1$, since $\alpha>1$), the polynomial $p(\sqrt{\alpha})$ could be expressed as:
$$p(\sqrt{\alpha})=a(\sqrt{\alpha})^4+b(\sqrt{\alpha})^3+c(\sqrt{\alpha})^2+d\sqrt{\alpha}+e = (a\alpha^2+c\alpha+e)+b\alpha\sqrt{\alpha}+d\sqrt{\alpha}$$
$$=(-b\alpha-d)+b\alpha\sqrt{\alpha}+d\sqrt{\alpha} = b\alpha(\sqrt{\alpha}-1)+d(\sqrt{\alpha}-1)$$
$$=(b\alpha+d)(\sqrt{\alpha}-1) = -(a\alpha^2+c\alpha+e)(\sqrt{\alpha}-1)$$

\hfill

Then, consider the sign of $(a\alpha^2+c\alpha+e)$:

First, we can express $(a\alpha^2+c\alpha+e)$ in terms of $p(\sqrt{\alpha})$ and $p(-\sqrt{\alpha})$:
$$p(\sqrt{\alpha})=a(\sqrt{\alpha})^4+b(\sqrt{\alpha})^3+c(\sqrt{\alpha})^2+d\sqrt{\alpha}+e = (a\alpha^2+c\alpha+e)+\sqrt{\alpha}(b\alpha+d)$$
$$p(-\sqrt{\alpha})=a(-\sqrt{\alpha})^4+b(-\sqrt{\alpha})^3+c(-\sqrt{\alpha})^2+d(-\sqrt{\alpha})+e = (a\alpha^2+c\alpha+e)-\sqrt{\alpha}(b\alpha+d)$$
Which, the following is true:
$$\frac{p(\sqrt{\alpha})+p(-\sqrt{\alpha})}{2}=\frac{2(a\alpha^2+c\alpha+e)}{2}=(a\alpha^2+c\alpha+e)$$

\hfill

Then for the first case, if for all $x\in\mathbb{R}$, $p(x)>0$, then $(a\alpha^2+c\alpha+e)=\frac{p(\sqrt{\alpha})+p(-\sqrt{\alpha})}{2}>0$ (since both $p(\sqrt{\alpha})$ and $p(-\sqrt{\alpha})$ are positive).

Yet, this implies that $p(\sqrt{\alpha})=-(a\alpha^2+c\alpha+e)(\sqrt{\alpha}-1)<0$ (since $\sqrt{\alpha}>1$, and the other term is positive), which is a contradiction.

\hfill

Furthermore, for the second case, if for all $x\in\mathbb{R}$, $p(x)<0$, then we can simply multiply the polynomial by $-1$ to yield the same result as the first case.

\hfill

Regardless of the case, we'll eventually run into a contradiction, therefore the initial assumption is false, given that $ax^2+(b+c)x+(d+e)=0$ has real roots greater than $1$, the equation $ax^4+bx^3+cx^2+dx+e=0$ has real root.

\end{document}