% Math_111B_HW5_Zih-Yu_Hsieh.tex

\documentclass{article}
\usepackage{graphicx} % Required for inserting images
\usepackage[margin = 2.54cm]{geometry}
\usepackage[most]{tcolorbox}

\newtcolorbox{myBox}[3]{
arc=5mm,
lower separated=false,
fonttitle=\bfseries,
%colbacktitle=green!10,
%coltitle=green!50!black,
enhanced,
attach boxed title to top left={xshift=0.5cm,
        yshift=-2mm},
colframe=blue!50!black,
colback=blue!10
}

\usepackage{amsmath}
\usepackage{amssymb}
\usepackage{verbatim}
\usepackage[utf8]{inputenc}
\linespread{1.2}

\newtheorem{definition}{Definition}
\newtheorem{proposition}{Proposition}
\newtheorem{theorem}{Theorem}
\newtheorem{question}{Question}

\title{Math 111B HW5}
\author{Zih-Yu Hsieh}

\begin{document}
\maketitle

\section*{1}
\begin{myBox}[]{}
    \begin{question}
        Let $R$ be a commutative ring. Prove or disprove that every non-constant monic
        polynomial $f(X)\in R[X]$ of degree $n$ has at most $n$ zeros in $R$.
    \end{question}
\end{myBox}

\textbf{Pf:}

Here is a counterexample: Consider $R=\mathbb{Z}_6$, and $f(X)\in R[X]$ defined as $f(X)=X+X^2$.

$f(X)$ is a degree 2 monic polynomial, yet the following is true:
$$f(0)=0+0^2=0$$
$$f(2)=2+2^2 = 2+4 = 6 \equiv 0\ (mod\ 6)$$
$$f(3)=3+3^2=3+9=12\equiv 0\ (mod\ 6)$$
$$f(5)=5+5^2=5+25=30\equiv 0\ (mod\ 6)$$
Hence, $0,2,3,5\in\mathbb{Z}_6$ are 4 distinct roots of $f(X)$, while it is only a degree 2 polynomial.
Hence, for $R$ being a commutative ring, it is still possible to find monic polynomial in $R[X]$ with more zeroes in $R$ than its degree.

\hfill

\section*{2}
\begin{myBox}[]{}
    \begin{question}
        Determine if the polynomial $f(X)=21X^2+2X-8\in\mathbb{Z}[X]$ is irreducible. If it is
        not irreducible, what are its factors?
    \end{question}
\end{myBox}

\textbf{Pf:}

Given $f(X)$ stated above. Notice the following: 
$$(7X-4)(3X+2) = 7X(3X+2)-4(3X+2)=(21X^2+14X)+(-12X-8) = 21X^2+2X-8 = f(X)$$
Also, since $\mathbb{Z}$ is an integral domain, then $(\mathbb{Z}[X])^\times = (\mathbb{Z})^\times$:

First, since $\mathbb{Z}\subseteq \mathbb{Z}[X]$, then it's clear that $(\mathbb{Z})^\times \subseteq (\mathbb{Z}[X])^\times$ (since if $a\in(\mathbb{Z})^\times$, $a^{-1}\in(\mathbb{Z})^\times$,
hence since $a,a^{-1}\in\mathbb{Z}[X]$ and $aa^{-1}=a^{-1}a=1$, then $a,a^{-1}\in(\mathbb{Z}[X])^\times$).

Then, suppose $f(X)\in (\mathbb{Z}[X])^\times$, there exists $g(X)\in(\mathbb{Z}[X])^\times$, with $f(X)g(X)=1$. Yet, since $1\neq 0$, then $f(X),g(X)\neq 0$; 
also, since $\mathbb{Z}$ is an integral domain, then: 
$$0=\deg(1)=\deg(f(X)g(X))=\deg(f(X))+\deg(g(X))$$
Since $\deg(f(X)),\deg(g(X))\geq 0$, then the only possibility is $\deg(F(X))=\deg(g(X))=0$.
Hence, both $f(X),g(X)$ are constant, hence $f(X),g(X)\in\mathbb{Z}$, while $f(X)g(X)=1$, showing that $f(X),g(X)\in(\mathbb{Z})^\times$.
Therefore, $(\mathbb{Z}[X])^\times \subseteq(\mathbb{Z})^\times$.

\hfill

The above two statements show that $(\mathbb{Z}[X])^\times=(\mathbb{Z})^\times$, then since $(7X-4),(3X+2)\notin \mathbb{Z}$,
then $(7X-4),(3X+2)\notin (\mathbb{Z})^\times = (\mathbb{Z}[X])^\times$.

Because $f(X)=21X^2+2X-8=(7X-4)(3X+2)$, while the two factors are not invertible, then $f(X)$ is reducible.

\hfill

\hfill

\section*{3}
\begin{myBox}[]{}
    \begin{question}
        Find the quotient and remainder for the division of $f(X) = 3X^4 + X^3 + 2X^2 + 1$ by
        $g(X) = X^2 + 4X + 2$ in $\mathbb{Z}_5[X].$
    \end{question}
\end{myBox}

\textbf{Pf:}

Notice that since the base field is $\mathbb{Z}_5$, we'll directly convert the coefficients without the modulo symbol.
We'll do the division recursively:

\hfill

\hfill

First, consider $3X^2\cdot g(X) = 3X^2(X^2+4X+2) = 3X^4+12X^3+6X^2 = 3X^4+2X^3+X^2$.
Which: 
$$f(X)-3X^2\cdot g(X)=(3X^4+X^3+2X^2+1)-(3X^4+2X^3+X^2) = -X^3+X^2+1 = 4X^3+X^2+1$$
So, $f(X)=3X^2\cdot g(X)+(4X^3+X^2+1)$.

\hfill

\hfill

Then, consider $4X\cdot g(X)=4X(X^2+4X+2)=4X^3+16X^2+8X=4X^3+X^2+3X$. Which:
$$(4X^3+X^2+1) - 4X\cdot g(X) = (4X^3+X^2+1)-(4X^3+X^2+3X) = -3X+1 = 2X+1$$
So, $(4X^3+X^2+1) = 4X\cdot g(X)+(2X+1)$. Plug into the previous equation:
$$f(X)=3X^2\cdot g(X)+(4X^3+X^2+1) = 3X^2\cdot g(X)+4X\cdot g(X)+(2X+1) = (3X^2+4X)g(X)+(2X+1)$$

\hfill

\hfill

Since $(2X+1)$ has degree 1, which is less than 2 the degree of $g(X)$, hence the division process ends here.

Which, let $q(X)=(3X^2+4X)$ and $r(X)=(2X+1)$, then $f(X)=q(X)g(X)+r(X)$.

The division of $f(X)$ by $g(X)$ has the quotient $q(X)=3X^2+4X$, and remainder $r(X)=2X+1$.

\break

\section*{4}
\begin{myBox}[]{}
    \begin{question}
        Find all zeros of the polynomial $f(X)=X^{25}-1 \in \mathbb{Z}_{37}[X]$.
    \end{question}
\end{myBox}

\textbf{Pf:}

First, notice that since $37$ is a prime, then the base ring $\mathbb{Z}_{37}$ is in fact a field.
Hence, every element except for $0$ is invertible, showing that $(\mathbb{Z}_{37})^\times = \mathbb{Z}_{37}\setminus\{0\}$.
Then, since $|\mathbb{Z}_{37}| = 37$, we have $|(\mathbb{Z}_{37})^\times| = |\mathbb{Z}_{37}\setminus\{0\}| = 37-1 = 36$.

\hfill

Then, suppose $a\in\mathbb{Z}_{37}$ is a zero of $f(X)=X^{25}-1$, which $f(a)=a^{25}-1 = 0$, showing that $a^{25}=1$.

Since $a\cdot a^{24} = a^{24}\cdot a=1$, then $a$ is invertible, hence $a\in(\mathbb{Z}_{37})^\times$.

Because $(\mathbb{Z}_{37})^\times$ is a group under multiplication, while $|(\mathbb{Z}_{37})^\times| = 36$, then $order(a)\ \bigm |\ 36$, 
dividing the order of the group.

Similarly, since $a^{25}=1$, then $order(a)\ \bigm | 25$ (since the power of 25 returns to the identity of the group).

Hence, $order(a)$ must be a common factor of $25 = 5^2$ and $36 = 2^2\cdot 3^2$, hence $order(a)\ \bigm |\ \gcd(25,36)=1$.

\hfill

So, the only possibility is $order(a)=1$, showing that $a^1=a=1$.

Therefore, the only zero for $f(X)=X^{25}-1$ in $\mathbb{Z}_{37}$ is $X=1$.

\hfill

\hfill

\section*{5}
\begin{myBox}[]{}
    \begin{question}
        Find the quotient and remainder for the division of $f(X) = 3X^5 + 5X^3 + X + 1$ by
        $g(X) = X^2 + 2X + 1$ in $\mathbb{Z}_7[X]$.
    \end{question}
\end{myBox}

\textbf{Pf:}

Notice that the base field is $\mathbb{Z}_7$, we'll again convert the coefficients without the modulo symbol.
We'll again do the division recursively:

\hfill

\hfill

First, consider $3X^3\cdot g(X)=3X^3(X^2+2X+1)=3X^5+6X^4+3X^3$. Which:
$$f(X)-3X^3\cdot g(X)=(3X^5 + 5X^3 + X + 1)-(3X^5+6X^4+3X^3)=-6X^4+2X^3+X+1 = X^4+2X^3+X+1$$
Hence, $f(X)=3X^3\cdot g(X)+(X^4+2X^3+X+1)$.

\hfill

\hfill

Then, consider $X^2\cdot g(X)=X^2(X^2+2X+1)=X^4+2X^3+X^2$. Which:
$$(X^4+2X^3+X+1)-X^2\cdot g(X)=(X^4+2X^3+X+1)-(X^4+2X^3+X^2) = -X^2+X+1=6X^2+X+1$$
Hence, $(X^4+2X^3+X+1) = X^2\cdot g(X)+(6X^2+X+1)$. Plug into the previous equation:
$$f(X)=3X^3\cdot g(X)+(X^4+2X^3+X+1)=3X^3\cdot g(X)+X^2\cdot g(X)+(6X^2+X+1) = (3X^3+X^2)g(X)+(6X^2+X+1)$$

\hfill

\hfill

Now, consider $6g(X)=6X^2+12X+6=6X^2+5X+6$. Which:
$$(6X^2+X+1)-6g(X)=(6X^2+X+1)-(6X^2+5X+6)=-4X-5=3X+2$$
Hence, $(6X^2+X+1)=6g(X)+(3X+2)$. Plug into the previous equation:
$$f(X)=(3X^2+X^2)g(X)+(6X^2+X+1) = (3X^3+X^2)g(X)+6g(X)+(3X+2) = (3X^3+X^2+6)g(X)+(3X+2)$$
Since $(3X+2)$ has degree $1$, which is less than $2$ the degree of $g(X)$, hence the division process ends here.

Which, let $q(X)=(3X^3+X^2+6)$ and $r(X)=(3X+2)$, then $f(X)=q(X)g(X)+r(X)$.
The division of $f(X)$ by $g(X)$ has the quotient $q(X)=3X^3+X^2+6$, and $r(X)=3X+2$.

\hfill

\hfill

\section*{6}
\begin{myBox}[]{}
    \begin{question}
        Let $p$ be a prime. prove or disprove that there exists a non-constant polynomial in
        $\mathbb{Z}_p[X]$ which has a multiplicative inverse.
    \end{question}
\end{myBox}

\textbf{Pf:}

Since $p$ is a prime, then the base ring $\mathbb{Z}_p$ is an integral domain. 

Hence, suppose $f(X)\in\mathbb{Z}_p[X]$ has a multiplicative inverse $g(X)\in\mathbb{Z}_p[X]$, then $f(X)g(X)=1$,
showing that $f(X),g(X)\neq 0$ (or else $1=0$ is a contradiction).

\hfill

Due to the property of integral domain, the following is true:
$$0=\deg(1)=\deg(f(X)g(X))=\deg(f(X))+\deg(g(X))$$
And, since $\deg(f(X)),\deg(g(X))\geq 0$, then the only possibility is $\deg(f(X))=\deg(g(X))=0$,
showing that $f(X),g(X)$ are constants.

Hence, there doesn't exist a non-constant polynomial in $\mathbb{Z}_p[X]$ that has a multiplicative inverse,
if $p$ is prime.

\break

\section*{7}
\begin{myBox}[]{}
    \begin{question}
        Let $k$ be a field and $R=k[X]$. Let $I=\left\{a_0+a_1X+...+a_nX^n\in R\ |\ \sum_{i=0}^{n}a_i=0\right\}$.
        Show that $I$ is an ideal of $R$. Is $I$ principal? If yes, find a generator of $I$.
    \end{question}
\end{myBox}

\textbf{Pf:}

\textbf{$I$ is an ideal:}

We'll first show that it is a subgroup under addition. Suppose $f(X),g(X)\in I$, which let $f(X)=a_0+a_1X+...+a_nX^n$, and $g(X)=b_0+b_1X+...+b_mX^m$.
They satisfy $\sum_{i=0}^{n}a_i=0 = \sum_{i=0}^{m}b_m$.

WLOG, assume that $n\geq m$. Then, $(f+g)(X)$ can be expressed as following:
$$(f+g)(X)=(a_0+a_1X+...+a_nX^n)+(b_0+b_1X+...+b_mX^m) = \sum_{i=0}^{m}(a_i+b_i)X^i + \sum_{j=m+1}^{n}a_jX^j$$
(Note: If $n=m$, then the second summation can be ignored).

Which, computing the sum of coefficients of $(f+g)(X)$, we get:
$$\sum_{i=0}^{m}(a_i+b_i)+\sum_{j=m+1}^{n}a_j = \left(\sum_{i=0}^{m}a_i+\sum_{j=m+1}^{n}a_j\right)+\sum_{i=0}^{m}b_i = \sum_{i=0}^{n}a_i + 0 = 0$$
Hence, $(f+g)(X)\in I$, showing that $I$ is closed under addition.

\hfill

Then, suppose $f(X)\in I$ (using the same expression as above). Which, $-f(X)=-(a_0+a_1X+...+a_nX^n) = -a_0-a_1X-...-a_nX^n$.
Which, sum up the coefficients of $-f(X)$, it is as follow:
$$\sum_{i=0}^{n}-a_i = (-1)\sum_{i=0}^{n}a_i = 0$$
(Note: recall that $f(X)\in I$ implies that $\sum_{i=0}^{n}a_i = 0$).

Hence, $-f(X)\in I$, showing that every element in $I$ has an additive inverse in $I$ also.

\hfill

Lastly, since $0\in k[X]$ has all the coefficient being $0$, the sum of coefficient is $0$. Hence, $0\in I$, showing that the zero element is in there.

Therefore, we can conclude that $I$ is a subgroup under addition.

\hfill

\hfill

Then, to show that $I$ is an ideal, it suffices to show that for all $f(X)\in I$, all $a\in k$, and all $l\in\mathbb{N}$, $aX^l\cdot f(X)\in I$.

Again, let $f(X)=a_0+a_1X+...+a_nX^n$, which $\sum_{i=0}^{n}a_i=0$. Then, consider the following: 
$$aX^l\cdot f(X)= aa_0X^l+a_1X^{1+l}+...+aa_nX^{n+l} = \sum_{i=0}^{l-1}0\cdot aX^i + \sum_{j=0}^{n}aa_jX^{j+l}$$
(Note: if $k=0$, then the first summation term above can be ignored).

The sum of coefficient of $X^k\cdot f(X)$ is as follow:
$$\sum_{i=0}^{l-1}0 + \sum_{j=0}^{n}aa_j = 0+a\sum_{j=0}^{n}a_j=0+0=0$$
Hence, given that $f(X)\in I$, every $a\in k$ and $l\in\mathbb{N}$ satisfies $aX^l\cdot f(X)\in I$.

\hfill

Which, for all $g(X)\in R$, $g(X)=b_0+b_1X+...+b_mX^m$ for some $b_0,b_1,...,b_m\in k$. Hence, given $f(X)\in I$, $g(X)\cdot f(X)$ is as follow:
$$g(X)\cdot f(X)= b_0f(X)+b_1X\cdot f(X)+...+b_mX^m\cdot f(X)$$
For all $i\in\{0,1,...,m\}$, since $b_i\in k$, then $b_iX^i\cdot f(X) \in I$; and since $I$ is a subgroup under addition, then $g(X)\cdot f(X)$ is a sum of elements in $I$, 
which $g(X)\cdot f(X)\in I$.

Hence, we can conclude that $I$ is in fact an ideal.

\hfill

\hfill

\textbf{$I$ is a Principal Ideal:}

Recall that given a commutative ring $R$, $R[X]$ is a Principal Ideal Domain if and only if $R$ is a field, hence since $k$ is a field, $k[X]$ must be a Principal Ideal Domain.
So, $I\subset k[X]$ is a principal ideal.

\hfill

\textbf{Generator of $I$:}

Now, consider the polynomial $X-1\ \in k[X]$: Its sum of coefficients is given as $1+(-1)=0$, hence $X-1\ \in I$, implying that $(X-1)\subseteq I$.

Also, for all $f(X)\in I$, let $f(X)=a_0+a_1X+...+a_nX^n$, which $\sum_{i=0}^{n}a_n=0$. Then, consider the following polynomial $g(X)\in k[X]$ defined as follow:
$$g(X)=-\sum_{i=0}^{n-1}\left(\sum_{j=0}^{i}a_j\right)X^i$$
Which, $(X-1)g(X)$ is given as follow:
$$(X-1)g(X)= X\cdot g(X)-g(X) = -\sum_{i=0}^{n-1}\left(\sum_{j=0}^{i}a_j\right)X^{i+1}+\sum_{i=0}^{n-1}\left(\sum_{j=0}^{i}a_j\right)X^i$$
$$ = -\left(\sum_{j=0}^{n-1}a_j\right)X^{(n-1)+1} - \sum_{i=0}^{n-2}\left(\sum_{j=0}^{i}a_j\right)X^{i+1} + \sum_{i=1}^{n-1}\left(\sum_{j=0}^{i}a_j\right)X^i + \left(\sum_{j=0}^{0}a_j\right)X^0$$
$$ = \left(a_n-a_n-\sum_{j=0}^{n-1}a_j\right)X^n - \sum_{i=1}^{n-1}\left(\sum_{j=0}^{i-1}a_j\right)X^i + \sum_{i=1}^{n-1}\left(\sum_{j=0}^{i}a_j\right)X^i + a_0$$
$$ = \left(a_n-\sum_{j=0}^{n}a_j\right)X^n + \sum_{i=1}^{n-1}\left(-\sum_{j=0}^{i-1}a_j+\sum_{j=0}^{i}a_j\right)X^i + a_0$$
$$ = (a_n-0)X^n + \sum_{i=1}^{n-1}a_iX^i + a_0$$
$$= \sum_{i=0}^{n}a_nX^n = f(X)$$
Hence, $f(X)=(X-1)g(X)$, showing that $f(X)\in (X-1)$, or $I\subseteq (X-1)$.

\hfill

With both containments being true, we can conclude that $I=(X-1)$, hence $X-1$ is a generator of $I$.


\end{document}