%Math_111B_HW3_Zih-Yu_Hsieh.tex

\documentclass{article}
\usepackage{graphicx} % Required for inserting images
\usepackage[margin = 2.54cm]{geometry}
\usepackage[most]{tcolorbox}

\newtcolorbox{myBox}[3]{
arc=5mm,
lower separated=false,
fonttitle=\bfseries,
%colbacktitle=green!10,
%coltitle=green!50!black,
enhanced,
attach boxed title to top left={xshift=0.5cm,
        yshift=-2mm},
colframe=blue!50!black,
colback=blue!10
}

\usepackage{amsmath}
\usepackage{amssymb}
\usepackage{verbatim}
\usepackage[utf8]{inputenc}
%\linespread{1.5}

\newtheorem{definition}{Definition}
\newtheorem{proposition}{Proposition}
\newtheorem{theorem}{Theorem}
\newtheorem{question}{Question}
\newtheorem{lemma}{Lemma}

\title{Math 111B HW3}
\author{Zih-Yu Hsieh}

\begin{document}
\maketitle

\section*{1}
\begin{myBox}[]{}
    \begin{question}
        Let $R$ be a finite commutative ring. Show that every element of $R$ is either a
        zero-divisor or a unit.
    \end{question}
\end{myBox}

\textbf{Pf:}

Suppose $R$ is a finite commutative ring, then for each element $a\in R$ there are two cases to consider:

\hfill

First, suppose there exists nonzero element $b\in R$ with $ab=ba=0$, then $a$ is a zero-divisor.

\hfill

Else, if for all nonzero element $b\in R$ satisfies $ab=ba\neq 0$, which also implies $a\neq 0$ (since $0\cdot b=0$ for all $b\in R$).
Then, for all $n\in\mathbb{N}$, $a^n\neq 0$: For base case $n=1$, $a^1 = a\neq 0$, and suppose for given $n\in\mathbb{N}$, it satisfies
$a^n\neq 0$, then by assumption, $a\cdot a^n = a^{n+1}\neq 0$, which by the principle of mathematical induction, $a^n\neq 0$ for all 
positive integer $n$.

Now, consider $S=\{a^n\ |\ n\in\mathbb{N}\}\subseteq R$, since $R$ is finite, the set $S$ is also finite. Thus, there must exists $m,n\in\mathbb{N}$
(assume $m>n$) with $a^m=a^n$. Which, $a^{n+(m-n)}-a^n=0$, or $a^n(a^{(m-n)}-1)=0$.

Notice that since $a$ is not a zero-divisor, then $(a^{(m-n)}-1)=0$ (if it's nonzero, then $a^n(a^{(m-n)}-1)\neq 0$).
Thus, $a^{(m-n)}=1$, which $a\cdot a^{(m-n-1)}=1$, showing that $a^{(m-n-1)}=a^{-1}$, thus $a$ is a unit.

\hfill

So for finite commutative ring $R$, if an element is not a zero-divisor, it is a unit.

\break


\section*{2}
\begin{myBox}[]{}
    \begin{question}
        Let $R$ be a ring. Prove or disprove that $Z(R[X])=Z(R)[X]$.
    \end{question}
\end{myBox}

\textbf{Pf:}

We'll prove that $Z(R)[X]=Z(R[X])$. Notice that if $R$ is commutative ($Z(R)=R$), then the polynomial ring $R[X]$ is also commutative ($Z(R[X])=R[X]$).
So, for commutative ring, $R[X]=Z(R)[X]=Z(R[X])$. So, the following proof is based on a non-commutative ring $R$.

\begin{itemize}
    \item[$\subseteq$:] For all polynomial $p\in Z(R)[X]$, there exists $p_0,p_1,...,p_n\in Z(R)$, with $p = p_0+p_1X+...+p_nX^n$.
    Which, for all $q\in R[X]$, there exists $q_0,q_1,...,q_m$, with $q = q_0+q_1X+...+q_mX^m$.
    Then, the multiplication is as follow:
    $$pq = c_0+c_1X+...+c_{m+n}X^{m+n},\quad c_k=\sum_{i,j,\ i+j=k}p_iq_j$$
    $$qp = c_0'+c_1'X+...+c_{m+n}'X^{m+n},\quad c_k' = \sum_{j,i,\ j+i=k}q_jp_i$$
    Since all $p_i\in Z(R)$, they commute with all elements in $R$, thus $c_k = c_k'$ for all index $k$,
    hence $pq=qp$. So, $p\in Z(R[X])$, indicating that $Z(R)[X]\subseteq Z(R[X])$.

    \hfill
    
    \item[$\supseteq$:] We'll prove by contradiction. Suppose $Z(R[X])\not\subseteq Z(R)[X]$, then there exists $p\in Z(R[X])$, such that some coefficient
    is not from $Z(R)$. Let $m\in\mathbb{N}$ be the largest index with $p_m\notin Z(R)$, which there exists $q\in R$, with $p_mq \neq qp_m$.

    Also, let $n\in\mathbb{N}$ be the largest power of $p$ (which $n \geq m$), then $p$ can be expressed as follow:
    $$p=p_0+p_1X+...+p_mX^m+p_{m+1}X^{m+1}+...+p_nX^n$$
    Then, by the assumption that $m$ is the largest index with $p_m\notin Z(R)$, which $p_{m+1},...,p_{n}\in Z(R)$.
    Thus, the polynomial $p_{m+1}X^{m+1}+...+p_nX^n\in Z(R)[X] \subseteq Z(R[X])$. Because $Z(R[X])$ itself is a ring, then:
    $$p-(p_{m+1}X^{m+1}+...+p_nX^n) = (p_0+p_1X+...+p_mX^m)\in Z(R[X])$$
    So, WLOG, we can assume $m$ is the largest power of $p$ (since we can subtract out all the powers larger than $m$).

    \hfill

    However, consider the following two expressions, $pq$ and $qp$:
    $$pq = (p_0+p_1X+...+p_mX^m)q = p_0q+p_1qX+...+p_mqX^m$$
    $$qp = q(p_0+p_1X+...+p_mX^m) = qp_0+qp_1X+...+qp_mX^m$$
    For $pq$, the degree $m$ coefficient is $p_mq$, while for $qp$, the degree $m$ coefficient is $qp_m$.
    Since $p_mq\neq qp_m$, then $pq\neq qp$. However, since $q\in R[X]$ while $p\in Z(R[X])$, $pq=qp$,
    so this is a contradiction.

    Thus, the assumption is false, $Z(R[X])\subseteq Z(R)[X]$.
\end{itemize}

With the above two statements, $Z(R)[X]=Z(R[X])$.

\break


\section*{3}
\begin{myBox}[]{}
    \begin{question}
        Let R be an integral domain. Prove that $(R[X])^\times = R^\times$.
    \end{question}
\end{myBox}

\textbf{Pf:}

Since $R \subseteq R[X]$, then for all $a\in R^\times$, $a^{-1}\in R^\times$, which $a,a^{-1}\in R[X]$ satisfy $aa^{-1}=a^{-1}a=1$,
indicating that $a\in (R[X])^\times$. So, $(R)^\times \subseteq (R[X])^\times$.

\hfill

Now, we'll use contradiction to prove that if $p\in R[X]$ has an inverse, then $p\in R$:
Suppose there exists a non-constant polynomial $p\in R[X]$ with an inverse, then there exists $q\in R[X]$,
with $pq=qp = 1$.

Let $p=p_0+p_1X+...+p_nX^n$ (which $n>0$, and $p_n \neq 0$), and $q = q_0+q_1X+...+q_mX^m$.

Then, we can use induction to prove that for all $k\in \{0,...,m\}$, $q_{m-k}=0$:

\hfill

For base case $k=0$, since $pq$ has the coefficient of $(n+m)$ degree being $p_nq_m$, because $(n+m)>0$, while $1$ is a constant polynomial,
then $(n+m)$ degree should have coefficient $0$, or $p_nq_m=0$; yet, since $p_n\neq 0$ by assumption, and $R$ is an integral domain,
then $q_m = q_{m-0}=0$.

Now, suppose for given $k\in \{0,...,m-1\}$, every integer $0\leq n \leq k$ satisfies $q_{m-n}=0$, then, $q$ can be expressed as follow:
$$q = q_0 + q_1X +... + q_{m-(k+1)}X^{m-(k+1)} + q_{m-k}X^{m-k}+...+q_mX^m$$
$$=q_0 + q_1X +... + q_{m-(k+1)}X^{m-(k+1)}$$
Which, $pq$ has the coefficient of $(n+(m-(k+1)))$ being $p_nq_{m-(k+1)}$, since $k\leq (m-1)$, the $(k+1)\leq m$, thus $(m-(k+1))\geq 0$.
So, since $n>0$, $(n+(m-(k+1)))>0$; however, since $pq=1$ a constant polynomial, so the coefficient of degree $(n+(m-(k+1)))>0$ is in fact $0$,
showing that $p_nq_{m-(k+1)}=0$. Again, since $p_n\neq 0$ by assumption, then $q_{m-(k+1)}=0$.

\hfill

So, by the Principle of Mathematical Induction, every $k\in \{0,...,m\}$ satisfies $q_{m-k}=0$, showing that all index $i\in\{0,...,m\}$ has $q_i=0$.

However, this implies $q = q_0+q_1X+...+q_mX^m = 0$, or $pq = 0$, which is a contradiction (since $pq=1$ by assumption).

So, the assumption is false, there doesn't exist a non-constant polynomial $p\in R[X]$ with an inverse.

Thus, for all $p\in (R[X])^\times$, $p$ is a constant polynomial, or $p\in R$.

Then, suppose $q\in R[X]$ is an inverse of $p$, based on the same logic, $q$ has an inverse implies $q\in R$, thus $p,q\in R^\times$, showing that $(R[X])^\times \subseteq R^\times$.

\hfill

With both statements above, $(R[X])^\times = R^\times$.


\break


\section*{4}
\begin{myBox}[]{}
    \begin{question}
        Let $R$ be a commutative ring. Prove or disprove that $(R[X])^\times = R^\times$.
    \end{question}
\end{myBox}

\textbf{Pf:}

Consider $R=\mathbb{Z}_4$, then consider $(3+2X)\in \mathbb{Z}_4[X]$:
$$(3+2X)^2 = (3+2X)(3+2X) = 3\cdot 3 + (3\cdot 2+2\cdot 3)X + 2\cdot 2X^2$$
$$ = (9\mod\ 4)+(12\mod\ 4)X + (4\mod\ 4)X^2 = 1+0X+0X^2 = 1$$
Which, since $(3+2X)\notin R$, then $(3+2X)\notin R^\times$; however, $(3+2X)$ has an inverse, namely itself,
so $(3+2X)\in (R[X])^\times$.

Hence, $(R[X])^\times \neq R^\times$ in this case.

\break


\section*{5 (Not done)}
\begin{myBox}[]{}
    \begin{question}
        Prove or disprove that only ideals of $M_2(\mathbb{R})$ are $(0)$ and $M_2(\mathbb{R})$.
    \end{question}
\end{myBox}

\textbf{Pf:}

\break


\section*{6 (Not done)}
\begin{myBox}[]{}
    \begin{question}
        Does there exist a field of order 6? Justify your answer.
    \end{question}
\end{myBox}

\textbf{Pf:}

There does not exist a field of order $6$.

\break


\section*{7}
\begin{myBox}[]{}
    \begin{question}
        Determine the smallest subring of $\mathbb{Q}$ that contains $1/2$. That is, describe the subring
        of $\mathbb{Q}$ which contains $1/2$ and every subring of $\mathbb{Q}$ containing $1/2$ also contains $S$.
    \end{question}
\end{myBox}

\textbf{Pf:}

Consider the set $S=\{\frac{m}{2^n}\in\mathbb{Q}\ |\ n\in\mathbb{N}\cup\{0\},\ m\in\mathbb{Z}\}$.

\hfill

\textbf{$S$ is a Subring:}
\begin{itemize}
    \item[(1)] For all $\frac{m_1}{2^{n_1}},\frac{m_2}{2^{n_2}}\in S$, the following are true:
    $$\frac{m_1}{2^{n_1}}+\frac{m_2}{2^{n_2}}=\frac{m_12^{n_2}+m_22^{n_1}}{2^{n_1+n_2}},\quad \frac{m_1}{2^{n_1}}\frac{m_2}{2^{n_2}}=\frac{m_1m_2}{w^{n_1+n_2}}$$
    Which, since $m_1,m_2$ are all integers while $n_1,n_2$ are natural numbers, then the numerators above are all integers, while the denominators are positive integer powers of $2$,
    thus the two elements belong to $S$, $S$ is closed under associative addition and multiplication (which, both are commutative and distributive, inherited from $\mathbb{Q}$).

    \hfill

    \item[(2)] Since $0 = \frac{0}{2^1}$ abd $1=\frac{2}{2^1}$, then $0,1\in S$, so both the zero and unity element of $\mathbb{Q}$ are in $S$.
    
    \hfill

    \item[(3)] Given any $\frac{m}{2^n}\in S$, the inverse $\frac{-m}{2^n}\in S$, thus the additive inverse also exists.
\end{itemize}

With the properties above, $S$ is a subring of $\mathbb{Q}$:
It is closed under commutative addition, has zero element and additive inverse for all element, thus $S$ is an abelian group under addition.
On the other hand, it's closed under multiplication and has unity element, thus $S$ is a monoid under multiplication.
With the distributive property, $S$ is a subring that contains $\frac{1}{2}$.

\hfill

\textbf{Every Subring $R\subseteq \mathbb{Q}$ containing $\frac{1}{2}$ contains $S$:}

Now, assume that $R\subseteq \mathbb{Q}$ is a subring containing $\frac{1}{2}$.

For all element $\frac{m}{2^n}\in S$ (with $m\in\mathbb{Z}$ and $n\in\mathbb{N}$), since $\frac{1}{2}\in R$, then its power $(\frac{1}{2})^n = \frac{1}{2^n}\in R$;
furthermore, because $\frac{1}{2^n}\in R$, then its integer multiple (sum of multiple $\frac{1}{2^n}$) is also contained in $R$, thus $\frac{m}{2^n}\in R$.

Hence, we can conclude that $S\subseteq R$, showing that $S$ is the smallest subring of $\mathbb{Q}$ containing $\frac{1}{2}$.

\break


\section*{8}
\begin{myBox}[]{}
    \begin{question}
        
    \end{question}
\end{myBox}

\break


\section*{9}
\begin{myBox}[]{}
    \begin{question}
        
    \end{question}
\end{myBox}

\break


\section*{10}
\begin{myBox}[]{}
    \begin{question}
        Let $R$ be an integral domain of characteristic $p > 0$. Let $A = \{x^p\ |\ x\in R\}$. Prove or
        disprove that $A$ is a subring of $R$.
    \end{question}
\end{myBox}

\textbf{Pf:}

We'll prove that $A$ is a subring of $R$. First, since $R$ is an integral domain, the its characteristic $p>0$ must be prime.

\hfill

Before starting, let's prove a lemma:
\begin{lemma}
    For all prime $p$, the binomial coefficient $\begin{pmatrix}
        p\\k
    \end{pmatrix}$ is divisible by $p$ for all integer $k$ satisfying $0<k<p$.
\end{lemma}

Given that $\begin{pmatrix}
    p\\k
\end{pmatrix}$ is an integer for all $k$ satisfying $0<k<p$, which it is written in the following form:
$$\begin{pmatrix}
    p\\k
\end{pmatrix}=\frac{p(p-1)...(p-k)}{k!},\quad k!\begin{pmatrix}
    p\\k
\end{pmatrix}=p(p-1)...(p-k)$$
The above equation indicates that $k!\begin{pmatrix}
    p\\k
\end{pmatrix}$ is divisible by $p$. Yet, since $k<p$, then $k!=1\cdot 2...(k-1)k$ is not divisible by $p$ (since it is a product of numbers coprime to $p$).
Then, in case for the numbe to be divisible by $p$, $\begin{pmatrix}
    p\\k
\end{pmatrix}$ must be a multiple of $p$ (or else if $\begin{pmatrix}
    p\\k
\end{pmatrix})$ is also coprime to $p$, the product $k!\begin{pmatrix}
    p\\k
\end{pmatrix}$ is also coprime to $p$, which is a contradiction). So, the lemma is true.

\hfill

\hfill

\textbf{$A$ is a Submonoid under Multiplication:}

Given that $R$ is an integral domain (which is commutative), for all $x,y\in R$, $x^p,y^p\in A$, which $x^py^p = (xy)^p$ while $xy\in R$.
Thus, $x^py^p = (xy)^p\in A$, showing that $A$ is closed under multiplication.

Furthermore, since $1^p=1\in A$, then the unity element is also in $A$, showing that $A$ is a submonoid of $R$ under multiplication.

\hfill

\textbf{$A$ is a Subroup under Addition:}

Given that $0^p = 0 \in A$, $A$ contains the zero element.

For all $x\in R$, there are two cases for the inverse:
\begin{itemize}
    \item If $p=2$, then $x^2\in R$ implies $x^2+x^2=0$ (by the definition of characteristic), thus $x^2=-x^2$, so $x^2\in A$ has an inverse in $A$.
    \item Else if $p\neq 2$, then $p$ is odd ($p=2k+1$ for some $k\in\mathbb{Z}$). Thus:
    $$(-x)^p = (-x)^{2k+1} = ((-x)^2)^k(-x) = (x^2)^k(-x) = -x^{2k}x = -x^{2k+1}=-x^p$$
    So, $x^p\in A$ while $-x^p\in A$, hence $x^p$ has an inverse in $A$.
\end{itemize}

Now, the only problem remain is addition: To prove that $A$ is closed under addition, consider arbitrary $x,y\in R$, and the expression $(x+y)^p$:
$$(x+y)^p = \sum_{k=0}^{p}\begin{pmatrix}
    p\\k
\end{pmatrix}x^ky^{p-k} = y^p + \sum_{k=1}^{p-1}\begin{pmatrix}
    p\\k
\end{pmatrix}x^ky^{p-k} + x^p$$
Notice that the binomial expansion is true because $R$ is an integral domain, which is commutative. 

Then, by \textbf{Lemma 1}, for $k\in\{1,...,p-1\}$, since $\begin{pmatrix}
    p\\k
\end{pmatrix}$ is a multiple of $p$, hence the expression $\begin{pmatrix}
    p\\k
\end{pmatrix}x^ky^{p-k}=0$ (since the integer multiple of $x^ky^{p-k}$ is some multiple of the characteristic of $R$, namely $p$).

So, $(x+y)^p = y^p + x^p$. For all $x,y\in R$, $x^p, y^p\in A$ satisfies $x^p+y^p = (x+y)^p \in A$, thus $A$ is closed under multiplication.

\hfill

\textbf{$A$ is a Subring of $R$:}

From the above proof, given that $A$ is an abelian subgroup of $R$ under addition, and it is also a submonoid of $R$ under multiplication,
with the distributive property inherited from $R$, we can conclude that $A$ is in fact a subring of $R$.


\end{document}