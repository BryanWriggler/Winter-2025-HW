% Math_111B_HW7_Zih-Yu_Hsieh.tex

\documentclass{article}
\usepackage{graphicx} % Required for inserting images
\usepackage[margin = 2.54cm]{geometry}
\usepackage[most]{tcolorbox}

\newtcolorbox{myBox}[3]{
arc=5mm,
lower separated=false,
fonttitle=\bfseries,
%colbacktitle=green!10,
%coltitle=green!50!black,
enhanced,
attach boxed title to top left={xshift=0.5cm,
        yshift=-2mm},
colframe=blue!50!black,
colback=blue!10
}

\usepackage{amsmath}
\usepackage{amssymb}
\usepackage{verbatim}
\usepackage[utf8]{inputenc}
\linespread{1.2}

\newtheorem{definition}{Definition}
\newtheorem{proposition}{Proposition}
\newtheorem{theorem}{Theorem}
\newtheorem{question}{Question}

\title{Math 111B HW7}
\author{Zih-Yu Hsieh}

\begin{document}
\maketitle

\section*{1}
\begin{myBox}[]{}
    \begin{question}
        Let $R$ be an integral domain in which every nonzero element is either irreducible
        or a unit. Prove or disprove that $R$ is a field. 
    \end{question}
\end{myBox}

\textbf{Pf:}

We'll prove by contradiction.

Suppose it is not a field, then there exists nonzero element $a\in R$ that is not invertible.
Since it's not a unit, $a$ must be irreducible.

Yet, since $a^2=a\cdot a$, then $a^2$ is not irreducible; and since $a\neq 0$, then because $R$ is an integral domain,
$a^2\neq 0$. Hence, $a^2$ must be a unit.

Now, consider $a^3$: Using the same argument above, since $a^3=a\cdot a\cdot a$, $a^3$ is not irreducible and is not $0$, 
therefore it is also a unit.

So, $a^3 = a^2\cdot a$, showing that $a = (a^2)^{-1}a^3$. It is a multiple of two units, therefore $a$ is also a unit, which reaches a contradiction.

Hence, the assumption is false, $R$ must be a field.

\hfil

\hfil

\section*{2}
\begin{myBox}[]{}
    \begin{question}
        Let $d\neq 1$ be an integer which is not divisible by the square of a prime number.
        Let $\mathbb{Z}[\sqrt{d}]=\{a+b\sqrt{d}\ |\ a,b\in\mathbb{Z}\}\subset \mathbb{C}$. 
        For any $a+b\sqrt{d}$, we let $N(a+b\sqrt{d})=|a^2-db^2|$. Show the following:
        \begin{itemize}
            \item[(1)] $\mathbb{Z}[\sqrt{d}]$ is a subring of $\mathbb{C}$.
            \item[(2)] $N(xy)=N(x)N(y)$ for $x,y\in\mathbb{Z}[\sqrt{d}]$.
            \item[(3)] $x\in\mathbb{Z}[\sqrt{d}]$ is zero iff $N(x)=0$.
            \item[(4)] $N(x)=1$ iff $x\in(\mathbb{Z}[\sqrt{d}])^\times$.
            \item[(5)] If $N(x)$ is a prime, then $x$ is irreducible in $\mathbb{Z}[\sqrt{d}]$.
        \end{itemize}
    \end{question}
\end{myBox}

\textbf{Pf:}

\begin{itemize}
    \item[(1)] \begin{itemize}
        \item First, it's closed under both addition and multiplication: for any $(a+b\sqrt{d}),(e+f\sqrt{d})\in \mathbb{Z}[\sqrt{d}]$, we have:
        $$(a+b\sqrt{d})+(e+f\sqrt{d})=(a+e)+(b+f)\sqrt{d}$$
        $$(a+b\sqrt{d})(e+f\sqrt{d})=(ae+bfd)+(af+be)\sqrt{d}$$
        Since $a,b,d,e,f\in\mathbb{Z}$, the above two expressions are having integer coefficients. Hence, both the addition and multiplication ends up in $\mathbb{Z}[\sqrt{d}]$,
        showing the set is closed under both operation.

        \item Then, both zero and unity element exists: Since $0=0+0\sqrt{d}$ and $1=1+0\sqrt{d}$, so $0,1\in\mathbb{Z}[\sqrt{d}]$;
        also, since all $x\in \mathbb{Z}[\sqrt{d}]\subset\mathbb{C}$, then $0+x = x$, and $1\cdot x = x$, showing that $0$ is the zero element,
        while $1$ is the unity element.

        \item Also, for all $a+b\sqrt{d}\in\mathbb{Z}[\sqrt{d}]$, since $a,b\in\mathbb{Z}$ satisfies $(-a),(-b)\in\mathbb{Z}$, so $-a-b\sqrt{d}\in\mathbb{Z}[\sqrt{d}]$,
        while $(a+b\sqrt{d})+(-a-b\sqrt{d})=(a-a)+(b-b)\sqrt{d}=0$, hence every element has its additive inverse in the set also.
    \end{itemize}

    Now, since associativity of the operations, commutativity of addition, and the distributive property are all inherited from $\mathbb{C}$,
    then we can say $(\mathbb{Z}[\sqrt{d}],+)$ is an abelian group (due to the closed operation, existence of zero element and additive inverse),
    while $(\mathbb{Z}[\sqrt{d}],\cdot)$ is a monoid (due to the closed operation, and the existence of unity element).

    Hence, the above properties guarantee $\mathbb{Z}[\sqrt{d}]\subset \mathbb{C}$ to be a subring.

    \hfil

    \item[(2)] For all $x,y\in\mathbb{Z}[\sqrt{d}]$, $x=a+b\sqrt{d}$, and $y=e+f\sqrt{d}$ for some $a,b,e,f\in\mathbb{Z}$.
    
    Hence, the following formulas are true:
    $$N(x)=|a^2-db^2|,\quad N(y)=|e^2-df^2|$$
    $$N(x)N(y)=|a^2-db^2|\cdot|e^2-df^2| = |(a^2-db^2)(e^2-df^2)| = |(ae)^2-d(af)^2-d(be)^2+(bfd)^2|$$
    $$= |((ae)^2+2abefd+(bfd)^2)+(-d(af)^2-d(be)^2-2abefd)|$$
    $$= |(ae+bfd)^2-d((af)^2+2abef+(be)^2)| = |(ae+bfd)^2-d(af+be)^2|$$

    $$xy=(a+b\sqrt{d})(e+f\sqrt{d})=(ae+bfd)+(af+be)\sqrt{d}$$
    $$N(xy)=|(ae+bfd)^2-d(af+be)^2|$$
    Then, since $N(x)N(y)=|(ae+bfd)^2-d(af+be)^2|=N(xy)$, the equation $N(x)N(y)=N(xy)$ is satisfied for all $x,y\in\mathbb{Z}[\sqrt{d}]$.

    \hfil

    \item[(3)] \begin{itemize}
        \item[$\implies:$] Suppose $x=0\in\mathbb{Z}[\sqrt{d}]$, then since $0=0+0\sqrt{d}$, then $N(0)=|0^2-d\cdot 0^2|=0$.
        \item[$\impliedby:$] Suppose $x=(a+b\sqrt{d})\in\mathbb{Z}[\sqrt{d}]$ satisfies $N(x)=|a^2-db^2|=0$, then $a^2=db^2$.
        
        Notice that since $d\neq 1$ is not divisible by any square of prime numbers, which $d\neq 0$ (because $0$ is divisible by all square of primes); so, if we do the unique prime factorization of $d$,
        $d=p_1^{q_1}...p_n^{q_n}$, we must have $q_1=...=q_n=1$ (if one $q_i>1$, then $p_i^2$ actually divides $d$, which is a contradiction). So, $d=p_1...p_n$, where each $p_i$ is a distinct prime.

        \hfil

        Now, suppose the contrary that $a\neq 0$ (which implies $x\neq 0$), then since $p_1\bigm | d$, then $p_1\bigm |a^2$ (Note: $a^2=db^2$).
        Which, this implies that $p_1\bigm |a$ (if not, by Euclid's Lemma, $(p_1,a)=1$, which implies that $(p_1,a^2)=1$, so $p_1$ no longer divides $a^2$, which is a contradiction).

        Then, under the prime factorization of $a^2$, $p_1$  must have an even power (since $a^2$ is a perfect square), which implies that in the prime factorization of $db^2=a^2$,
        $p_1$ must also have even power, say $2k$ for some $k\in\mathbb{N}$. 
        
        Yet, since $p_1$ has power $1$ in the prime factorization of $d$, then it can only have power of $(2k-1)$ in the prime factorization of $b^2$,
        which it appears odxd times. 

        Notice that since $b^2$ is a perfect square, all prime factors must have even power in prime factorization of $b^2$; however, now $p_1$ has an odd power in the prime factorization of $b^2$,
        which creates a contradiction.

        Therefore, our assumption must be false, $a=0$. Which further implies that $db^2=a^2=0$. Then, since $d\neq 0$, then since $\mathbb{Z}$ is an integral domain,
         $b^2=0$, and this implies $b=0$.

        So, $a=b=0$, hence $x=a+b\sqrt{d}=0$.
    \end{itemize}

    The above proves both direction, hence $x=0$ iff $N(x)=0$.

    \hfil

    \item[(4)] \begin{itemize}
        \item[$\implies:$] Suppose $N(x)=1$. Since $x=a+b\sqrt{d}$ for some $a,b\in\mathbb{Z}$ that satisfies $N(x)=|a^2-db^2|=1$. 
        Now, consider $x'=a-b\sqrt{d}$: Since $xx' = (a+b\sqrt{d})(a-b\sqrt{d})=a^2-db^2$, since $|a^2-db^2|=1$, then $a^2-db^2=1$ or $-1$.

        If $xx'=a^2-db^2=1$, then $x$ is already invertible; else if $xx'=a^2-db^2=-1$, then $(xx')^2 = x\cdot (x(x')^2)=1$, showing that $x$ is again invertible.

        Hence, regardless of the case, $x\in(\mathbb{Z}[\sqrt{d}])^\times$.

        \item[$\impliedby:$] Suppose $x\in (\mathbb{Z}[\sqrt{d}])^\times$, then $x^{-1}$ exists, and $xx^{-1}=1$.
        Then, by previous statements, the following is true:
        $$1=|1^2-d\cdot 0^2|=N(1)=N(xx^{-1})=N(x)N(x^{-1})$$
        Since $N(x)$ is a nonnegative integer with $N(x)\bigm | 1$, then the only possibility is $N(x)=1$.
    \end{itemize}

    Since above proves both direction, then we can conclude that $N(x)=1$ iff $x\in(\mathbb{Z}[\sqrt{d}])^\times$.

    \hfil

    \item[(5)] Suppose $x\in\mathbb{Z}[\sqrt{d}]$ satisfies $N(x)\in\mathbb{Z}$ being a prime number. Then, for all $y,z\in\mathbb{Z}[\sqrt{d}]$,
    if $yz = x$, then by the statement proven above, $N(yz)=N(y)N(z)=N(x)$ while $N(x)$ is a prime.

    Hence, since both $N(y),N(z)$ are nonnegative integers, because $N(y)N(z)$ is prime, at least one of them must be $1$.

    So, WLOG, assume $N(y)=1$. Which, by the statement proven in previous section, $y\in (\mathbb{Z}[\sqrt{d}])^\times$, this shows that $x$ is irreducible,
    since all $y,z\in\mathbb{Z}[\sqrt{d}]$ with $yz = x$ must have at least one of them being a unit.
\end{itemize}

\break

\section*{3}
\begin{myBox}[]{}
    \begin{question}
        Prove or disprove that 7 is irreducible in $\mathbb{Z}[\sqrt{5}]$.
    \end{question}
\end{myBox}

\textbf{Pf:}

Suppose $x,y\in\mathbb{Z}[\sqrt{5}]$ satisfies $xy=7$. Then, based on the norm problem in \textbf{Question 2},
the following is true:
$$49 = |7^2-5\cdot 0^2| = N(7) = N(xy) = N(x)N(y)$$
Since $N(x),N(y)$ are nonnegative integers dividing $49$, then they must be either $1,7$, or $49$.

\hfil

First, since $N(x)N(y)=49$, then one of them is $1$ iff the other one is $49$. But, since in \textbf{Question 2} we've proven how the norm is $1$ iff the element is invertible,
then this case provides no information about irreducibility (since one of the element in $x,y$ is already a unit, due to having a norm of $1$).

\hfil

Now, consider the case where both $N(x),N(y)\neq 1$, which $N(x)=N(y)=7$. Yet, this is not possible:
Suppose there exists $x=a+b\sqrt{5}\in\mathbb{Z}[\sqrt{5}]$ with $N(x)=|a^2-5b^2|=7$, then either $a^2-5b^2=7$, or $a^2-5b^2=-7$.

For the first case, ($a^2-5b^2=7$), if take modulo $5$ on both sides, we get $(a\mod\ 5)^2=(7\mod\ 5)=2$. Yet, this equation has no solution
(Since $0^2=0$, $1^2=1$, $2^2=4$, $3^2=9\equiv 4\ (mod\ 5)$, $4^2=16\equiv 1\ (mod\ 5)$, no element in $\mathbb{Z}_5$ satisfies $(a\mod 5)^2=2$).

Similarly, for the second case ($a^2-5b^2=-7$), if take modulo $5$ on both sides again, we get $(a\mod\ 5)^2=(-7\mod\ 5)=3$. However, by the same list above,
no element in $\mathbb{Z}_5$ satisfies $(a\mod\ 5)^2=3$, so this equation also has no solution.

Since for both possiblities we eventually reach a contradiction, then there doesn't exist $x\in\mathbb{Z}[\sqrt{5}]$, with $N(x)=7$.

\hfil

Since if $xy=7$, $N(x)N(y)=49$, which the above proves that $N(x)=N(y)=7$ is not possible, then WLOG, it must be the case $N(x)=1$ and $N(y)=49$.
Since $N(x)=1$, then $x\in\mathbb{Z}[\sqrt{5}]$ is invertible (by the statement in \textbf{Question 2}).

Because $xy=7$ implies one of the element is a unit, hence $7$ is irreducible in $\mathbb{Z}[\sqrt{5}]$.

\break

\section*{4}      
\begin{myBox}[]{}
    \begin{question}
        Find the smallest prime number which is irreducible in $\mathbb{Z}[\sqrt{-1}]$.
    \end{question}
\end{myBox}

\textbf{Pf:}

First, consider $2$: Since $(1+\sqrt{-1})(1-\sqrt{-1})=1-(\sqrt{-1})^2 = 1-(-1)=2$, while $N(1+i)=N(1-i)=|1^2-(-1)1^2| = 2$, 
this shows that both $(1+\sqrt{-1}),(1-\sqrt{-1})$ are not unit (since they have norm that's not $1$, which based on \textbf{Problem 2},
they're not units), while $(1+\sqrt{-1})(1-\sqrt{-1})=2$, which implies that $2$ is in fact reducible.

\hfil

Now, consider $3$: Recall that every prime element of a commutative ring is automatically irreducible, so to show if one element is irreducible,
it's sufficient to show that it's a prime element.

If consider $\mathbb{Z}[\sqrt{-1}]/(3)$, it is in the form $\mathbb{Z}_3[i]$, which is in fact a field (proven in \textbf{HW 2 Problem 4}), 
hence it is an integral domain.

Then, since $\mathbb{Z}[\sqrt{-1}]/(3)$ is an integral domain, it implies that $(3)$ is a prime ideal, hence $3$ is a prime element, which is irreducible.

So, he smallest prime which is irreducible in $\mathbb{Z}[\sqrt{-1}]$, is $3$.

\hfil

\hfil

\section*{5}
\begin{myBox}[]{}
    \begin{question}
        Show that there is a surjective ring homomorphism $\phi:\mathbb{Z}[X]\twoheadrightarrow \mathbb{Z}[\sqrt{5}]$. Find the
kernel of this ring homomorphism.
    \end{question}
\end{myBox}

\textbf{Pf:}

\textbf{The Ring Hmomorphism $\phi$:}

Define the ring homomorphism $\phi:\mathbb{Z}[X]\twoheadrightarrow \mathbb{Z}[\sqrt{5}]$ by $\phi(f(X))=f(\sqrt{5})$ (which is valid to perform,
since $\mathbb{Z}$ has an inclusion into $\mathbb{Z}[\sqrt{5}]$, so the map is first sending $f(x)\in\mathbb{Z}[X]$ to $f(x)\in(\mathbb{Z}[\sqrt{5}])[X]$, then take the evaluation $f(\sqrt{5})$).

Since evaluation map of a ring is always a ring homomorphism, so it remains to check surjectivity:
For all $a+b\sqrt{5}\in\mathbb{Z}[\sqrt{5}]$, the polynomial $a+bX\in\mathbb{Z}[X]$ satisfies $\phi(a+bX) = a+b\sqrt{5}$, showing that this map $\phi$ is in fact surjective.

\hfil

\textbf{The kernel of the map:}

Consider the ideal $I=(X^2-5)\subset \mathbb{Z}[X]$:

First, for all $f(X)\in I$, since $f(X)=(X^2-5)g(X)$ for some $g(X)\in\mathbb{Z}[X]$, then, the following is true:
$$\phi(f(X))=((\sqrt{5})^2-5)g(\sqrt{5}) = (5-5)g(\sqrt{5})=0$$
Hence, $f(X)\in \ker(\phi)$, showing that $(X^2-5)\subseteq \ker(\phi)$. Which, by Generalized First Isomorphism Theorem, there exists a well-defined ring homomorphism $\phi':\mathbb{Z}[X]/(X^2-5)\twoheadrightarrow \mathbb{Z}[\sqrt{5}]$,
such that composing with the projection map $\pi(f(X))=f(X)\mod (X^2-5)$, we have:
$$\phi(f(X))=\phi'(\pi(F(x)))=\phi'(f(X)\mod (X^2-5)) = f(\sqrt{5})$$

\hfil

Now, we'll prove that $(X^2-5)=\ker(\phi)$, by consider the relationship with $\mathbb{Q}[X]$ and $\mathbb{Q}[\sqrt{5}]$: 

\begin{itemize}
    \item[(1)] First, for $X^2-5\in \mathbb{Q}[X]$, by Rational Root Theorem, the only possible rational roots are $\pm 1,\pm 5$; yet, the following evaluations show that all of them are not a root:
    $$1^2-5 = (-1)^2-5 = -4,\quad 5^2-5 = (-5)^2-5 = 20$$
    Hence, $(X^2-5)$ has no roots in $\mathbb{Q}[X]$, and because it is degree $2$ while $\mathbb{Q}$ is a field, we can conclude that $X^2-5$ is irreducible, hence a prime element in $\mathbb{Q}[X]$ (which, $(X^2-5)\subseteq \mathbb{Q}[X]$ is a prime ideal, 
    and because $\mathbb{Q}[X]$ is a PID, the nonzero ideal $(X^2-5)$ is in fact maximal).

    \hfil

    \item[(2)] Now, define the evaluation map $\bar{\phi}:\mathbb{Q}[X]\twoheadrightarrow \mathbb{Q}[\sqrt{5}]$ by 
    $\bar{\phi}(f(X))=f(\sqrt{5})$, which this map is surjective, since all $a+b\sqrt{5}\in\mathbb{Q}[\sqrt{5}]$ has $\bar{\phi}(a+bX)=a+b\sqrt{5}$.
    
    We know that for $X^2-5\in \mathbb{Q}[X]$, $(X^2-5)\subseteq \ker(\bar{\phi})$ (since for any $g(X)\in\mathbb{Q}[X]$, $(X^2-5)g(X)$ has an evaluation at $\sqrt{5}$ being $0$); so, by Generalized First Isomorphism Theorem, 
    there exists a well-defined ring homomorphism $\bar{\phi}':\mathbb{Q}[X]/(X^2-5)\twoheadrightarrow \mathbb{Q}[\sqrt{5}]$, such that the composition with canonical map $\pi(f(X))=f(X)\mod (X^2-5)$, we have $\bar{\phi}(f(X))=\bar{\phi}'(\pi(f(X)))=f(\sqrt{5})$.

    Now, notice that since $(X^2-5)\subseteq \mathbb{Q}[X]$ is a maximal ideal, then $\mathbb{Q}[X]/(X^2-5)$ is actually a field;
    since $\bar{\phi}':\mathbb{Q}[X]/(X^2-5)\twoheadrightarrow \mathbb{Q}[\sqrt{5}]$ is surjective, then the domain is a field implies that $\ker(\bar{\phi}') = (0)$ (since the only ideal for a field is $(0)$ or itself, $\ker(\bar{\phi}')$ must be one of them; and since the map is surjective,
    then $\ker(\bar{\phi}')=(0)$ is enforced). 

    Since $\bar{\phi'}$ is surjective, while $\ker(\bar{\phi'})=(0)$, then it is also injective, proving that $\bar{\phi'}$ is in fact an isomorphism. 
    
    \hfil

    \item[(3)] Finally, consider the following diagram:
    \begin{center}
        \includegraphics*[width=60mm]{commute 3 111b hw7.png}
    \end{center}
    (Note: the $i$ indicates inclusion map, which is a ring homomorphism).

    Since we've proven that $\phi'$ is a surjective function, $\bar{\phi'}$ is an isomorphism, and both inclusion maps are injective, the only thing left is to prove that the diagram commutes.

    For all $f(X)\in\mathbb{Z}[X]\subset \mathbb{Q}[X]$, since the following is true: 
    $$\phi'(f(X)\mod (X^2-5)) = \phi(f(X))=f(\sqrt{5})\in \mathbb{Z}[X]$$
    Then, $i\circ \phi'(f(X)\mod (X^2-5))=f(\sqrt{5})\in\mathbb{Q}[X]$.

    Similarly, if do the inclusion first, $i\left(f(X)\mod (X^2-5)\right)\in \mathbb{Q}[X]/(X^2-5)$ satisfies: 
    $$\bar{\phi'}\circ i\left(f(X)\mod (X^2-5)\right) = \bar{\phi'}(f(X)\mod (X^2-5)) = \bar{\phi}(f(X))=f(\sqrt{5})\in\mathbb{Q}[X]$$
    Hence, $\phi'\circ i = i\circ \bar{\phi'}$, showing that the diagram commutes.

    \hfil

    Now, because $\bar{\phi'}\circ i$ is injective (since both $i$ and $\bar{\phi'}$ is injective), then $i\circ \phi'$ is injective.
    
    This enforces $\phi'$ to be injective
    (if there exists $f(X),g(X)\in\mathbb{Z}[X]$ with $\left(f(X)\mod (X^2-5)\right)\neq \left(g(X)\mod (X^2-5)\right)$, but $\phi'\left(f(X)\mod (X^2-5)\right)=\phi'\left(g(X)\mod (X^2-5)\right)$,
    then the composition $i\circ\phi'\left(f(X)\mod (X^2-5)\right)=i\circ\phi'\left(g(X)\mod (X^2-5)\right)$, showing that $i\circ\phi'$ is no longer injective, which contradicts).

    So, since $\phi'$ is injective, then for all $f(X)\in \ker(\phi)$ (which $\phi(f(X))=f(\sqrt{5})=0$), then:
    $$\phi'(f(X)\mod (X^2-5))=\phi(f(X))=0$$
    Which shows that $f(X)\mod (X^2-5)\in \ker(\phi')=(0)$ (since $\phi'$ is now injective), or $f(X)\in (X^2-5)$.

    Hence, $\ker(\phi)\subseteq (X^2-5)$, with the other inclusion proven before, we can conclude that $\ker(\phi)=(X^2-5)$.
\end{itemize}


%Now, because $X^2-5$ is irreducible over $\mathbb{Q}[X]$, it is also irreducible over $\mathbb{Z}[X]$,


\break

\section*{6}
\begin{myBox}[]{}
    \begin{question}
        Prove or disprove that $\mathbb{Z}[\sqrt{5}]$ is a pid.
    \end{question}
\end{myBox}

\textbf{Pf:}

We'll prove that $\mathbb{Z}[\sqrt{5}]$ is not a PID. Recall that for a PID, all irreducible elements are prime elements also, we'll use this to prove the statement.

\hfil

\textbf{$2$ is irreducible}

Now, consider $2\in\mathbb{Z}[\sqrt{5}]$: Suppose $x,y\in\mathbb{Z}[\sqrt{5}]$ satisfies $xy=2$, then by the statements in \textbf{Question 2},
using the same norm function, we get:
$$4=|2^2-5\cdot 0^2|=N(2)=N(x)N(y)$$
So, since $N(x),N(y)$ are nonnegative integers that multiply to be $4$, they must be either $1,2$, or $4$.

First, one of the element is $1$ iff the other element is $4$ (since $1\cdot 4 = 4$); if any element satisfies the norm being $1$ (for instance, if $N(x)=1$), 
then using the statements in \textbf{Question 2}, we know $x\in(\mathbb{Z}[\sqrt{5}])^\times$, hence it is a unit, which shows nothing about irreducibility.

\hfil

Now, consider the case where both $N(x),N(y)\neq 1$ (hence $N(x),N(y)=2$). Yet, this is not possible:
Suppose $x=a+b\sqrt{5}$ satisfies $N(x)=|a^2-5b^2| = 2$, then either $a^2-5b^2=2$, or $a^2-5b^2=-2$.

For the first possibility ($a^2-5b^2=2$), if we take modulo $5$ on both sides, we get that $(a\mod\ 5)^2 = (2\mod\ 5)$. However, this equation has no solution (since $0^2=0$, $1^2=1$, $2^2=4$, $3^2=9\equiv 4\ (mod\ 5)$, and $4^2=16\equiv 1\ (mod\ 5)$).

Similarly, for the second possibility ($a^2-5b^2=-2$), if we take modulo $5$ on both sides again, we get that $(a\mod\ 5)^2=(-2\mod 5)=(3\mod 5)$, which from the above list of equations,
we know it also has no solutions.

Hence, both possibilities are not satisfied, showing that $N(x)\neq 2$, which is a contradiction.

Therefore, we can conclude that if $xy=2$, then we can't have both $N(x),N(y)\neq 1$ (since this implies $N(x)=N(y)=2$, which is impossible),
so at least one of the element has norm $1$, which is equivalent to being a unit.

Since $xy=2$ implies one of the element is a unit, then $2$ is irreducible.

\hfil

\textbf{$2$ is not prime:}

However, $2$ is not prime: If we consider $\mathbb{Z}[\sqrt{5}]/(2)$, notice that since $1+\sqrt{5}$ has none of its coefficients divisible by $2$,
hence $(1+\sqrt{5})\notin (2)$, which under the quotient, $(1+\sqrt{5})\mod\ (2)$ is nonzero.

Yet, since $(1+\sqrt{5})^2 = 1+5+2\sqrt{5} = 6+2\sqrt{5}=2(3+\sqrt{5})$, then $(1+\sqrt{5})^2\mod\ (2)=0$,
showing that $(1+\sqrt{5})\mod\ (2)$ is actually a nonzero nilpotent element in $\mathbb{Z}[\sqrt{5}]/(2)$, hence this ring is not an integral domain.

\hfil

Because $\mathbb{Z}[\sqrt{5}]/(2)$ is not an Integral Domain, then $(2)$ is not a prime ideal, showing that $2$ is not a prime element.

Since there exists an irreducible element that's not prime, then $\mathbb{Z}[\sqrt{5}]$ cannot be a PID.

\end{document}