% Math_111B_HW6_Zih-Yu_Hsieh.tex

\documentclass{article}
\usepackage{graphicx} % Required for inserting images
\usepackage[margin = 2.54cm]{geometry}
\usepackage[most]{tcolorbox}

\newtcolorbox{myBox}[3]{
arc=5mm,
lower separated=false,
fonttitle=\bfseries,
%colbacktitle=green!10,
%coltitle=green!50!black,
enhanced,
attach boxed title to top left={xshift=0.5cm,
        yshift=-2mm},
colframe=blue!50!black,
colback=blue!10
}

\usepackage{amsmath}
\usepackage{amssymb}
\usepackage{verbatim}
\usepackage[utf8]{inputenc}
\linespread{1.2}

\newtheorem{definition}{Definition}
\newtheorem{proposition}{Proposition}
\newtheorem{theorem}{Theorem}
\newtheorem{question}{Question}

\title{Math 111B HW6}
\author{Zih-Yu Hsieh}

\begin{document}
\maketitle

\section*{1}
\begin{myBox}[]{}
    \begin{question}
        Let $k$ be an infinite field and let $f (X), g(X) \in k[X]$ be such that $a$ for
        all $a\in k^\times$. Prove or disprove that $f (X) = g(X)$.
    \end{question}
\end{myBox}

\textbf{Pf:}

We'll prove by contradiction, that $f(X)=g(X)$.

Suppose $f(X)\neq g(X)$, then $(f-g)(X)\neq 0$, hence $\deg(f-g)=n$ for some nonnegative integer $n$.

However, since $k$ is a field, a nonzero polynomial over a field has at most $n$ distinct zeroes, hence $(f-g)$ should have no more than $n$ distinct zeroes.

Yet, since for the infinite field $k$, $k^\times$ is also infinite, and all $a\in k^\times$ satisfies $f(a)=g(a)$, or $(f-g)(a)=0$, then $a$ is a zero of $(f-g)$,
showing that $(f-g)$ has infinite zeroes, which contradicts to the previous statement.

Hence, $f(X)=g(X)$ is enforced.

\hfill

\hfill

\section*{2}
\begin{myBox}[]{}
    \begin{question}
        Let $R$ be an integral domain such that the division algorithm holds in $R[X]$. Prove
        or disprove that $R$ is a field.  
    \end{question}
\end{myBox}

\textbf{Pf:}

Suppose $R$ is an integral domain such that the division algorithm holds in $R[X]$. Then, for all nonzero element $a\in R$,
consider $X^2$ and $aX$ in $R[X]$:

Because division algorithm works, there exists unique pair of polynomials $q(X),r(X)\in R[X]$, with $X^2 = q(X)\cdot aX + r(X)$, such that $r(X)=0$ or $\deg(r)<\deg(aX)=1$.

Since $r(X)=0$ or $0\leq \deg(r)<1$, then WLOG, can assume $r(X)$ is a constant, or $r(X)=\lambda\in R$. Hence, the above equation becomes:
$$X^2=q(X)\cdot aX + \lambda,\quad X^2-\lambda = q(X)\cdot aX$$
Because $X^2-\lambda \neq 0$, then $q(X)\neq 0$; hencee, $2=\deg(X^2) = \deg(q(X))+\deg(aX) = \deg(q(X))+1$, showing that $\deg(q(X))=1$.

Hence, there exists $b,c\in R$ (with $b\neq 0$), such that $q(X)=bX+c$. So, the above equation becomes:
$$X^2-\lambda = (bX+c)aX = abX^2 + acX$$
Because the two equations match up, then the leading coefficient also matches. Therefore, $1 = ab$, showing that $a$ is invertible.

\hfill

Because all nonzero element $a\in R$ is invertible, with the fact that $R$ is an integral domain (which is commutative), $R$ is a field.

\hfill

\hfill

\section*{3}
\begin{myBox}[]{}
    \begin{question}
        Prove or disprove that $f (X) = x^7 -X^5 + 2X^4 -3X^2 -X + 2 \in \mathbb{Q}[X]$ is irreducible.
    \end{question}
\end{myBox}

\textbf{Pf:}

Since $f(X)$ is a monic polynomial, then based on Rational Root Theorem, if there exists a rational root $q\in\mathbb{Q}$ of $f(X)$,
not only if $q$ is an integer, but also $q$ divides the constant term of $f(X)$, namely $2$.

\hfill

So, consider the divisors of $2$, the collection $\{\pm 1,\pm 2\}$:

Plug in $X=1$, we get $f(1)=1^7-1^5+2\cdot 1^4-3\cdot 1^2-1+2 = 1-1+2-3-1+2 = 0$, hence $1\in\mathbb{Q}$ is a root of $f(X)$.

Then, using the division algorithm, with the linaer term $(X-1)$, there exists unique polynomials $q(X),r(X)\in\mathbb{Q}[X]$,
with $f(X)=(X-1)q(X)+r(X)$, and either $r(X)=0$ or $0\leq \deg(r)<\deg((X-1))=1$. Hence, $r(X)$ is in fact a constant.

Also, since $f(1)=(1-1)q(1)+r(1)=0$, then $r(1)=0$, showing that $r(X)=0$. Hence, $f(X)=(X-1)q(X)$.

\hfill

Finally, since $f(X)\neq 0$, then $q(X)\neq 0$; also, because $\deg(f)=7$ and $\deg(f)=\deg((X-1))+\deg(q) = 1+\deg(q)$, then $\deg(q)=6$, showing that $q$ is a nonconstant polynomial in $\mathbb{Q}[X]$ 
(where $\mathbb{Q}$ is an Integral domain), hence nonconstant polynomials are not invertible.

Because $(X-1)$, $q(X)$ are both nonconstant polynomial, they're not invertible, hence $f(X)$ is a reducible element in $\mathbb{Q}[X]$.


\break

\section*{4 (Potentially need to change)}
\begin{myBox}[]{}
    \begin{question}
        Find all factors of $X^7-X\in \mathbb{Z}_7[X]$.
    \end{question}
\end{myBox}

\textbf{Pf:}

Recall that Fermat's Little Theorem states that given any prime $p$, all $n\in\mathbb{N}$ satisfies $n^p\equiv n\ (mod\ p)$.

Then, for all $n\in\mathbb{Z}_7$, it is also true that $n^7\equiv n\ (mod\ 7)$, showing that $n^7-n\equiv 0\ (mod\ 7)$.
Hence, $n$ is a zero of the equation $X^7-X \in \mathbb{Z}_7[X]$, which since $\mathbb{Z}_7$ is a field (due to the fact that $7$ is prime),
$(X-n)$ is a factor of $X^7-X$.

Also, since $\deg(X^7-X)=7$, then there are at most $7$ zeroes (counting multiplicity) for this equation. Since all $n\in\mathbb{Z}_7$ is a zero, 
then each $n$ has a multiplicity of $1$, showing that $X^7-X$ must be factored into distinct linear terms $(X-n)$.

Hence, $X^7-X = X(X-1)(X-2)(X-3)(X-4)(X-5)(X-6)$, and arbitrary product of these distinct linear terms would be factor of $X^7-X$.

\hfill

\hfill

\section*{5}
\begin{myBox}[]{}
    \begin{question}
        Find a prime $p > 5$ such that $X^2+1\in \mathbb{Z}_p[X]$ is irreducible.
    \end{question}
\end{myBox}

\textbf{Pf:}

Consider $p=7$: 

Recall that for a degree $2$ or $3$ polynomial in a polynomial ring $k[X]$ over a field $k$,
it is reducible implies there is a zero in the field $k$. Hence,since $\mathbb{Z}_7$ is a field,
to show that $X^2-1$ is irreducible in $\mathbb{Z}_7[X]$, it suffices to show that it has no zeroes in $\mathbb{Z}_7$.

Which, plug in all elements of $\mathbb{Z}_7$, we get:
$$0^2+1 = 1,\quad 1^2+1 = 2,\quad 2^2+1 = 5,\quad 3^2+1 = 10 \equiv 3\ (mod\ 7)$$
$$4^2+1 = 17\equiv 3\ (mod\ 7),\quad 5^2+1 = 26 \equiv 5\ (mod\ 7),\quad 6^2+1 = 37 \equiv 2\ (mod\ 7)$$
Hence, $X^2+1$ has no zeroes in $\mathbb{Z}_7$, showing that $X^2+1$ is irreducible in $\mathbb{Z}_7[X]$.

\break

\section*{6}
\begin{myBox}[]{}
    \begin{question}
        Let $f (X) = a_0 + a_1X + ... + a_{n-1}X^{n-1} + a_nX^n \in k[X]$, where $k$ is a field and $a_0\neq 0$.
        Let $g(X) = a_n + a_{n-1}X +... + a_1X^{n-1} + a_0X^n$. Suppose that $f (X)$ has a linear factor in
        $k[X]$. Prove or disprove that $g(X)$ has a linear factor in $k[X]$.
    \end{question}
\end{myBox}

\textbf{Pf:}

First, since $f(X)$ has a linear factor, then there exists $a\in k$, where $f(X)=(X-a)q(X)$ for some $q(X)\in k[X]$.
Hence, $f(a) = (a-a)q(a)=0$, showing that $a$ is a zero of $f$.

Notice that since $f(0)=a_0$, where by assumption $a_0\neq 0$, showing that $0$ is not a zero of $f$, hence $a\neq 0$.
Then, due to the fact that $k$ is a field and $a\neq 0$, then $a^{-1}\in k$ exists.

\hfill

Now, consider $g(a^{-1})$:
$$g(a^{-1}) = a_n+a_{n-1}a^{-1}+...+a_1(a^{-1})^{n-1}+a_0(a^{-1})^n=\sum_{i=0}^{n}a_{n-i}(a^{-1})^i$$
Which, multiply by $a^n$ on both sides, we get:
$$a^ng(a^{-1})=a^n\cdot \sum_{i=0}^{n}a_{n-i}(a^{-1})^i = \sum_{i=0}^{n}a_{n-i}\left((a^{-1})^i\cdot a^i\right)a^{n-i}=\sum_{i=0}^{n}a_{n-i}\left((a^{-1}\cdot a)^i\right)a^{n-i}$$
$$ = \sum_{i=0}^{n}a_{n-i}a^{n-i} = \sum_{j=0}^{n}a_ja^j$$
(Note: the second line is the change of index $j=n-i$).

Which, the final expression is the same as $f(a)$, which is $0$. Hence, $g(a^{-1})=f(a)=0$, showing that $a^{-1}$ is a zero of $g$.

Then, because it is a root, we can always factor out the term $(X-a^{-1})$ as a linear term of $g(X)$. Hence, $g(X)$ has a linear factor in $k[X]$.

\hfill

\hfill

\section*{7}
\begin{myBox}[]{}
    \begin{question}
        Prove or disprove that $f (X) = X^3 + 4X^2 + X - 1 \in \mathbb{Q}[X]$ is irreducible.
    \end{question}
\end{myBox}

\textbf{Pf:}

Notice that $f$ is a degree $3$ polynomial. Because $\mathbb{Q}$ is a field, then a degree $3$ polynomial is reducible implies there is a zero in the field.
Hence, if there is no zeroes in $\mathbb{Q}$, it implies that the polynomial $f$ is irreducible.

Now, by Rational Root Theorem, because $f$ is a monic polynomial, if $q\in\mathbb{Q}$ is a root of $f$, not only $q$ is an integer,
and $q$ divides the constant coefficient, namely $-1$.

Hence, the only possible rational roots are $\pm 1$. Yet, if plugin the values, we get:
$$f(1)=1^3+4\cdot 1^2+1-1 = 1+4 = 5,\quad f(-1)=(-1)^3+4\cdot (-1)^2+(-1)-1 = -1+4-1-1 = 1$$
Because the only possible rational numbers are not the root of $f$, then $f$ has no zeroes in $\mathbb{Q}$, showing that $f$ is irreducible over $\mathbb{Q}$.

\break

\section*{8}
\begin{myBox}[]{}
    \begin{question}
        Let $f (X) = a_0 + a_1X + ... + a_{n-1}X^{n-1} + a_nX^n \in\mathbb{Z}[X]$. Let $x,y\in\mathbb{Z}$ be such that
        $(x, y) = 1$ and $f (x/y) = 0$ when we consider $f (X)$ as a polynomial over $\mathbb{Q}$. Show that
        $y\bigm |a_n$.
    \end{question}
\end{myBox}

\textbf{Pf:}

If view $f$ as a polynomial over $\mathbb{Q}$, then $f(x/y)=0$ implies the following:
$$0=f(x/y)=a_0+a_1(x/y) + ... + a_{n-1}(x/y)^{n-1}+a_n(x/y)^n=\sum_{i=0}^{n}a_i(x/y)^i$$
Which, multiply both sides by $y^n$, we get:
$$0=y^n\cdot 0 = y^n\cdot \sum_{i=0}^{n}a_i(x/y)^i = \sum_{i=0}^{n}a_i\cdot x^i\cdot y^{n-i} = a_nx^n + \sum_{i=0}^{n-1}a_i\cdot x^i\cdot y^{n-i}$$
$$-\sum_{i=0}^{n-1}a_i\cdot x^i\cdot y^{n-i} = a_nx^n,\quad -y\cdot \sum_{i=0}^{n-1}a_ix^i\cdot y^{n-i-1} = a_nx^n$$
(Note: for $0\leq i\leq n-1$, $n-i-1 \geq 0$, hence for the last equation we can factor out a $y$).

Since the left side is divisible by $y$, then the right side is also divisible by $y$;
However, since $x,y$ are coprime, then $y$ cannot divide $x$, hence it cannot divide $x^n$. So, in case for $a_nx^n$ to be divisible by $y$, $y$ must divide $a_n$.

\end{document}