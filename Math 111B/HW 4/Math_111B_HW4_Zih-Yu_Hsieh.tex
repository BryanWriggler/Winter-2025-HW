%Math_111B_HW4_Zih-Yu_Hsieh.tex

\documentclass{article}
\usepackage{graphicx} % Required for inserting images
\usepackage[margin = 2.54cm]{geometry}
\usepackage[most]{tcolorbox}

\newtcolorbox{myBox}[3]{
arc=5mm,
lower separated=false,
fonttitle=\bfseries,
%colbacktitle=green!10,
%coltitle=green!50!black,
enhanced,
attach boxed title to top left={xshift=0.5cm,
        yshift=-2mm},
colframe=blue!50!black,
colback=blue!10
}

\usepackage{amsmath}
\usepackage{amssymb}
\usepackage{verbatim}
\usepackage[utf8]{inputenc}
\linespread{1.2}

\newtheorem{definition}{Definition}
\newtheorem{proposition}{Proposition}
\newtheorem{theorem}{Theorem}
\newtheorem{question}{Question}

\title{Latex Template}
\author{Zih-Yu Hsieh}

\begin{document}
\maketitle

\section*{1}
\begin{myBox}[]{}
    \begin{question}
        Prove or disprove the statement that if $k$ is any field, then $(X^2+1)$ is a maximal
        ideal of $k[X]$.
    \end{question}
\end{myBox}

\textbf{Pf:}

Consider $k=\mathbb{C}$. Then, notice that the following is true: 
$$(X+i)(X-i)=X(X-i)+i(X-i)=(X^2-ix)+(iX-i^2)=(X^2-(-1))=(X^2+1)$$
So, $(X^2+1)\subset (X+i)$, since $X^2+1\in (X+i)$.

\hfill

However, $(X^2+1)\subsetneq (X+i)$, since $X+i\notin (X^2+1)$:
Suppose $X+i=(X^2+1)h(X)$ for some $h(X)\in \mathbb{C}[X]$, then since $X+i\neq 0$, then $h(X)\neq 0$;
also, since $\mathbb{C}[X]$ is an integral domain, hence $1=\deg(X+i)=\deg(X^2+1)+\deg(h(X)) \geq \deg(X^2+1)=2$ (Note: since $\deg(h(x))\geq 0$).
However, this is a contradiction. Hence, $X+i\neq (X^2+1)h(X)$ for all $h(X)\in \mathbb{C}[X]$, showing that $X+i\notin (X^2+1)$.

\hfill

Furthermore, $(X+i)\neq \mathbb{C}[X]$: Suppose $(X+i)=\mathbb{C}[X]$, then $1\in (X+i)$, which there exists $h(X)\in\mathbb{C}[X]$,
such that $(X+i)h(X)=1$. However, since $1\neq 0$, then $h(X)\neq 0$; also, since $\mathbb{C}[X]$ is an integral domain, then
$0=\deg(1)=\deg(X+i)+\deg(h(X))\geq \deg(X+i)=1$, which is a contradiction. 
Hence, $(X+i)\neq \mathbb{C}[X]$.

\hfill

So, $(X+i)$ is an ideal strictly containing $(X^2+1)$, while $(X+i)\neq \mathbb{C}[X]$, showing that $(X^2+1)$ is not a maximal ideal of $\mathbb{C}[X]$.

\hfill

\hfill

\section*{2}
\begin{myBox}[]{}
    \begin{question}
        Prove that if $k$ is a field, then the map $\phi:k[X]\rightarrow k$ given by $\phi(f(X))=f(a)$ ($a\in k$) defines
        an isomorphism of rings, $\phi':\frac{k[X]}{(X-a)}\tilde\rightarrow k$.
    \end{question}
\end{myBox}

\textbf{Pf:}

(Note: Possibly need to show that it is a kernel)

\textbf{$(X-a)$ is the kernel:}

Given the ring homomorphism $\phi:k[X]\rightarrow k$ defined as $\phi(f(X))=f(a)$ ($a\in k$), 
for all $f(X)\in (X-a)$, since there exists $h(X)\in k[X]$, with $f(X)=(X-a)h(X)$. Hence:
$$\phi(f(X))=f(a)=(a-a)h(a)=0\cdot h(a)=0$$
This implies that $f(X)\in \ker(\phi)$, hence $(X-a)\subseteq \ker(\phi)$.

\hfill

Similarly, for all $f(X)\in \ker(\phi)$ (which $f(X)=f_0+f_1X+...+f_nX^n$ for some $f_0,f_1,...,f_n\in k$), since $\phi(f(X))=f(a)=0$, then the following is true:
$$f(a)=f_0+f_1a+...+f_na^n,\quad f(X)=f(X)-0=f(X)-f(a)=\sum_{j=0}^{n}f_jX^j-\sum_{j=0}^{n}f_ja^j$$
$$f(X)=\sum_{j=0}^{n}f_j(X^j-a^j)$$
(Note: the above equation is true, since $k[X]$ is commutative).

Now, notice that for all $m\in\mathbb{N}$ ($m\geq 2$), the following is true:
$$(X-a)\left(\sum_{j=0}^{m-1}X^ja^{(m-1)-j}\right)=X\sum_{j=0}^{m-1}X^ja^{(m-1)-j}-a\sum_{j=0}^{m-1}X^ja^{(m-1)-j}$$
$$=\sum_{j=0}^{m-1}X^{j+1}a^{(m-1)-j}-\sum_{j=0}^{m-1}X^ja^{(m-1)-j+1}$$
$$=X^ma^{(m-1)-(m-1)}+\sum_{j=0}^{m-2}X^{j+1}a^{(m-1)-j}-\sum_{j=1}^{m-1}X^ja^{(m-1)-j+1}-X^0a^{(m-1)-0+1}$$
$$=X^m+\sum_{j=1}^{m-1}X^{j}a^{(m-1)-(j-1)}-\sum_{j=1}^{m-1}X^ja^{m-j}-a^{m}$$
$$=X^m+\sum_{j=1}^{m-1}X^{j}a^{m-j}-\sum_{j=1}^{m-1}X^ja^{m-j}a^{m}$$
$$=X^m-a^m$$
Hence, for $m\geq 2$, $X^m-a^m=(X-a)h_m(X)$ for some $h_m(X)\in k[X]$. (And, for $m=1$, $(X-a)=(X-a)\cdot 1$, and for $m=0$, since $(X^0-a^0)=(1-1)=0$, which let $h_1(X)=1$ and $h_0(X)=0$, we can generalize it to $m=1$ and $m=0$ case).

So, the original function $f(X)$ can be rewrite as:
$$f(X)=\sum_{j=0}^{n}f_j(X^j-a^j)=\sum_{j=0}^{n}f_j(X-a)h_j(X) = (X-a)\left(\sum_{j=0}^{n}f_jh_j(X)\right)$$
Hence, $f(X)\in (X-a)$, showing that in fact $\ker(\phi)=(X-a)$.

\hfill

\textbf{Image of the map is $k$:}

For all $c\in k$, since $c\in k[X]$, then $\phi(c)=c$, showing that $\phi$ is surjective.

\hfill

Then, by First Isomorphism Theorem of Rings, we can conclude that $\frac{k[X]}{(X-a)}=\frac{k[X]}{\ker(\phi)}\cong \phi(k[X])=k$, 
which the ring homomorphism $\phi$ defines a ring isomorphism $\phi'$ (projection map) between $\frac{k[X]}{(X-a)}$ and $k$.

\break

\section*{3}
\begin{myBox}[]{}
    \begin{question}
        Let $R = \mathbb{R}[X_1, X_2]$. Prove or disprove that $(X_1^2+1)$ is a maximal ideal of $R$.
    \end{question}
\end{myBox}

\textbf{Pf:}

Consider the ideal $(X_1^2+1,X_2)$: Notice that since $X_1^2+1\in (X_1^2+1,X_2)$, so $(X_1^2+1)\subset (X_1^2+1,X_2)$;
yet, $X_2\notin (X_1^2+1)$:

Suppose $X_2\in (X_1^2+1)$, then $X_2=(X_1^2+1)h(X_1,X_2)$ for some $h(X_1,X_2)\in \mathbb{R}[X_1,X_2]$.
However, if evaluate $X_2=1$, then we get the following:
$$1=(X_1^2+1)h(X_1,1)$$
Notice that since $1\neq 0$, then $h(X_1,1)\neq 0$; hence, with $h(X_1,1), (X_1^2+1)\in \mathbb{R}[X_1]$, the following is true:
$$0=\deg(1)=\deg(X_1^2+1)+\deg(h(X_1,1))\geq \deg(X_1^2+1) = 2$$ 
Which is a contradiction. Hence, the assumption is false, $X_2\notin (X_1^2+1)$.

Hence, we can conclude that $(X_1^2+1)\subsetneq (X_1^2+1,X_2)$.

\hfill

\hfill

Also, notice that $(X_1^2+1,X_2)\neq \mathbb{R}[X_1,X_2]$: Suppose the two are the same, then $1\in (X_1^2+1,X_2)$, 
so there exists $h_1(X_1,X_2), h_2(X_1,X_2)\in \mathbb{R}[X_1,X_2]$ with $1=(X_1^2+1)h_1(X_1,X_2)+X_2h_2(X_1,X_2)$.

Yet, if evaluate $X_2=0$, we'll get the following:
$$1=(X_1^2+1)h_1(X_1,0)$$
Which $h_1(X_1,0)\in \mathbb{R}[X_1]$. Then, since $1\neq 0$, then $h_1(X_1,0)\neq 0$; then again, based on the degree of the polynomial, we yield:
$$0=\deg(1)=\deg(X_1^2+1)+\deg(h_1(X_1,0))\geq \deg(X_1^2+1)=2$$
Which again is a contradiction. Hence, we can't have $(X_1^2+1,X_2)=\mathbb{R}[X_1,X_2]$.

\hfill

So, the above shows that $(X_1^2+1)\subsetneq (X_1^2+1,X_2)\subsetneq \mathbb{R}[X_1,X_2]$, showing that $(X_1^2+1)$ is not a maximal ideal.

\break

\section*{4}
\begin{myBox}[]{}
    \begin{question}
        Let $n$ be a positive integer with decimal representation $a_ka_{k-1}...a_1a_0$. Show that $n$ is
divisible by 9 if and only if $\sum_{i=0}^{k}a_i$ is divisible by 9.
    \end{question}
\end{myBox}

\textbf{Pf:}

\textbf{Powers of $10$ modulo $9$:}

Notice that since $10\equiv 1\ (mod\ 9)$, then for all $n\in\mathbb{N}$, $10^n\equiv 1^n\ (mod\ 9)$, hence $10^n\equiv 1\ (mod\ 9)$.

\hfill

Now, notice that for any $n\in\mathbb{N}$, if the decimal representation is $a_ka_{k-1}...a_1a_0$ (with $a_0,a_1,...,a_k\in \{0,1,...,9\}$), 
it can also be rewritten as:
$$n=\sum_{j=0}^{k}a_j10^j$$
Hence, $n$ is divisible by $9$, if and only if $\sum_{j=0}^{k}a_j10^j$ is divisible by $9$, or $\sum_{j=0}^{k}a_j10^j \equiv 0\ (mod\ 9)$.

Then, based on the ring property of $\mathbb{Z}_9$, the following is true:
$$\left(\sum_{j=0}^{k}a_j10^j\right)\ mod\ 9=\sum_{j=0}^{k}(a_j\ mod\ 9)(10^j\ mod\ 9)=\sum_{j=0}^{k}(a_j\ mod\ 9) = \left(\sum_{j=0}^{k}a_j\right)\ mod\ 9$$
(Note: the above is true, since $10^j\equiv 1\ (mod\ 9)$ for all $j\in\mathbb{N}$).

Hence, we can conclude that $\sum_{j=0}^{k}a_j10^j \equiv 0\ (mod\ 9)$ if and only if $\left(\sum_{j=0}^{k}a_j\right)\equiv 0\ (mod\ 9)$, or $\left(\sum_{j=0}^{k}a_j\right)$ is divisible by $9$.

Therefore, we can conclude that $n$ is divisible by $9$, if and only if  $\left(\sum_{j=0}^{k}a_j\right)$ is divisible by $9$.


\break

\section*{5}
\begin{myBox}[]{}
    \begin{question}
        Let $m$ and $n$ be positive integers which are relative prime. Prove or disprove that
        the rings $\mathbb{Z}_{mn}$ and $\mathbb{Z}_m\times \mathbb{Z}_n$ are isomorphic.
    \end{question}
\end{myBox}

\textbf{Pf:}

Consider the following map $\phi:\mathbb{Z}\rightarrow \mathbb{Z}_m\times\mathbb{Z}_n$,
by $\phi(k)=(k\ mod\ m,k\ mod\ n)$.

It is a ring homomorphism, because for all $a,b\in\mathbb{Z}$, the following is true:
$$\phi(a+b)=((a+b)\ mod\ m,(a+b)\ mod\ n)=(a\ mod\ m,a\ mod\ n)+(b\ mod\ m,b\ mod\ n) = \phi(a)+\phi(b)$$
$$\phi(ab)=((ab)\ mod\ m, (ab)\ mod\ n)=(a\ mod\ m,a\ mod\ n)\cdot (b\ mod\ m,b\ mod\ n) = \phi(a)\cdot\phi(b)$$
(Note: the addition and multiplication is defined coordinate wise).

So, the map is in fact a ring homomorphism.

\hfill

\textbf{Kernel of $\phi$:}

Now, consider $\ker(\phi)$: For all $k\in\ker(\phi)$, since $(k\ mod\ m,k\ mod\ n)=(0,0)$, the $m\bigm |k$ and $n\bigm |k$,
hence $lcm(m,n)\bigm |k$; however, since $m,n$ are assumed to be coprime, then $lcm(m,n)=\frac{mn}{\gcd(m,n)}=mn$ (since $\gcd(m,n)=1$).
Hence, $mn\bigm |k$, showing that $\ker(\phi)\subseteq mn\mathbb{Z}$.

The converse is also true, since for all $k\in mn\mathbb{Z}$, $k=l\cdot mn$ for some $l\in\mathbb{Z}$, which:
$$\phi(k)=(l\cdot mn\ mod\ m,l\cdot mn\ mod\ n) = (0,0)$$
Hence, $k\in\ker(\phi)$, or $mn\mathbb{Z}\subseteq \ker(\phi)$. Then, $\ker(\phi)=mn\mathbb{Z}$.

\hfill

\textbf{$\phi$ is Surjective:}

Since $m,n$ are coprime, the by Bezout's Lemma, there exists $s,t\in\mathbb{Z}$, with $ms+tn=1$.
Then, $ms=-tn+1$, which $\phi(ms)=(ms\ mod\ m,ms\ mod\ n)=(0, -tn+1\ mod\ n)=(0,1)$;
also, since $tn=-ms+1$, which $\phi(tn)=(tn\ mod\ m, tn\ mod\ n)=(-ms+1\ mod\ m, 0)=(1,0)$.

Then, for all $(a,b)\in \mathbb{Z}_m\times\mathbb{Z}_n$, the following is true:
$$(a,b)=a(1,0)+b(0,1)=a\cdot\phi(tn)+b\cdot\phi(ms) = \phi(atn+bms)$$
Hence, we can conclude that $\phi$ is a surjective ring homomorphism.

\hfill

Now, with the above conditions, by First Isomorphism of Rings, we can conclude the following:
$$\mathbb{Z}_{mn}\cong \mathbb{Z}/mn\mathbb{Z}=\mathbb{Z}/\ker(\phi)\cong \phi(\mathbb{Z})=(\mathbb{Z}_m\times\mathbb{Z}_n)$$
Which, $\mathbb{Z}_{mn}$ and $\mathbb{Z}_m\times\mathbb{Z}_n$ are isomorphic as rings.

\break

\section*{6}
\begin{myBox}[]{}
    \begin{question}
        
    \end{question}
\end{myBox}

\textbf{Pf:}


\end{document}