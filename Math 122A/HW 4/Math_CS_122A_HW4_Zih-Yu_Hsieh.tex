%Math_CS_122A_HW4_Zih-Yu_Hsieh.tex

\documentclass{article}
\usepackage{graphicx} % Required for inserting images
\usepackage[margin = 2.54cm]{geometry}
\usepackage[most]{tcolorbox}

\newtcolorbox{myBox}[3]{
arc=5mm,
lower separated=false,
fonttitle=\bfseries,
%colbacktitle=green!10,
%coltitle=green!50!black,
enhanced,
attach boxed title to top left={xshift=0.5cm,
        yshift=-2mm},
colframe=blue!50!black,
colback=blue!10
}

\usepackage{amsmath}
\usepackage{amssymb}
\usepackage{verbatim}
\usepackage[utf8]{inputenc}
\linespread{1.2}

\newtheorem{definition}{Definition}
\newtheorem{proposition}{Proposition}
\newtheorem{theorem}{Theorem}
\newtheorem{question}{Question}

\title{Math CS 122A HW4}
\author{Zih-Yu Hsieh}

\begin{document}
\maketitle

\section*{1}
\begin{myBox}[]{}
    \begin{question}
        Ahlfors Pg. 96 Problem 2:

        Map the region between $|z|=1$ and $|z-\frac{1}{2}|=\frac{1}{2}$ on a half plane.
    \end{question}
\end{myBox}

\textbf{Pf:}

Consider the following transformation $g:\mathbb{C}\cup\{\infty\}\rightarrow\mathbb{C}\cup\{\infty\}$:
$$f(z)=\frac{z+1}{z-1}\cdot \frac{-i-1}{-i+1},\quad g(z)=e^{\pi f(z)}$$

First, if consider the points $-i,-1,1$ respectively on $|z|=1$, linear transformation $f$ maps the following:
$$f(-i)=\frac{-i+1}{-i-1}\cdot\frac{-i-1}{-i+1}=1,\quad f(-1)=\frac{-1+1}{-1-1}\cdot\frac{-i=1}{-i+1}=0,\quad f(1)=\infty$$
(Note: Since $f(1)$ is not defined under $\mathbb{C}$, it gets map to $\infty$).

Because the orientation of $|z|=1$ is $-i$ to $-1$ to $1$, going clockwise, and the orientation of the image is $1$ to $0$ to $\infty$, 
which on the right side is the half plane with positive imaginary parts. Hence, the right of $|z|=1$ under this orientation (which is the interior of $|z|=1$)
gets mapped to the half plane $Im(z)>0$.

\hfill

Now, consider the points $\frac{1}{2}(1-i), 0, 1$ on $|z-\frac{1}{2}|=\frac{1}{2}$, linear transformation $f$ maps the following:
$$f\left(\frac{1}{2}(1-i)\right)=\frac{(\frac{1}{2}-\frac{1}{2}i)+1}{(\frac{1}{2}-\frac{1}{2}i)-1}\cdot\frac{-i-1}{-i+1} = \frac{(1-i)+2}{(1-i)-2}\cdot\frac{-i-1}{-i+1}=\frac{3-i}{-1-i}\cdot\frac{-1-i}{1-i}$$
$$=\frac{3-i}{1-i}=\frac{(3-i)(1+i)}{(1-i)(1+i)}=\frac{3+1-i+3i}{2}=\frac{4+2i}{2}=2+i$$
$$f(0)=\frac{1}{-1}\cdot\frac{-i-1}{-i+1}=-\frac{-(1+i)^2}{(1-i)(1+i)}=\frac{2i}{2}=i,\quad f(1)=\infty$$

So, since the three points gets mapped to $(2+i), i, \infty$ respectively, and linear transformation maps circle to circle,
hence this is a circle passing through $\infty$, or a straight line passing through $i$ and $(2+i)$, which is the line $Im(z)=1$.
Then, with the orientation $\frac{1}{2}(1-i)$ to $0$ to $1$, the image has orientation $(2+i)$ to $i$ to $\infty$, which the left side is the half plane $Im(z)<1$.
Hence, the left of $|z-\frac{1}{2}|=\frac{1}{2}$ under this orientation (the exterior of $|z-\frac{1}{2}|=1$) gets mapped to the half plane $Im(z)<1$.

\hfill

With the above statements, all points in the region between $|z|=1$ and $|z-\frac{1}{2}|=\frac{1}{2}$ are in the interior of $|z|=1$, and in the exterior of $|z-\frac{1}{2}|=\frac{1}{2}$.
So, they are the intersection of $Im(z)>0$ and $Im(z)<1$.

Which, $\pi f(z)$ represents the region $0<Im(z)<\pi$.

So, for all $z_0$ in the given open region, $z_0=a+bi$, where $a\in \mathbb{R}$, and $0<b<\pi$. So:
$$e^{z_0}=e^{a+bi}=e^a\cdot e^{ib},\quad b\in (0,\pi)$$
Hence, $e^{z_0}$ satisfies $\arg(e^{z_0})=b\in (0,\pi)$, and $|e^{z_0}|=e^a>0$, hence the image of the region $0<Im(z)<\pi$ is in the half plane $Im(z)>0$
(in fact, the image is the whole half plane, since the choice of $a\in \mathbb{R}$ and $b\in (0,\pi)$ are arbitrary, hence $e^a\in (0,\infty)$ could be any value in the given region).

\hfill

Eventually, since $\pi f(z)$ maps the region between $|z|=1$ and $|z-\frac{1}{2}|=\frac{1}{2}$ onto the region $0<Im(z)<1$,
while $e^z_0$ maps this new region onto the half plane $Im(z)>0$, then the composition $e^{\pi f(z)}$ maps the desired region to the half plane $Im(z)>0$.

\hfill

\begin{figure}[h!]
    \begin{center}
        \includegraphics*[width=150mm]{moon to half plane.jpg}
        \caption{Transformation between regions}
    \end{center}
\end{figure}

\break

\section*{2}
\begin{myBox}[]{}
    \begin{question}
        Ahlfors Pg. 97 Problem 5:

        Map the inside of the right-hand branch of the hyperbola $x^2-y^2=a^2$ on the disk $|w|<1$
        so that the focus corresopnds to $w=0$ and the vertex to $w=-1$.
    \end{question}
\end{myBox}

\textbf{Pf:}

WLOG, assume $a>0$ (Note: $a<0$ can be replaced with $(-a)$ instead). Under this configuration, the vertex is when $y=0$, or $x=a$ for the right hand branch (the vertex is $z=a$).
Also, the focus is given by $(ka,0)$ with $k=\sqrt{1+\frac{b'^2}{a'^2}}$ when given the hyperbola $\frac{x^2}{a'^2}-\frac{y^2}{b'^2}=1$,
which under this configuration, $a'=b'=a$, hence $k=\sqrt{2}$ (so the focus is $z=\sqrt{2}a$).

(Note 2: under the requirement, the focus and vertex needs to be two distinct points, hence $a\neq 0$).

\hfill

\textbf{Map of $z^2$:}

Notice that for all $z\in\mathbb{C}$, since $z=x+iy$ for some $x,y\in\mathbb{R}$, then $z^2=(x^2-y^2)+i\cdot 2xy$.

If take the plane $Re(z)>0$ (where $x>0$), the map is injective: Suppoze $z^2=z_1^2$ for $z,z_1\in\mathbb{C}$, 
then $z^2-z_1^2=(z-z_1)(z+z_1)=0$, hence $z=z_1$ or $z=-z_1$. However, if restrict onto the plane $Re(z)>0$, then $z=-z_1$ is impossible for all values on this plane, hence $z=z_1$, showing it's injective.

\hfill

Now, consider the inside of the right-hand branch of the hyperbola $x^2-y^2=a^2$, which is restricted by the condition $x^2-y^2\geq a^2$:
For all $z=x+iy$ in the given region, $x^2-y^2\geq a^2$; hence, $z^2=(x^2-y^2)+i\cdot 2xy$ is in the half plane $Re(w)\geq a^2$.
Also, for all $w$ in the half plane $Re(w)\geq a^2$ ($a^2>0$), since it is in the domain of $\sqrt{z}$ (which is in $\mathbb{C}\setminus\{x\in\mathbb{R}\ |\ x\leq 0\}$),
then there exists $z=x+iy$ with $z^2=w$, hence $Re(w)=Re(z^2)=(x^2-y^2)\geq a^2$, showing that $z$ is in the given region.

Hence, we can conclude that the function $z^2$ restricting onto the inside of the right-hand branch of the given hyperbola (with condition $x^2-y^2\geq a^2$),
it is an injective function mapping the region onto the half plane $Re(z)\geq a^2$.

\hfill

\textbf{Mapping the Half Plane $Re(z)\geq a^2$ onto the Unit Disk:}

Consider the following linear transformation:
$$f(w)=1-\frac{2a^2}{w}$$
For the points $w_0$ on the line $Re(w)=a^2$, $w_0=a^2+iv$ for some $v\in\mathbb{R}$, hence the following is true:
$$f(w_0)=1-\frac{2a^2}{w_0}=\frac{w_0-2a^2}{w_0}=\frac{(a^2+iv)-2a^2}{a^2+iv}=\frac{-a^2+iv}{a^2+iv} = \frac{-(a^2-iv)}{a^2+iv} = -\frac{\bar{w_0}}{w_0}$$
Hence, $|f(w_0)|=\left|-\frac{\bar{w_0}}{w_0}\right|=\frac{|\bar{w_0}|}{|w_0|}=1$, the boundary or the half plane gets mapped to the boundary of the unit disk $|w|<1$;

Also, for all points $w_1$ in the plane $Re(w)>a^2$ (let $w=u+iv$ for $u,v\in \mathbb{R}$, hence $u>a^2$), there are two cases to conside.
The following is what $w_1$ gets mapped to:
$$f(w_1)=1-\frac{2a^2}{w_1}=\frac{w_1-2a^2}{w_1}=\frac{(u-2a^2)+iv}{u+iv}$$

First, if $u\leq 2a^2$, notice that since $0\leq |u-2a^2| = (2a^2-u) < (2a^2-a^2)=a^2 < u$, then, $|(u-2a^2)+iv|=\sqrt{(u-2a^2)^2+v^2}< \sqrt{u^2+v^2} = |w_1|$,
hence $|f(w_1)|=\frac{|(u-2a^2)+iv|}{|u+iv|} < 1$.

Else, if $u>2a^2$, then since $0<(u-2a^2)<u$, then again $|(u-2a^2)+iv|=\sqrt{(u-2a^2)^2+v^2}< \sqrt{u^2+v^2} = |w_1|$, hence $|f(w_1)|=\frac{|(u-2a^2)+iv|}{|u+iv|} < 1$ is still true.

So, we can conclude that the half plane $Re(w)\geq a^2$ gets mapped to the unit disk $|w|=1$, and since this is a linear transformation, the map is bijective.

\hfill

\textbf{Mapping Inside of Hyperbola to Unit Disk:}

If Compose the two functions above, consider the following transformation $\bar{f}(z)=f(z^2)=1-\frac{2a^2}{z^2}$:
First, for all $z$ in the inside of the given branch of hyperboala (in the region $x^2-y^2\geq a^2$), $z^2$ appears in the half plane $Re(w)\geq a^2$, and there is a one-to-one correspondence between the two regions under the map;
furthermore, since $f$ maps the half plane $Re(w)\geq a^2$ to the unit disk $|w|\leq 1$, and is also a one-to-one correspondence, then the composition $f(z^2)$ maps the interior of the hyperbola to the unit disk.

Also, computing the following, we get:
$$\bar{f}(a)=1-\frac{2a^2}{a^2}=1-2=-1,\quad \bar{f}(\sqrt{2}a)=1-\frac{2a^2}{(\sqrt{2}a)^2}=1-\frac{2a^2}{2a^2}=1-1=0$$
Which, since given the right branch of hyperbola $x^2-y^2=a^2$, $z_0=a$ is the vertex and $z_1=\sqrt{2}a$ is the focus,
then the vertex gets mapped to $-1$, and the focus gets mapped to $0$, hence this conformal map satisfies the given condition.

\hfill

\begin{figure}[h!]
    \begin{center}
        \includegraphics*[width=140mm]{hyperbola to circle.jpg}
        \caption{Transformation between Hyperbola and Circle}
    \end{center}
\end{figure}

\break

\section*{3}
\begin{myBox}[]{}
    \begin{question}
        Ahlfors Pg. 78 Problem 4:

        Show that any linear transformation which transforms the real axis
        into itself can be written with real coefficient.
    \end{question}
\end{myBox}

\textbf{Pf:}

Let $S:\mathbb{C}\cup\{\infty\}\rightarrow \mathbb{C}\cup\{\infty\}$ be arbitrary linear transformation that transforms the real axis to itself, then if restricted onto $\mathbb{R}$, 
the image of the function is also the real axis. 

Notice that since the transformation is bijective,
there exists distinct points $z_1,z_2,z_3\in\mathbb{C}\cup\{\infty\}$, with $S(z_1)=1$, $S(z_2)=0$, and $S(z_3)=\infty$.
Which, this indicates that $z_1,z_2,z_3$ is in fact on $\mathbb{R}\cup\{\infty\}$:

Suppose there exists a point not on $\mathbb{R}\cup\{\infty\}$, then the circle (or straight line if one of them is $\infty$) determined by $z_1,z_2,z_3$ is not on $\mathbb{R}\cup\{\infty\}$;
yet, since the image of $z_1,z_2,z_3$ is on the straight line $\mathbb{R}\cup\{\infty\}$,
that means the circle deteined by $z_1,z_2,z_3$ is mapped onto $\mathbb{R}\cup\{\infty\}$, which contradicts the fact that the preimage
of the real axis should be the real axis, under the given condition.

\hfill

Hence, $z_1,z_2,z_3\in \mathbb{R}\cup\{\infty\}$. Then, based on the formula for cross ratio,
the unique transformation $S$ with $S(z_1)=1, S(z_2)=0$, and $S(z_3)=\infty$, has the following formula:
$$S(z)=\frac{z-z_2}{z-z_3}\cdot\frac{z_1-z_3}{z_1-z_2}$$
Hence, since $z_1,z_2,z_3\in \mathbb{R}\cup\{\infty\}$, then the above transformation can be simplified to real coefficients.

\hfill

\textbf{For all three points being real:}

$S(z)$ is in the given form above, where every coefficients are real.

\hfill

\textbf{For one points being $\infty$:}

If $z_1=\infty$:
$$S(z)=\lim_{z_1\rightarrow\infty}\frac{z-z_2}{z-z_3}\cdot\frac{z_1-z_3}{z_1-z_2}=\frac{z-z_2}{z-z_3}$$
If $z_2=\infty$:
$$S(z)=\lim_{z_2\rightarrow\infty}\frac{z-z_2}{z-z_3}\cdot\frac{z_1-z_3}{z_1-z_2}=\frac{z_1-z_3}{z-z_3}$$
Else if $z_3=\infty$:
$$S(z)=\lim_{z_3\rightarrow\infty}\frac{z-z_2}{z-z_3}\cdot\frac{z_1-z_3}{z_1-z_2}=\frac{z-z_2}{z_1-z_2}$$



\break

\section*{4}
\begin{myBox}[]{}
    \begin{question}
        Ahlors Pg. 80 Problem 3:

        If the consecutive vertices $z_1,z_2,z_3,z_4$ of a quadrilateral lie on a circle, prove that
        $$|z_1-z_3|\cdot|z_2-z_4|=|z_1-z_2|\cdot|z_3-z_4|+|z_2-z_3|\cdot|z_1-z_4|$$
        and interpret the result geometrically.
    \end{question}
\end{myBox}

\textbf{Pf:}

First, consider the right hand side of the equation:
$$|z_1-z_2|\cdot|z_3-z_4|+|z_2-z_3|\cdot|z_1-z_4| = |z_2-z_3|\cdot|z_1-z_4|\cdot\left(\left|\frac{(z_1-z_2)\cdot(z_3-z_4)}{(z_2-z_3)\cdot(z_1-z_4)}\right|+1\right)$$
Then, recall that the cross ratio of $(z_1,z_3,z_2,z_4)$ can be expressed as:
$$(z_1,z_3,z_2,z_4)=\frac{(z_1-z_2)\cdot(z_3-z_4)}{(z_1-z_4)\cdot(z_3-z_2)}$$
Hence, the above expression can be rewritten as:
$$|z_1-z_2|\cdot|z_3-z_4|+|z_2-z_3|\cdot|z_1-z_4| = |z_2-z_3|\cdot|z_1-z_4|\cdot\left(\left|-\frac{(z_1-z_2)\cdot(z_3-z_4)}{(z_3-z_2)\cdot(z_1-z_4)}\right|+1\right)$$
$$=|z_2-z_3|\cdot|z_1-z_4|\cdot(|-(z_1,z_3,z_2,z_4)|+1)$$

\hfill

Notice that since $z_1,z_2,z_3,z_4$ is consecutive vertices on a circle, then the cross ratio is real; 
furthermore, by the statement in \textbf{Question 6}, since $z_1,z_3,z_4$ and $z_2,z_3,z_4$ have the same orientation,
hence the cross ratio $(z_1,z_2,z_3,z_4)>0$.

\begin{figure}[h!]
    \begin{center}
        \includegraphics*[width=80mm]{cross ratio 1.jpg}
    \end{center}
    \caption{Cross Ratio of $(z_1,z_2,z_3,z_4)$}
\end{figure}

\hfill

Similarly, when viewing in order of $z_1,z_3,z_2,z_4$, the orientation $z_1,z_3,z_4$ and $z_3,z_2,z_4$ are different,
hence the cross ratio $(z_1,z_3,z_2,z_4)<0$. (In \textbf{Figure 4})

\begin{figure}[h!]
    \begin{center}
        \includegraphics*[width=80mm]{cross ratio 2.jpg}
    \end{center}
    \caption{Cross Ratio of $(z_1,z_3,z_2,z_4)$}
\end{figure}

\hfill

Then, since $(z_1,z_3,z_2,z_4)<0$, $-(z_1,z_3,z_2,z_4)>0$, hence $|-(z_1,z_3,z_2,z_4)|=-(z_1,z_3,z_2,z_4)$. The above identity becomes:
$$|z_1-z_2|\cdot|z_3-z_4|+|z_2-z_3|\cdot|z_1-z_4|=|z_2-z_3|\cdot|z_1-z_4|\cdot(|-(z_1,z_3,z_2,z_4)|+1)$$
$$|z_2-z_3|\cdot|z_1-z_4|\cdot|(-(z_1,z_3,z_2,z_4)+1)|$$

Compute the third term in the equation, we get:
$$-(z_1,z_3,z_2,z_4)+1=-\frac{(z_1-z_2)\cdot(z_3-z_4)}{(z_3-z_2)\cdot(z_1-z_4)}+1$$
$$=\frac{(z_3-z_2)(z_1-z_4)-(z_1-z_2)(z_3-z_4)}{(z_3-z_2)(z_1-z_4)}$$
$$=\frac{(z_1z_3-z_1z_2-z_3z_4+z_2z_4)-(z_1z_3-z_1z_4-z_2z_3+z_2z_4)}{(z_3-z_2)(z_1-z_4)}$$
$$=\frac{-z_1z_2-z_3z_4+z_1z_4+z_2z_3}{(z_3-z_2)(z_1-z_4)}=\frac{z_1(z_4-z_2)+z_3(z_2-z_4)}{(z_3-z_2)(z_1-z_4)}$$
$$=\frac{(z_3-z_1)(z_2-z_4)}{(z_3-z_2)(z_1-z_4)}$$

Hence, plug back into the original equation, we get:
$$|z_1-z_2|\cdot|z_3-z_4|+|z_2-z_3|\cdot|z_1-z_4|=|z_2-z_3|\cdot|z_1-z_4|\cdot|(-(z_1,z_3,z_2,z_4)+1)|$$
$$=|z_2-z_3|\cdot|z_1-z_4|\cdot\left|\frac{(z_3-z_1)(z_2-z_4)}{(z_3-z_2)(z_1-z_4)}\right|=\left|(z_3-z_1)(z_2-z_4)\right|$$
So, the original original formula is true:
$$|z_3-z_1|\cdot|z_2-z_4|=|z_1-z_2|\cdot|z_3-z_4|+|z_2-z_3|\cdot|z_1-z_4|$$


\break

\section*{5}
\begin{myBox}[]{}
    \begin{question}
        Ahlfors Pg. 83 Problem 4:

        Find the linear transformation which carries the circle $|z|=2$ into $|z+1|=1$,
        the point $-2$ into the origin, and the origin into $i$.
    \end{question}
\end{myBox}

\textbf{Pf:}

\textbf{Symmetric points:}

First, for circle $|z|=2$, since the origin $0$ is not on the circle, then to find a precise map, we also need to consider
its symmetric point, namely $\infty$. (Note: the symmetric point of the center of a circle is always $\infty$).

\hfill

Now, consider the points they get mapped to: Since any linear transformation should preserve the symmetric points,
then as $0$ gets mapped to $i$, $\infty$ gets mapped to the symmetric point of $i$ on the circle $|z+1|=1$.
The following is the computation based on the formula given in the textbook. Let $z_0=i$, radius $r=1$, and the center $a=-1$, then its symmetric point $z_0^*$ is given by:
$$z_0^*=\frac{r^2}{(\bar{z_0}-a)}+a = \frac{1}{-i-(-1)}-1 = \frac{1}{1-i}-1 = \frac{(1+i)}{(1-i)(1+i)}-1$$
$$=\frac{1+i}{2}-1 = -\frac{1}{2}+\frac{1}{2}i$$
Hence, under the desired linear transformation, $\infty$ gets mapped to $z_0^*=-\frac{1}{2}+\frac{1}{2}i$.

\hfill

\textbf{Formula for Linear Transformation:}

Given that $-2\mapsto 0$, $0\mapsto i$, and $\infty\mapsto (-\frac{1}{2}+\frac{1}{2}i)$, consider the following map:
$$f(z) = \frac{-(1-i)z-2(1-i)}{2z+2(1+i)}$$
Which, it maps the given point as follow:
$$f(-2)=\frac{-(1-i)(-2)-2(1-i)}{2(-2)+2(1+i)}=\frac{0\cdot(1-i)}{-4+2+2i}=0$$
$$f(0)=\frac{-(1-i)\cdot 0-2(1-i)}{2\cdot 0+2(1+i)}=\frac{-2(1-i)}{2(1+i)}=-\frac{(1-i)^2}{(1-i)(1+i)} = -\frac{1-1-2i}{1+1}=\frac{2i}{2}=i$$
$$f(\infty)=\lim_{z\rightarrow\infty}\frac{-(1-i)z-2(1-i)}{2z+2(1+i)} = \frac{-(1-i)}{2} = -\frac{1}{2}+\frac{1}{2}i$$
Hence, the given linear transformation maps the three points to the correct locations.

\hfill

\textbf{Circle Maps to Circle:}

To verify that $|z|=2$ gets mapped to $|z+1|=1$, it suffices to show that three points on $|z|=2$ get mapped onto $|z+1|=1$.

First, we already have $-2 \mapsto 0$, which is a point satisfying the condition.

Now, consider the point $2i, -2i$ on the circle $|z|=2$:
$$f(2i)=\frac{-(1-i)2i-2(1-i)}{2\cdot 2i+2(1+i)} = \frac{-2(1+i)(1-i)}{2+6i}=\frac{-2}{1+3i}=\frac{-2(1-3i)}{(1+3i)(1-3i)}=\frac{-2+6i}{10}=\frac{-1+3i}{5}$$
$$f(-2i)=\frac{-(1-i)(-2i)-2(1-i)}{2(-2i)+2(1+i)}=\frac{-2(1-i)(1-i)}{2-2i}=-(1-i)$$
Then, consider the distance $|f(2i)+1|$ and $|f(-2i)+1|$, we get:
$$|f(2i)+1|=\left|\frac{-1+3i}{5}+1\right|=\left|\frac{4+3i}{5}\right|=\sqrt{\left(\frac{4}{5}\right)^2+\left(\frac{3}{5}\right)^2}=1$$
$$|f(-2i)+1|=|-(1-i)+1|=|i|=1$$
Hence, $f(2i),f(-2i)$ are two points on the circle $|z+1|=1$.

Since $-2,2i,-2i$ are three points on the circle $|z|=2$, and they get mapped to points on $|z+1|=1$ by the linear transformation $f$,
hence $|z|=2$ is mapped to $|z+1|=1$, showing that $f$ is in fact the desired linear transformation.

\hfill

\hfill

\section*{6}
\begin{myBox}[]{}
    \begin{question}
        Ahlfors Pg. 84 Problem 1:

        If $z_1,z_2,z_3,z_4$ are points on a circle, show that $z_1,z_3,z_4$ and $z_2,z_3,z_4$ determine the same orientation if and only If
        $(z_1,z_2,z_3,z_4)>0$.
    \end{question}
\end{myBox}

\textbf{Pf:}

Given four distinct points $z_1,z_2,z_3,z_4\in\mathbb{C}$, to determine the cross ratio $(z_1,z_2,z_3,z_4)$, it is given by the linear transformation
that gives $z_2\mapsto 1$, $z_3\mapsto 0$, and $z_4\mapsto \infty$.

If consider the orientation as $z_2,z_3,z_4$ respectively:

If $z_1,z_3,z_4$ has the same orientation as above, then $z_1,z_2$ needs to be on the same arc when the circle is separated by $z_3$ and $z_4$.

Which, this happens if the linear transformation would transform $z_1,z_2$ onto the same side of the real line, so $z_1$ gets mapped to a positive value.
Hence, $(z_1,z_2,z_3,z_4)>0$.

\hfill

Conversely, if $(z_1,z_2,z_3,z_4)>0$, then $z_1$ gets mapped to a positive value on the real axis.
Which, since the orienation is given by $z_2,z_3,z_4$ in order, and $z_1,z_2$ both get mapped to positive values while $z_3$ gets mapped to $0$,
hence $z_1,z_2$ must be on the same side of the circle when the circle is separated by $z_3,z_4$, the orientation $z_1,z_3,z_4$ must have the same orientation as $z_2,z_3,z_4$.

\hfill

\begin{figure}[h!]
    \begin{center}
        \includegraphics*[width=100mm]{Cross Ratio and Orientation.jpg}
        \caption{Cross Ratio and Orientation}
    \end{center}
\end{figure}

\break

\section*{7}
\begin{myBox}[]{}
    \begin{question}
       Ahlfors Pg. 88 Problem 6:

       Find all circles which are orthogonal to $|z|=1$ and $|z-1|=4$.
    \end{question}
\end{myBox}

\textbf{Pf:}

Given the two circles $|z|=1$ and $|z-1|=4$, notice that the relationship is similar to the Circle of Apollonius 
(which there exists a fixed ratio for any circle, such that the distance from any points on the circle to some two points $k,k_0$ always form that fixed ratio).
Then, the two limit points $k,k_0$ of the given circles, every circle in the system has a center collinear to the two limit points.
Hence, with both circles $|z|=1$ and $|z-1|=4$ that have center $0$ and $1$, we can conclude that the limit points both lie on the real axis ($k,k_0\in\mathbb{R}\cup\{\infty\}$).

\hfill

Also, the two limit points are in fact symmetric points under any circle in the given system, so they must satisfy the following relation:
$$r_1=1,\quad c_1=0,\quad k_0=\frac{r_1^2}{\bar{k}-\bar{c_1}}+c_1 = \frac{1}{k}$$
$$r_2=4,\quad c_2=1,\quad k_0=\frac{r_2^2}{\bar{k}-\bar{c_2}}+c_2=\frac{16}{k-1}+1$$
(Note: the above equations are based on the relation of symmetric points with respect to each circle, where $r$ is the radius and $c$ is the center; and, since $k\in\mathbb{R}\cup\{\infty\}$, can assume $\bar{k}=k$).

Hence, we can deduce the following:
$$\frac{1}{k}=\frac{16}{k-1}+1,\quad (k-1)=16k+k(k-1),\quad k^2+14k+1=0$$
$$k=-7\pm 4\sqrt{3}$$
Which, take $k=-7+4\sqrt{3}$, $k_0=\frac{1}{k}=-7-4\sqrt{3}$, so the two points satisfy the given condition.

(Note: in \textbf{HW 2, Ahlfors Pg. 33 Problem 4}, if the modulus of a circle after a rational function is constant,
then the pole $w^*$ and the zero $w$ satisfy the relation $w^*=\frac{1}{\bar{w}}$, the two points are the symmetric points under linear transformation.
Conversely, we can also conclude that since the given circles have $k,k_0$ being the symmetric points, the distance from the circle to the two points form a constrant ratio).

\hfill

Now, if we consider the linear transformation $w=\frac{z-k}{z-k_0}$, based on the previous logic, 
the circles $|z|=1$ and $|z-1|=4$ get mapped to concentric circles around origin (since every point on the same circle has a fixed ratio between distances from $k$ and $k_0$).
Hence, every circle that's orthogonal to both circles, must get mapped to circles that are orthogonal to the concentric circles through the origin in the image.

This could happen only if the image of the circle is a straight line through the origin, which is parametrized by $z=w_0t$ for nonzero number $w_0\in\mathbb{C}$, and $t\in\mathbb{R}$.

To reconstruct the preimage, consider the inverse of the above linear transformation, given by $z=\frac{k_0w-k}{w-1}$.



\end{document}