%Math_CS_122A_HW3_Zih-Yu_Hsieh.tex

\documentclass{article}
\usepackage{graphicx} % Required for inserting images
\usepackage[margin = 2.54cm]{geometry}
\usepackage[most]{tcolorbox}

\newtcolorbox{myBox}[3]{
arc=5mm,
lower separated=false,
fonttitle=\bfseries,
%colbacktitle=green!10,
%coltitle=green!50!black,
enhanced,
attach boxed title to top left={xshift=0.5cm,
        yshift=-2mm},
colframe=blue!50!black,
colback=blue!10
}

\usepackage{amsmath}
\usepackage{amssymb}
\usepackage{verbatim}
\usepackage[utf8]{inputenc}
%\linespread{1.5}

\newtheorem{definition}{Definition}
\newtheorem{proposition}{Proposition}
\newtheorem{theorem}{Theorem}
\newtheorem{question}{Question}

\title{Math CS 122A HW3}
\author{Zih-Yu Hsieh}

\begin{document}
\maketitle

\section*{1}
\begin{myBox}[]{}
    \begin{question}
        Ahlfors Pg. 44 Problem 2
    \end{question}
\end{myBox}

\textbf{Pf:}

\textbf{Expression of $\sinh, \cosh$:}

Given that $\cosh(z)=\frac{e^z+e^{-z}}{2}$ and $\sinh(z)=\frac{e^z-e^{-z}}{2}$. Then, given that $\cos(z)=\frac{e^{iz}+e^{-iz}}{2}$ and $\sin(z)=\frac{e^{iz}-e^{-iz}}{2i}$,
the following identities are true:
$$\cos(iz)=\frac{e^{i(iz)}+e^{-i(iz)}}{2}=\frac{e^{-z}+e^{z}}{2} = \cosh(z)$$ 
$$\sin(iz)=\frac{e^{i(iz)}-e^{-i(iz)}}{2i} =-i\frac{e^{-z}-e^{z}}{2} = i\left(\frac{e^z-e^{-z}}{2}\right) = i\sinh(z)$$
Thus, $\cosh(z)=\cos(iz)$, and $\sinh(z)=\frac{1}{i}\sin(iz) = -i\sin(iz)$.

\hfill

\textbf{Addition Formula:}

Then, according to the original trigonometry addition formulas, for all $a,b\in\mathbb{C}$, the following is true:
$$\sinh(a+b)=-i\sin(i(a+b)) = -i\sin(ia+ib) = -i(\sin(ia)\cos(ib)+\sin(ib)\cos(ia))$$
$$= (-i\sin(ia))\cos(ib) + (-i\sin(ib))\cos(ia) = \sinh(a)\cosh(b) + \sinh(b)\cosh(a)$$

$$\cosh(a+b)=\cos(i(a+b))=\cos(ia+ib) = \cos(ia)\cos(ib)-\sin(ia)\sin(ib)$$
$$= \cosh(a)\cosh(b) + (-i\sin(ia))(-i\sin(ib)) = \cosh(a)\cosh(b)+\sinh(a)\sinh(b)$$
Thus, the addition formula is given as:
$$\sinh(a+b)=\sinh(a)\cosh(b)+\sinh(b)\cosh(a),\quad \cosh(a+b)=\cosh(a)\cosh(b)+\sinh(a)\sinh(b)$$

\hfill

\textbf{Double Angle Formula:}
With the above formulas, for all $z\in\mathbb{C}$, $\sinh(2z),\cosh(2z)$ can be given as:
$$\sinh(2z) = \sinh(z+z)=\sinh(z)\cosh(z)+\sinh(z)\cosh(z)=2\sinh(z)\cosh(z)$$

$$\cosh(2z)=\cosh(z+z)=\cosh(z)\cosh(z)+\sinh(z)\sinh(z)=\cosh(z)^2+\sinh(z)^2$$

\break

\section*{2}
\begin{myBox}[]{}
    \begin{question}
        Ahlfors Pg. 47 Problem 6
    \end{question}
\end{myBox}

\textbf{Pf:}

\textbf{Case for $2^i$:}

Consider $2^i = e^{i\cdot \log(2)}$, which $\log(2) = \{\ln(2) + i(\arg(2)+k\cdot 2\pi)\ |\ k\in\mathbb{Z}\}$ (where $\arg(2)=0$, since $2\in \mathbb{R}$). 
Thus:
$$2^i=e^{i\cdot \log(2)} = e^{i(\ln(2)+ik\cdot 2\pi)} = e^{-k\cdot 2\pi + i\ln(2)} = e^{-k\cdot 2\pi}\cdot e^{i\ln(2)}$$
So, $2^i = \{e^{-k\cdot 2\pi} \cdot e^{iln(2)}\ |\ k\in\mathbb{Z}\}$.

\hfill

\textbf{Case for $i^i$:}

Consider $i^i = e^{i\cdot \log(i)}$, which $\log(i)=\{\ln|i|+i(\arg(i)+k\cdot 2\pi)\ |\ k\in\mathbb{Z}\}$ (where $\ln|i| = \ln(1)=0$, and $\arg(i)=\frac{\pi}{2}$).
Thus:
$$i^i=e^{i\cdot \log(i)} = e^{i\cdot i(\frac{\pi}{2}+k\cdot 2\pi)} = e^{-\frac{\pi}{2}-k\cdot 2\pi}$$
So, $i^i = \{e^{-\frac{\pi}{2}-k\cdot 2\pi}\ |\ k\in\mathbb{Z}\}$.

\hfill

\textbf{Case for $(-1)^{2i}$:}

Consider $(-1)^{2i} = e^{2i\cdot \log(-1)}$, which $\log(-1)=\{\ln|-1|+i(\arg(-1)+k\cdot 2\pi)\ |\ k\in\mathbb{Z}\}$ (where $\ln|-1|=\ln(1)=0$, and $\arg(-1)=\pi$). 
Thus:
$$(-1)^{2i}=e^{2i\cdot \log(-1)}=e^{2i\cdot i(\pi+k\cdot 2\pi)} = e^{-2(2k+1)\pi} = e^{-(4k+2)\pi}$$
So, $(-1)^{2i}=\{e^{-(4k+2)\pi}\ |\ k\in\mathbb{Z}\}$.

\hfill

\hfill

\section*{3}
\begin{myBox}[]{}
    \begin{question}
        Ahlfors Pg. 72 Problem 1
    \end{question}
\end{myBox}

\textbf{Pf:}

Define the region $\Omega = \mathbb{C}\setminus(-\infty,-1]\cup [1,\infty)$ (region excluding real numbers except for ones in between $(-1,1)$).
Which, since $\sqrt{z}$ is defined as a single-valued function on $\mathbb{C}\setminus(-\infty,0]$, by 
$\sqrt{z} = \sqrt{|z|}(\cos(\arg(z)/2)+i\sin(\arg(z)/2))$ (which, $\arg(z)\in (-\pi,\pi)$). 

\hfill

Then, given the function $f(z)=\sqrt{1+z}+\sqrt{1-z}$, for all $z\in \Omega$, since $z\notin (-\infty,-1]$, then $(1+z)\notin (-\infty, 0]$, thus $\sqrt{1+z}$ is well-defined;
similarly, since $z\notin [1,\infty)$, thus $-z \notin (-\infty,-1]$, or $(1-z)\notin (-\infty,0]$. Hence, $\sqrt{1-z}$ is also well-defined.

Now, with the function $f(z)$ being well-defined on $\Omega$ an open subset, based on the definition of square root above, the following is true:
$$\forall z\in\Omega,\quad f(z)=\sqrt{1+z}+\sqrt{1-z}$$
$$ = \sqrt{|1+z|}\left(\cos\left(\frac{\arg(1+z)}{2}\right)+i\sin\left(\frac{\arg(1+z)}{2}\right)\right)+\sqrt{|1-z|}\left(\cos\left(\frac{\arg(1-z)}{2}\right)+i\sin\left(\frac{\arg(1-z)}{2}\right)\right)$$

\hfill

Then, since $\sqrt{z}$ is analytic, and the polynomial $(1+z),(1-z)$ are both analytic, because the composition of analytic functions are analytic,
hence $\sqrt{1+z}$ and $\sqrt{1-z}$ are analytic; and since the sum of analytic function is again analytic, then $f(z)=\sqrt{1+z}+\sqrt{1-z}$ is again analytic.


\end{document}