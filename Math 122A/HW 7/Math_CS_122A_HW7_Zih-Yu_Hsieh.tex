% Math_CS_122A_HW7_Zih-Yu_Hsieh.tex

\documentclass{article}
\usepackage{graphicx} % Required for inserting images
\usepackage[margin = 2.54cm]{geometry}
\usepackage[most]{tcolorbox}

\newtcolorbox{myBox}[3]{
arc=5mm,
lower separated=false,
fonttitle=\bfseries,
%colbacktitle=green!10,
%coltitle=green!50!black,
enhanced,
attach boxed title to top left={xshift=0.5cm,
        yshift=-2mm},
colframe=blue!50!black,
colback=blue!10
}

\usepackage{amsmath}
\usepackage{amssymb}
\usepackage{verbatim}
\usepackage[utf8]{inputenc}
\linespread{1.2}

\newtheorem{definition}{Definition}
\newtheorem{proposition}{Proposition}
\newtheorem{theorem}{Theorem}
\newtheorem{question}{Question}

\title{Math CS 122A HW7}
\author{Zih-Yu Hsieh}

\begin{document}
\maketitle

\section*{1}
\begin{myBox}[]{}
    \begin{question}
        Apply the representation $f(z) = w_0+\zeta(z)^n$ to $\cos(z)$ with $z_0=0$.
        Determine $\zeta(z)$ explicitly.
    \end{question}
\end{myBox}

\textbf{Pf:}

Given $z_0=0$, which $\cos(z_0)=\cos(0)=1$. So, since $\cos(z)-1$ has zero ath $z=0$, and it is not identically $0$, there exists some order $n\in\mathbb{N}$ and analytic function $g(z)$ such that $g(0)\neq 0$,
with $\cos(z)-1 = (z-0)^ng(z) = z^ng(z)$.

Now, consider the derivatives of $\cos(z)-1$, and their evaluation at $z_0=0$:
$$\frac{d}{dz}(\cos(z)-1)=-\sin(z),\quad -\sin(0)=0$$
$$\frac{d}{dz}(-\sin(z))=-\cos(z),\quad -\cos(0)=1\neq 0$$
Notice that this implies the order of the zero is $2$, hence $\cos(z)-1 = z^2g(z)$, where the goal is to find the analytic branch $\zeta(z)$ such that $\cos(z)-1 = \zeta(z)^2$.

\hfill

Now, notice that $\cos(z)=\frac{e^{iz}+e^{-iz}}{2}$, hence: 
$$\cos(z)-1 = \frac{e^{iz}+e^{-iz}}{2} -1 = \frac{e^{iz}+e^{-zi}-2}{2} = \frac{(e^\frac{iz}{2}-e^{\frac{-iz}{2}})^2}{2}$$
So, define the branch $\zeta(z) = \frac{e^\frac{iz}{2}-e^{\frac{-iz}{2}}}{\sqrt{2}}$ would satisfy $\zeta(z)^2 = \cos(z)-1$, its negative representation is also fitting the desired condition.



\break

\section*{2}
\begin{myBox}[]{}
    \begin{question}
        Show by use of (36), or directly, that $|f(z)|\leq 1$ for $|z|\leq 1$ implies
        $$\frac{|f'(z)|}{(1-|f(z)|^2)}\leq \frac{1}{1-|z|^2}$$
    \end{question}
\end{myBox}

\textbf{Pf:}

In the textbook, given an analytic function $f$, with $w_0=f(z_0)$ with $|z_0|<R$ and $|w_0|<M$ for some $R,M>0$,
then for $|z|<R$, we have the following inequality:
$$\left|\frac{M(f(z)-w_0)}{M^2-\bar{w_0}f(z)}\right|\leq \left|\frac{R(z-z_0)}{R^2-\bar{z_0}z}\right|$$

\hfill

In the problem, the statement has $R=M=1$, hence fixing any $z_0$ with $|z_0|<1$, for all $z\neq z_0$ with $|z|<1$, it satisfies the following:
$$\left|\frac{f(z)-f(z_0)}{1-\overline{f(z_0)}f(z)}\right|\leq \left|\frac{z-z_0}{1-\bar{z_0}z}\right|$$
(Note: the original equation has $w_0=f(z_0)$).

Which, since $z\neq z_0$, $|z-z_0|\neq 0$. Modify the equation, we get:
$$\left|\frac{f(z)-f(z_0)}{z-z_0}\right|\cdot\left|\frac{1}{1-\overline{f(z_0)}f(z)}\right|\leq \frac{1}{|1-\bar{z_0}z|}$$
Now, notice that $\lim_{z\rightarrow z_0}\frac{f(z)-f(z_0)}{z-z_0} = f'(z_0)$, $\lim_{z\rightarrow z_0} (1-\overline{f(z_0)}f(z)) = (1-\overline{f(z_0)}f(z_0)) = (1-|f(z_0)|^2)$, and $\lim_{n\rightarrow\infty}(1-\bar{z_0}z) = (1-\bar{z_0}z_0) = (1-|z_0|^2)$.

Which, for $|z_0|<1$, then $(1-|z_0|^2) > 0$; similarly, by Maximal Principle, since $|z_0|<1$ $z_0$ is not on the boundary of the unit disk $\mathbb{D}$.
Hence, $|f(z_0)|$ cannot be the maximum, showing that $|f(z_0)|<1$ (since if $|f(z_0)|=1$, because it is not the maximum, there exists $z_1\in \mathbb{D}$, with $1=|f(z_0)| <|f(z_1)|$, 
which contradicts the fact that $|f(z_1)|\leq 1$ when $|z_1|\leq 1$). Hence, $(1-|f(z_0)|^2) > 0$.

So, the above inequality can be reduce to the following:
$$\lim_{z\rightarrow z_0}\left|\frac{f(z)-f(z_0)}{z-z_0}\right|\cdot\left|\frac{1}{1-\overline{f(z_0)}f(z)}\right|\leq \lim_{z\rightarrow z_0}\frac{1}{|1-\bar{z_0}z|}$$
$$|f'(z_0)|\cdot \frac{1}{|1-|f(z_0)|^2|}\leq \frac{1}{|1-|z_0|^2|}$$
Hence, for $|z|<1$, we can conclude the following:
$$\frac{|f'(z)|}{1-|f(z)|^2}\leq \frac{1}{1-|z|^2}$$

\break

\section*{3}
\begin{myBox}[]{}
    \begin{question}
        Prove that the arc of smallest noneuclidean length that joins two
        given points in the unit disk is a circular arc which is orthogonal to the unit
        circle. (Make use of a linear transformation that carries one end point
        to the origin, the other to a point on the positive real axis.)
        The shortest noneuclidean length is called the noneuclidean distance
        between the end points. Derive a formula for the noneuclidean distance
        between $z_1$ and $z_2$. Answer:
        $$\frac{1}{2}\log\frac{1+\left|\frac{z_1-z_2}{1-\bar{z_1}z_2}\right|}{1-\left|\frac{z_1-z_2}{1-\bar{z_1}z_2}\right|}$$
    \end{question}
\end{myBox}

\textbf{Pf:}

For any distinct $z_1,z_2\in\mathbb{D}$ (which $|z_1|,|z_2|<1$),consider the map $S(z)=\frac{z_1-z}{1-\bar{z_1}z}$:
Recall that in \textbf{HW 1 Question 4 Part (b)}, we've proven that given $|w|<1$, the map $T(z)=\frac{w-z}{1-\bar{w}z}$ is in fact a bijection of the unit disk $\mathbb{D}$,
hence the map $S(z)$ here is also a bijection of the unit disk, and specifically $S(z_1)=0$.

\hfill

Now, since $S$ is a linear transformation, the noneuclidean distance is preserved (based on a statement given in the previous problem in the textbook);
and, by the Maximum Principle, since the maximum of the function's modulus could only appear at the boundary, hence since $|z_2|<1$ (not on the boundary of $\mathbb{D}$),
then $|f(z_2)|<1$ (since $S$ is a bijection on $\mathbb{D}$, hence the maximum is given by $\max|S(z)|=1$).

So, we can conclude that the Noneuclidean distance is achieved by some path $\gamma$ connecting the origin and the point $S(z_2)\in\mathbb{D}$ (because the noneuclidean distance of a path is invariant under linear transformation, 
therefore it is sufficient to find such path after the transformation).

\hfill

\textbf{The Path $\gamma$ is a Straight Line:}

WLOG, let $\gamma:[0,1]\rightarrow\mathbb{D}$ be a differentiable path, such that $\gamma(0)=0$, and $\gamma(1)=S(z_2)$. 
Which, at every input, $\gamma(t)=r(t)e^{i\theta(t)}$ for some real-valued differentiable function $r(t)$ and $\theta(t)$ (Which, one can assume that $1>r(t)\geq 0$ for all $t\in [0,1]$ to fit $\gamma$ in the unit disk $\mathbb{D}$).
Also, it satisfies $r(0)=0$, and $r(1)=|S(z_2)|$.

Then, we get the following:
$$|\gamma(t)| = r(t),\quad \gamma'(t) = r'(t)e^{i\theta(t)}  + i\theta'(t)r(t)e^{i\theta(t)}=(r'(t)+i\theta'(t)r(t))e^{i\theta(t)}$$
$$|\gamma'(t)| = |r'(t)+i\theta'(t)r(t)| = \sqrt{(r'(t))^2+(\theta'(t)r(t))^2}$$
Which, consider the noneuclidean distance, it is given as follow:
$$\int_{\gamma}\frac{|dz|}{1-|z|^2} = \int_{0}^{1}\frac{|\gamma'(t)|}{1-|\gamma(t)|^2}dt = \int_{0}^{1}\frac{\sqrt{(r'(t))^2+(\theta'(t)r(t))^2}}{1-(r(t))^2}dt$$
And, since $\sqrt{(r'(t))^2+(\theta'(t)r(t))^2}\geq \sqrt{(r'(t))^2} = |r'(t)|\geq r'(t)$, the above integral satisfy the inequality:
$$\int_{0}^{1}\frac{\sqrt{(r'(t))^2+(\theta'(t)r(t))^2}}{1-(r(t))^2}dt \geq \int_{0}^{1}\frac{r'(t)}{1-(r(t))^2}dt$$
So, doing the substitution $u=r(t)$, $du = r'(t)dt$ (which $t=0$ satisfies $u=r(0)=0$, and $t=1$ satisfies $u=r(1)=|S(z_2)|$), we get the following:
$$\int_{\gamma}\frac{|dz|}{1-|z|^2}\geq \int_{0}^{|S(z_2)|}\frac{1}{1-u^2}du = \frac{1}{2}\int_{0}^{|S(z_2)|}\left(\frac{1}{1-u}+\frac{1}{1+u}\right)du$$
$$\int_{\gamma}\frac{|dz|}{1-|z|^2}\geq \frac{1}{2}\left(-\ln|1-u|+\ln|1+u|\right)\bigg |_{0}^{|S(z_2)|} = \frac{1}{2}\ln\left|\frac{1+u}{1-u}\right|\bigg |_{0}^{|S(z_2)|} = \frac{1}{2}\ln\left|\frac{1+|S(z_2)|}{1-|S(z_2)|}\right|$$

With $|S(z_2)| = \left|\frac{z_1-z_2}{1-\bar{z_1}z_2}\right| <1$, then $\frac{1+|S(z_2)|}{1-|S(z_2)|}>0$, the above inequality becomes:
$$\int_{\gamma}\frac{|dz|}{1-|z|^2}\geq \frac{1}{2}\ln\frac{1+\left|\frac{z_1-z_2}{1-\bar{z_1}z_2}\right|}{1-\left|\frac{z_1-z_2}{1-\bar{z_1}z_2}\right|}$$

Notice that for the straight path $\gamma(t)=S(z_2)t$ ($\gamma'(t)=S(z_2)$), the path integral produces the above value,
so we can claim that the shortest distance is given as the above value.

\hfill

Hence, we can claim that the smallest noneuclidean distance between two points, is given as:
$$\frac{1}{2}\ln\frac{1+\left|\frac{z_1-z_2}{1-\bar{z_1}z_2}\right|}{1-\left|\frac{z_1-z_2}{1-\bar{z_1}z_2}\right|}$$
Also, because this minimum is achieved by having a straight path $\gamma$ going through $0$ and $S(z_2)$ (which is a straight line through the center, and it is orthogonal to the unit circle).
Hence, the preimage $S^{-1}(\gamma)$ must be a circle that's orthogonal to the preimage of the unit circle, which is the unit circle itself (since the Mobius Transoformation $S$ with $|z_1|<1$ maps the boundary onto the boundary).

With $S(z_1)=0$ and $S(z_2)=S(z_2)$, $z_1$ and $z_2$ are both on the preimage $S^{-1}(\gamma)$, so the shortest noneuclidean path joining the two arbitrary points in $\mathbb{D}$,
is a circular arce orthogonal to the unit circe.

\end{document}