%name: Math_CS_122A_HW1_Zih-Yu_Hsieh.tex

\documentclass{article}
\usepackage{graphicx} % Required for inserting images
\usepackage[margin = 2.54cm]{geometry}

\usepackage{amsmath}
\usepackage{amssymb}
\usepackage{verbatim}
\usepackage[utf8]{inputenc}
\linespread{1.5}

\newtheorem{definition}{Definition}
\newtheorem{proposition}{Proposition}
\newtheorem{theorem}{Theorem}
\newtheorem{question}{Question}

\title{Math CS 122A HW1}
\author{Zih-Yu Hsieh}

\begin{document}
\maketitle

\subsection*{1}
\begin{question}
    \textbf{Ahlfors Pg. 9 Problem 5:
    Prove Lagrange's Identity in the complex form:
    $$\left|\sum_{i=1}^{n}a_ib_i\right|^2 = \sum_{i=1}^{n}|a_i|^2\sum_{i=1}^{n}|b_i|^2-\sum_{i\leq i<j\leq n}|a_i\overline{b_j}-a_j\overline{b_i}|^2$$}
\end{question}

\hfill

\textbf{Pf:}

\break

\subsection*{2}
\begin{question}
    \textbf{Ahlfors Pg. 11 Problem 1:
    Prove that:
    $$\left|\frac{a-b}{1-\bar{a}b}\right| < 1$$
    if $|a|<1$ and $|b|<1$.
    }
\end{question}

\hfill

\textbf{Pf:}

Suppose $a,b\in\mathbb{C}$ satisfy $|a|,|b| <1$. Then, since $|\bar{a}b| = |\bar{a}|\cdot |b| = |a|\cdot |b| <1$,
$|\bar{a}b| \neq |1| = 1$, then $\bar{a}b \neq 1$.
So, $1-\bar{a}b \neq 0$, the term $\frac{a-b}{1-\bar{a}b}$ is defined.

Now, consider the following identity:
$$\forall x,y\in \mathbb{C},\quad |x-y|^2 = |x|^2+|y|^2 - 2 Re(x\bar{y})$$
So, the following equations are true:
$$|a-b|^2 = |a|^2+|b|^2-2Re(a\bar{b})$$
$$|1-\bar{a}b|^2 = |1|^2+|\bar{a}b|^2-2Re(\bar{a}b)$$
Which, $|\bar{a}b| = |\bar{a}|\cdot|b| = |a|\cdot|b|$, and $Re(\bar{a}b) = \frac{\bar{a}b + \overline{\bar{a}b}}{2} = \frac{\bar{a}b + a\bar{b}}{2} = Re(a\bar{b})$,
so the equation can be simplified to:
$$|1-\bar{a}b|^2 = 1+|a|^2\cdot|b|^2-2Re(a\bar{b})$$

Then, consider the term $(1+|a|^2\cdot|b|^2)-(|a|^2+|b|^2)$:
$$(1+|a|^2\cdot|b|^2)-(|a|^2+|b|^2) = (1-|b|^2)+|a|^2(|b|^2-1)$$
$$ = (1-|b|^2)-|a|^2(1-|b|^2)=(1-|a|^2)(1-|b|^2)$$
Since both $|a|,|b|<1$, then $|a|^2,|b|^2 <1$, which $(1-|a|^2),(1-|b|^2)>0$. 
Hence, we can conclude that $(1+|a|^2\cdot|b|^2)-(|a|^2+|b|^2) = (1-|a|^2)(1-|b|^2) >0$, which: 
$$(1+|a|^2\cdot|b|^2) > (|a|^2+|b|^2),\quad (1+|a|^2\cdot|b|^2-2Re(a\bar{b})) > (|a|^2+|b|^2-2Re(a\bar{b}))$$
Replace the terms with the original form of absolute value, we get:
$$|1-\bar{a}b|^2 > |a-b|^2$$
Because $(1-\bar{a}b)\neq 0$, then $|1-\bar{a}b|^2 >0$. So:
$$1 > \frac{|a-b|^2}{|1-\bar{a}b|^2} = \left|\frac{a-b}{1-\bar{a}b}\right|^2,\quad \left|\frac{a-b}{1-\bar{a}b}\right| <1$$
Which the inequality is true.

\break

\subsection*{3}
\begin{question}
    \textbf{Ahlfors Pg. 11 Problem 3:
    If $|a_i|<1,\ \lambda_i \geq 0$ for all $i = 1,...,n$, and $\lambda_1 + ...+\lambda_n=1$, show that:
    $$|\lambda_1a_1 + ... + \lambda_na_n| <1$$
    }
\end{question}

\textbf{Pf:}

Given that $|a_i|<1,\ \lambda_i \geq 0$ for all $i = 1,...,n$, and $\lambda_1 + ...+\lambda_n=1$. By the Triangle Inequality:
$$|\lambda_1a_1 + ... + \lambda_na_n| \leq |\lambda_1a_1|+...+|\lambda_na_n| = \lambda_1|a_1|+...+\lambda_n|a_n|$$
(Note: above is true since each coefficient $\lambda_i \geq 0$).
Then, let $M = \max\{|a_1|,...,|a_n|\}$, which for all $i\in\{1,...,n\}$, $|a_i|\leq M$; and since $|a_i| <1$ for all index $i$, $M <1$.
Thus, the following is true:
$$\lambda_1|a_1|+...+\lambda_n|a_n| \leq \lambda_1\cdot M + ... +\lambda_n\cdot M = M(\lambda_1+...+\lambda_n) = M$$
Which, combining all the inequalities above, we get:
$$|\lambda_1a_1 + ... + \lambda_na_n| \leq \lambda_1|a_1|+...+\lambda_n|a_n| \leq M <1$$
$$|\lambda_1a_1 + ... + \lambda_na_n| <1$$
Which, the given inequality is true.

\break

\subsection*{4}
\begin{question}
    \textbf{Ahlfors Pg. 16 Problem 4:
    If $w = \cos(\frac{2\pi}{n})+i\sin(\frac{2\pi}{n})$ for some positive integer $n$, prove that:
    $$1+w^h+...+w^{(n-1)h}=0$$
    For any integer $h$ which is not a multiple of $n$.
    }
\end{question}

\textbf{Pf:}
Given that $w = \cos(\frac{2\pi}{n})+i\sin(\frac{2\pi}{n})$ and $h\in\mathbb{Z}$ is not a multiple of $n$. 
Then, the term $\frac{h}{n}$ is not an integer, which consider $w^h$:
$$w^h = \cos\left(\frac{h\cdot 2\pi}{n}\right)+i\sin\left(\frac{h\cdot 2\pi}{n}\right) \neq 1$$
Since $\frac{h}{n}$ is not an integer, $\frac{h\cdot 2\pi}{n}$ is not an integer multiple of $2\pi$, thus $\cos\left(\frac{h\cdot 2\pi}{n}\right)\neq 1$, or $w^h \neq 1$.

Hence, $(1-w^h) \neq 0$, division with this number is defined. Now, consider the following:
$$1+w^h+...+w^{(n-1)h} = \frac{(1-w^h)(1+w^h+...+w^{(n-1)h})}{(1-w^h)} = \frac{1-w^{nh}}{1-w^h}$$
Which, since $w^n = \cos(\frac{n\cdot 2\pi}{n})+i\sin(\frac{n\cdot 2\pi}{n}) = \cos(2\pi)+i\sin(2\pi) = 1$, then $w^{nh} = (w^n)^h = 1^h = 1$. Thus:
$$1+w^h+...+w^{(n-1)h} = \frac{1-w^{nh}}{1-w^h} = \frac{1-1}{1-w^h}=0$$
Which, the given equality is true.

\break

\subsection*{5}
\begin{question}
    \textbf{Ahlfors Pg. 17 Problem 5:
        What is the value of:
        $$1-w^h + w^{2h}+...+(-1)^{(n-1)}w^{(n-1)h}$$
    }
\end{question}

\textbf{Pf:}

Using the same condition of \textbf{Question 4}, there are two cases to consider:

First, if $h=\frac{2k+1}{2}n$ for some $k\in\mathbb{Z}$, then: 
$$w^h=\cos(\frac{h\cdot 2\pi}{n})+i\sin(\frac{h\cdot 2\pi}{n})$$
$$= \cos(\frac{(2k+1)n\pi}{n})+i\sin(\frac{(2k+1)n\pi}{n}) = \cos((2k+1)\pi)+i\sin((2k+1)\pi) = -1$$

Which the sum is as follow:
$$1-w^h + w^{2h}+...+(-1)^{(n-1)}w^{(n-1)h} = \sum_{j=0}^{n-1}(-1)^j(w^h)^j = \sum_{j=0}^{n-1}(-1)^j(-1)^j$$
$$1-w^h + w^{2h}+...+(-1)^{(n-1)}w^{(n-1)h} = \sum_{j=0}^{n-1}(1)^j = \sum_{j=0}^{n-1}1 = n$$
Under the case of $n=2h$, the sum yields a value of $n$.

\hfill

Else, if $h\neq\frac{2k+1}{2}n$, for all $k\in\mathbb{Z}$, then since $\frac{h\cdot2\pi}{n} \neq \frac{(2k+1)n\pi}{n} = (2k+1)\pi$ for all $k\in\mathbb{Z}$, then $\cos(\frac{h\cdot 2\pi}{n}) \neq -1$,
which $w^h=\cos(\frac{h\cdot 2\pi}{n})+i\sin(\frac{h\cdot 2\pi}{n}) \neq -1$, or $(1+w^h) \neq 0$.

Thus, the sum can be expressed as follow:
$$1-w^h + w^{2h}+...+(-1)^{(n-1)}w^{(n-1)h} = \frac{(1+w^h)(1-w^h + w^{2h}+...+(-1)^{(n-1)}w^{(n-1)h})}{(1+w^h)} = \frac{1+w^{nh}}{1+w^h}$$
$$1-w^h + w^{2h}+...+(-1)^{(n-1)}w^{(n-1)h} = \frac{2}{1+w^h}$$
Since $w^{nh}=1$ is proven in \textbf{Question 4}.

\break




\end{document}