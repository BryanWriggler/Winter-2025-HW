%name: Math_CS_122A_HW1_Zih-Yu_Hsieh.tex

\documentclass{article}
\usepackage{graphicx} % Required for inserting images
\usepackage[margin = 2.54cm]{geometry}

\usepackage{amssymb}
\usepackage{amsmath}
\usepackage{verbatim}
\usepackage[utf8]{inputenc}
\linespread{1.5}

\newtheorem{definition}{Definition}
\newtheorem{proposition}{Proposition}
\newtheorem{theorem}{Theorem}
\newtheorem{question}{Question}

\title{Math CS 122A HW1}
\author{Zih-Yu Hsieh}

\begin{document}
\maketitle

\begin{comment}
\section*{1 (Unsolved)}
\begin{question}
    \textbf{Ahlfors Pg. 9 Problem 5}
\end{question}

\hfill

\textbf{Pf:}

\break


\section*{2}
\begin{question}
    \textbf{Ahlfors Pg. 11 Problem 1
    }
\end{question}

\hfill

\textbf{Pf:}

Suppose $a,b\in\mathbb{C}$ satisfy $|a|,|b| <1$. Then, since $|\bar{a}b| = |\bar{a}|\cdot |b| = |a|\cdot |b| <1$,
$|\bar{a}b| \neq |1| = 1$, then $\bar{a}b \neq 1$.
So, $1-\bar{a}b \neq 0$, the term $\frac{a-b}{1-\bar{a}b}$ is defined.

Now, consider the following identity:
$$\forall x,y\in \mathbb{C},\quad |x-y|^2 = |x|^2+|y|^2 - 2 Re(x\bar{y})$$
So, the following equations are true:
$$|a-b|^2 = |a|^2+|b|^2-2Re(a\bar{b})$$
$$|1-\bar{a}b|^2 = |1|^2+|\bar{a}b|^2-2Re(\bar{a}b)$$
Which, $|\bar{a}b| = |\bar{a}|\cdot|b| = |a|\cdot|b|$, and $Re(\bar{a}b) = \frac{\bar{a}b + \overline{\bar{a}b}}{2} = \frac{\bar{a}b + a\bar{b}}{2} = Re(a\bar{b})$,
so the equation can be simplified to:
$$|1-\bar{a}b|^2 = 1+|a|^2\cdot|b|^2-2Re(a\bar{b})$$

Then, consider the term $(1+|a|^2\cdot|b|^2)-(|a|^2+|b|^2)$:
$$(1+|a|^2\cdot|b|^2)-(|a|^2+|b|^2) = (1-|b|^2)+|a|^2(|b|^2-1)$$
$$ = (1-|b|^2)-|a|^2(1-|b|^2)=(1-|a|^2)(1-|b|^2)$$
Since both $|a|,|b|<1$, then $|a|^2,|b|^2 <1$, which $(1-|a|^2),(1-|b|^2)>0$. 
Hence, we can conclude that $(1+|a|^2\cdot|b|^2)-(|a|^2+|b|^2) = (1-|a|^2)(1-|b|^2) >0$, which: 
$$(1+|a|^2\cdot|b|^2) > (|a|^2+|b|^2),\quad (1+|a|^2\cdot|b|^2-2Re(a\bar{b})) > (|a|^2+|b|^2-2Re(a\bar{b}))$$
Replace the terms with the original form of absolute value, we get:
$$|1-\bar{a}b|^2 > |a-b|^2$$
Because $(1-\bar{a}b)\neq 0$, then $|1-\bar{a}b|^2 >0$. So:
$$1 > \frac{|a-b|^2}{|1-\bar{a}b|^2} = \left|\frac{a-b}{1-\bar{a}b}\right|^2,\quad \left|\frac{a-b}{1-\bar{a}b}\right| <1$$
Which the inequality is true.

\break

\section*{3}
\begin{question}
    \textbf{Ahlfors Pg. 11 Problem 3
    }
\end{question}

\textbf{Pf:}

Given that $|a_i|<1,\ \lambda_i \geq 0$ for all $i = 1,...,n$, and $\lambda_1 + ...+\lambda_n=1$. By the Triangle Inequality:
$$|\lambda_1a_1 + ... + \lambda_na_n| \leq |\lambda_1a_1|+...+|\lambda_na_n| = \lambda_1|a_1|+...+\lambda_n|a_n|$$
(Note: above is true since each coefficient $\lambda_i \geq 0$).
Then, let $M = \max\{|a_1|,...,|a_n|\}$, which for all $i\in\{1,...,n\}$, $|a_i|\leq M$; and since $|a_i| <1$ for all index $i$, $M <1$.
Thus, the following is true:
$$\lambda_1|a_1|+...+\lambda_n|a_n| \leq \lambda_1\cdot M + ... +\lambda_n\cdot M = M(\lambda_1+...+\lambda_n) = M$$
Which, combining all the inequalities above, we get:
$$|\lambda_1a_1 + ... + \lambda_na_n| \leq \lambda_1|a_1|+...+\lambda_n|a_n| \leq M <1$$
$$|\lambda_1a_1 + ... + \lambda_na_n| <1$$
Which, the given inequality is true.

\hfill

\end{comment}

\section*{1}
\begin{question}
    \textbf{Ahlfors Pg. 16 Problem 4
    }
\end{question}

\textbf{Pf:}
Given that $w = \cos(\frac{2\pi}{n})+i\sin(\frac{2\pi}{n})$ and $h\in\mathbb{Z}$ is not a multiple of $n$. 
Then, the term $\frac{h}{n}$ is not an integer, which consider $w^h$:
$$w^h = \cos\left(\frac{h\cdot 2\pi}{n}\right)+i\sin\left(\frac{h\cdot 2\pi}{n}\right) \neq 1$$
Since $\frac{h}{n}$ is not an integer, $\frac{h\cdot 2\pi}{n}$ is not an integer multiple of $2\pi$, thus $\cos\left(\frac{h\cdot 2\pi}{n}\right)\neq 1$, or $w^h \neq 1$.

Hence, $(1-w^h) \neq 0$, division with this number is defined. Now, consider the following:
$$1+w^h+...+w^{(n-1)h} = \frac{(1-w^h)(1+w^h+...+w^{(n-1)h})}{(1-w^h)} = \frac{1-w^{nh}}{1-w^h}$$
Which, since $w^n = \cos(\frac{n\cdot 2\pi}{n})+i\sin(\frac{n\cdot 2\pi}{n}) = \cos(2\pi)+i\sin(2\pi) = 1$, then $w^{nh} = (w^n)^h = 1^h = 1$. Thus:
$$1+w^h+...+w^{(n-1)h} = \frac{1-w^{nh}}{1-w^h} = \frac{1-1}{1-w^h}=0$$
Which, the given equality is true.

\break

\section*{2}
\begin{question}
    \textbf{Ahlfors Pg. 17 Problem 5
    }
\end{question}

\textbf{Pf:}

The sum $1-w^h+w^{2h}-...+(-1)^{(n-1)}w^{(n-1)h} = \sum_{i=0}^{(n-1)}(-w^h)^{i}$. Which, there are two cases to consider:

\hfill

First, if $h\neq \frac{(2k+1)}{2}n$ for all $k\in\mathbb{Z}$. Thus, $h\cdot\frac{2\pi}{n}\neq\frac{(2k+1)n}{2}\cdot\frac{2\pi}{n} = (2k+1)\pi$ for all $k\in\mathbb{Z}$, $\cos\left(\frac{h\cdot 2\pi}{n}\right) \neq \cos((2k+1)\pi)= -1$, so $w^h=\cos\left(\frac{h\cdot 2\pi}{n}\right)+i\sin\left(\frac{h\cdot 2\pi}{n}\right) \neq -1$. Hence, $(1-(-w^h)) = (1+w^h) \neq 0$.

Then, the sum could be expressed as:
$$\sum_{i=0}^{(n-1)}(-w^h)^{i} = \frac{(1-(-w^h))(\sum_{i=0}^{(n-1)}(-w^h)^{i})}{(1-(-w^h))} = \frac{1-(-w^h)^n}{1+w^h}$$
Which, there are two possibilities:

\begin{itemize}
    \item If $n$ is odd, then $(-w^h)^n = (-1)^nw^{nh} = -w^{nh}$, while $w^{nh}=1$ (proven in \textbf{Question 1}). Thus, $(-w^h)^n = -1$, and the sum is as follow:
    $$\sum_{i=0}^{(n-1)}(-w^h)^{i}=\frac{1-(-w^h)^n}{1+w^h} = \frac{1-(-1)}{1+w^h} = \frac{2}{1+w^h}$$

    \item Else if $n$ is even, then $(-w^h)^n = (-1)^nw^{nh} = w^{nh}$, while $w^{nh}=1$. Thus, the sum is as follow:
    $$\sum_{i=0}^{(n-1)}(-w^h)^{i}=\frac{1-(-w^h)^n}{1+w^h} = \frac{1-1}{1+w^h}=0$$
\end{itemize}

\hfill

Else, if $h=\frac{(2k+1)}{2}n$ for some $k\in\mathbb{Z}$, then $\frac{h\cdot 2\pi}{n} = \frac{(2k+1)n}{2}\cdot\frac{2\pi}{n} = (2k+1)\pi$. 

Thus, $w^h = \cos\left(\frac{h\cdot 2\pi}{n}\right)+i\sin\left(\frac{h\cdot 2\pi}{n}\right) = \cos((2k+1)\pi)+i\sin((2k+1)\pi) = -1$. So, the sum is expressed as:
$$\sum_{i=0}^{(n-1)}(-w^h)^{i} = \sum_{i=0}^{(n-1)}(-(-1))^{i}=\sum_{i=0}^{(n-1)}1^{i} = \sum_{i=0}^{(n-1)}1 = n$$



\break

\section*{3}
\begin{question}
    \textbf{Ahlfors Pg. 28 Problem 4
    }
\end{question}

\textbf{Pf:}

Suppose $f(z)$ is an analytic function that has constant norm. Which, let $z=x+iy$ for any $x,y\in\mathbf{R}$, and $f(x+iy) = u(x,y)+iv(x,y)$ for first-order differentiable real-valued functions $u, v$.

Since $f$ is analytic, $u$ and $v$ satisfy the Cauchy-Riemann Equations:
$$\frac{\partial u}{\partial x} = \frac{\partial v}{\partial y},\quad \frac{\partial v}{\partial x} = -\frac{\partial u}{\partial y}$$

\hfill

Now, consider $|f| = |u+iv| = \sqrt{u^2+v^2}$, which is assumed to be a constant. Then, there are two cases to consider:

First, if $|f|=0$ for all $z\in Dom(f)$, then $\sqrt{u^2+v^2} = 0$, which $u^2+v^2=0$ while $u,v$ are real-valued function. This only happens if $u,v=0$, thus $f(z)=u(x,y)+iv(x,y)=0$, which $f$ is a constant function.

\hfill

Else if $|f|=c$ for some $c>0$ for all $z\in Dom(f)$. Then, consider the partial derivative of $|f|$:
$$\frac{\partial}{\partial x}(|f|)=\frac{\partial}{\partial x}(\sqrt{u^2+v^2})=\frac{1}{2\sqrt{u^2+v^2}}\left(2u\frac{\partial u}{\partial x}+2v\frac{\partial v}{\partial x}\right) = \frac{1}{c}\left(u\frac{\partial u}{\partial x}+v\frac{\partial v}{\partial x}\right)$$
$$\frac{\partial}{\partial y}(|f|)=\frac{\partial}{\partial y}(\sqrt{u^2+v^2})=\frac{1}{2\sqrt{u^2+v^2}}\left(2u\frac{\partial u}{\partial y}+2v\frac{\partial v}{\partial y}\right) = \frac{1}{c}\left(u\frac{\partial u}{\partial y}+v\frac{\partial v}{\partial y}\right)$$
Since $|f|=c$ is a constant, then the partial derivatives are all $0$. Thus:
$$u\frac{\partial u}{\partial x}+v\frac{\partial v}{\partial x}=0,\quad u\frac{\partial u}{\partial y}+v\frac{\partial v}{\partial y}=0$$
Which multiplying the second equation by $i$, we get:
$$iu\frac{\partial u}{\partial y}+iv\frac{\partial v}{\partial y}=0, \quad iu\frac{\partial u}{\partial y}+iv\frac{\partial v}{\partial y}=u\frac{\partial u}{\partial x}+v\frac{\partial v}{\partial x}$$
Reorganize the equation, we get:
$$v\left(-\frac{\partial v}{\partial x}+i\frac{\partial v}{\partial y}\right)=u\left(\frac{\partial u}{\partial x}-i\frac{\partial u}{\partial y}\right)$$
Based on Cauchy-Riemann Equation $\frac{\partial v}{\partial x} = -\frac{\partial u}{\partial y}$, substitute the variables containing these two terms, we get:
$$v\left(\frac{\partial u}{\partial y}+i\frac{\partial v}{\partial y}\right)=u\left(\frac{\partial u}{\partial x}-i\left(-\frac{\partial v}{\partial x}\right)\right)$$
Which, since $\frac{\partial u}{\partial y}+i\frac{\partial v}{\partial y} = i\left(-i\frac{\partial u}{\partial y}+\frac{\partial v}{\partial y}\right)$, the following is true:
$$iv\left(-i\frac{\partial u}{\partial y}+\frac{\partial v}{\partial y}\right)=u\left(\frac{\partial u}{\partial x}+i\frac{\partial v}{\partial x}\right)$$

Now, recall that for analytic function, $\frac{\partial f}{\partial z} = \left(\frac{\partial u}{\partial x}+i\frac{\partial v}{\partial x}\right) = \left(\frac{\partial v}{\partial y}-i\frac{\partial u}{\partial y}\right)$. Thus the above equation could also be written as:
$$iv\frac{\partial f}{\partial z}=u\frac{\partial f}{\partial z},\quad (u-iv)\frac{\partial f}{\partial z} = 0$$
However, since $|f(z)| = |u+iv| = c >0$, then $(u+iv) \neq 0$, which its conjugate $(u-iv) \neq 0$ also. Thus, in case for the above equation to be true, $\frac{\partial f}{\partial z} = 0$, showing that $f$ is a constant function.

\hfill

Since for all analytic function, having constant norm implies the function itself is constant, then any analytic function that's not constant cannot have constant norm.


\break


\begin{comment}
\section*{7 (Unsolved)}
\begin{question}
    \textbf{Ahlfors Pg. 28 Problem 5}
\end{question}

\textbf{Pf:}

Let $z=x+iy$ for $x,y\in\mathbf{R}$, which $f(x+iy) = u(x,y)+iv(x,y)$ for some real-valued function $u$ and $v$. 

Then, $\bar{z}=x-iy$, with $\overline{f(\bar{z})} = \overline{f(x-iy)} = \overline{u(x,-y)+iv(x,-y)} = u(x,-y)-iv(x,-y)$. Which, let $u_1(x,y) = u(x,-y)$, and $v_1(x,y)=-v(x,-y)$, then $\overline{f(\bar{z})} = u_1(x,y)+iv_1(x,y)$.

\hfill

By the definition of analytic function, $f(z)$ is analytic if and only if Cauchy-Riemann Equation is satisfied with $u,v$:
$$\frac{\partial u}{\partial x}=\frac{\partial v}{\partial y},\quad \frac{\partial v}{\partial x}=-\frac{\partial u}{\partial y}$$

Similarly, $\overline{f(\bar{z})}$ is analytic if and only if Cauchy-Riemann Equation is satisfied with $u_1,v_1$:
$$\frac{\partial u_1}{\partial x}=\frac{\partial v_1}{\partial y},\quad \frac{\partial v_1}{\partial x}=-\frac{\partial u_1}{\partial y}$$
Which, let $y_1=-y$ (which $\frac{\partial y_1}{\partial y}=-1$, and $\frac{\partial y_1}{\partial x}=0$), we have $u_1(x,y) = u(x,y_1)$ and $v_1(x,y)=-v(x,y_1)$. Then, the following is true:
$$\frac{\partial u_1}{\partial x}=\frac{\partial}{\partial x}(u(x,y_1)) = \frac{\partial u}{\partial x}\frac{\partial x}{\partial x}+\frac{\partial u}{\partial y_1}\frac{\partial y_1}{\partial x} = \frac{\partial u}{\partial x}\cdot 1+\frac{\partial u}{\partial y_1}\cdot 0 = \frac{\partial u}{\partial x}$$
$$\frac{\partial u_1}{\partial y}=\frac{\partial}{\partial y}(u(x,y_1)) = \frac{\partial u}{\partial x}\frac{\partial x}{\partial y}+\frac{\partial u}{\partial y_1}\frac{\partial y_1}{\partial y} = \frac{\partial u}{\partial x}\cdot 0+\frac{\partial u}{\partial y_1}\cdot (-1) = -\frac{\partial u}{\partial y_1}$$

$$\frac{\partial v_1}{\partial x}=\frac{\partial}{\partial x}(-v(x,y_1)) = -\left(\frac{\partial v}{\partial x}\frac{\partial x}{\partial x}+\frac{\partial v}{\partial y_1}\frac{\partial y_1}{\partial x}\right) = -\left(\frac{\partial v}{\partial x}\cdot 1+\frac{\partial v}{\partial y_1}\cdot 0\right) = -\frac{\partial v}{\partial x}$$
$$\frac{\partial v_1}{\partial y}=\frac{\partial}{\partial y}(-v(x,y_1)) = -\left(\frac{\partial v}{\partial x}\frac{\partial x}{\partial y}+\frac{\partial v}{\partial y_1}\frac{\partial y_1}{\partial y}\right) = -\left(\frac{\partial v}{\partial x}\cdot 0+\frac{\partial v}{\partial y_1}\cdot (-1)\right) = \frac{\partial v}{\partial y_1}$$


\break


\break

\end{comment}


\section*{4}
\begin{question}
    \textbf{Stein and Shakarchi Pg. 26 Problem 7}
\end{question}

\textbf{Pf:}

\begin{itemize}
    \item[(a)]
    Given $z,w\in\mathbb{C}$ such that $\bar{z}w\neq 1$ (which implies that $\overline{\bar{z}w}=z\bar{w}\neq 1$, since $\overline{\bar{z}w}\neq\bar{1}=1$).

    First, for all $u,v\in\mathbb{C}$, the following identity is true:
    $$|u-v|^2 = |u|^2+|v|^2-2Re(\bar{u}v)$$
    Which, apply it to $(w-z)$ and $(1-\bar{w}z)$, we get:
    $$|w-z|^2 = |w|^2+|z|^2 - 2Re(\bar{w}z)$$
    $$|1-\bar{w}z|^2=|1|^2+|\bar{w}z|^2 - 2Re(\bar{1}\cdot(\bar{w}z)) = 1+|w|^2\cdot|z|^2-2Re(\bar{w}z)$$

    \hfill

    \textbf{When $|z|,|w| <1$:}
    
    Given that $|z|,|w|<1$, we just need to compare $|w|^2+|z|^2$ and $1+|w|^2\cdot |z|^2$. Which, if we take the difference, it is as follow:
    $$(1+|w|^2\cdot|z|^2)-(|w|^2+|z|^2) = |w|^2(|z|^2-1) + (1-|z|^2)$$
    $$= -|w|^2(1-|z|^2)+(1-|z|^2) = (1-|w|^2)(1-|z|^2)$$
    Since both $|z|,|w|<1$, then $|z|^2,|w|^2<1$, which $0 < (1-|z|^2), (1-|w|^2)$, thus $(1-|w|^2)(1-|z|^2) >0$.

    From this, we can conclude the following:
    $$0<(1-|w|^2)(1-|z|^2)=(1+|w|^2\cdot|z|^2)-(|w|^2+|z|^2),\quad(|w|^2+|z|^2) < (1+|w|^2\cdot|z|^2)$$
    $$|w|^2+|z|^2-2Re(\bar{w}z) < 1+|w|^2\cdot|z|^2-2Re(\bar{w}z)$$
    Substitute the original modulus form, we get:
    $$|w-z|^2 < |1-\bar{w}z|^2$$
    $$\frac{|w-z|^2}{|1-\bar{w}z|^2}<1,\quad \left|\frac{w-z}{1-\bar{w}z}\right|^2<1,\quad \left|\frac{w-z}{1-\bar{w}z}\right|<1$$
    
    \hfill

    \textbf{When $|z|=1$ or $|w|=1$:}

    Suppose $|z|=1$ or $|w|=1$.

    If $|z|=1$ (or $|z|^2=1$), the following is true:
    $$|w-z|^2 = |w|^2+|z|^2 - 2Re(\bar{w}z) = |w|^2\cdot|z|^2+1-2Re(\bar{w}z) = |1-\bar{w}z|^2$$

    Else if $|w|=1$ (or $|w|^2=1$), the following is true:
    $$|w-z|^2 = |w|^2+|z|^2 - 2Re(\bar{w}z) = 1+|w|^2\cdot|z|^2-2Re(\bar{w}z) = |1-\bar{w}z|^2$$
    Which, regardless of the case, $|w-z|^2 = |1-\bar{w}z|^2$, or $|w-z| = |1-\bar{w}z|$. Hence, since $\bar{w}z \neq 1$ by assumption ($1-\bar{w}z\neq 0$, or $|1-\bar{w}z|\neq 0$), the following is true:
    $$\frac{|w-z|}{|1-\bar{w}z|} = 1,\quad \left|\frac{w-z}{1-\bar{w}z}\right|=1$$

    \hfill
    
    \item[(b)]
    Given a fixed complex number $w\in \mathbb{D}$, where $\mathbb{D} \subset \mathbb{C}$ is the unit disc, and the map $F:z\mapsto\frac{w-z}{1-\bar{w}z}$. 

    In case for the function $F$ to be defined on $w$, we need $|w| <1$: Since $w$ is in the unit disc, then $|w| \leq 1$; yet, if $|w|=1$, then $(1-\bar{w}w) = (1-|w|^2) = (1-1) = 0$, which the function is not defined since $(1-\bar{w}w)$ is in the denominator. So, we need $|w| <1$.
    
    Then, there are some conditions to check:
    \begin{itemize}
        \item[(i)] For all $z\in\mathbb{D}$, $|z| \leq 1$. And, since $|w|<1$, then $|\bar{w}z|=|w|\cdot|z| < 1$, thus $\bar{w}z \neq 1$, or $(1-\bar{w}z)\neq 0$. Thus, the value $F(z)=\frac{w-z}{1-\bar{w}z}$ is defined.
        
        First, if $|z|=1$, by the statement proven in part $(a)$, the following is true:
        $$|F(z)|=\left|\frac{w-z}{1-\bar{w}z}\right|=1$$
        Thus, $F(z)\in\mathbb{D}$.

        Else, if $|z|<1$, then since $|w|<1$ is proven beforehand, again by the statement proven in part $(a)$, the following is true:
        $$|F(z)|=\left|\frac{w-z}{1-\bar{w}z}\right|<1$$
        Thus again, $F(z)\in\mathbb{D}$.

        Regardless of the case, for all $z\in\mathbb{D}$, $F(z)\in\mathbb{D}$, thus restricting the domain to $\mathbb{D}$, $F(\mathbb{D}) \subseteq \mathbb{D}$. So, $F:\mathbb{D}\rightarrow \mathbb{D}$.

        To prove that it is analytic (holomorphic), recall that the function $z$ is holomorphic, which $(w-z)$ and $(1-\bar{w}z)$ are both holomorphic functions (while given that $(1-\bar{w}z)\neq 0$ for all $z\in\mathbb{D}$). Thus, the quotient of two functions $\frac{(w-z)}{(1-\bar{w}z)}$ is holomorphic.

        \hfill

        \item[(ii)] Consider $F(0)$ and $F(w)$:
        $$F(0) = \frac{w-0}{1-\bar{w}0} = \frac{w}{1-0} = w$$
        $$F(w) = \frac{w-w}{1-\bar{w}w}=\frac{0}{1-|w|^2}=0$$
        Note: the second equation is defined, since we've proven that $|w|<1$, which $|w|^2<1$, or $(1-|w|^2)>0$, the function is defined.

        \hfill

        \item[(iii)] If $|z| = 1$, then from what we've proven in part (a):
        $$|F(z)| = \left|\frac{w-z}{1-\bar{w}z}\right| = 1$$

        \hfill

        \item[(iv)]
        Consider $F\circ F$: For all $z\in\mathbb{D}$, the following is true:
        $$F\circ F(z) = F\left(\frac{w-z}{1-\bar{w}z}\right) = \frac{w-\frac{w-z}{1-\bar{w}z}}{1-\bar{w}\frac{w-z}{1-\bar{w}z}} = \frac{w(1-\bar{w}z)-(w-z)}{1(1-\bar{w}z)-\bar{w}(w-z)}$$
        $$=\frac{w-w\bar{w}z-w+z}{1-\bar{w}z-\bar{w}w+\bar{w}z} = \frac{-|w|^2z+z}{1-|w|^2} = \frac{z(1-|w|^2)}{1-|w|^2} = z$$
        (Note: since $(1-|w|^2)\neq 0$, the above equation is defined).

        Which, $F\circ F$ is actually an Identity map from $\mathbb{D}$ to $\mathbb{D}$, it is bijective.

        Thus, $F$ is surjective: If $F$ is not surjective, then there exists $u\in\mathbb{D}$, such that any $z\in\mathbb{D}$ cannot satisfy $F(z)=u$. However, that means for all $z\in \mathbb{D}$, since $F(z)\in\mathbb{D}$, $F\circ F(z) \neq u$, which $F\circ F(u) = u$ is a contradiction. Therefore, $F$ must be surjective.

        Also, $F$ is injective: For all $z_1,z_2\in\mathbb{D}$ with $F(z_1)=F(z_2)$, since $z_1=F\circ F(z_1)=F(F(z_1)) = F(F(z_2)) = F\circ F(z_2) = z_2$, then $z_1=z_2$, showing that $F$ is injective.
    \end{itemize}
\end{itemize}

\break

\section*{5}
\begin{question}
    \textbf{Stein and Shakarchi Pg. 26 Problem 9
    }
\end{question}

\textbf{Pf:}

Given that $u(x,y), v(x,y)$ are two real-valued functions, with $x=r\cos(\theta)$ and $y=r\sin(\theta)$. Then, the following is true:
$$\frac{\partial x}{\partial r}=\cos(\theta),\quad \frac{\partial x}{\partial \theta} = -r\sin(\theta),\quad \frac{\partial y}{\partial r}=\sin(\theta),\quad \frac{\partial y}{\partial \theta} = r\cos(\theta)$$
Then, the partial derivative of $u$ and $v$ can be rewritten as:
$$\frac{\partial u}{\partial r} = \frac{\partial u}{\partial x}\frac{\partial x}{\partial r}+\frac{\partial u}{\partial y}\frac{\partial y}{\partial r} = \frac{\partial u}{\partial x}\cos(\theta)+\frac{\partial u}{\partial y}\sin(\theta)$$
$$\frac{\partial u}{\partial \theta} = \frac{\partial u}{\partial x}\frac{\partial x}{\partial \theta}+\frac{\partial u}{\partial y}\frac{\partial y}{\partial \theta} = -\frac{\partial u}{\partial x}r\sin(\theta)+\frac{\partial u}{\partial y}r\cos(\theta)$$

$$\frac{\partial v}{\partial r} = \frac{\partial v}{\partial x}\frac{\partial x}{\partial r}+\frac{\partial v}{\partial y}\frac{\partial y}{\partial r} = \frac{\partial v}{\partial x}\cos(\theta)+\frac{\partial v}{\partial y}\sin(\theta)$$
$$\frac{\partial v}{\partial \theta} = \frac{\partial v}{\partial x}\frac{\partial x}{\partial \theta}+\frac{\partial v}{\partial y}\frac{\partial y}{\partial \theta} = -\frac{\partial v}{\partial x}r\sin(\theta)+\frac{\partial v}{\partial y}r\cos(\theta)$$

If Cauchy-Riemann Equation is satisfied:
$$\frac{\partial u}{\partial x}=\frac{\partial v}{\partial y},\quad \frac{\partial v}{\partial x}=-\frac{\partial u}{\partial y}$$
Substitute into the previous equation, we can yield:
$$\frac{\partial u}{\partial r}=\frac{\partial u}{\partial x}\cos(\theta)+\frac{\partial u}{\partial y}\sin(\theta) = \frac{\partial v}{\partial y}\cos(\theta)-\frac{\partial v}{\partial x}\sin(\theta)$$
$$= \frac{1}{r}(\frac{\partial v}{\partial y}r\cos(\theta)-\frac{\partial v}{\partial x}r\sin(\theta)) = \frac{1}{r}\frac{\partial v}{\partial \theta}$$

$$\frac{1}{r}\frac{\partial u}{\partial \theta} = \frac{1}{r}(-\frac{\partial u}{\partial x}r\sin(\theta)+\frac{\partial u}{\partial y}r\cos(\theta)) = -\frac{\partial u}{\partial x}\sin(\theta)+\frac{\partial u}{\partial y}\cos(\theta)$$
$$ = -\frac{\partial v}{\partial y}\sin(\theta)-\frac{\partial v}{\partial x}\cos(\theta) = -(\frac{\partial v}{\partial y}\sin(\theta)+\frac{\partial v}{\partial x}\cos(\theta)) = -\frac{\partial v}{\partial r}$$
Which, $\frac{\partial u}{\partial r}=\frac{1}{r}\frac{\partial v}{\partial \theta}$, and $\frac{1}{r}\frac{\partial u}{\partial \theta}= -\frac{\partial v}{\partial r}$.

\hfill

\hfill

Now, given logarithm function $\ln(z)=\ln(r) + i\theta$, with $r>0$ and $-\pi <\theta <\pi$. Then, let $u(r,\theta)=\ln(r)$, and $v(r,\theta) = \theta$, then $\ln(z) = u+iv$.

Consider the first-order partial derivative of $u$ and $v$:
$$\frac{\partial u}{\partial r} = \frac{1}{r},\quad \frac{\partial u}{\partial \theta} = 0$$
$$\frac{\partial v}{\partial r} = 0,\quad \frac{\partial v}{\partial \theta} = 1$$
Which, the following are true:
$$\frac{\partial u}{\partial r} = \frac{1}{r}\cdot 1 = \frac{1}{r}\frac{\partial v}{\partial \theta}$$
$$\frac{1}{r}\frac{\partial u}{\partial \theta} = \frac{1}{r}\cdot 0 = 0 = -0 = -\frac{\partial v}{\partial r}$$
Thus, the Cauchy-Riemann Equation in polar coordinates is satisfied, proving that the logarithmic function is holomorphic on $r>0$ and $-\pi <\theta <\pi$.
\end{document}