% Math_CS_122A_HW2_Zih-Yu_Hsieh.tex

\documentclass{article}
\usepackage{graphicx} % Required for inserting images
\usepackage[margin = 2.54cm]{geometry}

\usepackage{amsmath}
\usepackage{amssymb}
\usepackage{verbatim}
\usepackage[utf8]{inputenc}
%\linespread{1.5}

\newtheorem{definition}{Definition}
\newtheorem{proposition}{Proposition}
\newtheorem{theorem}{Theorem}
\newtheorem{question}{Question}

\title{Math CS 122A HW1}
\author{Zih-Yu Hsieh}

\begin{document}
\maketitle

\section*{1}
\begin{question}
    Ahlfors Pg. 33 Problem 4
\end{question}

\textbf{Pf:}

Suppose $R(z)$ is a rational function such that the numerator and denominator have no common roots, and it satisfies $|R(z)|=1$ whenever $|z|=1$. 
For simplicity, $R(z)$ is in the following form:
$$R(z)=\frac{a_0+a_1z+...+a_nz^n}{b_0+b_1z+...+b_mz^m},\quad m,n\in\mathbb{N},\quad a_n,b_n\neq 0$$
Notice that without loss of generality, we can assume $m=n$: If the two are not equal, multiply $R(z)$ by $z^{m-n}$ would form the 
same degree on both the numerator and denominator (if $m>n$, the numerator has highest degree of $z^{m-n}\cdot z^n=z^m$; else if $m<n$, 
it's the same as the denominator multiplied by $z^{n-m}$, which the highest degree of the denominator is $z^{n-m}\cdot z^m=z^n$).

Also, since for all $z\in\mathbb{C}$ with $|z|=1$, $|z^{m-n}|=1$, thus $R_1(z)=z^{m-n}R(z)$
still fulfills the given property (if $|z|=1$, $|z^{m-n}R(z)|=|z|^{m-n}|R(z)|=1$).

\hfill

Now, for all $z\in\mathbb{C}$ with $|z|=1$, $|z|^2=z\bar{z}=1$, thus $z=1/\bar{z}$. Similarly, since $|R(z)|=1$, then 
$|R(z)|^2=R(z)\overline{R(z)}=1$. Which, substitute $z$ by $1/\bar{z}$ would get the following:
$$|z|=1\implies R(z)\overline{R(1/\bar{z})}=1$$
Notice that $\overline{R(1/\bar{z})}$ itself is also a rational function:
$$\overline{R(1/\bar{z})}=\overline{\left(\frac{a_0+a_1(1/\bar{z})+...+a_n(1/\bar{z})^n}{b_0+b_1(1/\bar{z})+...+b_n(1/\bar{z})^n}\right)} = \overline{\left(\frac{a_0\bar{z}^n+a_1\bar{z}^{n-1}+...+a_n}{b_0\bar{z}^n+b_1\bar{z}^{n-1}+...+b_n}\right)}$$
$$\overline{R(1/\bar{z})}=\frac{\bar{a_0}z^n+\bar{a_1}z^{n-1}+...+\bar{a_n}}{\bar{b_0}z^n+\bar{b_1}z^{n-1}+...+\bar{b_n}}$$
Thus, the product $R(z)\overline{R(1/\bar{z})}$ is also a rational function.

\hfill

Then, consider $R(z)\overline{R(1/\bar{z})}-1$: From the above equation, every $z\in\mathbb{C}$ with $|z|=1$ satifies the following:
$$R(z)\overline{R(1/\bar{z})}-1=1-1=0$$
Thus, every $z$ on the unit circle is a zero of the rational function $R(z)\overline{R(1/\bar{z})}-1$, it has infinite zeroes; yet, suppose the rational function has order $m>0$, it has at most $m$ distinct zeroes, which is a contradiction. Therefore, $R(z)\overline{R(1/\bar{z})}-1$ must have order $0$, indicating that it is a constant function.

Also, since every $z$ on the unit circle has $R(z)\overline{R(1/\bar{z})}-1=0$, then the function itself (as a constant) must be $0$, which implies the function $R(z)\overline{R(1/\bar{z})}=1$.

\hfill

Finally, since $R(z)\overline{R(1/\bar{z})}=1$, then for all $\alpha\in\mathbb{C}$ that is a zero of $R(z)$ ($R(\alpha)=0$), must also be the pole of $\overline{R(1/\bar{z})}$:
Supopse $\alpha\in\mathbb{C}$ is a zero of $R(z)$, but not a pole of $\overline{R(1/\bar{z})}$, then $\overline{R(1/\bar{\alpha})}\in\mathbb{C}$ and $R(\alpha)=0$. Then $R(\alpha)\overline{R(1/\bar{\alpha})}=0\cdot\overline{R(1/\bar{\alpha})} = 0 \neq 1$.
Which, the function $R(z)\overline{R(1/\bar{z})}$ is defined on $\alpha$, but has an output of $0$ instead of $1$, indicating the function is not a constant function.
Yet, this contradicts the previous statement, so $\alpha$ must a pole of $\overline{R(1/\bar{z})}$, or $1/\bar{\alpha}$ is a pole of $R(z)$.

\hfill

Given the rational function with the condition $|z|=1$ implies $|R(z)|=1$, if $\alpha\neq 0$ is a zero of $R(z)$, then $1/\bar{\alpha}$ must be a pole of $R(z)$.

For the special case $\alpha=0$ ($R(0)=0$), since as $z$ approaches $0$, $\overline{R(1/\bar{z})}=1/R(z)$ diverges, indicating that as $1/\bar{z}$ goes unbounded (approaching $\infty$ on extended complex plane),
$\overline{R(1/\bar{z})}$ diverges, hence $R(z)$ has a pole at $\infty$.

And, for the other special case $\alpha=\infty$, the function $R(1/z)$ approaches $0$ as $z$ approaches $0$ (or $\frac{1}{z}$ goes unbounded, approaching $\infty$ on the extended complex plane),
which $\overline{R(1/\overline{(1/z)})} = \overline{R(\bar{z})}$ would diverge when $z$ approaches $0$, indicating that $R(z)$ has a pole at $0$.

\hfill

\section*{2}
\begin{question}
    Ahlfors Pg. 37 Problem 2
\end{question}

\textbf{Pf:}

Suppose $\lim_{n\rightarrow\infty}z_n=A$, then for all $\epsilon>0$, there exists $N$, with $n\geq N\implies |z_n-A|<\epsilon$.

Also, because the sequence converges, it is also bounded. Thus, there exists $M>0$, such that for every $n\in\mathbb{N}$, $|z_n-A|<M$.

\hfill

Which, for all $\epsilon>0$, since $\frac{\epsilon}{2}>0$, there exists $N_1\in\mathbb{N}$, with $n\geq N_1$ implies $|z_n-A|<\frac{\epsilon}{2}$.

Then, for the given $\epsilon$, since $\frac{\epsilon}{2}>0$, by Archimedean's Property, there exists $N_2\in\mathbb{N}$, with $N_1M<N_2\frac{\epsilon}{2}$
(Or, $\frac{N_1M}{N_2}<\frac{\epsilon}{2}$). 

\hfill

Now, let $N=\max\{N_1,N_2\}+1$, for all $n\geq N$, it is clear that $n>N_1$ and $n>N_2$. Which, consider the following difference:
$$\left|\frac{\sum_{i=1}^{n}z_i}{n}-A\right| = \left|\frac{\sum_{i=1}^{n}(z_i-A)}{n}\right|=\left|\sum_{i=1}^{N_1}\frac{(z_i-A)}{n}+\sum_{i=N_1+1}^{n}\frac{(z_i-A)}{n}\right|$$
$$\left|\frac{\sum_{i=1}^{n}z_i}{n}-A\right|\leq \sum_{i=1}^{N_1}\frac{|z_i-A|}{n}+\sum_{i=N_1+1}^{n}\frac{|z_i-A|}{n}$$
Which, by the construction beforehand, for index $i\in\{1,...,N_1\}$, $|z_i-A|<M$; and for index $j\in\{N_1+1,...,n\}$, $|z_j-A|<\frac{\epsilon}{2}$ (since $j>N_1$).
Thus, the above inequality can be expressed as:
$$\left|\frac{\sum_{i=1}^{n}z_i}{n}-A\right|\leq \sum_{i=1}^{N_1}\frac{M}{n}+\sum_{i=N_1+1}^{n}\frac{\epsilon/2}{n} = \frac{N_1M}{n}+\frac{(n-N_1)\epsilon}{2n}$$
$$\left|\frac{\sum_{i=1}^{n}z_i}{n}-A\right|\leq \frac{N_1M}{n}+\frac{n\epsilon}{2n}\leq \frac{N_1M}{n}+\frac{\epsilon}{2}$$
(Note: the second inequality holds since $(n-N_1)<n$).

Now, since $n>N_2$, then $\frac{1}{n}<\frac{1}{N_2}$. So, $\frac{N_1M}{n}<\frac{N_1M}{N_2}<\frac{\epsilon}{2}$. 
Then, the above inequality becomes:
$$\left|\frac{\sum_{i=1}^{n}z_i}{n}-A\right|\leq \frac{N_1M}{n}+\frac{\epsilon}{2} < \frac{\epsilon}{2}+\frac{\epsilon}{2}=\epsilon$$
Hence, for any $\epsilon>0$, there exists $N$, with $n\geq N$ implies $\left|\sum_{i=1}^{n}\frac{z_i}{n}-A\right|<\epsilon$,
which implies: 
$$\lim_{n\rightarrow \infty}\sum_{i=1}^{n}\frac{z_i}{n}=A$$

\break

\section*{3}
\begin{question}
    Ahlfors Pg. 41 Poblem 7
\end{question}

\textbf{Pf:}

Given that $\lim_{n\rightarrow\infty}\frac{|a_n|}{|a_{n+1}|}=R$ ($R\in [0,\infty]$). Without Loss of Generality, one can assume after some sufficiently large index $n$, $|a_n|>0$ for the limit of ratio to be well defined, and all the proof below would 
assume for chosen index $n$, $|a_n|>0$.

\hfill

\textbf{When $0<R<\infty$:}

Since $\frac{1}{R}$ is well-defined, then $\lim_{n\rightarrow\infty}\frac{|a_{n+1}|}{|a_n|}=\lim_{n\rightarrow\infty}\frac{1}{|a_{n}|/|a_{n+1}|}=\frac{1}{R}$.
Now, the goal is to prove $\lim_{n\rightarrow\infty}\sqrt[n]{|a_n|}=\frac{1}{R}$:
\begin{itemize}
    \item[(1)] $\lim\sup \{\sqrt[n]{|a_n|}\}\leq \frac{1}{R}:$ To approach this, consider any $U>\frac{1}{R}$. Since $(U-\frac{1}{R})>0$, by the definition of convergence,
    there exists $N$, with $n\geq N$ implies $\left|\frac{|a_{n+1}|}{|a_n|}-\frac{1}{R}\right|<(U-\frac{1}{R})$. Thus:
    $$\left(\frac{1}{R}-U\right)<\frac{|a_{n+1}|}{|a_n|}-\frac{1}{R}<\left(U-\frac{1}{R}\right),\quad \frac{|a_{n+1}|}{|a_n|}<U$$
    Then, for the fixed $N$ and $U$ constructed above, consider arbitrary $n>N$, the term $|a_n|$ could be expressed as:
    $$|a_n|=\frac{|a_n|}{|a_{n-1}|}\cdot...\cdot\frac{|a_{N+1}|}{|a_N|}\cdot|a_N| = |a_N|\cdot\prod_{k=N}^{n-1}\frac{|a_{k+1}|}{|a_k|}$$
    Notice that for index $k\in\{N,N+1,...,n-1\}$, since $k\geq N$, then $0< \frac{|a_{k+1}|}{|a_k|}<U$, thus:
    $$|a_n| = |a_N|\cdot\prod_{k=N}^{n-1}\frac{|a_{k+1}|}{|a_k|} < |a_N|\prod_{k=N}^{n-1}U =  |a_N|U^{n-N}$$
    Now, let $M=|a_N|U^{-N}>0$, for all $n>N$, $|a_n|<U^n\cdot M$, or $\sqrt[n]{|a_n|}<\sqrt[n]{U^n\cdot M} = U\sqrt[n]{M}$.

    Based on this inequality, define the two quantities as follow:
    $$\alpha_n = \sup\{\sqrt[k]{|a_k|}\ |\ k\geq n\},\quad \beta_n = \sup\{U\sqrt[k]{M}\ |\ k\geq n\}$$
    Since for all $k\geq n$, $\sqrt[k]{|a_k|}<U\sqrt[k]{M}\leq \beta_n$, thus $\beta_n$ is the upper bound of the set $\{\sqrt[k]{|a_k|}\ |\ k\geq n\}$, hence $\alpha_n\leq \beta_n$;
    and, since $\lim_{n\rightarrow\infty}\sqrt[n]{M}=1$ for $M>0$, then $\lim_{n\rightarrow\infty}U\sqrt[n]{M}=U$, which all subsequential limit is $U$. Thus, the following is true:
    $$\lim_{n\rightarrow\infty}\beta_n =  \lim\sup\{U\sqrt[n]{M}\}=\lim_{n\rightarrow\infty}U\sqrt[n]{M}=U$$
    Which, since for all $n>N$, $\alpha_n\leq \beta_n$, the following is true:
    $$\lim\sup\{\sqrt[n]{|a_n|}\}=\lim_{n\rightarrow\infty}\alpha_n \leq \lim_{n\rightarrow\infty}\beta_n = U$$
    Thus, $\lim\sup\{\sqrt[n]{|a_n|}\}\leq U$ for all $U>\frac{1}{R}$, hence $\lim\sup\{\sqrt[n]{|a_n|}\}\leq \frac{1}{R}$.

    \hfill

    \item[(2)] $\lim\inf\{\sqrt[n]{|a_n|}\}\geq \frac{1}{R}$: Similarly, consider any $0<L<\frac{1}{R}$. Since $(\frac{1}{R}-L)>0$, there exists $N$, with $n\geq N$ implies $\left|\frac{|a_{n+1}|}{|a_n|}-\frac{1}{R}\right|<(\frac{1}{R}-L)$. Thus:
    $$\left(\L-\frac{1}{R}\right)<\frac{|a_{n+1}|}{|a_n|}-\frac{1}{R}<\left(\frac{1}{R}-L\right),\quad 0<L<\frac{|a_{n+1}|}{|a_n|}$$
    Then, for the fixed $N$ and $L$, any $n>N$ satisfies the following:
    $$|a_n|=|a_N|\cdot\prod_{k=N}^{n-1}\frac{|a_{k+1}|}{|a_k|} > |a_N|\cdot\prod_{k=N}^{n-1}L = |a_N|\cdot L^{n-N}$$
    Now, let $m=|a_N|\cdot L^{-N}>0$, for all $n>N$, $|a_n|>L^n\cdot m$, thus $\sqrt[n]{|a_n|}>\sqrt[n]{L^n\cdot m} = L\sqrt[n]{m}$.

    Again, define the following two quantities:
    $$\gamma_n=\inf\{\sqrt[k]{|a_k|}\ |\ k\geq n\},\quad \delta_n = \inf\{L\sqrt[k]{m}\ |\ k\geq n\}$$
    Since for all $k\geq n$, $\sqrt[k]{|a_k|}>L\sqrt[k]{m} \geq \delta_n$, thus $\delta_n$ is a lower bound of $\{\sqrt[k]{|a_k|}\ |\ k\geq n\}$, hence $\gamma_n \geq \delta_n$.
    And, since $m>0$, $\lim_{n\rightarrow\infty}\sqrt[n]{m}=1$, thus $\lim_{n\rightarrow\nLeftarrow}L\sqrt[n]{m}=L$. Thus:
    $$\lim_{n\rightarrow\infty}\delta_n = \lim\inf\{L\sqrt[n]{m}\} =\lim_{n\rightarrow\nLeftarrow}L\sqrt[n]{m}=L$$
    Which, since for all $n>N$, $\gamma_n\geq \delta_n$, the following is true:
    $$\lim\inf\{\sqrt[n]{|a_n|}\}=\lim_{n\rightarrow\infty}\gamma_n \geq \lim_{n\rightarrow\infty}\delta_n=L$$
    Hence, $\lim\inf\{\sqrt[n]{|a_n|}\}\geq L$ for all $L$ satisfying $0<L<\frac{1}{R}$, which $\lim\inf\{\sqrt[n]{|a_n|}\}\geq \frac{1}{R}$.
\end{itemize}

From the above 2 statements, the following is true:
$$\frac{1}{R}\leq\lim\inf\{\sqrt[n]{|a_n|}\}\leq \lim\sup\{\sqrt[n]{|a_n|}\}\leq \frac{1}{R}$$
Thus, $\lim\inf\{\sqrt[n]{|a_n|}\}= \lim\sup\{\sqrt[n]{|a_n|}\}=\frac{1}{R}$, so the radius of convergence $\frac{1}{\lim\sup\{\sqrt[n]{|a_n|}\}}=R$.

\hfill

\hfill

\textbf{When $R=\infty$:}

Now, given that $\lim_{n\rightarrow\infty}\frac{|a_{n}|}{|a_{n+1}|}=R=\infty$, which for all $M>0$, there exists $N$, with $n\geq N$ implies $\frac{|a_n|}{|a_{n+1}|}>M$.

\hfill

We'll prove by contradiciton. Suppose the radius of convergence $R'<R$, which $R'\in[0,\infty)$. 
Then, choose $r\in(R',\infty)$, and consider $\sum_{n=1}^{\infty}a_nr^n$:

For all $n\in\mathbb{N}$, the ratio $\frac{|a_{n+1}r^{n+1}|}{|a_nr^n|}=\frac{r}{|a_n|/|a_{n+1}|}$. Now, for all $\epsilon>0$ (which $\frac{\epsilon}{r}>0$), since there exists $M\in\mathbb{N}$ with $1<M\frac{\epsilon}{r}$,
then $\frac{1}{M}<\frac{\epsilon}{r}$. For the chosen $M$, there exists $N$, such that $n\geq N$ implies $\frac{|a_n|}{|a_{n+1}|}>M$, thus the ratio $\frac{1}{|a_n|/|a_{n+1}|}<\frac{1}{M}$. So:
$$\frac{|a_{n+1}r^{n+1}|}{|a_nr^n|}=\frac{r}{|a_n|/|a_{n+1}|}<\frac{r}{M} < r\frac{\epsilon}{r}=\epsilon$$
So, for all $\epsilon>0$, there exists $N$ with $n\geq N$ implies $\left|\frac{|a_{n+1}r^{n+1}|}{|a_nr^n|}-0\right|<\epsilon$, thus $\lim_{n\rightarrow\infty}\frac{|a_{n+1}r^{n+1}|}{|a_nr^n|}=0<1$.
Then, by Ratio Test, we can conclude that $\sum_{n=1}^{\infty}a_nr^n$ converges.  Yet, since $|r|=r>R'$, it is outside of the radius of convergence,
so the given series should diverge, and this is a contradiction.

So, the radius of convergence $R'\geq R$, which since $R=\infty$, $R'=\infty$ is the radius of convergence.

\hfill

\hfill

\textbf{When $R=0$:}

Now, given that $\lim_{n\rightarrow\infty}\frac{|a_{n}|}{|a_{n+1}|}=R=0$, which for all $\epsilon>0$, there exists $N$, with $n\geq N$ implies $\left|\frac{|a_n|}{|a_{n+1}|}-0\right|<\epsilon$.

\hfill

We'll approach by contradiction again. Suppose the radius of convergence $R'>R=0$, which $R'\in(0,\infty]$.
Then, choose $r\in (0,R')$, and consider $\sum_{n=1}^{\infty}a_nr^n$:

Again, for all $n\in\mathbb{N}$, the ratio $\frac{|a_{n+1}r^{n+1}|}{|a_nr^n|}=\frac{r}{|a_n|/|a_{n+1}|}$. Which, for all $M>0$ ($\frac{r}{M}>0$), since there exists $N$,
with $n\geq N$ implies $\left|\frac{|a_n|}{|a_{n+1}|}-0\right|=\frac{|a_n|}{|a_{n+1}|}<\frac{r}{M}$. Then, $\frac{1}{|a_n|/|a_{n+1}|}>\frac{M}{r}$.

So, for any $n\geq N$, the following is true:
$$\frac{|a_{n+1}r^{n+1}|}{|a_nr^n|}=\frac{r}{|a_n|/|a_{n+1}|} > r\frac{M}{r}=M$$
Since the choice of $M>0$ is arbitrary, then the sequence $\frac{|a_{n+1}r^{n+1}|}{|a_nr^n|}$ is not bounded, which according to ratio test, the series $\sum_{n=1}^{\infty}a_nr^n$ diverges.


Yet, since $0<|r|=r<R'$, it is in the radius of convergence, $\sum_{n=1}^{\infty}a_nr^n$ should converge, which is a contradiction.

So, the radius of convergence $R'\leq R=0$, which indicates that $R'=0$ is the radius of convergence.

\hfill

\hfill

\hfill

Regardless of the case, $R$ is always the radius of convergence, thus we can also define radius of convergence as $R=\lim_{n\rightarrow\infty}\frac{|a_n|}{|a_{n+1}|}$, if the limit is well-defined.

\break

\section*{4}
\begin{question}
    Ahlfors Pg. 41 Problem 9
\end{question}

\textbf{Pf:}

Given the following series $\sum_{n=1}^{\infty}\frac{z^n}{1+z^{2n}}$, to indicate the values of $z\in\mathbb{C}$ that lead to convergence, there are three cases:

\hfill

\textbf{(i): When $|z|<1$:}

For all $z\in\mathbb{C}$ with $|z|<1$, since $\lim_{n\rightarrow\infty}|z|^n=0$, choose $\epsilon=\frac{1}{2}$, there exists $N\in\mathbb{N}$, such that 
$n\geq N$ implies $|z|^n < \frac{1}{2}$ (or $-|z|^n > -\frac{1}{2}$). Then, for all $n\geq N$ (which $2n\geq N$), the following is true:
$$|1+z^{2n}| = |1-(-z^{2n})| \geq \left||1|-|-z^{2n}|\right| = 1-|z^{2n}| > 1-\frac{1}{2}=\frac{1}{2}$$
Thus, for $n\geq N$, $\frac{1}{2} < |1+z^{2n}|$, which indicates the following:
$$\frac{1}{|1+z^{2n}|}<2,\quad \left|\frac{z^n}{1+z^{2n}}\right| =\frac{|z^n|}{|1+z^{2n}|}<2|z^n|=2|z|^n$$
Now, consider $\sum_{n=N}^{\infty}\left|\frac{z^n}{1+z^{2n}}\right|$, since every term satisfies $0\leq\left|\frac{z^n}{1+z^{2n}}\right|<2|z|^n$, and the series $\sum_{n=N}^{\infty}2|z|^n$ 
converges due to the assumption that $|z|<1$, then by comparison test, the series $\sum_{n=N}^{\infty}\left|\frac{z^n}{1+z^{2n}}\right|$ converges, which implies
$\sum_{n=N}^{\infty}\frac{z^n}{1+z^{2n}}$ absolutely converges.

Thus, given $|z|<1$, the series $\sum_{n=1}^{\infty}\frac{z^n}{1+z^{2n}}$ converges.

\hfill

\hfill

\textbf{(ii): When $|z|=1$:}

For all $z\in\mathbb{C}$ with $|z|=1$, and any $n\in\mathbb{N}$, consider $\left|\frac{z^n}{1+z^{2n}}\right|$:
Suppose $1+z^{2n}\neq 0$, by the Triangle Inequality, since $|1+z^{2n}| \leq |1|+|z^{2n}|=2$, then:
$$\frac{1}{|1+z^{2n}|} \geq \frac{1}{2},\quad \left|\frac{z^n}{1+z^{2n}}\right|\geq\frac{|z^n|}{2}=\frac{1}{2}$$
This indicates that $\lim_{n\rightarrow\infty}\frac{z^n}{1+z^{2n}}\neq 0$, since choosing $\epsilon=\frac{1}{2}$, every $n\in\mathbb{N}$ with $(1+z^{2n})\neq 0$,
satisfies $\left|\frac{z^n}{1+z^{2n}}-0\right|\geq \frac{1}{2}=\epsilon$. 

Then, since the sequence $\frac{z^n}{1+z^{2n}}$ does not converge to $0$ for all $z$ with $|z|=1$, the series $\sum_{n=1}^{\infty}\frac{z^n}{1+z^{2n}}$ diverges.

\hfill

\hfill

\textbf{(iii): When $|z|>1$:}

For all $z\in\mathbb{C}$ with $|z|>1$, then for all $n\in\mathbb{N}$, $|z|^{2n}>|z|^n>1$. Also, since the sequence $|z|^n$ is strictly increasing and not bounded, there exists $N$, such that $n\geq N$ implies $|z|^{2n}>|z|^n>2$, or $\frac{1}{2}|z|^{2n}>1$.

Thus, the following is true:
$$|1+z^{2n}|=|z^{2n}-(-1)|\geq \left||z^{2n}|-|-1|\right| = |z|^{2n}-1 >|z|^{2n}-\frac{1}{2}|z|^{2n}=\frac{1}{2}|z|^{2n}$$
Which, the above inequality indicates the following:
$$\frac{1}{|1+z^{2n}|}<\frac{1}{\frac{1}{2}|z|^{2n}}=\frac{2}{|z|^{2n}},\quad \left|\frac{z^n}{1+z^{2n}}\right|=\frac{|z|^n}{|1+z^{2n}|}<\frac{2|z|^n}{|z|^{2n}} = \frac{2}{|z|^n} = 2\left|\frac{1}{z}\right|^n$$
If we consider the series $\sum_{n=N}^{\infty}\left|\frac{z^n}{1+z^{2n}}\right|$, since every term satisfies $0\leq \left|\frac{z^n}{1+z^{2n}}\right|<2\left|\frac{1}{z}\right|^n$, and the series $\sum_{n=N}^{\infty}2\left|\frac{1}{z}\right|^n$ converges since $\left|\frac{1}{z}\right|<1$, then by comparison test, the series $\sum_{n=N}^{\infty}\left|\frac{z^n}{1+z^{2n}}\right|$ converges.

Thus, the original series $\sum_{n=1}^{\infty}\frac{z^n}{1+z^{2n}}$ is absolutely converging.

\break

\section*{5}
\begin{question}
    Stein and Shakarchi Pg. 28 Problem 16 (e)
\end{question}

Given the hypergeometric series as:
$$F(\alpha,\beta,\gamma; z)=1+\sum_{n=1}^{\infty}\frac{\alpha(\alpha+1)...(\alpha+n-1)\beta(\beta+1)...(\beta+n-1)}{n!\gamma(\gamma+1)...(\gamma+n-1)}z^n$$
With $\alpha,\beta\in\mathbb{C}$, and $\gamma\notin \{-n\ |\ n\in\mathbb{N}\}$.

For all positive integer $n$, define the coefficient $a_n$ as follow:
$$a_n=\frac{\alpha(\alpha+1)...(\alpha+n-1)\beta(\beta+1)...(\beta+n-1))}{n!\gamma(\gamma+1)...(\gamma+n-1)}$$

\hfill

\textbf{(i): If $\alpha$ or $\beta$ are non-positive integers:}

Without Loss of Generality, can assume $\alpha$ is a non-positive integer (since interchanging $\alpha$ and $\beta$ doesn't affect the coefficient). Then, $\alpha=-k$ for some $k\in\mathbb{N}$. Which, for all index $n> k$, the coefficient's numerator involves a term $(\alpha+k) = (-k+k)=0$, which the coefficient $a_n=0$.

For all $N>k$, $a_N=0$, which the following partial sum can be expressed as:
$$\sum_{n=1}^{N}a_nz^n = \sum_{n=1}^{k}a_nz^n+\sum_{n=(k+1)}^{N}a_nz^n=\sum_{n=1}^{k}a_nz^n$$
Thus, the sequence of series $s_N = \sum_{n=1}^{N}a_nz^n = \sum_{n=1}^{k}a_nz^n = s_k$ for all $N>k$, which is eventually a constant sequence. So, the series $\sum_{n=1}^{\infty}a_nz^n$ converges.

Thus, for all $z\in\mathbb{C}$, $F(\alpha,\beta,\gamma;z) = 1+\sum_{n=1}^{\infty}a_nz^n$ is defined, which the radius of convergence is $R=\infty$.

\hfill

\textbf{(ii): If both $\alpha,\beta$ are not non-positive integers:}

Now, in \textbf{Question 3} it has proven, if the limit $\lim_{n\rightarrow\infty}\frac{|a_n|}{|a_{n+1}|}=R$ for some $R\in[0,\infty]$, then $R$ is precisely the radius of convergence.

Which, for all $n\in\mathbb{N}$, the ratio $\frac{|a_{n}|}{|a_{n+1}|}$ is defined as follow:
$$\frac{\alpha(\alpha+1)...(\alpha+n-1)\beta(\beta+1)...(\beta+n-1)}{n!\gamma(\gamma+1)...(\gamma+n-1)}\cdot \frac{(n+1)!\gamma(\gamma+1)...(\gamma+n-1)(\gamma+n)}{\alpha(\alpha+1)...(\alpha+n-1)(\alpha+n)\beta(\beta+1)...(\beta+n-1)(\beta+n)}$$
$$=\frac{(n+1)(\gamma+n)}{(\alpha+n)(\beta+n)} = \frac{n^2+(\gamma+1)n+\gamma}{n^2+(\alpha+\beta)n+\alpha\beta} = \frac{1+(\gamma+1)/n+\gamma/n^2}{1+(\alpha+\beta)/n+\alpha\beta/n^2}$$
Then, since $\lim_{n\rightarrow\infty}\frac{1}{n}=0$, then the following limit is defined as:
$$\lim_{n\rightarrow\infty}\frac{|a_{n}|}{|a_{n+1}|} = \lim_{n\rightarrow\infty}\frac{1+(\gamma+1)/n+\gamma/n^2}{1+(\alpha+\beta)/n+\alpha\beta/n^2} = \frac{1+(\gamma+1)\cdot 0+\gamma\cdot 0}{1+(\alpha+\beta)\cdot 0+\alpha\beta\cdot 0}=1$$
Which, the radius of convergence of hypergeometric series is $R=1$.

\break

\section*{6}
\begin{question}
    Stein and Shakarchi Pg. 29 Problem 19 (c)
\end{question}

\textbf{Pf:}

For all $z\in\mathbb{C}$ ($z\neq 1$) satisfying $|z|=1$, consider the following partial sum:
$$A_n = \sum_{i=0}^{n}z^n = \frac{1-z^{n+1}}{1-z}$$
Notice that for all $n\in\mathbb{N}$, the following inequality is true:
$$|A_n|=\left|\frac{1-z^{n+1}}{1-z}\right|\leq \frac{|1|+|z^{n+1}|}{|1-z|} = \frac{2}{|1-z|}$$
Thus, for given $z$ with $z\neq 1$ and $|z|=1$, the geometric partial sum $A_n$ is always bounded by $\frac{2}{|1-z|}$.

\hfill

\textbf{Summation by Part Formula:}

Given sequence $(a_n)_{n\in\mathbb{N}},(b_n)_{n\in\mathbb{N}}$, and let $A_N=\sum_{n=1}^{N}a_n$ (with $A_0=0$),
then for all $p,q\in\mathbb{N}$ (with $p<q$), the following formula is true:
$$\sum_{n=p}^{q}a_nb_n = \sum_{n=p}^{q}\left(\sum_{k=1}^{n}a_k-\sum_{k=1}^{n-1}a_k\right)b_n = \sum_{n=p}^{q}(A_n-A_{n-1})b_n$$
$$ = \sum_{n=p}^{q}A_nb_n-\sum_{n=p}^{q}A_{n-1}b_n = \sum_{n=p}^{q}A_nb_n-\sum_{n=(p-1)}^{(q-1)}A_{n}b_{n+1} $$
$$ = \sum_{n=p}^{q-1}A_n(b_n-b_{n+1})+A_qb_q-A_{p-1}b_{p}$$

\hfill

\textbf{Convergence of Series of Products:}

Now, suppose $(a_n)_{n\in\mathbb{N}}$ is a complex sequence, and $(b_n)_{n\in\mathbb{N}}$ is a real sequence, such that the partial sum of $a_n$ are all bounded
(i.e. there exists $M>0$, such that every $N\in\mathbb{N}$ satisfies $A_N=\sum_{n=1}^{N}a_n$ has $|A_N|<M$), and $b_n$ is a monotonic 
non-increasing sequence that converges to $0$ (i.e. for all $n\in\mathbb{N}$, $b_n\geq b_{n+1}$, and $\lim_{n\rightarrow\infty}b_n=0$; this also implies $b_n\geq 0$). Then, $\sum_{n=1}^{\infty}a_nb_n$ converges.

\hfill

To prove this, let $s_N = \sum_{n=1}^{N}a_nb_n$, the goal is to prove that the sequence $(s_N)_{N\in\mathbb{N}}$ is Cauchy.

First, by the convergence of $b_n$, for all $\epsilon>0$, since $\frac{\epsilon}{2M}>0$, there exists $N$, with $n\geq N$ implies $|b_n-0|=b_n<\frac{\epsilon}{2M}$. (Note: $M>0$ is the bound of $A_N$).

Then, for the same $\epsilon$ given, any $p,q> N$ with $p<q$ (Note: with $(p-1)\geq N$) satisfy the following:
$$|s_q-s_{p-1}| = \left|\sum_{n=1}^{q}a_nb_n-\sum_{n=1}^{p-1}a_nb_n\right| = \left|\sum_{n=p}^{q}a_nb_n\right|$$
$$=\left|\sum_{n=p}^{q-1}A_n(b_n-b_{n+1})+A_qb_q-A_{n-1}b_{n}\right| \leq \sum_{n=p}^{q-1}|A_n(b_n-b_{n+1})|+|A_qb_q|+|A_{n-1}b_n|$$
Which, since every $n\in\mathbb{N}$ satisfies $|A_n|<M$, then:
$$|s_q-s_{p-1}| \leq \sum_{n=p}^{q-1}|A_n(b_n-b_{n+1})|+|A_qb_q|+|A_{p-1}b_p| \leq \sum_{n=p}^{q-1}M|(b_n-b_{n+1})|+M|b_q|+M|b_p|$$
Also, since $b_n\geq b_{n+1}$ for all $n\in\mathbb{N}$, thus $(b_n-b_{n+1})\geq 0$; along with the condition that $b_n\geq 0$, the following is true:
$$|s_q-s_{p-1}|\leq \sum_{n=p}^{q-1}M|(b_n-b_{n+1})|+M|b_q|+M|b_p| = M\left(\sum_{n=p}^{q-1}(b_n-b_{n+1})+b_q+b_p\right)$$ 
$$|s_q-s_{p-1}|\leq M\left(\sum_{n=p}^{q-1}b_n - \sum_{n=p}^{q-1}b_{n+1}+b_q+b_p\right)$$
$$|s_q-s_{p-1}|\leq M\left(\sum_{n=p}^{q-1}b_n - \sum_{n=p+1}^{q}b_{n}+b_q+b_p\right)$$
$$|s_q-s_{p-1}|\leq M(b_p-b_q+b_q+b_p) = 2Mb_p$$
Now, since $p\geq N$, then by the convergence of $b_n$ constructed beforehand, $b_p<\frac{\epsilon}{2M}$. Thus:
$$|s_q-s_{p-1}|\leq 2Mb_p < 2M\frac{\epsilon}{2M} = \epsilon$$
Hence, the sequence $(s_N)_{N\in\mathbb{N}}$ is Cauchy, thus converges.

\hfill

\textbf{Convergence of $\sum_{n=1}^{\infty}z^n/n$ on unit circle:}

For any $z\neq 1$ with $|z|=1$, let $a_n=z^n$ and $b_n=\frac{1}{n}$ for all $n\in\mathbb{N}$.

From the first part, the partial sum of $a_n$ is bounded (proven that $|A_n|\leq \frac{2}{|1-z|}$), and $b_n=\frac{1}{n}$ is a nonincreasing sequence
that converges to $0$. Then, by the above statement, the series of product $\sum_{n=1}^{\infty}a_nb_n$ converges. Thus, the following series converges, given that $z\neq 1$ and $|z|=1$:
$$\sum_{n=1}^{\infty}\frac{z^n}{n}=L\in\mathbb{C}$$


\end{document}