% Math_CS_122A_HW8_Zih-Yu_Hsieh.tex

\documentclass{article}
\usepackage{graphicx} % Required for inserting images
\usepackage[margin = 2.54cm]{geometry}
\usepackage[most]{tcolorbox}

\newtcolorbox{myBox}[3]{
arc=5mm,
lower separated=false,
fonttitle=\bfseries,
%colbacktitle=green!10,
%coltitle=green!50!black,
enhanced,
attach boxed title to top left={xshift=0.5cm,
        yshift=-2mm},
colframe=blue!50!black,
colback=blue!10
}

\usepackage{amsmath}
\usepackage{amssymb}
\usepackage{verbatim}
\usepackage[utf8]{inputenc}
\linespread{1.2}

\newtheorem{definition}{Definition}
\newtheorem{proposition}{Proposition}
\newtheorem{theorem}{Theorem}
\newtheorem{question}{Question}

\title{Math CS 122A HW8}
\author{Zih-Yu Hsieh}

\begin{document}
\maketitle

\section*{1}
\begin{myBox}[]{}
    \begin{question}
        Ahlfors Pg. 148 Problem 2:

        Prove that the region obtained from a simply connected region by
        removing $m$ points has the connectivity $m + 1$, and find a homology basis.
    \end{question}
\end{myBox}

\textbf{Pf:}

Given the open region $\Omega$ is simply connected, which the complement $\Omega^C$ is connected in the extended complex plane.
Now, let $z_1,...,z_m\in \Omega$ denote the $m$ distinct points being reomoved. Then, the new region $\Omega' = \Omega\setminus\{z_1,...,z_m\}$,
and $(\Omega')^C = \Omega^C \cup \{z_1,...,z_m\}$.

First, all $i\in\{1,...,m\}$ has $\{z_i\}$ being disjoint from $\Omega^C$: Since $z_i\in\Omega$, while $\Omega$ is open, then there exists $\epsilon_i>0$,
with $B_{\epsilon_i}(z_i)\subseteq \Omega$. Hence, for all $a\in \Omega^C$, $d(z_i,a)\geq \epsilon_i$, showing that $z_i$ and $\Omega^C$ are disjoint.

Then, since $\{z_i\}$ and $\Omega^C$ are both closed under standard topology, the two being disjoint implies the two are not connected, hence belong to different connected components.

Furthermore, all distinct $i,j\in\{1,...,m\}$ have $\{z_i\},\{z_j\}$ being disjoint, since they're distinct points by assumption. Then, the two sets are also not connected, hence they belong to different connected components.

Then, because each set $\Omega^C, \{z_1\},...,\{z_m\}$ all belong to distinct connected component, while each set is connected ($\Omega^C$ is connected in the extended complex plane by assumption, while a singleton is always connected under standard topology),
then, there are $m+1$ connected components for the above collection. Hence, $(\Omega')^C$ has $m+1$ connected components, showing that the new region $\Omega'$ has connectivity $m+1$.

\hfil

\textbf{Homology Basis:}

Recall the above definition, each $i\in \{1,...,m\}$ exists $\epsilon_i>0$, with $d(z_i,a)\geq \epsilon_i$ for all $a\in\Omega^C$;
also, since for $j\neq i$, $d(z_i,z_j)>0$, then let $d = \min\{\epsilon_i\ |\ 1\leq i\leq m\} \cup \{d(z_i,z_j)\ |\ 1\leq i<j\leq m\}$.
Which, $d>0$, and $B_d(z_i)$ contains no points from other connected components
(for $a\in \Omega^C$, $d(z_i,a)\geq \epsilon_i\geq d$, showing that $a\notin B_d(z_i)$; and for $j\neq i$, $d(z_i,z_j)\geq d$, so $z_j\notin B_d(z_i)$ also). Therefore, $B_d(z_i)\setminus\{z_i\}$ is disjoint from $(\Omega')^C = \Omega^C\cup \{z_1,...,z_m\}$.

Then, for each index $i$, let cycle $\gamma_i$ be defined as the parametrization $z_i+\frac{d}{2}e^{i\theta}$ for $\theta\in [0,2\pi]$. 
Then, since for all $a\in\gamma_i$, $|a-z_i| = |\frac{d}{2}e^{i\theta}| = \frac{d}{2}<d$, then $a\in B_d(z_i)\setminus\{z_i\}$. showing that $\gamma_i\subset B_d(z_i)\setminus\{z_i\}$.
Therefore, $\gamma_i$ is disjoint from $(\Omega')^C$, which is contained fully in $\Omega'$.

Also, since $\gamma_i\subset B_d(z_i)$ (which is simply connected), for all $j\neq i$, since $z_j\notin B_d(z_i)$, then $n(\gamma_i,z_j)=0$; the same applies for all $a\in \Omega^C$ also.
And, if do the following integration, we get:
$$n(\gamma_i,z_i)=\frac{1}{2\pi i}\int_{\gamma_i}\frac{dz}{z-z_i} = \frac{1}{2\pi i}\int_{0}^{2\pi}\frac{1}{(z_i+\frac{d}{2}e^{i\theta})-z_i}i\frac{d}{2}e^{i\theta}d\theta = \frac{1}{2\pi i}\int_{0}^{2\pi}id\theta = 1$$
Hence, $\gamma_i$ has winding number $1$ for component $\{z_i\}$, while winding number $0$ for other components for the complement. Therefore, the collection $\gamma_1,...,\gamma_m$ forms a homology basis for $\Sigma'$.

\hfil

\hfil

\section*{2}
\begin{myBox}[]{}
    \begin{question}
        Ahlfors Pg. 148 Problem 4:

        Show that single-valued analytic branches of $\log z$, $z^\alpha$ and $z^z$ can be
        defined in any simply connected region which does not contain the origin.
    \end{question}
\end{myBox}

\textbf{Pf:}

\textbf{Single-Valued Branch of $\log(z)$:}

Given that $\Omega$ is a simply connected region that doesn't contain $0$, then since $z\neq 0$ in the region,
then based on \textbf{Corollary 2} of Generalized Cauchy's Theorem in the textbook (Ahlfors Pg. 142),
one can define a single-valued branch of $\log(z)$ in this region.

\hfil

\textbf{Single-Valued Branch of $z^\alpha$ and $z^z$:}

From the above part, since the single-valued branch of $\log(z)$ can be defined, then for all $\alpha\in\mathbb{R}$, $\alpha\log(z)$ and $z\log(z)$ both have single-valued branch.

Hence, $z^\alpha = e^{\alpha \log(z)}$ and $z^z = e^{z\log (z)}$ are also well-defined.

\break

\section*{3}
\begin{myBox}[]{}
    \begin{question}
        Ahlfors Pg. 148 Problem 5:

        Show that a single-valued analytic branch of $\sqrt{1-z^2}$ can be
        defined in any region such that the points $\pm 1$ are in the same component
        of the complement. What are the possible values of
        $$\int\frac{dz}{\sqrt{1-z^2}}$$
        over a closed curve in the region?
    \end{question}
\end{myBox}

\textbf{Pf:}

Assume $\Omega$ is the open region, where $1,-1$ are in the same connected component of the complement.
Which, for all cycle $\gamma\subset \Omega$, the winding number $n(\gamma,1)=n(\gamma,-1)$.

\hfil

\textbf{Single-Valued Branch:}

First, consider the analytic function $\left(\frac{1}{z+1}-\frac{1}{z-1}\right)$ on $\Omega$: For all cycle $\gamma\subset \Omega$,
the following integral is true:
$$\int_\gamma\left(\frac{1}{z+1}-\frac{1}{z-1}\right)dz = \left(\int_\gamma\frac{1}{z+1}dz - \int_\gamma\frac{1}{z-1}dz\right) = 2\pi i(n(\gamma,1)-n(\gamma,-1)) = 0$$
Then, this implies that an antiderivative $F(z)$ exists on $\Omega$ (with $F'(z)=\frac{1}{2}\left(\frac{1}{z+1}-\frac{1}{z-1}\right)$). Then, since the original function can be rewrite as:
$$F'(z)=\left(\frac{1}{z+1}-\frac{1}{z-1}\right)=\frac{(z-1)-(z+1)}{(z+1)(z-1)}=\frac{-2}{z^2-1} = \frac{2}{1-z^2}$$
Hence, $F(z)$ is an antiderivative of $\frac{2}{1-z^2}$.

(Note: In real numbers, the antiderivative of the above equation is also written as $\ln(x+1)-\ln(x-1)=\ln\frac{x+1}{x-1}$, which will be a tool for guess here).

\hfil

Now, consider the equation $\frac{z+1}{z-1}e^{-F(z)}$, and its derivative:
$$\frac{z+1}{z-1}=1+\frac{2}{z-1},\quad \frac{d}{dz}\left(\frac{z+1}{z-1}\right)=\frac{d}{dz}\left(\frac{2}{z-1}\right) = \frac{-2}{(z-1)^2}$$

$$\frac{d}{dz}\left(\frac{z+1}{z-1}e^{-F(z)}\right) = \frac{-2}{(z-1)^2}e^{-F(z)}+\frac{z+1}{z-1}\left(-\frac{dF}{dz}\right)e^{-F(z)}$$
$$= \frac{-2}{(z-1)^2}e^{-F(z)}+\frac{z+1}{z-1}\left(\frac{-2}{1-z^2}\right)e^{-F(z)}$$
$$= \frac{-2}{(z-1)^2}e^{-F(z)}+\frac{z+1}{z-1}\left(\frac{-2}{(1-z)(1+z)}\right)e^{-F(z)}$$
$$= \frac{-2}{(z-1)^2}e^{-F(z)}+\frac{2}{(z-1)^2}e^{-F(z)} = 0$$
Then, since the derivative is $0$, then the function is in fact a constant over $\Omega$; and, since $\Omega$ excludes both $\pm 1$, then the value of $\frac{z+1}{z-1}e^{-F(z)}$ is always nonzero.
Hence, $C=\frac{z+1}{z-1}e^{-F(z)}\neq 0$.

Which, rewrite the function, we get:
$$C=\frac{z+1}{z-1}e^{-F(z)} = \frac{(z+1)^2}{(z^2-1)}e^{-F(z)} = -\frac{(z+1)^2}{1-z^2}e^{-F(z)}$$
Hence, we can rewrite it as the follow:
$$1-z^2 = -\frac{1}{C}(z+1)^2e^{-F(z)}$$
Then, define the function $\frac{i}{\sqrt{C}}(z+1)e^{-\frac{F(z)}{2}}$, we get:
$$\left(\frac{i}{\sqrt{C}}(z+1)e^{-\frac{F(z)}{2}}\right)^2 = -\frac{1}{C}(z+1)^2e^{-F(z)} = 1-z^2$$
Hence, define this branch as $\sqrt{1-z^2}=\frac{i}{\sqrt{C}}(z+1)e^{-\frac{F(z)}{2}}$, it is a well-defined single-valued branch.

\hfil

\textbf{Integral of $1/\sqrt{1-z^2}$:}


\begin{comment}
From the above part, we can define $\sqrt{1-z^2}$ as $D(z+1)e^{-\frac{F(z)}{2}}$ for constant $D=\frac{i}{\sqrt{C}}$ as defined above. Then, for any closed curve $\gamma\in\Omega$, the following integral can be rewrite as:
$$\int_\gamma \frac{1}{\sqrt{1-z^2}}dz = \int_\gamma\sqrt{1-z^2}\cdot \frac{1}{1-z^2}dz = \int_\gamma D(z+1)e^{-\frac{F(z)}{2}}\cdot \frac{1}{2}\cdot\frac{2}{1-z^2}dz = \int_\gamma D(z+1)e^{-\frac{F(z)}{2}}\cdot \frac{1}{2}\cdot F'(z)dz$$
(Note: Recall that $F'(z)=\frac{1}{1-z^2}$ from the construction in the first part).

Then, let define the parametrization $\gamma:[a,b]\rightarrow\mathbb{C}$ for some nonempty closed interval $[a,b]$, we get:
$$\int_\gamma \frac{1}{\sqrt{1-z^2}}dz = \int_{a}^{b}D(\gamma(t)+1)e^{-\frac{F(\gamma(t))}{2}}\cdot\frac{F'(\gamma(t))}{2}\cdot\gamma'(t)dt$$ 
$$ = -D(\gamma(t)+1)e^{-\frac{F(\gamma(t))}{2}}\bigg|_{a}^{b} + \int_{a}^{b}D\cdot\gamma'(t)\cdot e^{-\frac{F(\gamma(t))}{2}}dt$$
$$=\int_\gamma De^{-\frac{F(z)}{2}}dz$$
(Note: The first line is parametrization, the second line is done through integration by part, and the third line is gotten by the fact that $\gamma$ is a closed curve, hence $\gamma(b)=\gamma(a)$, which the first part evaluated to be $0$).

Then, since $\sqrt{1-z^2}=D(z+1)e^{-\frac{F(z)}{2}}$, then $De^{-\frac{F(z)}{2}}=\frac{\sqrt{1-z^2}}{(1+z)}=\frac{\sqrt{1-z^2}(1-z)}{1-z^2}=\frac{1-z}{\sqrt{1-z^2}}$. Hence, the above integral becomes:
$$\int_\gamma \frac{1}{\sqrt{1-z^2}}dz=\int_\gamma De^{-\frac{F(z)}{2}}dz = \int_\gamma \frac{1-z}{\sqrt{1-z^2}}dz$$
\end{comment}

\end{document}