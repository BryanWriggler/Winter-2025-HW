% Math_CS_122A_HW5_Zih-Yu_Hsieh.tex
% Note: HW 5 is not required to turn in

\documentclass{article}
\usepackage{graphicx} % Required for inserting images
\usepackage[margin = 2.54cm]{geometry}
\usepackage[most]{tcolorbox}

\newtcolorbox{myBox}[3]{
arc=5mm,
lower separated=false,
fonttitle=\bfseries,
%colbacktitle=green!10,
%coltitle=green!50!black,
enhanced,
attach boxed title to top left={xshift=0.5cm,
        yshift=-2mm},
colframe=blue!50!black,
colback=blue!10
}

\usepackage{amsmath}
\usepackage{amssymb}
\usepackage{verbatim}
\usepackage[utf8]{inputenc}
\linespread{1.2}

\newtheorem{definition}{Definition}
\newtheorem{proposition}{Proposition}
\newtheorem{lemma}{Lemma}
\newtheorem{theorem}{Theorem}
\newtheorem{question}{Question}

\title{Math CS 122A HW6}
\author{Zih-Yu Hsieh}

\begin{document}
\maketitle

\section*{1}
\begin{myBox}[]{}
    \begin{question}
        Ahlfors Pg. 123 Problem 2:

        Prove that a function which is analytic in the whole plane and
        satisfies an inequality $|f(z)| < |z|^n$ for some nand all sufficiently large $|z|$
        reduces to a polynomial.
    \end{question}
\end{myBox}

\textbf{Pf:}

Given that for $z\in\mathbb{C}$ with $|z|$ being sufficiently large, $|f(z)|< |z|^n$ is satisfied, 
then there exists a radius $r>0$, such that $|z|\geq r$ implies $|f(z)|<|z|^n$. Which, we'll consider the $n^{th}$ derivative,
$f^{(n)}(z)$. (Note: Since $f$ is analytic on the whole plane, then all of its derivative exists, and is analytic on the whole plane).

\hfill

First, consider the disk $D_{2r} = \{z\in\mathbb{C}\ |\ |z|\leq 2r\}$: Since it is a closed and bounded set,
then it is compact. Hence, since $|f^{(n)}|$ is also continuous due to the analytic nature of $f^{(n)}$, 
then $|f^{(n)}|(D_{2r}) \subseteq \mathbb{R}$ is also compact, there exists $M>0$, such that for all $z\in D_{2r}$,
$|f^{(n)}(z)| \leq M$.

\hfill

Then, for all $z\in \mathbb{C}\setminus D_{2r}$, we'll consider $f^{(n)}(z)$ using Cauchy's Integral Formula:
Let $\gamma$ be the curve of the circle $|z|=r$, then for all $z$ not on the given circle, the following is true:
$$f^{(n)}(z) = \frac{n!}{2\pi i}\int_{\gamma}\frac{f(\zeta)}{(\zeta-z)^{n+1}}d\zeta$$
Which, for $z\in \mathbb{C}\setminus D_{2r}$ , since $|z| > 2r > r$, then for all $\zeta \in \gamma$ (which $|\zeta|=r$), the following is true:
$$|\zeta-z| \geq ||\zeta|-|z|| = |r-|z|| = |z|-r > 2r-r = r,\quad |\zeta-z|^{n+1}> r^{n+1},\quad \frac{1}{|\zeta-z|^{n+1}} < \frac{1}{r^{n+1}}$$
Similarly, since $|\zeta|\geq r$, then based on the assumption, $|f(\zeta)| < |\zeta|^n = r^n$.
Hence, the following inequality is true:
$$|f^{(n)}(z)| = \left|\frac{n!}{2\pi i}\int_{\gamma}\frac{f(\zeta)}{(\zeta-z)^{n+1}}d\zeta\right| \leq \frac{n!}{2\pi}\int_{\gamma}\frac{|f(\zeta)|}{|\zeta-z|^{n+1}}\cdot|d\zeta| < \frac{n!}{2\pi}\int_{\gamma}\frac{r^n}{r^{n+1}}\cdot|d\zeta|$$
$$|f^{(n)}(z)| < \frac{n!}{2\pi}\cdot\frac{1}{r}\cdot 2\pi r = n!$$
(Note: the first inequality is true, based on the statement that $|f(\zeta)|< r^n$ and $\frac{1}{|\zeta-z|^{n+1}}<\frac{1}{r^{n+1}}$).

Hence, take $M'=\max\{M, n!\}$, then for all $z\in\mathbb{C}$, if $z\in D_{2r}$, then $|f^{(n)}(z)| \leq M \leq M'$; else if $z\in \mathbb{C}\setminus D_{2r}$, then $|f^{n}(z)| \leq n! \leq M'$.
So, the analytic function $f^{(n)}(z)$ is bounded on the whole plane, which by Liouville's Theorem, $f^{(n)}(z)$ must be a constant function.

Then, since the $n^{th}$ derivative of $f$ is a constant, then $f$ must be a polynomial (in fact, with degree at most $n$).

\break

\section*{2}
\begin{myBox}[]{}
    \begin{question}
        Ahlfors Pg. 123 Problem 5:

        Show that the successive derivatives of an analytic function at a
        point can never satisfy $|f^{(n)}(z)| > n!n^n$. Formulate a sharper theorem of
        the same kind.
    \end{question}
\end{myBox}

\textbf{Pf:}

Let the analytic function $f$ be defined on an open set $\Omega$, which for all $z_0\in\Omega$, 
there exists $r'>0$, such that the open disk $|z-z_0| < r'$ is within $\Omega$. If we let $r=\frac{r'}{2}>0$,
then the closed disk $|z-z_0|\leq r$ is fully contained in $|z-z_0|<r'$, which is within $\Omega$.

\hfill

Now, let $\gamma$ be the circle $|z-z_0|=r$, since it is a compact set where $|f|$ is continuous since $f$ is analytic,
then $|f|(\gamma)\subseteq \mathbb{R}$ has a maximum, there exists $M>0$, such that for all $z\in \gamma$, $|f(z)|\leq M$ (For simplicity, choose $M\geq 1$).

Hence, based on Cauchy's Integral Formula, for all $n\in\mathbb{N}$, the following formula is true:
$$f^{(n)}(z_0) = \left|\frac{n!}{2\pi i}\int_{\gamma}\frac{f(\zeta)}{(\zeta-z_0)^{n+1}}\right| \leq \frac{n!}{2\pi}\int_{\gamma}\frac{|f(\zeta)|}{|\zeta-z_0|^{n+1}}\cdot|d\zeta| \leq \frac{n!}{2\pi}\int_{\gamma}\frac{M}{r^{n+1}}\cdot|d\zeta|$$
$$f^{(n)}(z_0) \leq \frac{n!}{2\pi} \cdot \frac{M}{r^{n+1}}\cdot 2\pi r = \frac{n!M}{r^n}$$
(Note: For all $\zeta\in \gamma$, $|\zeta-z_0|=r$, and $|f(\zeta)|\leq M$).

Notice that since $\frac{M}{r}>0$, then by Archimedean's Property, there exists $k\in\mathbb{N}$, with $k> \frac{M}{r}$,
which since $M\geq 1$ is assumed, the following inequality is true:
$$k^k > \left(\frac{M}{r}\right)^k = \frac{M^k}{r^k} \geq \frac{M}{r^k},\quad |f^{(k)}(z_0)| \leq \frac{k!M}{r^k} < k!k^k$$
Also, for all integer $n\geq k$, the following is satisfied:
$$n^n \geq k^n > \left(\frac{M}{r}\right)^n = \frac{M^n}{r^n} \geq \frac{M}{r^n},\quad |f^{(n)}(z_0)| \leq \frac{n!M}{r^n} < n!n^n$$
So, for all $z_0\in\Omega$, there exists $k\in\mathbb{N}$, such that $n\geq k$ implies $|f^{(n)}(z_0)| < n!n^n$, showing that $|f^{(n)}(z)| > n!n^n$ can never be satisfied by any point $z$ and for all but finitely many $n\in\mathbb{N}$.

\hfill

\textbf{Stronger Condition:}

Recall that for all $r_0>0$, by Archimedean's Property, there exists $N\in\mathbb{N}$ with $N>r_0$. Therefore, for $n\geq N$,
$\frac{r_0^{n+1}}{(n+1)!}  = \frac{r_0}{(n+1)}\cdot \frac{r_0^n}{n!} < \frac{r_0}{N}\cdot \frac{r_0^n}{n!}$, which for all positive integer $k$, we can inductively prove that
$\frac{r_0^{N+k}}{(N+k)!} < \left(\frac{r_0}{N}\right)^k\cdot\frac{r_0^N}{N!}$. 

Hence, since $\frac{r_0}{N}<1$, then the following is true:
$$0 < \frac{r_0^{N+k}}{(N+k)!} < \left(\frac{r_0}{N}\right)^k\cdot \frac{r_0^N}{N!}$$
$$0 \leq \lim_{k\rightarrow \infty} \frac{r_0^{N+k}}{(N+k)!} \leq \lim_{k\rightarrow \infty} \left(\frac{r_0}{N}\right)^k\cdot \frac{r_0^N}{N!} = 0$$
Which, $\lim_{n\rightarrow \infty}\frac{r_0^n}{n!} = 0$ based on the above inequality, so there exists $N\in\mathbb{N}$, such that $n\geq N$ implies $\frac{r_0^n}{n!} <1$, or $r_0^n < n!$.

Then, looking back to the inequality $|f^{(n)}(z_0)| \leq \frac{n!M}{r^n}$, since $\frac{M^{1/n}}{r}>0$, there exists $N$, such that $n\geq N$ implies $\frac{M}{r^n}=\left(\frac{M^{1/n}}{r}\right)^n < n!$. Hence, the following inequality is true:
$$|f^{(n)}(z_0)| \leq \frac{n!M}{r^n}  < n!\cdot n! = (n!)^2$$
So, we can conclude that for some $N\in\mathbb{N}$, $n\geq N$ implies $|f^{(n)}(z_0)| < (n!)^2$,
which is a stricter condition than $n!n^n$, since $\lim_{n\rightarrow\infty}\frac{n!}{n^n}=0$ (Because this implies that for all sufficiently large $n$, $n!<n^n$).

\hfill

\hfill

\section*{3}
\begin{myBox}[]{}
    \begin{question}
        Ahlfors Pg. 130 Problem 2:

        Show that a function which is analytic in the whole plane and has
        a nonessential singularity at $\infty$ reduces to a polynomial.
    \end{question}
\end{myBox}

\textbf{Pf:}

Before starting the proof, we'll first prove a lemma for this problem:

\begin{lemma}
    Suppose $f$ is analytic in the whole plane, and $\lim_{z\rightarrow\infty}f(z)$ converges to some value in $\mathbb{C}$, then $f$ is a constant function.
\end{lemma}

To show this, since $\lim_{z\rightarrow\infty}f(z)=L$ for some $L\in\mathbb{C}$, then for all $\epsilon>0$, there exists $M_1>0$,
such that $|z|>M_1$ implies $|f(z)-L|<\epsilon$, hence $|f(z)|=|(f(z)-L)+L| \leq |f(z)-L|+|L| < |L|+\epsilon$.

Also, consider the closed disk $\mathbb{D}_{M_1}$: Since it is a bounded closed set, then it is compact. Hence, since $|f|$ is continuous,
$|f|(\mathbb{D}_{M_1})\subseteq \mathbb{R}$ is also a compact set, hence there exists $M_2>0$, such that for all $z\in\mathbb{D}_{M_1}$,
$|f(z)| \leq M_2$.

Then, fix $\epsilon>0$ and take $M=\max\{|L|+\epsilon,M_2\}$, for all $z\in\mathbb{C}$, if $z\in\mathbb{D}_{M_1}$, then $|f(z)|\leq M_2\leq M$; else if $z\notin \mathbb{D}_{M_1}$,
then $|z|>M_1$, hence $|f(z)| < |L|+\epsilon \leq M$. This shows that $f$ is bounded on the whole plane,
which by Liouville's Theorem, $f$ must be a constant function, and the Lemma has been proven.

\hfill

\textbf{Function is a Polynomial:}

Given that $f$ only has a nonessential singularity at $\infty$, while analytic on the whole plane, there are two cases to consider: First, if $f$ is a removable singularity (which $\lim_{z\rightarrow\infty}f(z)=L$ for some $L\in\mathbb{C}$),
or $f$ has a pole at $\infty$.

\hfill

For the first case, since $f$ is analytic on the whole plane, while $lim_{z\rightarrow\infty}f(z)$ converges to some $L\in\mathbb{C}$, then apply \textbf{Lemma 1},
we get that $f$ is a constant function, which is a polynomial.

\hfill

For the second case, take $g(z)=f(\frac{1}{z})$ as the analytic function, then since $\lim_{z\rightarrow 0}g(z)= \lim_{z\rightarrow 0}f(\frac{1}{z})=\infty$ ($f(z)$ has a pole at $\infty$, which $g(z)=f(\frac{1}{z})$ has a pole at $0$),
then suppose $g$ has the pole at $0$ with order $h\in\mathbb{N}$, there exists some analytic function $g_h(z)$ that is analytic at $0$,
with $g(z)=\frac{1}{z^h}g_h(z)$.

Notice that since $g(z)=f(\frac{1}{z})=\frac{1}{z^h}g_h(z)$ is analytic for all $z\neq 0$ (because $\frac{1}{z}\in\mathbb{C}$, while $f$ is analytic on the whole plane),
then $g_h(z) = z^hg(z)$ is analytic at all $z\neq 0$, while also analytic at $0$ from the above statement, hence $g_h(z)$ is analytic on the whole plane $\mathbb{C}$.

Also, $\lim_{z\rightarrow\infty}g(z)$ must also converge to some value in $\mathbb{C}$, since it is equivalent to $\lim_{z\rightarrow\infty}f(\frac{1}{z})=\lim_{z\rightarrow 0}f(z)=f(0)$.
Then, this implies for all $\epsilon>0$, there exists $M>0$, such that $|z|>M$ implies $|g(z)-f(0)| < \epsilon$,
hence, the following is true:
$$|g(z)| = |(g(z)-f(0))+f(0)| \leq |g(z)-f(0)|+|f(0)| < |f(0)|+\epsilon$$
$$|g(z)|=\left|\frac{1}{z^h}g_h(z)\right| < |f(0)|+\epsilon,\quad |g_h(z)| < (|f(0)|+\epsilon)|z|^h$$
Which, let $C=|f(0)|+\epsilon>0$, for all $z$ with $|z|>M$, we have $|g_h(z)| < C|z|^h$; which let $M'=\max\{M, C\}$, when $|z|>M'$,
since $|z|>C$ and $|z|>M$, the following is true:
$$|g_h(z)|<C|z|^h < |z|^{h+1}$$
Now, apply the statement proven in \textbf{Question 1}, since $g_h(z)$ is analytic on the whole plane, while for sufficiently large $|z|$, $|g_h(z)|< |z|^{h+1}$ (with $(h+1)\in\mathbb{N}$ being fixed),
then $g_h(z)$ reduces to a polynomial, and with degree at most $(h+1)$. Hence, $g_h(z)=a_0+a_1z+...+a_{h+1}z^{h+1}$ for some $a_0,a_1,...,a_{h+1}\in\mathbb{C}$.

Then, recall the fact that $f(\frac{1}{z})=g(z)$, then $f(z)=g(\frac{1}{z})=\frac{1}{(1/z)^h}g_h(\frac{1}{z})$, we get the following:
$$f(z)=z^h\cdot \left(a_0+a_1\frac{1}{z}+...+a_{h+1}\frac{1}{z^{h+1}}\right) = a_0z^h + a_1z^{h-1}+...+a_h+a_{h+1}\frac{1}{z}$$
Also, notice that since $f(0)$ is defined, then we must have $a_{h+1}=0$ (or else the above function wouldn't be defined at $0$).

Hence, $f(z)=a_0z^h + a_1z^{h-1}+...+a_h$, which is a polynomial.

\hfill

\hfill

Regardless of the case, we always get the fact that $f(z)$ is a polynomial.

Hence, having $f$ being analytic on the whole plane,
while having a nonessential singularity at $\infty$ implies that $f(z)$ must be a polynomial.


\break

\section*{4}
\begin{myBox}[]{}
    \begin{question}
        Ahflors Pg. 130 Problem 6:

        Show that an isolated singularity of $f(z)$ cannot be a pole of $\exp(f(z))$.
    \end{question}
\end{myBox}

\textbf{Pf:}

Suppose $f(z)$ has an isolated singularity at $a\in\mathbb{C}$, there are three cases to consider:
Whether $a$ is a removable singularity, an essential singularity, or a pole at $a$.

\hfill

\textbf{When $f$ has a Removable Singularity at $a$:}

Since it's removable at $a$, $\lim_{z\rightarrow a}f(z) = L$ for some $L\in\mathbb{C}$, hence $\lim_{z\rightarrow a}e^{f(z)}=e^L\in\mathbb{C}$,
so it is also a removable singularity for $e^f$, not a pole.

\hfill

\textbf{When $f$ has an Essential Singularity at $a$:}

Since it is an essential singularity at $a$, for all $z\in\mathbb{C}$ (except possibly finitely many points), every neighborhood of $a$ has value of $f$
being arbitrarily close to $z$, then take $\log(z)\in\mathbb{C}$ (except for $z=0$), since $f$ can get arbitrarily close to $\log(z)$ in any of the neighborhood of $a$,
then $e^f$ can get arbitrarily close to $e^{log(z)} = z$ for every neighborhood around $a$, showing that $e^f$ also has an essential singularity at $a$, instead of a pole.

\hfill

\textbf{When $f$ has a Pole at $a$:}

Suppose $f$ has a pole of order $h$ at $a$, then $f(z)=\frac{1}{(z-a)^h}f_h(z)$ for some analytic function $f_h(z)$ that's analytic at $a$ (so, $f_h(a)$ is defined, and $f_h(a)\neq 0$).

First, for all $w\in\mathbb{C}$, $w=|w|\cdot e^{i\cdot \arg(w)}$, for $k\in\mathbb{N}$, define $w^\frac{1}{k} = |w|^\frac{1}{k}\cdot e^{i\cdot\frac{\arg(w)}{k}}$ for simplicity.

Now, for all $z\in\mathbb{C}$ (where $z\neq 0$), since $e^{f_h(a)}\neq 0$, then $z\cdot e^{f_h(a)}$ can be arbitrary member of $\mathbb{C}\setminus\{0\}$. Consider the sequence $(z_n)_{n\in\mathbb{N}}$ defined by:
$$\forall n\in\mathbb{N},\quad z_n=\left(\frac{1}{\ln|z|+i\cdot(\arg(z)+2n\pi)}\right)^\frac{1}{h}+a$$
Then, since as $\lim_{n\rightarrow\infty}z_n=a$ (because the first fraction term converges to $0$ due to an unbounded denominator), then $\lim_{n\rightarrow\infty}f(z_n)=\infty$.

However, when consider $e^{f(z_n)}$, we yield the following:
$$e^{f(z_n)}=\exp\left(\frac{1}{(z_n-a)^h}f_h(z_n)\right)=\exp\left(\frac{1}{\left(\left(\frac{1}{\ln|z|+i\cdot(\arg(z)+2n\pi)}\right)^\frac{1}{h}+a-a\right)^h}f_h(z_n)\right)$$
$$=\exp\left(\frac{1}{1/(\ln|z|+i\cdot(\arg(z)+2n\pi))}f_h(z_n)\right) = \exp\left(\ln|z|+i\cdot(\arg(z)+2n\pi)\right)\cdot e^{f_h(z_n)}$$
$$= z\cdot e^{f_h(z_n)}$$
Which, taking $n\rightarrow\infty$, we get the following:
$$\lim_{n\rightarrow\infty}e^{f(z_n)} = \lim_{n\rightarrow\infty}ze^{f_h(z_n)} = z\cdot e^{f_h(a)}$$
Since $z\cdot e^{f_h(a)}$ can be arbitrary within $\mathbb{C}\setminus\{0\}$, then for any neighborhood of $a$, the function $e^{f(z)}$ can be close to arbitrary values besides $0$
(since there exists a sequence $(z_n)_{n\in\mathbb{N}}$ such that $e^{f(z_n)}$ converging to such arbitrary value),
then $e^f$ has an essential singularity at $a$, instead of having a pole.

\hfill

Hence, in all cases, when $f$ has an isolated singularity at $a$, $e^f$ has either a removable, or essential singularity at $a$, showing that an isolated singularity of $f(z)$ cannot be a pole of $e^{f(z)}$.

\hfill

\hfill

\section*{5}
\begin{myBox}[]{}
    \begin{question}
        Stein and Shakarchi Pg. 66 Problem 7:
        
        Suppose $f:\mathbb{D}\rightarrow \mathbb{C}$ is holomorphic. Show that the diameter $d=\sup_{z,w\in\mathbb{D}}|f(z)-f(w)|$ 
        of the image of $f$ satisfies $2|f'(0)| \leq d$.

        Moreover, it can be shown that equality holds precisely when $f$ is linear, $f(z) =
        a_0 + a_1z$.
    \end{question}
\end{myBox}

\textbf{Pf:}

Fix an $r\in(0,1)$, and let $\gamma$ be the circle $|z|=r$. Then, based on Cauchy's Integral Formula, $f'(0)$ can be extracted as:
$$f'(0)=\frac{1}{2\pi i}\int_{\gamma}\frac{f(\zeta)}{\zeta^2}d\zeta$$
Which, if consider $f(-z)$ instead, its derivative is given by $-f'(-z)$, hence with $z=0$, the following is true:
$$-f'(0)=\frac{1}{2\pi i}\int_{\gamma}\frac{f(-\zeta)}{\zeta^2}d\zeta,\quad f'(0)=-\frac{1}{2\pi i}\int_{\gamma}\frac{f(-\zeta)}{\zeta^2}d\zeta$$
Which, adding the two equations together, we yield:
$$2f'(0)=\frac{1}{2\pi i}\int_{\gamma}\frac{f(\zeta)-f(-\zeta)}{\zeta^2}d\zeta$$
Given that $g(z)=f(z)-f(-z)$ is an analytic function (sum and composition of analytic functions), and $\gamma$ is a compact set (boundary of a bounded circle),
then since $|g|$ is continuous, $|g|(\gamma)\subseteq \mathbb{R}$ is a compact set. Hence, there exists a maximum on $\gamma$, for some $z_0\in\gamma$, all $z\in\gamma$ satisfies:
$$|f(z)-f(-z)|=|g(z)|\leq |g(z_0)|=|f(z_0)-f(-z_0)|$$
And, notice that since $z_0, -z_0$ have modulus $r$ (since they're on $\gamma$), then $z_0,-z_0\in \mathbb{D}$. Therefore,
$|g(z_0)|=|f(z_0)-f(-z_0)| \leq \sup_{z,w\in\mathbb{D}}|f(z)-f(w)| = d$, thus for all $z\in\gamma$, $|g(z)|\leq d$.

\hfill

Now, going back to the third equation, since for all $\zeta\in\gamma$, it satisfies $|g(\zeta)|\leq d$ and $|\zeta| = r$, we can conclude the following inequality:
$$2|f'(0)| = |2f'(0)| = \left|\frac{1}{2\pi i}\int_{\gamma}\frac{f(\zeta)-f(-\zeta)}{\zeta^2}d\zeta\right|\leq\frac{1}{2\pi}\int_{\gamma}\frac{|g(\zeta)|}{|\zeta|^2}\cdot|d\zeta| \leq \frac{1}{2\pi}\int_{\gamma}\frac{d}{r^2}\cdot |d\zeta|$$
$$2|f'(0)| \leq \frac{1}{2\pi}\cdot \frac{d}{r^2}\cdot 2\pi r = \frac{d}{r}$$
Also, notice that the above inequality is true for all $r\in (0,1)$.

(Note: since for all $r\in (0,1)$, $\frac{1}{r}>1$, then $\frac{d}{r}>d$; on the other hand, for all $\epsilon>0$, since $d+\epsilon/2 = d(1+\frac{\epsilon}{2d}) = \frac{d}{1/(1+\frac{\epsilon}{2d})}$,
because $(1+\frac{\epsilon}{2d})>1$, then $0<1/(1+\frac{\epsilon}{2d})<1$, hence $(1+\frac{\epsilon}{2d})\in (0,1)$, or $d+\epsilon/2=\frac{d}{(1+\frac{\epsilon}{2d})}\in \{\frac{d}{r}\ |\ r\in(0,1)\}$.
However, since $d+\epsilon/2 < d+\epsilon$, then $d+\epsilon$ is not a lower bound of the set. Hence, $d=\inf\{\frac{d}{r}\ |\ r\in(0,1)\}$).

Since $d = \inf\{\frac{d}{r}\ |\ r\in(0,1)\}$,then, since $2|f'(0)| \leq \frac{d}{r}$ for all $r\in(0,1)$, then $2|f'(0)|$ is a lower bound of $\{\frac{d}{r}\ |\ r\in(0,1)\}$,
hence $2|f'(0)| \leq \inf\{\frac{d}{r}\ |\ r\in(0,1)\} = d$.

\hfill

\textbf{Equality under Linear Functions:}

Let $f(z)=a_0+a_1z$ for $a_0,a_1\in\mathbb{C}$, then $f'(z) = a_1$, hence $2|f'(0)| = 2|a_1|$. (Note: if $|a_1|=0$, then $f'=0$, which $f$ is a constant, hence $d=\sup_{z,w\in\mathbb{D}}|f(z)-f(w)|=0$ regardless; so, assume $|a_1|\neq 0$).

Now, for all $z,w\in\mathbb{D}$ (which $|z|,|w|<1$), $f(z)-f(w)=(a_0+a_1z)-(a_0+a_1w) = a_1(z-w)$, which: 
$$|f(z)-f(w)| = |a_1|\cdot |z-w| \leq |a_1|\cdot (|z|+|w|) < 2|a_1|$$
So, we can first conclude that $d=\sup_{z,w\in\mathbb{D}}|f(z)-f(w)| \leq 2|a_1|$ (since $2|a_1|$ is an upper bound).

Then, for all $\epsilon>0$ (and let $\epsilon<4|a_1|$ for simplicity), consider $2|a_1|-\epsilon$:
Let $z=(1-\frac{\epsilon}{4|a_1|}) \in \mathbb{D}$ (which $0<z<1$, since $0<\frac{\epsilon}{4|a_1|}<\frac{4|a_1|}{4|a_1|}=1$), consider $|f(z)-f(-z)|$:
$$|f(z)-f(-z)| = |a_1|\cdot |z-(-z)| = |a_1|\cdot |2z| = 2|a_1|\cdot|z| = 2|a_1|\cdot z = 2|a_1|\left(1-\frac{\epsilon}{4|a_1|}\right)$$
$$|f(z)-f(-z)| = 2|a_1|-\frac{\epsilon}{2} > 2|a_1|-\epsilon$$
Since $z,-z\in \mathbb{D}$, then $|f(z)-f(-z)|>(2|a_1|-\epsilon)$ indicates that $(2|a_1|-\epsilon)$ is no longer an upper bound.
Hence, $2|a_1|$ is in fact $\sup_{z,w\in\mathbb{D}}|f(z)-f(w)|$.

\hfill

So, recall that $2|f'(0)|=2|a_1|$, then $2|f'(0)| = \sup_{z,w\in\mathbb{D}}|f(z)-f(w)| = d$, which the equality is true.

\break

\section*{6}
\begin{myBox}[]{}
    \begin{question}
        Stein and Shakarchi Pg. 66 Problem 8:

        If $f$ is a holomorphic function on the strip $-1 <y< 1$, $x\in\mathbb{R}$ with
        $$|f(z)|\leq A(1+|z|)^\eta$$
        $\eta$ a fixed real number and for all z in that strip.
        
        Show that for each integer $n\geq 0$ there exists $A_n\geq 0$ so that
        $$|f^{(n)}(x)|\leq A_n(1+|x|)^\eta$$
    \end{question}
\end{myBox}

\textbf{Pf:}

First, WLOG, we can assume $\eta\geq 0$: For all $z$ in the given region, since $|z|\geq 0$, then $(1+|z|)\geq 1$.
Hence, if $\eta<0$ (which $|\eta|>0$, and $\eta=-|\eta|$), since $(1+|z|)^{|\eta|} \geq (1+|z|)\geq 1$, the following inequality is true:
$$|f(z)| \leq A(1+|z|)^\eta\leq A(1+|z|)^{-|\eta|}\cdot(1+|z|)^{|\eta|}=A(1+|z|)^0$$
Hence, choose $\eta'=0$, $|f(z)| \leq A(1+|z|)^{\eta'}$ is still valid. Also, since $|f(z)|\geq 0$, and $(1+|z|)^\eta> 0$ for all $z$ in the given strip,
then $A\geq 0$ is also enforced.

\hfill

Now, given any $x\in\mathbb{R}$, it is also within the given region (since $y=Im(x)=0$). Then, choose radius $r=\frac{1}{2}$, 
construct the circle $\gamma_x$ by $|z-x|=r$: Since for all $z\in\gamma_x$, $|Im(z)| = |Im(z)-Im(x)| \leq |z-x| =\frac{1}{2}$,
then $z$ is also within the given region (since $-1<-\frac{1}{2}<Im(z)<\frac{1}{2}<1$), so it is a valid circle to use or Cauchy's Integral Formula.

\hfill

\textbf{Construction of $A_n$:}

Based on Cauchy's Integral Formula, the $n^{th}$ derivative can be governed by the following:
$$f^{(n)}(x)=\frac{n!}{2\pi i}\int_{\gamma_{x}}\frac{f(\zeta)}{(\zeta-x)^{n+1}}d\zeta$$
Which, for all $\zeta\in \gamma_x$, $|\zeta-x|^{n+1}=r^{n+1}$; furthermore, since $|f(\zeta)|\leq A(1+|\zeta|)^\eta$, while $\zeta = x+re^{i\theta}$ for some $\theta\in [0,2\pi)$,
hence by Triangle Inequality, the following is true:
$$|\zeta| = |x+re^{i\theta}| \leq |x|+|re^{i\theta}|=|x|+r$$
Hence, $(1+|\zeta|)\leq (1+|x|+r) = (\frac{3}{2}+|x|)$ (Note: $r=\frac{1}{2}$), and since $|x|\geq 0$, then $(\frac{3}{2}+|x|)\leq(\frac{3}{2}+\frac{3}{2}|x|)= \frac{3}{2}(1+|x|)$.
Therefore, with $\eta\geq 0$, we have $(1+|\zeta|)^\eta \leq (\frac{3}{2})^\eta \cdot (1+|x|)^\eta$.

\hfill

Now, apply these inequalities into the Integral Formula, we get the following inequality:
$$|f^{(n)}(x)| = \left|\frac{n!}{2\pi i}\int_{\gamma_{x}}\frac{f(\zeta)}{(\zeta-x)^{n+1}}d\zeta\right| \leq \frac{n!}{2\pi}\int_{\gamma_x}\frac{|f(\zeta)|}{|\zeta-x|^{n+1}}\cdot|d\zeta|\leq \frac{n!}{2\pi}\int_{\gamma_x}\frac{A(1+|\zeta|)^\eta}{r^{n+1}}\cdot|d\zeta|$$
$$|f^{(n)}(x)|\leq \frac{n!}{2\pi}\int_{\gamma_x}\frac{A(1+|\zeta|)^\eta}{r^{n+1}}\cdot|d\zeta| \leq \frac{n!}{2\pi r^{n+1}}\int_{\gamma_x}A(3/2)^\eta\cdot(1+|x|)^\eta\cdot |d\zeta| $$
$$|f^{(n)}(x)|\leq \frac{n!\cdot A(3/2)^\eta}{2\pi r^{n+1}}\cdot (1+|x|)^\eta\cdot 2\pi r = \frac{n! A(3/2)^\eta}{r^n}(1+|x|)^\eta$$
Recall that $r=\frac{1}{2}$, then the above expression can be simplified to $|f^{(n)}(x)|\leq 2^nn!A(3/2)^\eta\cdot (1+|x|)^\eta$.
Hence, let $A_n = 2^n\cdot n!\cdot A(3/2)^\eta$, all $x\in\mathbb{R}$ and $n\in\mathbb{N}$ satisfies $|f^{(n)}(x)|\leq A_n(1+|x|)^\eta$.

\end{document}