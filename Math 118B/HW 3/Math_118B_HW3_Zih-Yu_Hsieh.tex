%Math_118B_HW3_Zih-Yu_Hsieh.tex

\documentclass{article}
\usepackage{graphicx} % Required for inserting images
\usepackage[margin = 2.54cm]{geometry}
\usepackage[most]{tcolorbox}

\newtcolorbox{myBox}[3]{
arc=5mm,
lower separated=false,
fonttitle=\bfseries,
%colbacktitle=green!10,
%coltitle=green!50!black,
enhanced,
attach boxed title to top left={xshift=0.5cm,
        yshift=-2mm},
colframe=blue!50!black,
colback=blue!10
}

\newtcolorbox{myBox2}[3]{
arc=5mm,
lower separated=false,
fonttitle=\bfseries,
%colbacktitle=green!10,
%coltitle=green!50!black,
enhanced,
attach boxed title to top left={xshift=0.5cm,
        yshift=-2mm},
colframe=green!50!black,
colback=green!10
}

\usepackage{amsmath}
\usepackage{amssymb}
\usepackage{verbatim}
\usepackage[utf8]{inputenc}
\linespread{1.2}

\newtheorem{definition}{Definition}
\newtheorem{proposition}{Proposition}
\newtheorem{theorem}{Theorem}
\newtheorem{question}{Question}

\title{Math 118B HW3 - Lebesgue Criterion of Riemann Integrability}
\author{Zih-Yu Hsieh}

\begin{document}
\maketitle

\textbf{The goal of this Homework is to understand and prove the Lebesgue Criterion of Riemann Integrability.}

\section*{Setup 1}
\begin{myBox}[]{}
    \begin{definition}
        A set $E\subset \mathbb{R}$ is said to be of measure zero if given $\epsilon>0$ there is a
        countable collection of open intervals $\{I_j\}_{j\in\mathbb{Z}_+}$ which covers $E$, i.e. 
        $E\subseteq \bigcup_{j\in\mathbb{Z}_+}I_j$ and such that
        $$\sum_{j=1}^{\infty}|I_j|<\epsilon$$
    \end{definition}
\end{myBox}

\subsection*{1}
\begin{myBox2}[]{}
    \begin{question}
        Prove that every countable set of $\mathbb{R}$ is a set of measure zero.
    \end{question}
\end{myBox2}

\textbf{Pf:}

Given $E\subset \mathbb{R}$ that is countable, then there exists a bijection $f:E\rightarrow \mathbb{N}$,
which generates an index for all element $a\in E$. 

Then, given any $\epsilon>0$, for all $a\in E$, let $j=f(a)$, consider the open inverval $I_j=(a-\frac{\epsilon}{2^{j+2}},a+\frac{\epsilon}{2^{j+2}})$: 
the collection $\{I_j\}_{j\in\mathbb{Z}_+}$ is countable, and $E\subseteq \bigcup_{j\in\mathbb{Z}_+}I_j$,
since for all $a\in E$, let $j=f(a)\in\mathbb{Z}_+$, we have $f(a)\in I_j=(a-\frac{\epsilon}{2^{j+2}},a+\frac{\epsilon}{2^{j+2}})$.

On the other hand, the following is true for the length of the countable set:
$$\forall j\in\mathbb{Z}_+,\quad |I_j| = \left|(a+\frac{\epsilon}{2^{j+2}})-(a-\frac{\epsilon}{2^{j+2}})\right|=\left|2\cdot \frac{\epsilon}{2^{j+2}}\right|=\frac{\epsilon}{2^{j+1}}$$
$$\sum_{i=1}^{\infty}|I_j|=\sum_{i=1}^{\infty}\frac{\epsilon}{2^{j+1}}=\frac{\epsilon}{2}<\epsilon$$
The above series is converging since it's a geometric series with radius $\frac{1}{2}<1$. Hence, for all $\epsilon>0$,
there is a countable collection of open intervals covering $E$ with the series of length bounded by $\epsilon$, proving that $E$ the countable set has measure $0$.

\hfill

\rule{15.5cm}{0.1mm}

\hfill

\subsection*{2}
\begin{myBox2}[]{}
    \begin{question}
        Prove that the countable union of sets of measure zero has measure zero.
    \end{question}
\end{myBox2}

\textbf{Pf:}

Let $\{E_n\}_{n\in\mathbb{Z}_+}$ be a countable collection of sets, each with measure $0$. Then, for any given $\epsilon>0$, for every $n\in\mathbb{Z}_+$, since $\frac{\epsilon}{2^n}>0$,
there exists a countable collection of open interval $\{I_j^n\}_{j\in\mathbb{Z}_+}$, with $E_n \subseteq \bigcup_{j\in\mathbb{Z}_+}I_j^n$, and $\sum_{j=1}^{\infty}|I_j^n|<\frac{\epsilon}{2^n}$.

Now, consider the collection $\mathcal{F}=\bigcup_{n\in\mathbb{Z}_+}\{I_j^n\}_{j\in\mathbb{Z}_+}$, a countable union of "countable collection of open intervals", which is again countable.
Which, since $E_n \subseteq \bigcup_{j\in\mathbb{Z}_+}I_j^n$ for all $n\in\mathbb{Z}_+$, then:
$$\bigcup_{n\in\mathbb{Z}}E_n \subseteq \bigcup_{n\in\mathbb{Z}_+}\left(\bigcup_{j\in\mathbb{Z}_+}I_j^n\right)$$
Which the left side is countable union of sets with measure $0$, while the right side is the union of open intervals in family $\mathcal{F}$.

Then, to consider the length of $\mathcal{F}$, since it is countable, there exists a bijection $f:\mathcal{F}\rightarrow \mathbb{N}$ that generates the index.
Which, for the first $k$ elements in this index of $\mathcal{F}$, the elements are $\{I_{j_1}^{n_1},...,I_{j_k}^{n_k}\}$. let $J=\max\{j_1,...,j_k\}$ and $N=\max\{n_1,...,n_k\}$,
then these elements are in the collection $\bigcup_{n=1}^{N}\{I_j^n\}_{j=1}^{J}$. Which, the collection has the length being bounded:
$$\forall n\in\{1,...,N\},\quad \sum_{j=1}^{J}|I_j^n|\leq \sum_{j=1}^{\infty}|I_j^n|=\frac{\epsilon}{2^n}$$
$$\sum_{n=1}^{N}\left(\sum_{j=1}^{J}|I_j^n|\right) \leq \sum_{j=1}^{\infty}|I_j^n| \leq \sum_{n=1}^{N}\frac{\epsilon}{2^n}\leq \sum_{n=1}^{\infty}\frac{\epsilon}{2^n}=\epsilon$$
The above two inequalities are true, since each partial sum is monotonically non-decreasing, and bounded above.
Hence, the sum of length $s_k=\sum_{i=1}^{k}|I_{j_i}^{n_i}| \leq \sum_{n=1}^{N}\left(\sum_{j=1}^{J}|I_j^n|\right) \leq \epsilon$ for any positive integer $k$, while this partial sum of length $s_k$
is also monotonically non-decreasing, hence the series of length converges, and the following is true:
$$\lim_{k\rightarrow\infty}s_k=\sum_{k=1}^{\infty}|I_{j_k}^{n_k}| = \sup\{s_k\} \leq \epsilon$$
(Note: since $\epsilon$ is the upper bound of the partial sums, hence the above inequality is true).
Then, since $\mathcal{F}$ covers the $\bigcup_{n\in\mathbb{Z}_+}E_n$, and the series of $\mathcal{F}$ elements' length satisfy $\sum_{k=1}^{\infty}|I_{j_k}^{n_k}|\leq \epsilon$, then we can conclude that $\bigcup_{n\in\mathbb{Z}_+}E_n$ has measure $0$.

\hfill

\rule{15.5cm}{0.1mm}

\hfill

\subsection*{3}
\begin{myBox2}[]{}
    \begin{question}
        Let $ E$ be the set of all $x\in[0,1]$ whose decimal expansion contains
        only the digits 4 and 7. We have seen that $E$ is uncountable. Prove that $E$ is a
        set of measure zero.
    \end{question}
\end{myBox2}

\textbf{Pf:}

From the description, $E=\{x\in[0,1]\ |\ x=0.a_1a_2...a_n...,\ \forall n\in\mathbb{N}, a_n\in\{4,7\}\}$. 
For each $n\in\mathbb{N}$, let $E_n=\{x\in[0,1]\ |\ x=0.a_1a_2...a_n...,\ \forall i\in\{1,...,n\}, a_i\in\{4,7\}\}$ (set of reals in $[0,1]$ with the first $n$ decimals being $4$ or $7$).

Notice that for $n\geq 2$, there are $2^{(n-1)}$ distinct cases for $0.a_1a_2...a_{(n-1)}$ (first $(n-1)$ decimals) in $E_n$, then for each case, if $x\in E_n$ has this arrangement for the first $(n-1)$ decimals:
$$0.a_1a_2...a_{(n-1)}3 < x < 0.a_1a_2...a_{(n-1)}8$$
Hence, for each arrangement, they're contained in the open interval $(0.a_1a_2...a_{(n-1)}3,0.a_1a_2...a_{(n-1)}8)$, which has length
$(0.a_1a_2...a_{(n-1)}8-0.a_1a_2...a_{(n-1)}3)=\frac{8-3}{10^n} = \frac{5}{10^n}$.

All $2^{(n-1)}$ collection of these open intervals would cover $E_n$, since the first $(n-1)$ decimals for each $x\in E_n$ must be some arrangement of $4$ and $7$.
Hence, $E_n$ can be covered by unions of $2^{(n-1)}$ open intervals, each with length $\frac{5}{10^n}$, hence the total length of the open cover is $2^{(n-1)}\cdot 5\cdot \frac{1}{10^n}=\frac{1}{2}\cdot(\frac{2}{10})^{(n-1)}=\frac{1}{2}(\frac{1}{5})^{(n-1)}$.

\hfill

Now, since for each $x\in E$, the first $n$ decimals are consist of $4$ and $7$, hence $x\in E_n$, or $E\subseteq E_n$.
From the previous part, since for each $n\in\mathbb{N}$, the set $E_n$ could be covered by a finite collection of open intervals with sum of length $\frac{1}{2}(\frac{1}{5})^{(n-1)}$, so does the set $E$.

Then, becase $\lim_{n\rightarrow \infty}\frac{1}{2}(\frac{1}{5})^{(n-1)}=0$, then for all $\epsilon>0$, there exists $n$, with $\frac{1}{2}(\frac{1}{5})^{(n-1)}<\epsilon$.
Hence, choose the collection of open intervals for $E_n$ (along with countable empty sets), $E$ can be covered with the given collection of open intervals, with total length $\frac{1}{2}(\frac{1}{5})^{(n-1)}<\epsilon$,
showing that $E$ in fact has measure $0$.

\break

\section*{Setup 2}
\begin{myBox}[]{}
    \begin{definition}
        Let $f : [a, b]\rightarrow \mathbb{R}$ be a bounded function, $(b-a)<\infty$. For $x\in[a,b]$
        and $\eta>0$ define
        $$\Omega(f,x,\eta)=\sup\{|f(x_1)-f(x_2)|\ :\ x_1,x_2\in (x-\eta,x+\eta) \cap [a,b]\}$$
        and the oscillation of $f$ at a point $x\in[a,b]$
        $$\omega_f(x)=\lim_{\eta\rightarrow 0^+}\Omega(f,x,\eta)=\inf_{\eta>0}\{\Omega(f,x,\eta)\}$$
    \end{definition}
\end{myBox}

\subsection*{4}
\begin{myBox2}[]{}
    \begin{question}
        Prove that $\omega_f(x)$ is defined for any $x\in[a,b]$.
    \end{question}
\end{myBox2}

\textbf{Pf:}

Given any $x\in[a,b]$, and $\eta_1,\eta_2>0$ with $\eta_1>\eta_2$, since $(x-\eta_2,x+\eta_2)\subset (x-\eta_1,x+\eta_1)$, hence:
$$\{|f(x_1)-f(x_2)|\ :\ x_1,x_2\in (x-\eta_2,x+\eta_2) \cap [a,b]\}\subseteq \{|f(x_1)-f(x_2)|\ :\ x_1,x_2\in (x-\eta_1,x+\eta_1) \cap [a,b]\}$$
This implies $\Omega(f,x,\eta_2)\leq \Omega(f,x,\eta_1)$, since the supremum of the set on the right, is also an upper bound of the set on the left.

On the other hand, since the set $\{|f(x_1)-f(x_2)|\ :\ x_1,x_2\in (x-\eta,x+\eta) \cap [a,b]\}$ for all $\eta>0$ is a collection of distance in $\mathbb{R}$,
hence $0$ is always a lower bound of the set, showing that $0\leq \Omega(f,x,\eta)$.

\hfill

Now, since for all $\eta>0$, the value $\Omega(f,x,\eta)$ is bounded below by $0$ for all $\eta>0$, this implies $\inf_{\eta>0}\{\Omega(f,x,\eta)\}$ exists.
Then, to prove that $\lim_{\eta\rightarrow 0^+}\Omega(f,x,\eta)=\inf_{\eta>0}\{\Omega(f,x,\eta)\}=\omega$, for all $\epsilon>0$, since $\omega+\epsilon$ is no longer a lower bound of the set,
there exists $\eta>0$, with $\omega \leq \Omega(f,x,\eta)<\omega+\epsilon$. Then, choose $\delta = \eta>0$, 
for all $\mu'>0$ with $\mu'<\mu = \delta$, from the previous section, $\omega \leq \Omega(f,x,\eta')\leq \Omega(f,x,\eta) < \omega+\epsilon$, hence:
$$|\omega-\Omega(f,x,\eta')| < \epsilon$$
This demonstrates that $\lim_{\eta\rightarrow 0^+}\Omega(f,x,\eta)=\omega=\inf_{\eta>0}\{\Omega(f,x,\eta)\}$, hence $\omega_f(x)$ is defined.

\hfill

\rule{15.5cm}{0.1mm}

\hfill

\subsection*{5}
\begin{myBox2}[]{}
    \begin{question}
        Prove that $f$ is continuous at $x_0$ if and only if $\omega_f(x_0)=0$.
    \end{question}
\end{myBox2}

\textbf{Pf:}

\begin{itemize}
    \item[$\implies:$] Suppose $f$ is continuous at $x_0$, for all $\epsilon>0$ (since $\frac{\epsilon}{2}>0$), there exists $\delta>0$ such that $|x-x_0|<\delta$ implies $|f(x)-f(x_0)|<\epsilon$.
    Then, choose $\eta=\delta$, consider $\Omega(f,x_0,\delta)=\sup\{|f(x_1)-f(x_2)|\ :\ x_1,x_2\in (x_0-\delta,x_0+\delta) \cap [a,b]\}$:

    For all $x_1,x_2\in (x_0-\delta,x_0+\delta)\cap [a,b]$, since $x_1,x_2\in B_\delta(x_0)$, then by assumption $|f(x_1)-f(x_0)|,|f(x_2)-f(x_0)|<\frac{\epsilon}{2}$. Hence, the following is true:
    $$|f(x_1)-f(x_2)| = |(f(x_1)-f(x_0))+(f(x_0)-f(x_2))| \leq |f(x_1)-f(x_0)|+|f(x_2)-f(x_0)|<\frac{\epsilon}{2}+\frac{\epsilon}{2}=\epsilon$$
    Hence, $\epsilon$ is an upper bound of the set $\{|f(x_1)-f(x_2)|\ :\ x_1,x_2\in (x_0-\delta,x_0+\delta) \cap [a,b]\}$, showing that $\Omega(f,x_0,\delta)\leq \epsilon$.

    \hfill

    Because the choice $\epsilon>0$ is arbitrary, then with the corresponding $\delta>0$, $\omega_f(x_0) \Omega(f,x_0,\delta)\leq \epsilon$. 
    Then, $\omega_f(x_0)\leq \epsilon$ for all $\epsilon>0$, showing that $\omega_f(x_0) \leq 0$;
    also, $\omega_f(x_0)$ is an infimum of all nonnegative numbers (in \textbf{4} we've proven $0 \leq \Omega(f,x,\eta)$ for all $\eta>0$),
    hence $\omega_f(x_0)\geq 0$. The two statements imply $\omega_f(x_0)=0$.
    
    \hfill

    \item[$\impliedby:$] Suppose $\omega_f(x_0)=0$. Then, by definition, for all $\epsilon>0$, since $\epsilon=0+\epsilon$ is no longer a lower bound of the set $\{\Omega(f,x_0,\eta)\ |\ \eta>0\}$,
    there exists $\delta = \eta>0$, such that $0\leq \Omega(f,x_0,\delta) < \epsilon$. 

    Hence, for all $x\in B_\delta(x_0)\cap [a,b]$, $|f(x)-f(x_0)| \leq \Omega(f,x_0,\delta)<\epsilon$, proving that $f$ is continuous at $x_0$.
\end{itemize}
The above two implications shows that $f$ is continuous at $x_0$ if and only if $\omega_f(x_0)=0$.

\hfill

\rule{15.5cm}{0.1mm}

\hfill

\subsection*{6}
\begin{myBox2}[]{}
    \begin{question}
        Prove that for any $\mu>0$ the set $A_\mu=\{x\in[a,b]\ :\ \omega_f(x)\geq \mu\}$ is
        compact.   
    \end{question}
\end{myBox2}

\textbf{Pf:}

Since $A_\mu \subseteq [a,b]$ while $[a,b]$ is compact, then to prove that $A_\mu$ is compact, it suffices to show that $A_\mu$ is closed, or $A_\mu'\subseteq A_\mu$.

\hfill

For all $x_0\in A_\mu'$, for every radius $r>0$, there exists $x_1\in B_r(x_0)\setminus\{x_0\}\cap A_\mu$. Hence, $\omega_f(x_1)\geq \mu$.

Now, take $\eta = r-|x_0-x_1|>0$, for all $x\in B_\eta(x_1)$, since $|x-x_1|<\eta = r-|x_0-x_1|$, then:
$$|x-x_0|=|(x-x_1)+(x_1-x_0)|\leq |x-x_1|+|x_1-x_0| < (r-|x_0-x_1|)+|x_1-x_0| = r$$
This indicates that $x\in B_r(x_0)$, or $B_\eta(x_1)\subseteq B_r(x_0)$.

Hence, for all $x_c,x_d\in (B_\eta(x_1)\cap [a,b])\subseteq (B_r(x_0)\cap [a,b])$, since the following is true: 
$$|f(x_c)-f(x_d)|\leq \Omega(f,x_0,r)=\sup\{|f(x)-f(x')|\ :\ x,x'\in B_r(x_0)\cap [a,b]\}$$
Hence, $\Omega(f,x_0,r)$ is an upper bound of the set $\{|f(x)-f(x')|\ :\ x,x'\in B_\eta(x_1)\cap [a,b]\}$, which implies the following:
$$\Omega(f,x_0,r)\geq \Omega(f,x_1,\eta)=\sup\{|f(x)-f(x')|\ :\ x,x'\in B_\eta(x_1)\cap [a,b]\}$$
Thus, we can further conclude that $\Omega(f,x_0,r)\geq \Omega(f,x_1,\eta)\geq \omega_f(x_1) \geq \mu$.

\hfill

Now, because for all $r>0$, $\Omega(f,x_0,r)\geq \mu$, then $\mu$ is the lower bound of the set $\{\Omega(f,x_0,r)\ |\ r>0\}$, 
showing that $\mu \leq \omega_f(x_0)=\inf\{\Omega(f,x_0,r)\ |\ r>0\}$. Hence, $x_0 \in A_\mu$, showing that $A_\mu'\subseteq A_\mu$.

This proves that $A_\mu$ is closed, and since $A_mu\subseteq[a,b]$ a compact set, then $A_\mu$ is also compact.

\hfill

\rule{15.5cm}{0.1mm}

\hfill

\subsection*{7}
\begin{myBox2}[]{}
    \begin{question}
        Prove that the set of discontinuities of $f$ can be written as
        $$D_f=\bigcup_{j\in\mathbb{Z}_+}A_{1/j}=\bigcup_{j\in\mathbb{Z}_+}\left\{x\in[a,b]\ :\ \omega_f(x)\geq \frac{1}{j}\right\}$$
    \end{question}
\end{myBox2}

\textbf{Pf:}

In \textbf{Question 5}, we've proven the equivalence of continuity at $x_0$ and $\omega_f(x_0)=0$, hence $x\in[a,b]$ is a discontinuity of $f$
iff $\omega_f(x)\neq 0$ (which actually is $\omega_f(x)>0$). Hence, $D_f=\{x\in[a,b]\ |\ \omega_f(x)>0\}$.

Now, for all $x\in D_f$, since $\omega_f(x)>0$, by Archimedean's Property, there exists $j\in\mathbb{Z}_+$, with $\omega_f(x)>\frac{1}{j}>0$,
this implies $x\in A_{1/j}=\left\{x\in[a,b]\ :\ \omega_f(x)\geq \frac{1}{j}\right\}$, hence $x\in \bigcup_{j\in\mathbb{Z}_+}A_{1/j}$. This implies $D_f\subseteq \bigcup_{j\in\mathbb{Z}_+}A_{1/j}$.

\hfill

\rule{15.5cm}{0.1mm}

\hfill

\subsection*{8}
\begin{myBox2}[]{}
    \begin{question}
        Prove that if for some $\epsilon>0$, $\omega_f(x)<\epsilon$ for any $x\in[a,b]$, then
        there exists $\eta>0$ such that for all $x\in[a,b]$,
        $$\Omega(f,x,\eta)<\epsilon$$
    \end{question}
\end{myBox2}

\textbf{Pf:}

Suppose there exists $\epsilon>0$, with $\omega_f(x)<\epsilon$ for all $x\in[a,b]$, then since $\epsilon$ is not a lower bound of the set $\{\Omega(f,x,\eta)\ |\ \eta>0\}$,
then there exists $\eta_x>0$, such that $\omega_f(x)\leq \Omega(f,x,\eta_x)<\epsilon$.

Now, consider the collection of open intervals $\mathcal{F}=\{(x-\eta_x/2,x+\eta_x/2)\ |\ x\in[a,b]\}$: Since $[a,b]\subseteq \bigcup\mathcal{F}$, then $\mathcal{F}$ is an open cover of $[a,b]$;
hence, by the compactness of $[a,b]$, there exists $x_1,...,x_n\in [a,b]$, such that $[a,b]\subseteq \bigcup_{i=1}^{n}(x_i-\mu_{x_i}/2,x_i+\mu_{x_i}/2)$.

\hfill

Then, let $\eta=\min\{\frac{1}{2}\eta_{x_1},...,\frac{1}{2}\eta_{x_n}\}>0$. For all $x\in[a,b]$, from the above construction, there exists $i\in\{1,...,n\}$ with $x\in B_{\eta_{x_i}/2}(x_i)$.
Now, consider the set $S=\{|f(x_c)-f(x_d)|\ :\ x_c,x_d\in (x-\eta,x+\eta)\cap [a,b]\}$:
For all $x_c,x_d\in (x-\eta,x+\eta)\cap [a,b]$, they satisfy $|x_c-x|,|x_d-x|<\mu \leq \frac{1}{2}\eta_{x_i}$. Hence, the following inequalities are true:
$$|x_c-x_i| = |(x_c-x)+(x-x_i)| \leq |x_c-x|+|x-x_i| < \frac{1}{2}\eta_{x_i}+\frac{1}{2}\eta_{x_i}=\eta_{x_i}$$
$$|x_d-x_i| = |(x_d-x)+(x-x_i)| \leq |x_d-x|+|x-x_i| < \frac{1}{2}\eta_{x_i}+\frac{1}{2}\eta_{x_i} = \eta_{x_i}$$
These two inequalities imply $x_c,x_d\in (x_i-\eta_{x_i},x_i+\eta_{x_i})$, which $|f(x_c)-f(x_d)|\leq\Omega(f,x_i,\eta_{x_i})$.
Hence, $\Omega(f,x_i,\eta_{x_i})$ is an upper bound of the set $S$, showing that $\sup(S)=\Omega(f,x,\eta)\leq \Omega(f,x_i,\eta_{x_i}) < \epsilon$ regarding the initial construction.

So, this $\eta>0$ satisfies the desired condition.

\break

\section*{Proof of the Main Theorem}

\begin{myBox}[]{}
    \begin{theorem}
        Let $f:[a,b]\rightarrow \mathbb{R}$ be a bounded function $(b-a)<\infty$. Then $f$ is Riemann
        integrable on $[a, b]$ if and only if the set of discontinuities of $f$, $D_f$ is a set of
        measure zero.
    \end{theorem}
\end{myBox}

\subsection*{9}
\begin{myBox2}[]{}
    \begin{question}
        Prove the theorem.
    \end{question}
\end{myBox2}

\textbf{Pf:}

\begin{itemize}
    \item[$\implies:$] We'll approach this by contradiction. Suppose $f:[a,b]\rightarrow \mathbb{R}$ is Riemann Integrable, yet $D_f$ has measure greater than $0$.
    
    In \textbf{Quesion 7}, we've proven that $D_f = \bigcup_{j\in\mathbb{Z}_+}A_{1/j}$ (Countable union of $A_j$), with $A_{1/j}$ being defined in \textbf{Question 6}.
    With the assumption that $D_f$ has measure $0$, there exists $j_0\in\mathbb{Z}_+$, with $A_{1/j_0}$ having measure greater than $0$: If all $j\in\mathbb{Z}_+$ has measure $0$,
    then by the statement proven in \textbf{Question 2}, the countable union $D_f=\bigcup_{j\in\mathbb{Z}_+}A_{1/j}$ should also have measure $0$, which contradicts the assumption.

    \hfill

    With the given $j_0\in\mathbb{Z}_+$, consider any partition $P=\{x_0=a,x_1,...,x_{n-1},x_n=b\}$ with $x_{i-1}<x_i$ for all index $i$.
    Since $A_{1/j_0}\subseteq [a,b]$, there exists some intervals in the partition $I_{n_1},...,I_{n_i}$ that covers $A_{1/j_0}$ (here, assume every chosen interval covered some part of $A_{1/j_0}$, not disjoint with it).

    WLOG, we can assume that for each interval $I_{n_j}$, there exists a point of discontinuity $x\in A_{1/j_0}$ that is an interior point of $I_{n_j}$:
    If $x=x_i$ for some $i\neq 0$ and $i\neq n$, then we can combine two intervals $I'=I_{i}\cup I_{i+1}=[x_{i-1},x_{i+1}]$, which $x_i$ becomes the interior point of the modified interval;
    else if $x=a$ or $x=b$, since all the definition only consider the cases in $[a,b]$, then $a,b$ is could be considered as the interior point of $[a,b]$ under subspace topology.

    \hfill
    
    Because $A_{1/j_0}$ has nonzero measure, then there exists $\epsilon>0$, such that for any open interval covering $\{I_j\}_{j\in\mathbb{Z}_+}$ of $A_{1/j_0}$, the sum of length of the intervals $\sum_{j=1}^{\infty}|I_{j}|\geq \epsilon$, regardless of the collection of open intervals
    (in particular, $\sum_{j=1}^{i}|I_{n_j}|\geq \epsilon$, since each interval has the same length with its interior).
    
    Furthermore, from the previous assumption, there exists $x\in A_{1/j_0}$ that is an interior point of $I_{n_j}$ for each $j\in\{1,...,i\}$,
    hence there exists radius $r_j>0$, with $B_{r_j}(x)\cap [a,b]\subseteq I_{n_j}$.
    Now, by definition, since for all $x_1,x_2\in B_{r_j}(x)\cap [a,b]\subseteq I_{n_j}$ satisfies $\inf_{x\in I_{n_j}}\{f(x)\}\leq f(x_1),f(x_2)\leq \sup_{x\in I_{n_j}}\{f(x)\}$, hence:
    $$|f(x_1)-f(x_2)|\leq \left(\sup_{x\in I_{n_j}}\{f(x)\}-\inf_{x\in I_{n_j}}\{f(x)\}\right)$$
    This implies the following: 
    $$\sup\{|f(x_1)-f(x_2)|\ :\ x_1,x_2\in B_{r_j}(x)\cap [a,b]\}=\Omega(f,x,r_j)\leq \left(\sup_{x\in I_{n_j}}\{f(x)\}-\inf_{x\in I_{n_j}}\{f(x)\}\right)$$
    $$\frac{1}{j_0}\leq \omega_f(x)\leq \Omega(f,x,r_j)\leq \left(\sup_{x\in I_{n_j}}\{f(x)\}-\inf_{x\in I_{n_j}}\{f(x)\}\right)$$
    Hence, consider the difference in upper and lower sum, we yield:
    $$U(f,P)-L(f,P)=\sum_{k=1}^{n}\left(\sup_{x\in I_{k}}\{f(x)\}-\inf_{x\in I_{k}}\{f(x)\}\right)\cdot |I_k| \geq \sum_{j=1}^{i}\left(\sup_{x\in I_{n_j}}\{f(x)\}-\inf_{x\in I_{n_j}}\{f(x)\}\right)\cdot |I_{n_j}|$$
    $$U(f,P)-L(f,P)\geq \sum_{j=1}^{i}\left(\sup_{x\in I_{n_j}}\{f(x)\}-\inf_{x\in I_{n_j}}\{f(x)\}\right)\cdot |I_{n_j}| \geq \sum_{j=1}^{i}\frac{1}{j_0}|I_{n_j}|$$
    $$U(f,P)-L(f,P)\geq \frac{1}{j_0}\sum_{j=1}^{i}|I_{n_j}| \geq \frac{1}{j_0}\cdot\epsilon$$
    (Note: The above is true, since the collection $I_{n_1},...,I_{n_j}$ is part of the partition).

    Hence, the difference in the upper and lower sum for any $P$ is at least $\epsilon/j_0>0$, showing that $f$ is not Riemann Integrable.
    Yet, this contradicts our initial assumption; hence, the assumption is false, $f$ is Riemann Integrable implies $D_f$ has measure $0$.
    
    \hfill

    \item[$\impliedby:$] 
    
\end{itemize}

\end{document}