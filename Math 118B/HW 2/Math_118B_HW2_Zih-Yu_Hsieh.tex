%Math_118B_HW2_Zih-Yu_Hsieh.tex

\documentclass{article}
\usepackage{graphicx} % Required for inserting images
\usepackage[margin = 2.54cm]{geometry}

\usepackage{amsmath}
\usepackage{amssymb}
\usepackage{verbatim}
\usepackage[utf8]{inputenc}
%\linespread{1.5}

\newtheorem{definition}{Definition}
\newtheorem{proposition}{Proposition}
\newtheorem{theorem}{Theorem}
\newtheorem{question}{Question}

\title{Latex Template}
\author{Zih-Yu Hsieh}

\begin{document}
\maketitle

\section*{1}
\begin{question}
    Let $f : (a, b) \rightarrow \mathbb{R}$ be differentiable on $(a, b)$.
    
    Prove : if $\forall x\in(a,b),\ f'(x)\neq 0$, then $f$ is one-to-one on $(a, b)$.
    
    Give an example showing that the converse statement is in general not true.
\end{question}

\textbf{Pf:}

Suppose $\forall x\in(a,b),\ f'(x)\neq 0$:

\hfill

\textbf{(1) $f'(x)$ is strictly less than or greater than $0$ on $(a,b)$:}

We'll prove by contradiction: Suppose $f'(x)$ is neither strictly less than $0$ nor strictly greater than $0$ on $(a,b)$,
then there exists $x_0, x_1\in (a,b)$, with $f'(x_0)\leq 0$ and $f'(x_1)\geq 0$, and by the assumption that $f'(x)\neq 0$,
the strict inequality $f'(x_0)< 0$ and $f'(x_1)> 0$ is applied. (This also implies $x_0 \neq x_1$, since derivatives are different
 at the two points).

\hfill

Recall that for function $f:[a,b]\rightarrow \mathbb{R}$ be differentiable on $(a,b)$, if $a<c<d<b$ and 
$f'(c)\neq f'(d)$, for any $\lambda$ strictly in between $f'(c)$ and $f'(d)$ (either $f'(c)<\lambda<f'(d)$ or $f'(c)>\lambda>f'(d)$),
there exists $x\in (c,d)$ with $f'(x)=\lambda$.

Then, first suppose $x_0<x_1$: $f$ is differentiable on $(a,b)$ and $f'(x_0) < 0 < f'(x_1)$ implies there exists $x\in(x_0,x_1)$ with
$f'(x)=0$, which contradicts the assumption;

then suppose $x_1<x_0$: again, $f$ is differentiable on $(a,b)$ and $f'(x_1)>0>f'(x_0)$ implies there exists $x\in (x_1,x_0)$ with
$f'(x)=0$, which again contradicts the assumption.

So, the assumption is false, $f'(x)$ must be strictly greater than $0$ or less than $0$ for all $x\in (a,b)$.

\hfill

\textbf{(2) $f$ is strictly increasing or decreasing on $(a,b)$:}

Based on \textbf{(1)}, $f'(x)$ is strictly less than $0$ or strictly greater than $0$.

Suppose $f'(x)>0$ for all $x\in(a,b)$, then for any $x,y\in (a,b)$ with $x<y$, by the Mean Value Theorem, 
there exists $c\in (x,y)\subseteq (a,b)$, such that the following is true:
$$\frac{f(y)-f(x)}{y-x}=f'(c) >0,\quad (f(y)-f(x))=f'(c)(y-x)$$
Since $(y-x), f'(c)>0$ by assumption, the $(f(y)-f(x))=f'(c)(y-x)>0$, thus $f(y)>f(x)$, showing that $f$ is strictly increasing.

Similarly, suppose $f'(x)<0$ for all $x\in (a,b)$, with the same $x,y$ above, by Mean Value Theorem, there exists $c\in(x,y)\subseteq (a,b)$,
such that the following is true:
$$\frac{f(y)-f(x)}{y-x}=f'(c) <0,\quad (f(y)-f(x))=f'(c)(y-x)$$
Since $(y-x)>0$ and $f'(c)<0$, then $(f(y)-f(x))=f'(c)(y-x)<0$, of $f(y)<f(x)$, showing that $f$ is strictly decreasing.

\break

\section*{2}
\begin{question}
    
\end{question}

\break

\section*{3}
\begin{question}
    
\end{question}

\break

\section*{4}
\begin{question}
    
\end{question}

\break

\section*{5}
\begin{question}
    
\end{question}

\end{document}