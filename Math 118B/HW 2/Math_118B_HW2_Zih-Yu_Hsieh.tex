%Math_118B_HW2_Zih-Yu_Hsieh.tex

\documentclass{article}
\usepackage{graphicx} % Required for inserting images
\usepackage[margin = 2.54cm]{geometry}

\usepackage{amsmath}
\usepackage{amssymb}
\usepackage{verbatim}
\usepackage[utf8]{inputenc}
%\linespread{1.5}
\usepackage[most]{tcolorbox}

\newtcolorbox{myBox}[3]{
arc=5mm,
lower separated=false,
fonttitle=\bfseries,
%colbacktitle=green!10,
%coltitle=green!50!black,
enhanced,
attach boxed title to top left={xshift=0.5cm,
        yshift=-2mm},
colframe=blue!50!black,
colback=blue!10
}

\newtheorem{definition}{Definition}
\newtheorem{proposition}{Proposition}
\newtheorem{theorem}{Theorem}
\newtheorem{question}{Question}

\title{Math 111B HW2}
\author{Zih-Yu Hsieh}

\begin{document}
\maketitle

\section*{1}
\begin{myBox}[]{}
    \begin{question}
        Let $f : (a, b) \rightarrow \mathbb{R}$ be differentiable on $(a, b)$.
        
        Prove : if $\forall x\in(a,b),\ f'(x)\neq 0$, then $f$ is one-to-one on $(a, b)$.
        
        Give an example showing that the converse statement is in general not true.     
    \end{question}
\end{myBox}


\textbf{Pf:}

Suppose $\forall x\in(a,b),\ f'(x)\neq 0$:

\hfill

\textbf{(1) $f'(x)$ is strictly less than or greater than $0$ on $(a,b)$:}

We'll prove by contradiction: Suppose $f'(x)$ is neither strictly less than $0$ nor strictly greater than $0$ on $(a,b)$,
then there exists $x_0, x_1\in (a,b)$, with $f'(x_0)\leq 0$ and $f'(x_1)\geq 0$, and by the assumption that $f'(x)\neq 0$,
the strict inequality $f'(x_0)< 0$ and $f'(x_1)> 0$ is applied. (This also implies $x_0 \neq x_1$, since derivatives are different
 at the two points).

\hfill

Recall that for function $f:[a,b]\rightarrow \mathbb{R}$ be differentiable on $(a,b)$, if $a<c<d<b$ and 
$f'(c)\neq f'(d)$, for any $\lambda$ strictly in between $f'(c)$ and $f'(d)$ (either $f'(c)<\lambda<f'(d)$ or $f'(c)>\lambda>f'(d)$),
there exists $x\in (c,d)$ with $f'(x)=\lambda$.

Then, first suppose $x_0<x_1$: $f$ is differentiable on $(a,b)$ and $f'(x_0) < 0 < f'(x_1)$ implies there exists $x\in(x_0,x_1)$ with
$f'(x)=0$, which contradicts the assumption;

then suppose $x_1<x_0$: again, $f$ is differentiable on $(a,b)$ and $f'(x_1)>0>f'(x_0)$ implies there exists $x\in (x_1,x_0)$ with
$f'(x)=0$, which again contradicts the assumption.

So, the assumption is false, $f'(x)$ must be strictly greater than $0$ or less than $0$ for all $x\in (a,b)$.

\hfill

\textbf{(2) $f$ is strictly increasing or decreasing on $(a,b)$:}

Based on \textbf{(1)}, $f'(x)$ is strictly less than $0$ or strictly greater than $0$.

Suppose $f'(x)>0$ for all $x\in(a,b)$, then for any $x,y\in (a,b)$ with $x<y$, by the Mean Value Theorem, 
there exists $c\in (x,y)\subseteq (a,b)$, such that the following is true:
$$\frac{f(y)-f(x)}{y-x}=f'(c) >0,\quad (f(y)-f(x))=f'(c)(y-x)$$
Since $(y-x), f'(c)>0$ by assumption, the $(f(y)-f(x))=f'(c)(y-x)>0$, thus $f(y)>f(x)$, showing that $f$ is strictly increasing.

Similarly, suppose $f'(x)<0$ for all $x\in (a,b)$, with the same $x,y$ above, by Mean Value Theorem, there exists $c\in(x,y)\subseteq (a,b)$,
such that the following is true:
$$\frac{f(y)-f(x)}{y-x}=f'(c) <0,\quad (f(y)-f(x))=f'(c)(y-x)$$
Since $(y-x)>0$ and $f'(c)<0$, then $(f(y)-f(x))=f'(c)(y-x)<0$, of $f(y)<f(x)$, showing that $f$ is strictly decreasing.

\hfill

With the above condition, since $f$ is either strictly increasing or strictly decreasing on $[a,b]$, 
then for all $x,y\in(a,b)$, $x\neq y$ implies $f(x)\neq f(y)$ (or else it's no longer strictly increasing or decreasing).
Thus, $f$ is in fact one-to-one on $(a,b)$.

\hfill

\textbf{Counterexample of Converse:}

Let $f:[-1,1]\rightarrow \mathbb{R}$ be $f(x)=x^3$, which $f'(x)=3x^2$, which $f'(0)=0$. 
Yet, suppose $x,y\in (-1,1)$ has $x^3=y^3$, then:
$$x^3-y^3=(x-y)(x^2+xy+y^2)=0$$
Which, the only real solution is $x=y$ (since if treating $y$ as constant, $y^2-4y^2=-3y^2\leq 0$; the only time with real solution is when $y=0$, which implies $x^3=0$, or $x=0$).

So, $f(x)=x^3$ is one-to-one on the region $(-1,1)$, but still has $f'(0)=0$, which is a counterexample.

\hfill

\hfill

\section*{2}
\begin{myBox}[]{}
    \begin{question}
        Let $f : (a, b) \rightarrow R$ be a function such that:
        $$\exists M>0,\exists\alpha>0,\ \forall x,y\in(a,b),\ |f(x)-f(y)|<M|x-y|^\alpha$$
        If $\alpha\in(0,1)$, then $f$ is Holder of order $\alpha$ in $(a, b)$. If $\alpha$ = 1, then $f$ is Lipschitz.
        Prove :
    
        (a) If $\alpha>1$, then $f$ is constant.
    
        (b) If $\alpha\in(0,1]$, then $f$ is uniformly continuous on $(a, b)$.
    
        (c) Give an example such that $f$ is Lipschitz, but not differentiable.
    
        (d) If $f$ is differentiable on $(a, b)$ and $f(x)$ is bounded on (a, b), then f is Lipschitz.
    \end{question}
\end{myBox}

\textbf{Pf:}

\begin{itemize}
    \item[(a)] Suppose $\alpha>1$, then there exists $\epsilon>0$, such that $\alpha=1+\epsilon$. Which, for all $x,y\in (a,b)$ (with $x\neq y$), the following is true:
    $$|f(x)-f(y)|<M|x-y|^\alpha = M|x-y|^{1+\epsilon} = M|x-y|\cdot|x-y|^{\epsilon}$$
    $$\left|\frac{f(x)-f(y)}{x-y}\right| = \frac{|f(x)-f(y)|}{|x-y|}<M|x-y|^\epsilon$$
    Which, fix arbitrary $x_0\in (a,b)$, for all $y\in(a,b)$ with $y\neq x_0$, the following is true:
    $$0\leq \left|\frac{f(x_0)-f(y)}{x_0-y}\right|<M|x_0-y|^\epsilon,\quad -M|x_0-y|^\epsilon<\frac{f(x_0)-f(y)}{x_0-y}<M|x_0-y|^\epsilon$$
    Since $\epsilon>0$, then $\lim_{y\rightarrow x_0}|x_0-y|^\epsilon=0$. Which, by Squeeze Theorem, the following is true:
    $$0=\lim_{y\rightarrow x_0}-M|x_0-y|^\epsilon\leq \lim_{y\rightarrow x_0}\frac{f(x_0)-f(y)}{x_0-y} \leq \lim_{y\rightarrow x_0}M|x_0-y|^\epsilon = 0$$
    Thus, $\lim_{y\rightarrow x_0}\frac{f(x_0)-f(y)}{x_0-y}=0$, or $f'(x_0)=0$.

    \hfill

    This implies that $f(x)$ is a constant function: Suppose $f(x)$ is not a constant function, then there exists $c,d\in (a,b)$ with $c<d$, such that $f(c)\neq f(d)$.

    Notice that since $f'(x_0)$ exists for all $x_0\in(a,b)$, then by Mean Value Theorem, there exists $x\in (c,d)$, such that $f'(x)(d-c)=f(d)-f(c)$.

    Yet, since $f'(x)=0$, while $f(d)-f(c)\neq 0$, $0 = f'(x)(d-c)\neq f(d)-f(c)$, which it is a contradiction.

    Thus, $f(x)$ must be a constant function.

    \hfill
    
    \item[(b)] Suppose $\alpha\in (0,1]$, notice that for all $x,y\in(a,b)$, the following is true:
    $$a<x<b,\quad -b<-y<-a,\quad (a-b)=-(b-a)<(x-y)<(b-a),\quad |x-y|<|b-a|$$
    Which, since $\alpha>0$, then $|x-y|^\alpha<|b-a|^\alpha$. Now, for any $\epsilon>0$, define $\delta = (\frac{\epsilon}{M})^\frac{1}{\alpha}>0$, 
    then for all $x,y\in (a,b)$, if $|x-y|<\delta$, the following is true:
    $$|f(x)-f(y)|<M|x-y|^\alpha < M\cdot \delta^\alpha$$
    (Note: the above inequality is true, since $\alpha>0$, then $0\leq |x-y|<|b-a|$ implies $|x-y|^\alpha < |b-a|^\alpha$).
    Thus, it can be rewritten as:
    $$|f(x)-f(y)| < M\cdot \delta^\alpha = M\cdot \left(\left(\frac{\epsilon}{M}\right)^\frac{1}{\alpha}\right)^\alpha = M \cdot \frac{\epsilon}{M} = \epsilon$$
    Thus, since for all $\epsilon>0$, there exists $\delta>0$ with $|x-y|<\delta$ implies $|f(x)-f(y)|<\epsilon$, $f$ is uniformly continuous.

    \hfill

    \item[(c)] Consider the function $f:(-1,1)\rightarrow\mathbb{R}$ by $f(x)=|x|$.
    
    Choose $M=1.01$ and $\alpha=1$, then the following is true:
    $$\forall x,y\in (-1,1),\quad |f(x)-f(y)| = \left||x|-|y|\right| \leq |x-y| = |x-y|^\alpha<1.01|x-y|^\alpha = M|x-y|^\alpha$$
    Thus, $f$ is Lipschitz continuous.

    Yet, $f$ is not differentiable at $x=0$: For all $x<0$ and $y>0$ (with $x,y\in(-1,1)$), the following is true:
    $$\frac{f(x)-f(0)}{x-0} = \frac{|x|-0}{x-0} = \frac{-x}{x} = -1$$
    $$\frac{f(y)-f(0)}{y-0}=\frac{|y|-0}{y-0}=\frac{y}{y}=1$$
    Which, $\lim_{x\rightarrow 0}\frac{f(x)-f(0)}{x-0}$ is not defined, since the left and right limits as $x$ approaches $0$ are different.

    \hfill
    
    \item[(d)] Suppose $f$ is differentiable on $(a,b)$ and $f'(x)$ is bounded on $(a,b)$, then there exists $M>0$, with $|f'(x)|<M$ for all $x\in (a,b)$.
    Which, for all $x,y\in (a,b)$ with $x< y$, by the Mean Value Theorem, there exists $c\in (x,y)$, such that
    $f(y)-f(x) = f'(c)(y-x)$. Which, the following is true:
    $$|f(y)-f(x)| = |f'(c)|\cdot|y-x| < M|y-x|$$
    Thus, $f$ is Lipschitz continuous.
    

\end{itemize}

\break

\section*{3}
\begin{myBox}[]{}
    \begin{question}
        For any $a\geq 0$, define $f_a:\mathbb{R}\rightarrow\mathbb{R}$ as:
        $$f_a(x)=\begin{cases}
            x^asin(\frac{1}{x}) & x> 0\\
            0 & x\leq 0
        \end{cases}$$
        (a) For which values of $a$ is $f_a$ continuous at 0.
        
        (b) For which values of $a$ is $f_a'(0)$ defined.
        
        (c) For which values of $a$ is $f_a'$ continuous at 0.
        
        (d) For which values of $a$ is $f_a''(0)$ defined.
    \end{question}
\end{myBox}

\textbf{Pf:}

\begin{itemize}
    \item[(a)] For $a=0$, the function $f_a(x)$ is not continuous: Choose the sequence $(x_n)_{n\in\mathbb{N}}$ by $x_n=\frac{1}{(2n+1/2)\pi} >0$, 
    then $\lim_{n\rightarrow\infty}\frac{1}{(2n+1/2)\pi}=0$, thus $x_n$ converges to $0$; but, consider $(f_a(x_n))_{n\in\mathbb{N}}$:
    $$\forall n\in\mathbb{N},\quad f_a(x_n) = x_n^0\sin\left(\frac{1}{x_n}\right) = \sin\left(\frac{1}{1/(2n+1/2)\pi}\right) = \sin((2n+1/2)\pi) = 1$$
    Which, $\lim_{n\rightarrow\infty}f_a(x_n) = 1\neq 0 = f_a(0)$, thus $f_a(x_n)$ doesn't converge to $f_a(0)$, showing it's not continuous.
    
    \hfill

    Now, for all $a>0$,
    \item[(b)]
    \item[(c)]
    \item[(d)]   
\end{itemize}

\break

\section*{4}
\begin{question}
    
\end{question}

\break

\section*{5}
\begin{question}
    
\end{question}

\end{document}