% Math_118B_HW6_Zih-Yu_Hsieh.tex

\documentclass{article}
\usepackage{graphicx} % Required for inserting images
\usepackage[margin = 2.54cm]{geometry}
\usepackage[most]{tcolorbox}

\newtcolorbox{myBox}[3]{
arc=5mm,
lower separated=false,
fonttitle=\bfseries,
%colbacktitle=green!10,
%coltitle=green!50!black,
enhanced,
attach boxed title to top left={xshift=0.5cm,
        yshift=-2mm},
colframe=blue!50!black,
colback=blue!10
}

\usepackage{amsmath}
\usepackage{amssymb}
\usepackage{verbatim}
\usepackage[utf8]{inputenc}
\linespread{1.2}

\newtheorem{definition}{Definition}
\newtheorem{proposition}{Proposition}
\newtheorem{theorem}{Theorem}
\newtheorem{question}{Question}

\title{Math 118B HW6}
\author{Zih-Yu Hsieh}

\begin{document}
\maketitle

\section*{1}
\begin{myBox}[]{}
    \begin{question}
        Rudin Chapter 5 Exercise 22:

        Suppose $f$ is a real function on $\mathbb{R}$. Call $x$ a \textit{fixed point} of $f$ if $f(x)=x$.
        \begin{itemize}
            \item[(a)] If $f$ is differentiable and $f'(t)\neq 1$ for every real $t$, prove that $f$ has at most one fixed point.
            \item[(b)] Show that the function $f$ defined by 
            $$f(t)=t+(1+e^t)^{-1}$$
            has no fixed point, although $0<f'(t)<1$ for all real $t$.
            \item[(c)] However, if there is a constant $A<1$ such that $|f'(t)|\leq A$ for all real $t$, prove that a fixed point $x$ of $f$ exists,
            and that $x=\lim x_n$, where $x_1$ is an arbitrary real number and
            $$x_{n+1}=f(x_n)$$
            for $n=1,2,3,....$
            \item[(d)] Show that the process described in (c) can be visualized by the zig-zag path
            $$(x_1,x_2)\rightarrow (x_2,x_2)\rightarrow (x_2,x_3)\rightarrow (x_3,x_3)\rightarrow (x_3,x_4)\rightarrow ...$$
        \end{itemize}
    \end{question}
\end{myBox}

\textbf{Pf:}

\begin{itemize}
    \item[(a)] Given $f$ is differentiable and $f'(t)\neq 1$ for all real $t$. Suppose the contrary that $f$ has more than one fixed point,
    there exists distinct $x,y\in\mathbb{R}$ (and WLOG, assume $x<y$), with $f(x)=x$ and $f(y)=y$. However, by Mean Value Theorem, there exists $c\in (x,y)$, such that $f'(c)=\frac{f(y)-f(x)}{y-x}=\frac{y-x}{y-x}=1$,
    which contradicts the assumption that all $t\in\mathbb{R}$ satisfies $f'(t)\neq 1$.

    Hence, the assumption is wrong, $f$ couldn't have more than one fixed point.

    \hfil

    \item[(b)] Given $f(t)=t+(1+e^t)^{-1}$, apply the differentiation rules, we get:
    $$f'(t)=1-(1+e^t)^{-1}\cdot e^t = 1-\frac{e^t}{(1+e^t)^2}$$
    Since for all $t\in\mathbb{R}$, $e^t>0$, so $(1+e^t)>1$ and $(1+e^t)>e^t$. Hence, $0<\frac{e^t}{(1+e^t)^2}<1$ (since everything is positive, while $e^t<(1+e^t)<(1+e^t)^2$).

    Yet, there doesn't exists a fixed point: If consider $f(t)-t$, we get $(1+e^t)^{-1}$. Since $e^t>0$ for all $t\in\mathbb{R}$, then $(1+e^t)>0$, so does $(1+e^t)^{-1}$. Therefore, there doesn't exists $t\in\mathbb{R}$, 
    with $(1+e^t)^{-1} = f(t)-t=0$, so there doesn't exist any fixed point for this function.

    \hfil

    \item[(c)] Suppose there exists $0\leq A<1$ such that $|f'(t)|\leq A$ for all real $t$. Then, for al distinct $x,y\in\mathbb{R}$ (WLOG, assume $x<y$), by Mean Value Theorem,
    there exists $c\in (x,y)$, with $f'(c)(x-y)=(f(x)-f(y))$. So, the following is true:
    $$|f(x)-f(y)| = |f'(c)|\cdot |x-y|\leq A|x-y|$$

    \hfil

    Now, for any $x_1\in\mathbb{R}$, we'll prove by induction that all $n\in\mathbb{N}$, $|x_{n+1}-x_n|\leq A^{n-1}|x_2-x_1|$.

    For base case $n=1$, it's clear that $|x_{1+1}-x_1| = |x_2-x_1| \leq A^{1-1}|x_2-x_1|$.

    Now, suppose for given $n\in\mathbb{N}$, $|x_{n+1}-x_n|\leq A^{n-1}|x_2-x_1|$, then for case $(n+1)$:
    $$|x_{(n+1)+1}-x_{n+1}| = |f(x_{n+1})-f(x_n)| \leq A|x_{n+1}-x_n| \leq A\cdot A^{n-1}|x_2-x_1| = A^{(n+1)-1}|x_2-x_1|$$
    Which, this completes the induction, showing that all $n\in\mathbb{N}$ satisfies $|x_{n+1}-x_n|\leq A^{n-1}|x_2-x_1|$.

    \hfil

    Now, we can prove that the sequence $(x_n)_{n\in\mathbb{N}}$ is a Cauchy sequence in $\mathbb{R}$, therefore converges:

    Given that $0\leq A<1$, then $\frac{1}{1-A}>0$. Now, since $A^{n-1}|x_2-x_1|$ defines a geometric sequence with ratio $0\leq A<1$,
    then $\lim_{n\rightarrow\infty}A^{n-1}|x_2-x_1|=0$. So, for all $\epsilon>0$, since $(1-A)\epsilon>0$, there exists $N$,
    with $n\geq N$ implies $A^{n-1}|x_2-x_1|<(1-A)\epsilon$.

    Now, for all $m>n\geq N$, the following is true:
    $$|x_m-x_n| = \left|\sum_{k=0}^{m-n-1}(x_{n+(k+1)}-x_{n+k})\right| \leq \sum_{k=0}^{m-n-1}|x_{n+(k+1)}-x_{n+k}|$$
    $$|x_m-x_n|\leq \sum_{k=0}^{m-n-1}|x_{n+(k+1)}-x_{n+k}| \leq \sum_{k=0}^{m-n-1}A^{n+k-1}|x_2-x_1|$$
    $$|x_m-x_n|\leq A^{n-1}|x_2-x_1|\sum_{k=0}^{m-n-1}A^{k} \leq A^{n-1}|x_2-x_1|\sum_{k=0}^{\infty}A^k$$
    $$|x_m-x_n|\leq A^{n-1}|x_2-x_1|\cdot\frac{1}{1-A} < (1-A)\epsilon \cdot \frac{1}{1-A} = \epsilon$$
    Since for all $\epsilon>0$, there exists $N$, with $m>n\geq N$ implies $|x_m-x_n|<\epsilon$, hence $(x_n)_{n\in\mathbb{N}}$ is a cauchy sequence,
    which converges to some $x\in \mathbb{R}$.

    Then, since $f$ is differentiable, then $f$ is continuous; hence, the following is true:
    $$\lim_{n\rightarrow\infty}f(x_n)=f\left(\lim_{n\rightarrow\infty}x_n\right)=f(x),\quad \lim_{n\rightarrow\infty}f(x_n)=\lim_{n\rightarrow\infty}x_{n+1}=x$$
    Hence, $f(x)=x$, which any $x_1\in\mathbb{R}$ with $x_{n+1}=f(x_n)$, has the sequential limit being a fixed point $x\in\mathbb{R}$.

    Also, based on the previous part, since all $t\in\mathbb{R}$ has $|f'(t)|\leq A<1$, then by \textbf{Part (a)}, since $f'(t)\neq 1$ for all $t$, $f$ has at most one fixed point.
    Hence, this fixed point is unique, all such sequence $(x_n)_{n\in\mathbb{N}}$ converges to a unique fixed point $x\in\mathbb{R}$.

    \hfil

    \item[(d)] Start with $x_1$, since $x_1$ gets mapped to $x_2$, the point on the graph is $(x_1,x_2)$; then, to let $x_2$ be the next input, we need to match the x-coordinate to $x_2$,
    hence draw a line from $(x_1,x_2)$ to $(x_2,x_2)$. This can then connect to the next point (since the input coordinate is now $x_2$).

    Then, using similar concept, for all integer $k\geq 2$, every input $x_k$ corresponds to output $x_{k+1}$, which we can connect the previous point $(x_k,x_k)$ to $(x_k,x_{k+1})$, 
    which then can connect to $(x_{k+1},x_{k+1})$ so the next input is $x_{k+1}$.

    Under the actual graph, the pattern looks like the following:

    \begin{figure}[h!]
        \begin{center}
            \includegraphics*[width = 100mm]{118 hw6 p1 pd.jpg}
            \caption{The Path Converging to the Fixed Point}
        \end{center}
    \end{figure}
\end{itemize}

\break

\section*{2}
\begin{myBox}[]{}
    \begin{question}
        For $f(x) = \cos(x)$, show that $x_{n+1}=f(x_n)$ defines a convergent sequence
        for arbitrary $x_0\in\mathbb{R}$. Calculate the root $\alpha=\cos(\alpha)$, $\alpha>0$, with an error less than
        $10^{-2}$.
    \end{question}
\end{myBox}

\textbf{Pf:}

For all $x_0\in\mathbb{R}$, since $|x_1|=|cos(x_0)|\leq 1$, then WLOG, we just need to consider the properties of $\cos(x)$ on the domain $[-1,1]$.

For all distinct $x,y\in[-1,1]$ (WLOG, assume $x<y$), since $\cos(x)$ is differentiable on $\mathbb{R}$ (with derivative $-\sin(x)$), by Mean Value Theorem, 
there exists $c\in (x,y)$, such that $(\cos(x)-\cos(y))=-\sin(c)(x-y)$. Also, notice on $[-1,1]$, $|\sin(x)|$ has a maximum at $1$ 
(since $\sin(x)$ is strictly increasing on this domain, hence $-\sin(1)=\sin(-1)\leq \sin(x)\leq \sin(1) < 1$; so $|\sin(x)|\leq \sin(1)$ on $[-1,1]$). Hence:
$$|\cos(x)-\cos(y)|=|-\sin(c)|\cdot|x-y| \leq \sin(1)\cdot|x-y|$$
Then, using this, we can prove inductively that all integer $n\geq 2$ has $|x_{n+1}-x_n| \leq (\sin(1))^{n-1}|x_2-x_1|$:

For base case $n=2$, $|x_{2+1}-x_2| = |\cos(x_2)-\cos(x_1)| \leq \sin(1)\cdot |x_2-x_1|$.

Now, for give $n\geq 2$, if $|x_{n+1}-x_n|\leq (\sin(1))^{n-1}|x_2-x_1|$, then for case $(n+1)$:
$$|x_{(n+1)+1}-x_{(n+1)}|=|\cos(x_{n+1})-\cos(x_n)|\leq \sin(1)\cdot|x_{n+1}-x_n|\leq \sin(1)\cdot (\sin(1))^{n-1}|x_2-x_1|$$
So, this finishes the induction, all $n\geq 2$ satisfies $|x_{n+1}-x_n|\leq (\sin(1))^{n-1}|x_2-x_1|$.

Since $0<\sin(1)<1$, then $\lim_{n\rightarrow\infty}(\sin(1))^{n-1}|x_2-x_1|=0$. Hence, for all $\epsilon>0$ (which $(1-\sin(1))\epsilon>0$), there exists $N$, with $n\geq N$ implies $|x_{n+1}-x_n|\leq (\sin(1))^{n-1}|x_2-x_1|<(1-\sin(1))\epsilon$.

Then, for all $m>n\geq N$, the following summation is true:
$$|x_m-x_n| = \left|\sum_{k=0}^{m-n-1}(x_{n+k+1}-x_{n+k})\right|\leq\sum_{k=0}^{m-n-1}|x_{n+k+1}-x_{n+k}| \leq \sum_{k=0}^{m-n-1}(\sin(1))^{(n+k)-1}|x_2-x_1|$$
$$\leq (\sin(1))^{n-1}|x_2-x_1|\sum_{k=0}^{m-n-1}(\sin(1))^k\leq (\sin(1))^{n-1}|x_2-x_1|\sum_{k=0}^{\infty}(\sin(1))^k $$
$$= (\sin(1))^{n-1}|x_2-x_1|\cdot\frac{1}{1-\sin(1)} < (1-\sin(1))\epsilon\cdot\frac{1}{1-\sin(1)}=\epsilon$$
Since all $\epsilon>0$ has a corresponding $N$, with $m>n\geq N$ implies $|x_m-x_n|<\epsilon$, then the sequence defined by $x_{n+1}=\cos(x_n)$ is in fact a Cauchy Sequence,
hence in the complete space $\mathbb{R}$, it is a convergent sequence.

So, regardless of the choice $x_0$, using the definition $x_{n+1}=\cos(x_n)$ for all $n\in\mathbb{N}$, the sequence $(x_n)_{n\in\mathbb{N}}$ converges.


\hfil

\textbf{Approximation of $\alpha=\cos(\alpha)$:}

Since $\cos(x)$ is infinitely differentiable, while its derivatives are always $\pm\cos(x)$ or $\pm\sin(x)$, then the derivatives are always bounded by $1$.

If do the $4^{th}$ degree taylor polynomial of $\cos(x)$ about $0$, namely $P_4(x)=1-\frac{x^2}{2!}+\frac{x^4}{4!}$, we know for all $y\in\mathbb{R}$,
there exists $c\in \overline{xy}$, with $\cos(y)-P_4(y)=\frac{f^{(5)}(c)}{5!}y^5$. Hence:
$$|\cos(y)-P_4(y)| \leq \left|\frac{f^{(5)}(c)}{5!}y^5\right| \leq \frac{1}{5!}|y^5| \leq 10^{-2}|y^5|$$
(Note: $\frac{1}{5!}=\frac{1}{120}\leq 10^{-2}$).

Then, given $\alpha=\cos(\alpha)$ (which $|\alpha|\leq 1$) and $\alpha>0$, we have $|\alpha-P_4(\alpha)| = |\cos(\alpha)-P_4(\alpha)| \leq 10^{-2}|\alpha^5| \leq 10^{-2}$.
If calculate the root of $x-P_4(x)$ on the domain $[0,1]$, we'll be able to approximate the solution.

Running a Python Program using Newton's Method with 50 iterations, the root is approximately $r=0.73922$.

Which, putting into $\cos(x)$, we get:
$$\cos(0.73922)\approx 0.73899$$
This root is nearly a fixed point for $\cos(x)$.

\begin{figure}[h!]
    \begin{center}
        \includegraphics*[width=120mm]{newtons method.png}
        \caption{Code for Calculation}
    \end{center}
\end{figure}


\break

\section*{3}
\begin{myBox}[]{}
    \begin{question}
        Let $A\subseteq \mathbb{R}^n$ be a convex and bounded set such that $\bar{0}\in A$. 
        Let $T:\overline{A}\rightarrow\overline{A}$ be a function such that 
        $$\forall x,y\in\overline{A},\quad \|T(x)-T(y)\|\leq \|x-y\|$$
        \begin{itemize}
            \item[(a)] Prove that the set of fixed point of $T$, $\{x\in\mathbb{R}^n\ :\ T(x)=x\}$ is convex and nonempty.
            \item[(b)] Give examples showing that the hypotheses of convexity and boundedness of $A$ are essential.
            \item[(c)] Deduce a weaker condition than convexity under which the result still holds.  
        \end{itemize}
    \end{question}
\end{myBox}

\textbf{Pf:}

\begin{itemize}
    \item[(a)] \textbf{Existence of Fixed Point:}
    
    For all $\lambda\in (0,1)$ (so $0<\lambda<1$), consider the function $\lambda T$: 
    
    First, it is well-defined, since for all $x\in \overline{A}$, because $T(x),\bar{0}\in \overline{A}$,
    then by convexity, any $t\in [0,1]$ has $t\cdot T(x)+(1-t)\bar{0} = t\cdot T(x)\in \overline{A}$.
    Hence, since $\lambda \in [0,1]$, then $\lambda T(x)\in \overline{A}$.

    \hfil
    
    Since for all $x,y\in \overline{A}$, the following is satisfied:
    $$\|\lambda T(x)-\lambda T(y)\| = \lambda\|T(x)-T(y)\| \leq \lambda \|x-y\|$$
    Then, by Contraction Principle, $\lambda T$ has a unique fixed point in $\overline{A}$, each $\lambda$ corresponds to a unique $x_\lambda\in \overline{A}$, with $\lambda T(x_\lambda)=x_\lambda$.

    \hfil

    Now, consider a sequence $(\lambda_n)_{n\in\mathbb{N}}\subset (0,1)$ that converges to $1$. If we consider the sequence of fixed point $(x_{\lambda_n})_{n\in\mathbb{N}}\subset \overline{A}$,
    (with respect to each $\lambda_n$), since $\overline{A}$ is closed and bounded, then by Bolzano Weierstrass Theorem, 
    there exists a convergent subsequence $(x_{\lambda_{n_k}})_{k\in\mathbb{N}}$ that converges to some $x_1$, and $x_1\in\overline{A}$ since $\overline{A}$ is closed, it contains all its limit points.

    \hfil

    Now, we can prove that $x_1$ is a fixed point of $T$: Because $T$ is continuous (for all $\epsilon>0$, let $\delta=\epsilon$, then all $x,y\in \overline{A}$ with $\|x-y\|<\delta=\epsilon$ has $\|T(x)-T(y)\|\leq \|x-y\|<\epsilon$),
    then since $\lim_{k\rightarrow\infty}x_{\lambda_{n_k}}=x_1$, then $\lim_{k\rightarrow\infty}T(x_{\lambda_{n_k}})=T(x_1)$.

    Also, recall that each $x_{\lambda_{n_k}}$ is a fixed point for the function $\lambda_{n_k}T$, so $x_{\lambda_{n_k}}=\lambda_{n_k}T(x_{\lambda_{n_k}})$, or $T(x_{\lambda_{n_k}})=\frac{1}{\lambda_{n_k}}x_{\lambda_{n_k}}$.
    (Note: $\frac{1}{\lambda_{n_k}}$ is well-defined, since it is contained in $(0,1)$, so it's never $0$).

    Then, because $(\lambda_n)_{n\in\mathbb{N}}$ converges to $1$, so does its subsequence $(\lambda_{n_k})_{k\in\mathbb{N}}$. Then, since the sequence is never $0$, while the limit is also nonzero, then:
    $$\lim_{k\rightarrow\infty}\frac{1}{\lambda_{n_k}}=\frac{1}{\lim_{k\rightarrow\infty}\lambda_{n_k}} = 1$$
    Because $\overline{A}$ is bounded, there exists $M>0$ (choose $M>2$, and sufficiently large), with all $x\in\overline{A}$, $\|x\|\leq M$.
    
    So, by the above two limits, for all $\epsilon>0$ (for simplicity, modify $M$ from above such that $1>\frac{\epsilon}{2M}>0$), there exists $N_1,N_2$, such that:
    $$k\geq N_1 \implies \|x_{\lambda_{n_k}}-x_1\|<\frac{\epsilon}{2M},\quad k\geq N_2 \implies \left|\frac{1}{\lambda_{n_k}}-1\right|<\frac{\epsilon}{2M}$$
    Which, the second part above also implies that $0<\frac{1}{\lambda_{n_k}}<1+\frac{\epsilon}{2M}$.
    So, for $N=\max\{N_1,N_2\}$, for all $k\geq N$ (so $k\geq N_1,N_2$), we have:
    $$\left\|T(x_{\lambda_{n_k}})-x_1\right\| = \left\|\frac{1}{\lambda_{n_k}}x_{\lambda_{n_k}}-x_1\right\| = \left\|\left(\frac{1}{\lambda_{n_k}}x_{\lambda_{n_k}}-\frac{1}{\lambda_{n_k}}x_1\right)+\left(\frac{1}{\lambda_{n_k}}x_1-x_1\right)\right\|$$
    $$\leq \left|\frac{1}{\lambda_{n_k}}\right|\cdot\left\|x_{\lambda_{n_k}}-x_1\right\|+\left|\frac{1}{\lambda_{n_k}}-1\right|\cdot \|x_1\| < \left(1+\frac{\epsilon}{2M}\right)\frac{\epsilon}{2M} + M\cdot \frac{\epsilon}{2M}$$
    $$\leq 2\cdot \frac{\epsilon}{2M}+\frac{\epsilon}{2} = \frac{\epsilon}{M}+\frac{\epsilon}{2} < \frac{\epsilon}{2}+\frac{\epsilon}{2}=\epsilon$$
    (Note: For the second line, recall that $\frac{\epsilon}{2M}<1$ by our choice; and for the last line, recall that $M>2$).
    Hence, we can conclude that $\lim_{k\rightarrow\infty}T(x_{\lambda_{n_k}})=x_1$ (since all $\epsilon>0$, there exists $N$, with $\|T(x_{\lambda_{n_k}})-x_1\| <\epsilon$). 

    Hence, $\lim_{k\rightarrow\infty}T(x_{\lambda_{n_k}})=x_1$.

    So, by the uniqueness of limit in metric space, $\lim_{k\rightarrow\infty}T(x_{\lambda_{n_k}}) = T(x_1)=x_1$, showing that $x_1$ is a fixed point of $T$.
    Hence, the set of fixed point of $T$ is nonempty.

    \hfil

    \textbf{Convexity of the Set of Fixed Point:}

    Given that two points $x,y\in\overline{A}$ are fixed points of $T$ (i.e. $T(x)=x$ and $T(y)=y$), then for all $t\in [0,1]$, the point $z=tx+(1-t)y$ (on the line segmant $\overline{xy}$), it satisfies the following:
    $$\|T(z)-y\| = \|T(z)-T(y)\| \leq \|z-y\|=\|(tx+(1-t)y)-y\| = t\|x-y\|$$
    $$\|T(z)-x\|=\|T(z)-T(x)\| \leq \|z-x\|=\|(tx+(1-t)y)-x\| = \|(t-1)x+(1-t)y\| = (1-t)\|x-y\|$$
    Now, since distance between $x,y$ satisfies:
    $$\|x-y\| \leq \|T(z)-x\|+\|T(z)-y\| \leq (1-t)\|x-y\| + t\|x-y\| = \|x-y\|$$
    Notice that if $T(z)$ is not colinear with $x,y$, then $\|x-y\|<\|T(z)-x\|+\|T(z)-y\|$ (since triangle inequality is equality iff the two components are nonnegative multiple of each other under vector space;
    if $T(z)$ is not in $\overline{xy}$, then $T(z)-x$ and $T(z)-y$ cannot be scalar nonnegative multiple of each other).
    Hence, $T(z)\in \overline{xy}$, or $T(z)=\lambda x+(1-\lambda)y$ for some $\lambda\in [0,1]$. Which:
    $$\|T(z)-y\| = \|(\lambda x+(1-\lambda)y)-y\| = \lambda\|x-y\|,\quad \|T(z)-x\| = \|(\lambda x+(1-\lambda)y)-x\|=(1-\lambda)\|x-y\|$$

    On the other hand, we can also conclude:
    $$(1-t)\|x-y\| \leq \|x-y\|-\|T(z)-y\| \leq \|T(z)-x\| \leq (1-t)\|x-y\|,\quad \|T(z)-x\| = (1-t)\|x-y\|$$
    $$t\|x-y\| = (1-(1-t))\|x-y\| \leq \|x-y\|-\|T(z)-x\| \leq \|T(z)-y\| \leq t\|x-y\|,\quad \|T(z)-y\|=t\|x-y\|$$
    It implies that $\|T(z)-y\|=\lambda\|x-y\|=t\|x-y\|$, and $\|T(z)-x\|=(1-\lambda)\|x-y\|=(1-t)\|x-y\|$, which enforce the condition $\lambda =t$ (unless $\|x-y\|=0$, but if $x=y$, $z=x=y$ is trivial).
    So, $T(z)=\lambda x+(1-\lambda)y = tx+(1-t)y=z$.

    Hence, $z$ is also a fixed point, which $\overline{xy}$ is a subset of this fixed point. Therefore, we can conclude that this set of fixed point is in fact convex.

    
    \hfil

    \item[(b)] \textbf{Counterexample without convexity:}
    
    Consider the unit circle $S^1 \subset \mathbb{R}^2$, which is not convex (since $(1,0), (-1,0)\in S^1$, yet $\frac{1}{2}(1,0)+(1-\frac{1}{2})(-1,0)=(0,0)\notin S^1$).

    Now, consider $T:S^1\rightarrow S^1$ by $T(x,y)=-(x,y)$. For all $a,b\in S^1$, the given condition is true:
    $$\|T(a)-T(b)\| = \|(-a)-(-b)\| \leq \|a-b\|$$
    Yet, it has no fixed point, since the only point $v\in \mathbb{R}^2$ with $-v = v$ is $v=\bar{0}$, yet $\bar{0}\notin S^1$.
    So, this function has no fixed point.

    \hfil

    \textbf{Couterexample without Boundedness:}

    Consider any $n\in\mathbb{N}$, and any nonzero element $a\in\mathbb{R}^n$. Define $T:\mathbb{R}^n\rightarrow\mathbb{R}^n$ by $T(x)=x+a$.
    For all $x,y\in \mathbb{R}^n$, the condition is satisfied:
    $$\|T(x)-T(y)\| = \|(x+a)-(y+a)\| = \|x-y\|$$
    Yet, since $a\neq \bar{0}$, all $x\in\mathbb{R}^n$ couldn't satisfy $T(x)=x$ (or else $a=\bar{0}$ is a contradiction). Hence, this function also has no fixed point.

    \hfil

    \item[(c)] Instead of convexity, consider a Star-Shaped domain centered at $\bar{0}$: Given a nonempty set $A\subset \mathbb{R}^n$, it is Star-Shaped,
    if there exists a point $a_0\in A$, such that for all $b\in A$, the line segmant $\overline{a_0b}\subset A$. Which, call the point $a_0$ being a center of $A$.

    If given a bounded Star-Shaped domain $A$ that's centered at $\bar{0}$, and $T:\overline{A}\rightarrow\overline{A}$ satisfied $\|T(x)-T(y)\|\leq \|x-y\|$ for all $x,y\in \overline{A}$.
    Then, for all $\lambda\in (0,1)$, $\lambda T$ is a well-defined function again (since for all $x\in \overline{A}$, $\lambda T(x) = \lambda T(x)+(1-\lambda)\cdot \bar{0} \in \overline{A}$, due to the Star-Shaped assumption).
    Since $\lambda T$ satisfied $\|\lambda T(x)-\lambda T(y)\|= \lambda \|T(x)-T(y)\|\leq \lambda\|x-y\|$, then by contraction principle, there exists unique fixed point $x_\lambda\in\overline{A}$ (or $\lambda T(x_\lambda)=x_\lambda$).

    Then, using the same method in \textbf{Part (a)} (since beside the Star-Shaped condition, closed and boundedness of $\overline{A}$ follows), choose a sequence $(\lambda_n)_{n\in\mathbb{N}}\subset (0,1)$ that converges to $1$,
    the corresponding fixed point sequence $(x_{\lambda_n})_{n\in\mathbb{N}}\subset \overline{A}$, which has a convergent subsequence $(x_{\lambda_{n_k}})_{k\in\mathbb{N}}$ converging to some $x_1\in \overline{A}$ due to the closed and boundedness of $\overline{A}$.

    Again, by the continuity of $T$, we have $\lim_{k\rightarrow\infty}T(x_{\lambda_{n_k}})=T(x_1)$, and since $\lambda_{n_k}T(x_{\lambda_{n_k}})=x_{\lambda_{n_k}}$, then $T(x_{\lambda_{n_k}})=\frac{1}{\lambda_{n_k}}x_{\lambda_{n_k}}$.
    Using the same method in \textbf{Part (a)}, we can again deduce:
    $$\lim_{k\rightarrow\infty}T(x_{\lambda_{n_k}}) = \lim_{k\rightarrow\infty}\frac{1}{\lambda_{n_k}}x_{\lambda_{n_k}} = \left(\lim_{k\rightarrow\infty}\frac{1}{\lambda_{n_k}}\right)(\lim_{k\rightarrow\infty}x_{\lambda_{n_k}}) = x_1$$
    Hence, $T(x_1)=x_1$, showing that the fixed point is nonempty.

    So, given a bounded Star-Shaped domain centered at $\bar{0}$, we can still deduce the fact that the function has a nonempty collection of fixed point (though convexity is not guaranteed).
\end{itemize}

\break

\section*{4}
\begin{myBox}[]{}
    \begin{question}
        Let $X=C([0,1]:\mathbb{R})$ be the space of continuous real-valued functions defined in the interval $[0,1]$.
        Prove that for any $\lambda\in (0,1)$ the functional equation
        $$f(t)=\int_{0}^{1}e^{-st}\cos(\lambda f(s))ds$$
        has a unique solution in $X$. Extend this result to the case $\lambda=1$.
    \end{question}
\end{myBox}

\textbf{Pf:}

\textbf{Unique Solution for $\lambda\in (0,1)$:}

First, recall that $X=C([0,1]:\mathbb{R})$ is a Banach Space, a complete normed vector space, so the Contraction Principle works in here.
Define a transformation $T_\lambda:X\rightarrow X$ by $(T_\lambda(f))(t)=\int_{0}^{1}e^{-st}\cos(\lambda f(s))ds$ for all $\lambda\in (0,1)$ and $f\in X$.

Now, notice that by Mean Value Theorem, for all distinct $x,y\in\mathbb{R}$ (assume $x<y$), since the derivative of $\cos(t)$ is given by $-\sin(t)$,
then there exists $c\in (x,y)$, such that $(\cos(x)-\cos(y))=-\sin(c)(x-y)$. Hence, $|\cos(x)-\cos(y)|\leq |-\sin(c)|\cdot|x-y| \leq |x-y|$ (so this inequality can be generalized to all $x,y\in\mathbb{R}$).

Then, for any $f,g\in X$, any $t\in [0,1]$ satisfies the following:
$$|(T_\lambda(f))(t)-(T_\lambda(g))(t)| = \left|\int_{0}^{1}e^{-st}\cos(\lambda f(s))ds-\int_{0}^{1}e^{-st}\cos(\lambda g(s))ds\right|$$
$$ = \left|\int_{0}^{1}e^{-st}(\cos(\lambda f(s))-\cos(\lambda g(s)))ds\right| \leq \int_{0}^{1}|e^{-st}|\cdot |\cos(\lambda f(s))-\cos(\lambda g(s))|ds$$
$$\leq \int_{0}^{1}|\lambda f(s)-\lambda g(s)|ds = \lambda \int_{0}^{1}|f(s)-g(s)|ds$$
Since the variable $s\in [0,1]$ (domain of all functions in the function space $X$), then $|f(s)-g(s)| \leq \|f-g\|_\infty$, hence the above inequality can be rewrite as:
$$|(T_\lambda(f))(t)-(T_\lambda(g))(t)| \leq \lambda\int_{0}^{1}|f(s)-g(s)|ds \leq \lambda\int_{0}^{1}\|f-g\|_\infty ds = \lambda\|f-g\|_\infty$$
And, since the above inequality is true for all $t\in [0,1]$, then in fact $\|T_\lambda(f)-T_\lambda(g)\| \leq \lambda \|f-g\|_\infty$.

Now, because all $f,g\in X$ satisfies $\|T_\lambda(f)-T_\lambda(g)\|_\infty \leq \lambda\|f-g\|_\infty$ while $\lambda<1$, then by contraction principle, there exists a unique $f_\lambda \in X$,
such that $T_\lambda(f_\lambda)=f_\lambda$. Or, $f_\lambda\in X$ is the unique equation satisfying:
$$f_\lambda(t)=\int_{0}^{1}e^{-st}\cos(\lambda f_\lambda(s))ds$$

\hfil

\textbf{Extension to $\lambda=1$:}

Before starting, since for all $\lambda\in (0,1)$ and all $t\in [0,1]$, the following inequality is satisfied:
$$|f_\lambda(t)| = \left|\int_{0}^{1}e^{-st}\cos(\lambda f_\lambda(s))ds\right|\leq\int_{0}^{1}|e^{-st}|\cdot |\cos(\lambda f_\lambda(s))|ds \leq \int_{0}^{1}1 ds = 1$$
Then, we can conclude that $\|f_\lambda\|_\infty \leq 1$.

\hfil

Now, consider a sequence $(\lambda_n)_{n\in\mathbb{N}}\subset (0,1)$ that converges to $1$, and consider the corresponding sequence of functions $\{f_{\lambda_n}\}_{n\in\mathbb{N}}$,
our goal is to apply Arzela-Ascoli Theorem.

First, we know the domain $[0,1]$ for all these functions are bounded, and the above statement proved that the sequence of function is uniformly bounded by $1$, so the remaining condition is to prove that the sequence of function is equicontinuous.

Recall that the function $e^{x}$ is a continuous function at $x=0$, so for all $\epsilon>0$, there exists $\delta>0$, such that $|x|<\delta$ implies $|1-e^x|<\epsilon$.
Then, using the same $\delta$, for all $n\in\mathbb{N}$, any $t_1,t_2\in [0,1]$ satisfying $|t_1-t_2|<\delta$, we have:
$$|f_{\lambda_n}(t_1)-f_{\lambda_n}(t_2)| = \left|\int_{0}^{1}e^{-st_1}\cos(\lambda_n f_{\lambda_n}(s))ds-\int_{0}^{1}e^{-st_2}\cos(\lambda_n f_{\lambda_n}(s))ds\right|$$
$$ = \left|\int_{0}^{1}(e^{-st_1}-e^{-st_2})\cos(\lambda_n f_{\lambda_n}(s))ds\right| \leq \int_{0}^{1}|e^{-st_1}-e^{-st_2}|\cdot|\cos(\lambda_n f_{\lambda_n}(s))|ds$$
$$\leq \int_{0}^{1}|e^{-st_1}|\cdot|1-e^{st_1-st_2}|ds \leq \int_{0}^{1}|1-e^{s(t_1-t_2)}|ds$$
For all $s\in [0,1]$, since $0\leq s|t_1-t_2|\leq |t_1-t_2|<\delta$, then by continuity of $e^{x}$, we know $|1-e^{s(t_1-t_2)}| < \epsilon$.
Hence, the above inequality becomes:
$$|f_{\lambda_n}(t_1)-f_{\lambda_n}(t_2)| \leq \int_{0}^{1}|1-e^{s(t_1-t_2)}|ds < \int_{0}^{1}\epsilon ds = \epsilon$$
Since regardless of $n\in\mathbb{N}$, every $\epsilon>0$ has a corresponding $\delta>0$, with $|t_1-t_2|<\delta$ implies $|f_{\lambda_n}(t_1)-f_{\lambda_n}(t_2)|<\epsilon$,
then this concludes that the sequence of functions $\{f_{\lambda_n}\}_{n\in\mathbb{N}}$ is in fact equicontinuous.

Since all three conditions are satisfied, by Arzela-Ascoli Theorem, there exists a convergent subsequence $\{f_{\lambda_{n_k}}\}_{k\in\mathbb{N}}\subset \{f_{\lambda_n}\}_{n\in\mathbb{N}}$, define $f\in X$ to be the subsequential limit of $\{f_{\lambda_{n_k}}\}_{k\in\mathbb{N}}$.

\hfil

Finally, we can prove that $f(t)=\int_{0}^{1}e^{-st}\cos(f(s))ds$ (a solution for $\lambda=1$):

Since $\{f_{\lambda_{n_k}}\}_{k\in\mathbb{N}}$ converges to $f$ uniformly, then for all $\epsilon>0$ (which $\frac{\epsilon}{2}>0$), there exists $K$, with $k\geq K$ implies $\|f-f_{\lambda_{n_k}}\|_\infty <\frac{\epsilon}{2}$.

Also, since $\{\lambda_n\}_{n\in\mathbb{N}}$ converges to $1$, then the subseuqnce $\{\lambda_{n_k}\}_{k\in\mathbb{N}}$ also converges to $1$. Hence, for the given $\epsilon>0$ above, there exists $K'$,
with $k\geq K'$ implies $|1-\lambda_{n_k}|<\frac{\epsilon}{2}$.

Now, let $g\in X$ be defined as $g(t)=\int_{0}^{1}e^{-st}\cos(f(s))ds$. 

Choose $N=\max\{K,K'\}$, for all $k\geq N$ (which $k\geq K',K$), then for all $t\in [0,1]$, it satisfies:
$$|g(t)-f_{\lambda_{n_k}}(t)| = \left|\int_{0}^{1}e^{-st}\cos(f(s))ds - \int_{0}^{1}e^{-st}\cos(\lambda_{n_k}f_{\lambda_{n_k}}(s))ds\right|$$
$$= \left|\int_{0}^{1}e^{-st}(\cos(f(s))-\cos(\lambda_{n_k}f_{\lambda_{n_k}}(s)))ds\right|\leq \int_{0}^{1}|e^{-st}|\cdot |\cos(f(s))-\cos(\lambda_{n_k}f_{\lambda_{n_k}}(s))|ds$$
$$\leq \int_{0}^{1}|f(s)-\lambda_{n_k}f_{\lambda_{n_k}}f(s)|ds \leq \int_{0}^{1}\|f-\lambda_{n_k}f_{\lambda_{n_k}}\|_\infty ds = \|f-\lambda_{n_k}f_{\lambda_{n_k}}\|_\infty$$
Which, the above term can be rewrite as:
$$\|f-\lambda_{n_k}f_{\lambda_{n_k}}\|_\infty = \|(f-f_{\lambda_{n_k}})+(f_{\lambda_{n_k}}-\lambda_{n_k}f_{\lambda_{n_k}})\|_\infty$$
$$\leq \|f-f_{\lambda_{n_k}}\|_\infty +\|(1-\lambda_{n_k})f_{\lambda_{n_k}}\|_\infty < \frac{\epsilon}{2}+|1-\lambda_{n_k}|\cdot \|f_{\lambda_{n_k}}\|_\infty$$
$$ < \frac{\epsilon}{2} + \frac{\epsilon}{2}\cdot \|f_{\lambda_{n_k}}\|_\infty \leq \frac{\epsilon}{2}+\frac{\epsilon}{2}=\epsilon$$
(Note: Recall that $\|f_{\lambda_{n_k}}\|_\infty \leq 1$).

Hence, the above combined to be the below inequality:
$$|g(t)-f_{\lambda_{n_k}}(t)| \leq \|f-\lambda_{n_k}f_{\lambda_{n_k}}\|_\infty < \epsilon$$
Which further implies that $\|g-f_{\lambda_{n_k}}\|_\infty \leq \epsilon$. Since all $\epsilon>0$ has an $N$, such that $k\geq N$ implies $\|g-f_{\lambda_{n_k}}\|_\infty \leq \epsilon$,
then $g$ is the limit of the subsequence $\{f_{\lambda_{n_k}}\}_{k\in\mathbb{N}}$. Which, because under metric space, the limit is unique, therefore $g=f$.

Hence, we can conclude that $f(t)=g(t)=\int_{0}^{1}e^{-st}\cos(f(s))ds$. For, $\lambda=1$, $f$ is a solution for the given functional equation.

\hfil

\hfil

\section*{5}
\begin{myBox}[]{}
    \begin{question}
        Let $K\subset \mathbb{R}^n$ be a compact set. Suppose that $T:K\rightarrow K$ satisfies
        $$\forall x,y\in K,\quad \|T(x)-T(y)\|<\|x-y\|$$
        Show that there exists a unique $x_0\in K$ such that $T(x_0)=x_0$.
    \end{question}
\end{myBox}

\textbf{Pf:}

\textbf{Existence:}

First, we'll verify that the map $D:K\rightarrow\mathbb{R}$ by $D(x)=\|x-T(x)\|$ is continuous:

For all $x,y\in K$, given any $\epsilon>0$, if chosen $\delta=\frac{\epsilon}{2}>0$, then for $\|x-y\|<\delta=\frac{\epsilon}{2}$, we have $\|T(x)-T(y)\|<\|x-y\|<\frac{\epsilon}{2}$.
Hence, the above function $D$ satisfies:
$$\|D(x)-D(y)\|=\|(x-T(x))-(y-T(y))\| = \|(x-y)+(T(y)-T(x))\|$$
$$\|D(x)-D(y)\| \leq \|x-y\| + \|T(x)-T(y)\| < 2\|x-y\| < 2\cdot\frac{\epsilon}{2}=\epsilon$$
Hence, we can conclude that $D$ is continuous on $K$, a compact set. Which, the image $D(K)\subset \mathbb{R}$ is compact, therefore a minmum $\lambda=\min D(K)$ exists.

Notice that $\lambda \geq 0$, since $D(x)\geq 0$ for all $x\in K$; also, we know there exists $x_0\in K$, with $D(x_0)=\lambda$ by the definition of minimum.

Now, to prove that $\lambda=0$, suppose $\lambda \neq 0$ (which $\lambda>0$) for the sake of contradiction. Then, notice that $x_0$ and $T(x_0)$ satisfies:
$$D(T(x_0))=\|T(x_0)-T(T(x_0))\|<\|x_0-T(x_0)\| = D(x_0)=\lambda$$
Which, $D(T(x_0))<D(x_0)$, while $D(x_0)=\lambda$ is assumed to be the minimum of the set $D(K)$ (which $D(T(x_0)))\in D(K)$). So, this is a contradiction,
hence the initial assumption must be false. Therefore, $\lambda=0$. Hence, there exists $x_0\in K$, with $D(x_0)=\|x_0-T(x_0)\|=0$, so $T(x_0)=x_0$.

\hfil

\textbf{Uniqueness:}

Suppose the contrary that there exists more than one fixed point (let $x_0,y_0\in K$ be two distinct fixed points). Then, $\|x_0-y_0\|=\|T(x_0)-T(y_0)\|<\|x_0-y_0\|$ is a contradiction.
Therefore, the assumption is false, there must have at most one fixed point. And by the existence argument, we know there exists a unique fixed point.

\hfil

\hfil

\section*{6}
\begin{myBox}[]{}
    \begin{question}
        Let $K\subset \mathbb{R}^n$
        be a compact set and $f : K \rightarrow K$ be a function such that
        $$\|f(x)-f(y)\|=\|x-y\|,\quad \forall x,y\in K$$
        Show that $f$ is a bijection.
    \end{question}
\end{myBox}

\textbf{Pf:}

\textbf{$f$ is Injective:}

For all $x,y\in K$, suppose $f(x)=f(y)$, then since $0=\|f(x)-f(y)\|=\|x-y\|$, then $x=y$ is enforced. Hence, this proves injectivity.

\hfil

\textbf{$f$ is Surjective:}

Suppose the contrary, that $f$ is not surjective (so, $f(K)\subsetneq K$).

First, since for all $\epsilon>0$, choose $\delta=\epsilon$, all $x,y\in K$ with $\|x-y\|<\delta=\epsilon$ satisfies $\|f(x)-f(y)\|=\|x-y\|<\epsilon$,
hence $f$ is uniformly continuous on $K$. Then, because $K$ is compact, then $f(K)$ is also compact, which is closed and bounded.

Now, since $K\setminus f(K)\neq \emptyset$ based on assumption, there exists $x_0\in K\setminus f(K)$. Which, because the sets $\{x_0\}$ and $f(K)$ are both compact (which are both closed),
while the two sets are disjoint, then by \textbf{HW 1 Question 3} (part from \textbf{Rudin Chapter 4 Question 21}), in any metric space,
disjoint closed set $C$ and compact set $K$ always have $\inf\{d(x,y)\ |\ x\in C,y\in K\}>0$ (a positive distance between sets $C$ and $K$).
So, apply this to the two sets, there exists $\lambda>0$, such that all $y\in f(K)$ satisfies $\|x_0-y\| = d(x_0,y) \geq \lambda$.

\hfil

Then, define $f_0(x)=x$, $f_1(x)=f(x)$, and for all integer $n\geq 1$, $f_{n+1}(x)=f(f_n(x))$. 
(Note: inductively, we can also prove that all $m,n\in\mathbb{N}$ satisfies $f_m(f_n(x))=f_{m+n}(x) = f_n(f_m(x))$).

With this definition, we can prove by induction that for all $n\in\mathbb{N}$,
all $y\in f_{n}(K)$ satisfies $\|f_n(x_0)-f(y)\|\geq \lambda$.

For base case $n=1$, recall that for all $y\in f_1(K) = f(K)$, because all $y\in f(K)$ satisfies $\|x_0-y\|\geq \lambda$, since $f$ preserves distance, we have:
$$\|f_1(x_0)-f(y)\| =\|f(x_0)-f(y)\| = \|x_0-y\| \geq \lambda$$
Hence, all $y\in f_1(K)$ satisfies $\|f_1(x_0)-f(y)\|\geq \lambda$, the claim is true for $n=1$.

Now, suppose for given $n\in\mathbb{N}$, all $y\in f_{n}(K)$ satisfies $\|f_n(x_0)-f(y)\|\geq \lambda$.
Then, for all $y\in f_{n+1}(K) = f(f_n(K))$, there exists $x\in f_n(K)$, with $f(x)=y$. Which, by induction hypothesis, $\|f_n(x_0)-y\|=\|f_n(x_0)-f(x)\| \geq \lambda$. 
Hence, the following inequality is true:
$$\|f_{n+1}(x_0)-f(y)\| = \|f(f_n(x_0))-f(y)\| = \|f_n(x_0)-y\| \geq \lambda$$
Which, all $y\in f_{n+1}(K)$ satisfies $\|f_{n+1}(x_0)-f(y)\|\geq \lambda$, completing the induction.

\hfil

Lastly, consider the sequence defined recursively as $x_n=f_n(x_0)$ for all $n\in\mathbb{N}$. Then, since $f$ restrict the element to still be in $K$, then $(x_n)_{n\in\mathbb{N}}\subset K$, a compact set (which is closed and bounded).
Hence, by Bolzano Weierstrass Theorem, since $(x_n)_{n\in\mathbb{N}}$ is bounded, there exists a convergent subsequence $(x_{n_k})_{k\in\mathbb{N}}$, which this subsequence is Cauchy.

Then, given $\lambda>0$, there exists $N\in\mathbb{N}$, such that all $p\geq N$ implies $\|x_{n_p}-x_{n_{p+1}}\| < \lambda$, by the definition of Cauchy Sequence.

However, since $n_{p+1}=n_p+k$ for some $k\in\mathbb{N}$, looking back at the definition, $x_{n_p}=f_{n_p}(x_0)$, while $x_{n_{p+1}}=x_{n_p+k}=f_{n_p+k}(x_0) = f_k(f_{n_p}(x_0))$.

Because $k\geq 1$, then $f_k(x) = f(f_{k-1}(x))$, so $x_{n_{p+1}}=f_k(f_{n_p}(x_0))=f(f_{k-1}(f_{n_p}(x_0))) = f(f_{n_p}(f_{k-1}(x_0)))$.

So, let $y=f_{n_p}(f_{k-1}(x_0))\in f_{n_p}(K)$, by the previous claim, the following inequality is true:
$$\|x_{n_p}-x_{n_{p+1}}\| = \|f_{n_p}(x_0)-f(f_{n_p}(f_{k-1}(x_0)))\| = \|f_{n_p}(x_0)-f(y)\| \geq \lambda$$
Yet, this contradicts the statement that $\|x_{n_p}-x_{n_{p+1}}\| < \lambda$.

\hfil

Since we eventually reach a contradiction, then the assumption must be false, so $f$ needs to be surjective.

\hfil

The above two sections proved that $f$ is in fact a bijection.

\end{document}